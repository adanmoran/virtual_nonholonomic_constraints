%! TEX root = main.tex

%/========== Virtual Nonholonomic Constraints ==========/%

\chapter{Development of Virtual Nonholonomic Constraints}
\section{Preliminaries on Analytical Mechanics}
A mechanical system can be represented by \(N\) point masses, where each point
represents the center of mass of a physical body, along with \(r\)
\textit{equations of constraint} (EOC) which model the physical restrictions
between these masses.
The position of each point mass is described using three cartesian coordinates (one
for each spatial axis), so the system as a whole can be described by a vector in
\(\R^{3N}\) with \(r\) EOC. 
The dynamics of the system are computed by deriving the \(3N\)
\textit{equations of motion} (EOM) produced by Newton's second law \(F = m a\).
While this technique works for simple systems, it is impossible to
apply to a majority of mechanical systems since most of the forces 
on the system are not explicitly known. 

Rather than modeling a mechanical system with point masses and constraints,
it is often feasible to represent the position of the system using \(n\)
independent scalar-valued variables \(q_1,\ldots,q_n\) called 
\textit{generalized coordinates}, where \(n = 3N - m\) is the number of
\textit{degrees of freedom} (DOF) of the system \cite{greenwood_dynamics}.

Each generalized coordinate \(q_i\) takes values in \(\Rt{T_i}\), where
\(T_i = \infty\) if \(q_i\) represents a length or \(T_i = 2\pi\) if \(q_i\)
represents an angle.
It is convention to collect the coordinates into a \textit{configuration} 
\(q = (q_1,\ldots,q_n) \in \mathcal{Q}\) 
where the \textit{configuration manifold} \(\mathcal{Q}\) of the system has the
following structure:
\[
    \mathcal{Q} = \Rt{T_1} \times \cdot \times \Rt{T_n}
\] 
The derivative \(\dot{q} = (\dot{q}_1,\ldots,\dot{q}_n)\) of a configuration
is called a \textit{generalized velocity}, and the combined vector
\((q,\dot{q})\) is called a \textit{state} of the system.

The field of analytical mechanics provides a computational method for finding
the EOM of a system in generalized coordinates. The two most common analytical
methods for modelling robotic systems are \textit{Lagrangian} and
\textit{Hamiltonian} mechanics.

% ---------- Lagrangian Mechanics ---------- % 
\subsection{Lagrangian Mechanics}

Lagrangian mechanics uses the kinetic energy \(T(q,\dot{q})\) and potential
energy \(P(q)\) of the system to define the Lagrangian 
\(\mathcal{L} : \mathcal{Q}\times\R^n \rightarrow \R\) defined by
(\ref{eqn:lagrangian-general}) \cite{greenwood_dynamics}.
\begin{equation}\label{eqn:lagrangian-general}
    \mathcal{L}(q,\dot{q}) = T(q,\dot{q}) - P(q)
\end{equation}
When the mechanical system is actuated, the EOM are described by \(n\) second-order
ordinary differential equations (ODEs) obtained from the \textit{Euler-Lagrange
equations} with \textit{generalized input forces} \(\tau \in \R^k\) 
(\ref{eqn:el-eqns-general}). 
\begin{equation}\label{eqn:el-eqns-general}
    \diff{}{t}\left\{ \pdiff{\mathcal{L}}{\dot{q}_i} \right\}
    - \pdiff{\mathcal{L}}{q_i} = B_i\tpose(q) \tau
\end{equation}
The vector \(B_i\tpose: \mathcal{Q} \rightarrow \R^{1\times k}\) describes how
the input forces shape the dynamics of \(q_i\).
The matrix  \(B: \mathcal{Q} \rightarrow \R^{n \times k}\) with
\[
    B(q) = [B_i\tpose(q)]
\]
is called the \textit{input matrix} for the system.
If \(k < n\), we say the system is \textit{underactuated} with degree of
underactuation \(n - k\).

Many actuated mechanical systems have quadratic kinetic energies, so that the
Lagrangian can be written explicitly as
\begin{equation}\label{eqn:lagrangian}
    \mathcal{L}(q,\dot{q}) = \frac{1}{2} \dot{q}\tpose D(q) \dot{q} - P(q)
\end{equation}
where the \textit{inertia matrix} \(D: \mathcal{Q} \rightarrow \R^{n\times n}\) 
is a symmetric, positive definite matrix for all \(q \in \mathcal{Q}\) and the
potential function \(P : \mathcal{Q} \rightarrow \R\) is smooth. 
%If this is the case, the Euler-Lagrange equations reduce to (\ref{eqn:el-eqns}),
%\begin{equation}\label{eqn:el-eqns}
%    D(q)\ddot{q} + C(q,\dot{q})\dot{q} + \nabla P(q) = B(q)\tau
%\end{equation}
%where the \textit{Coriolis matrix} \(C(q,\dot{q})\) is of the form
%\[
%    [C]_{i,j} = \frac{1}{2}\sum\limits_{k = 1}^n 
%    \left(\pdiff{D_{i,j}}{q_k}  +
%     \pdiff{D_{i,k}}{q_j} -
%     \pdiff{D_{k,j}}{q_i}\right)\dot{q}_k
%\]

% ---------- Hamiltonian Mechanics ---------- %
\subsection{Hamiltonian Mechanics}
Hamiltonian mechanics provides an equivalent representation of the EOM
by converting the \(n\) second-order ODEs generated by Lagrangian mechanics into
\(2n\) first-order ODEs.

To do this, we first define the \textit{conjugate of momentum \(p_i\) to \(q_i\)} by
\begin{equation}\label{eqn:p-i}
    p_i(q,\dot{q}) = \pdiff{\mathcal{L}}{\dot{q}_i}(q,\dot{q})
\end{equation}
To ease notation, we will write \(p = (p_1, \ldots, p_n)\) and say that \(p\) is
the \textit{conjugate of momenta to \(q\)}. 
Note that each \(p_i\) is a linear function of \(\dot{q}\), so one can typically
solve for \(\dot{q}(q,p)\) by inverting the expressions from (\ref{eqn:p-i}).
The combined vector \((q,p)\) is called the \textit{phase} of the system.

Next we define the \textit{Hamiltonian} (\ref{eqn:hamiltonian-general}) 
as the Legendre transform of the Lagrangian \cite{landau_mechanics}.
\begin{equation}\label{eqn:hamiltonian-general}
    \mathcal{H}(q,p) = p\tpose \dot{q}(q,p) - \mathcal{L}(q,\dot{q}(q,p))
\end{equation}
The EOM in this framework can be shown to be the \(2n\)
first-order equations called \textit{Hamilton's equations}
\begin{equation}\label{eqn:hamiltons-eqns}
    \begin{cases}
        \dot{q} = \nabla_p\mathcal{H} \\
        \dot{p} = -\nabla_q\mathcal{H} + B(q)\tau \\
    \end{cases}
\end{equation}
where \(B(q) \in \R^{n\times k}\) is the same input matrix used by the
Lagrangian and \(\tau \in \R^k\) is the vector of generalized input forces.

If the kinetic energy is quadratic as in (\ref{eqn:lagrangian}), the conjugate of
momenta to \(q\) can be computed explicitly:
\[
    p = D(q)\dot{q}
\]
The resulting Hamiltonian system reduces to (\ref{eqn:hamiltonian}).
\begin{align}\label{eqn:hamiltonian}
    \mathcal{H}(q,p) &= \frac{1}{2} p\tpose D^{-1}(q) p + P(q) \\
                     &\begin{cases}
        \dot{q} = D^{-1}(q)p \\
        \dot{p} = -\frac{1}{2} \left[p\tpose \pdiff{D^{-1}}{q_i}(q) p\right] 
        - \nabla_q P(q) + B(q) \tau \\
    \end{cases} 
\end{align}

A pair of coordinates \((q,p)\) which satisfy Hamilton's equations 
under the Hamiltonian \(\mathcal{H}\) are
said to be \textit{canonical coordinates for} \(\mathcal{H}\).  A change of
coordinates \((q,p) \rightarrow (Q,P)\) is said to be a \textit{canonical
transformation} if \((Q,P)\) are canonical coordinates for
\(H(Q,P) = \mathcal{H}\left(q(Q,P), p(Q,P)\right)\).


\section{Simply Actuated Hamiltonian Systems}
% TODO: Describe the change of coordinates to get into q_u/q_a mode and show
% that the actuator directly affects pa but not pu. Use M for simply actuated
% inertia, D for normal coordinates
Given a Hamiltonian mechanical system (\ref{eqn:hamiltonian}), it is not obvious
how the input forces \(\tau\) affect the conjugate of momenta \(p_i\). 
This is because \(\tau\) is transformed by the input matrix \(B(q)\), which may
be quite complicated. 

We will define a new class of Hamiltonian systems where the effect of the input
forces on the conjugate of momenta is made obvious. This class of systems will
form the backbone for the rest of the theory developed in this thesis.

\begin{defn}
    A pair of canonical coordinates \((q,p)\) for \(\mathcal{H}\) are said to be
    \textit{simply actuated coordinates} if, in these coordinates, the input
    matrix \(B(q) \in \R^{n \times k}\) is of the form
    \[
        B(q) = \begin{bmatrix}
            \Zmat{(n-k)\times k} \\
            \Id{k}
        \end{bmatrix}
    \]
\end{defn}
\begin{defn}
    A Hamiltonian system is said to be \textit{simply actuated} if there exists
    a canonical transformation into simply actuated coordinates for
    \(\mathcal{H}\).
\end{defn}

Under the following assumptions on the input matrix, we will show that the
Hamiltonian system (\ref{eqn:hamiltonian}) is simply actuated.

\begin{assm}\label{assm:B-const}
    The input matrix \(B(q) \equiv B \in \R^{n\times k}\) is constant,
    full rank, and \(k \leq n\).
\end{assm}
\begin{assm}\label{assm:B-perp}
    There exists a matrix 
    \(B^\perp \in \R^{(n-k)\times n}\)
    which is right semi-orthogonal 
    (\ie \(B^\perp(B^\perp)\tpose = \Id{(n-k)}\))
    and which is a left-annihilator for \(B\)
    (\ie \(B^\perp B = \Zmat{(n-k) \times k}\)).
\end{assm}

Note that if \(k = (n-1)\), the existence of a left annihilator \(A^0\) for
\(B\) implies \(B^\perp := A^0/\norm{A^0}\) satisfies Assumption
\ref{assm:B-perp}.

\begin{assm}\label{assm:B-orthogonal}
    Assume without loss of generality that the input matrix \(B\) is left
    semi-orthogonal.
    That is, assume \(B\tpose B = \Id{k}\).
\end{assm}
\begin{proof}
Since \(B\) is a constant matrix, 
it has a singular-value decomposition 
\(B = U \Sigma V\tpose\) where \(U^{-1} = U\tpose \in \R^{n \times n}\), 
\(V^{-1} = V\tpose \in \R^{k \times k}\), and \(\Sigma \in \R^{n \times k}\) is
defined by
\[
    \Sigma = \begin{bmatrix}
        \sigma_1 & 0 & \cdots & 0 & 0 \\
        0 & \sigma_2 & \cdots & 0 & 0 \\
        \vdots & & \ddots &  & \vdots \\
        0 & 0 & \cdots & 0 & \sigma_k \\
          & - & \Zmat{(n-k)\times k} & -  &\\
    \end{bmatrix}
\]
where \(\sigma_i \neq 0\) because \(B\) is full-rank.
\textbf{SOURCE}
Defining \(T \in \R^{k \times k}\) by
\[
    T = \begin{bmatrix}
        \frac{1}{\sigma_1} & 0 & \cdots & 0 \\
        0 & \frac{1}{\sigma_2} & \cdots & 0 \\
    \vdots & & \ddots & \vdots \\
    0 & 0 & \cdots & \frac{1}{\sigma_k} \\
    \end{bmatrix}
\]
and assigning the input forces to \(\tau = V T \hat{\tau}\), we get a new input
matrix for \(\hat{\tau} \in \R^k\) given by \(\hat{B} = B V T = U \Sigma T\) 
which is still constant and full-rank. In particular, 
\(\hat{B}\tpose \hat{B} = T\tpose \Sigma\tpose \Sigma\tpose T = \Id{k}\).
\end{proof}

Let \(\mathbf{B} \in \R^{n\times n}\) be the following matrix:
\[
    \mathbf{B} = 
    \begin{bmatrix}
        B^\perp \\
        B\tpose \\
    \end{bmatrix}
\]
Since \(B^\perp\) is a left annihilator of \(B\) and both \(B^\perp\) and
\(B\tpose\) are right semi-orthogonal, it is easy to show that \(\mathbf{B}\) is
orthogonal:
\[
    \mathbf{B}\mathbf{B}\tpose = 
    \begin{bmatrix}
        B^\perp (B^\perp)\tpose & B^\perp B \\
        (B^\perp B)\tpose & B\tpose B
    \end{bmatrix} = \Id{n}
    \Rightarrow
    \mathbf{B}^{-1} = \mathbf{B}\tpose
\]

\begin{defn}
    Define the coordinate transformation 
    \((q_u,q_a) = \mathbf{B}q\), where \(q_u = B^\perp q\)
    and \(q_a = B\tpose q\).
    We call \(q_u\) the \textit{unactuated coordinates} and \(q_a\) the
    \textit{actuated coordinates} of the system. 
    Their corresponding conjugate of momenta are 
    \((p_u,p_a) = \mathbf{B}p\).
\end{defn}

The following theorem shows that these new coordinates are simply actuated
coordinates, so that \(\tau_i\) only affects \(p_{a,i}\)
and all \(p_u\) are unaffected by the input forces. 

\begin{thm}
    Under Assumptions \ref{assm:B-const},\ref{assm:B-perp}, and
    \ref{assm:B-orthogonal}, the change of coordinates 
    \((Q = (q_u, q_a), P = (p_u,p_a))\) 
    is a canonical transformation of the
    original Hamiltonian system (\ref{eqn:hamiltonian}) into
    the simply actuated Hamiltonian system (\ref{eqn:simple-hamiltonian}),
    \begin{align}\label{eqn:simple-hamiltonian}
        \hat{H}(Q,P) &= 
        \frac{1}{2} P\tpose \Minv(Q) P + V(Q) \\
       &\begin{cases}
            \dot{Q} = \Minv(Q)P \\
            \dot{P} = -\frac{1}{2}
                \left[P\tpose \pdiff{\Minv}{Q_i}(Q) P \right] 
                - \nabla_Q V(Q) + 
                \begin{bmatrix}
                    \Zmat{(n-k)\times k} \\
                    \Id{k} \\
                \end{bmatrix} \tau
        \end{cases} \nonumber
    \end{align}
    where
    \begin{align*}
        \Minv(Q) &= \mathbf{B}D^{-1}(\mathbf{B}\tpose Q)\mathbf{B}\tpose \\
        V(Q) &= P(\mathbf{B}\tpose Q) \\
    \end{align*}
\end{thm}
\begin{proof}
    For any constant matrix \(A\), the transformation
    \((Q = Aq, P = Ap)\) satisfies
    \(\pdiff{Q_i}{p_m} = \pdiff{P_i}{q_m} = 0\) for all 
    \(i,m \in \{1,\ldots,n\}\). Hence, the Poisson brackets between the
    new coordinates are zero:
    \begin{align*}
        [Q_i,Q_j] &:= \sum\limits_{m = 1}^n \pdiff{Q_i}{p_m}\pdiff{Q_j}{q_m} - 
        \pdiff{Q_i}{q_m}\pdiff{Q_j}{p_m} = 0 \\
        [P_i,P_j] &:= \sum\limits_{m=1}^n \pdiff{P_i}{p_m}\pdiff{P_j}{q_m} -
        \pdiff{P_i}{q_m}\pdiff{P_j}{p_m} = 0
    \end{align*}
    Since the matrix \(A = \mathbf{B}\) is invertible and orthogonal,
    \((A_i)\tpose A\tpose_j = (A_i)\tpose (A^{-1})_j = \delta_{i,j}\). Using this
    fact we see that the Poisson brackets between \(P\) and \(Q\) are given by:
    \begin{align*}
        [P_i,Q_j] &= \sum\limits_{m=1}^n\pdiff{P_i}{p_m}\pdiff{Q_j}{q_m}
        - \pdiff{P_i}{q_m}\pdiff{Q_j}{p_m} \\
                  &= \sum\limits_{m=1}^n A_{i,m}A_{j,m} - 0 \\
                  &= \sum\limits_{m=1}^n A_{i,m}A\tpose_{m,j} \\
                  &= (A_i)\tpose A\tpose_j \\
                  &= \delta_{i,j}
    \end{align*}

    By (45.10) in \cite{landau_mechanics}, the coordinate change
    \((Q = \mathbf{B}q, P = \mathbf{B}p)\) is a canonical transformation with
    new Hamiltonian 
    \(\hat{H}(Q,P) = \mathcal{H}(\mathbf{B}\tpose Q,\mathbf{B}\tpose P)\).

    Furthermore, since \(\dot{P} = \mathbf{B}\dot{p}\), the new input matrix is
    given by 
    \[
        \mathbf{B}B = \begin{bmatrix}
            B^\perp B \\
            B\tpose B \\
        \end{bmatrix} = 
        \begin{bmatrix}
            \Zmat{(n-k)\times k} \\
            \Id{k}
        \end{bmatrix}
    \]
    so the coordinates \((q_u,q_a,p_u,p_a)\) are simply actuated coordinates for 
    \(\hat{H}\) as desired.
\end{proof}

\section{Virtual Nonholonomic Constraints}
% TODO: Perform the full development of VNHCs, including its stabilizing
% controller.
%TODO: Special case: when dM/dqu = 0, show that we have a nice form and that we
%can solve for p_a and the closed-loop dynamics (qu,pu)_dot



























%\section{Virtual Nonholonomic Constraints}\label{sec:vnhcs}
%
%%----- Motivation -----%
%\subsection{Motivation}
%\textbf{TODO: Why do we bother with Hamiltonian? Why can't we do virtual nonholonomic 
%constraints in Lagrangian? What are some use-cases where VHCs don't work?}
%
%%---------- Hamiltonian from Lagrangian ----------%
%\subsection{Review: Hamiltonian Systems}
%\textbf{TODO: What is a Hamiltonian system and why does it matter?}
%
%One can compute the Hamiltonian of a system by performing a Legendre transform
%on its Lagrangian \textbf{TODO: citation for legendre transform}.
%First, define the conjugate of momenta for \(q\) by 
%\begin{equation*}
%p = \frac{\partial \mathcal{L}}{\partial \dot{q}} \in \mathbb{R}^n
%\end{equation*}
%Then, the Legendre transform is performed by taking
%\begin{equation*}
%\mathcal{H}(q,p) = p^T \dot{q} - \mathcal{L}(q,\dot{q})
%\end{equation*}
%
%For a mechanical system (\ref{eqn:lagrangian}), the conjugate of momenta for
%\(q\) is given by 
%\begin{equation*}
%    p = M(q)\dot{q}
%\end{equation*}
%which means the Hamiltonian of the system is given by
%the total mechanical energy \(E\) (\ref{eqn:hamiltonian} in 
%\((q,p)\) coordinates.
%\begin{equation}\label{eqn:hamiltonian}
%\mathcal{H}(q,p) = E(q,p) = \frac{1}{2} p^T \Minv(q) p + V(q)
%\end{equation}
%Note that \(M(q)\) is the inertia matrix and \(V(q)\) is the potential for
%the Hamiltonian system. This is simply a trick to distinguish them notationally
%from the Lagrangian \(D(q)\) and \(P(q)\); they are, in fact, identical in their
%contents.
%
%The equations of motion for the system in Hamiltonian coordinates is given by
%\begin{align}\label{eqn:hamiltionian_eom}
%\begin{split}
%\dot{q} &= \frac{\partial \mathcal{H}}{\partial p} = \Minv(q) p \\
%\dot{p} &= -\frac{\partial \mathcal{H}}{\partial q} + B \tau
%\end{split}
%\end{align}
%
%\textbf{Why bother looking at Hamiltonian systems? What is the intuition behind
%these equations of motion?}
%
%%---------- Hamiltonian VNHCs ----------%
%\subsection{Hamiltonian Virtual Nonholonomic Constraints}
%While VHCs are still possible in the Hamiltonian framework, the assumptions
%required to make this work are slightly different. Rather than deriving
%Hamiltonian VHCs directly, we will produce results for nonholonomic constraints
%first as VHCs are a special case of this new framework.
%
%Suppose the mechanical system has degree of underactuation one, so that coordinates of the system can be split into an unactuated component \(q_u \in [\mathbb{R}]_T, \, T \in \mathbb{R}_{>0}\) which is not influence by control, along with an actuated component \(q_a\); that is, suppose \(B\) is of the form \(B(q) = [0_m, B_1^T(q), \ldots, B_n^T(q)]^T, \, B_i^T(q) \in \mathbb{R}^{n-1}\) and \(\tau \in \mathbb{R}^{n-1}\). In this case, \(q = (q_u, q_a)^T\) and the equations of motion become
%\begin{align}\label{eqn:unactuated_actuated_eom}
%\begin{split}
%\dot{q_u} &= e_1^T \Minv(q) p \\
%\dot{p_u} &= -p^T\frac{\partial M}{\partial q_u} p - \partial_{q_u}V(q) \\
%\dot{q_a} &= 
%\begin{bmatrix}
%0 \cdots 0 \\
%I_{n-1} \\
%\end{bmatrix} \Minv(q) p \\
%\dot{p_a} &= -p^T\frac{\partial M}{\partial q_a} p - \nabla{q_a}V(q) + 
%\begin{bmatrix}
%B_1(q) \\
%\vdots \\
%B_n(q)
%\end{bmatrix} \tau
%\end{split}
%\end{align}
%
%Now we can begin to talk about Virtual Nonholonomic Constraints. In a similar fashion to what was defined for VHCs, let us first define the goal of these new virtual constraints.
%
%\begin{defn}
%A relation \(h \in C^2\left(\mathcal{Q}\times \mathbb{R}^n ; \mathbb{R}^k\right)\) with \(h(q,p) = 0\) is a \textbf{virtual nonholonomic constraint (VNHC) of order k} if there exists a feedback control \(\tau(q,p)\) which stabilizes the constraint manifold
%\[
%\Gamma = \left\{(q,p) | h(q,p) = 0, dh_q \dot{q} + dh_p \dot{p} = 0\right\}
%\]
%\end{defn}
%Define the output of the system to be \(e = h(q,p)\). We would like to find \(\tau(q,p)\) which drives \(e\) to zero to stabilize our constraint manifold \(\Gamma\). To accomplish this, we will input-output linearize (\ref{eqn:unactuated_actuated_eom}) to find \(\ddot{e} = -k_p e - k_d \dot{e}\) with \(k_p, k_d \in \mathbb{R}_{> 0}\).
%
%To characterize a certain class of VNHCs, let us make the following assumption.
%\begin{assm}\label{assm:vnhc_is_on_qu_pu}
%We assume our relation \(h\) is of the form \(h(q,p) = q_a - f(q_u,p_u)\) for some \(f \in C^2\left([\mathbb{R}]_T \times \mathbb{R} ; \mathbb{R}^{n - 1}\right)\).
%\end{assm}
%
%Now we solve for \(\tau\) by finding \(\ddot{e}\).
%\begin{align*}
%    e &= h(q,p) = qa - f(q_u,p_u)\\
%    \Rightarrow \dot{e} &= \dot{q_a} - df_{q_u}\dot{q_u} -df_{p_u}\dot{p_u} \\
%    &= [-df_{q_u} I_{n-1}]\dot{q} - df_{p_u} \dot{p_u} \\
%    &= dh_q \Minv(q) p - df_{p_u}\left( -\frac{1}{2}p^T \frac{\partial \Minv(q)}{\partial q_u} p - \partial_{q_u}V(q) \right) \\
%    &= dh_q \Minv(q) p + \frac{1}{2}df_{p_u} p^T \frac{\partial \Minv(q)}{\partial q_u} p + df_{p_u}\partial_{q_u}V(q)
%\end{align*}
%The control input \(\tau\) only appears in \(\dot{p_a}\) (see (\ref{eqn:unactuated_actuated_eom})). To simplify the analysis, terms in \(\ddot{e}\) which do not depend on \(\dot{p_a}\) explicitly are lumped together under the symbol \((*)\):
%\begin{align*}
%    \ddot{e} &= dh_q \Minv(q) \dot{p} + df_{p_u} p^T \frac{\partial \Minv(q)}{\partial q_u} \dot{p} + (*) \\
%    &= (dh_q \Minv(q) + df_{p_u} p^T \frac{\partial \Minv(q)}{\partial q_u})B\tau + (*) \\
%    &= (dh_q \Minv(q) + dh_{p_u} p^T \frac{\partial \Minv(q)}{\partial q_u})B\tau + (*)
%\end{align*}
%
%From the derivations above, one can solve for \(\tau\) iff the matrix on the left of \(\tau\) is full rank. Thus, for systems with degree of underactuation one we give the following definition.
%\begin{defn}
%A VNHC \(h(q,p) = 0\) of order \(n - 1\) is \textbf{regular} if \(dh_{p_a} = 0\), \(dh_{q_a} = (1 \ldots 1)^T\), and 
%\[
%\text{rank}\left\{ (dh_q \Minv(q) + dh_{p_u} p^T \frac{\partial \Minv(q)}{\partial q_u})B\right\} = n - 1
%\]
%everywhere on the constraint manifold \(\Gamma\). Equivalently, a VNHC \(h\) of order \(n - 1\) is regular if it satisfies Assumption \ref{assm:vnhc_is_on_qu_pu} and system (\ref{eqn:unactuated_actuated_eom}) with output \(e = h(q,p)\) is of relative degree \(\{2,2,\ldots,2\}\) everywhere on \(\Gamma\).
%\end{defn}
%
%In general, \(\dot{e}\) is a function of \(q_u\) and \(p = (p_u,p_a)^T\). Since the purpose of a regular VNHC is to fully parameterize \(\Gamma\) by \((q_u,p_u)\), it is essential that one can solve for \(p_a = p_a(q_u,p_u)\). Unfortunately this often cannot be done, since \(\dot{e}\) contains the quadratic term
%\[
%\frac{1}{2} df_{p_u} p^T \frac{\partial \Minv(q)}{\partial q_u} p 
%\]
%We can solve for \(p_a\) if this quadratic term does not exist.
%\begin{assm}\label{assm:M_is_Mqa}
%Assume \(\partial M(q) / \partial q_u = 0 \Leftrightarrow \partial \Minv(q) / \partial q_u = 0\)
%\end{assm}
%
%Under Assumption \ref{assm:M_is_Mqa}, we get that the rank condition for \(h(q,p)\) to be a regular VNHC reduces to \(\text{rank}\left(dh_q \Minv B\right) = n - 1\). This is the same rank condition as required for Virtual Holonomic Constraints.
%
%Now we solve for \(p_a\) on the constraint manifold (when \(e = \dot{e} = 0\)):
%\begin{align*}
%    \dot{e} = dh_q \Minv(q)p + df_{p_u} \partial_{q_u}V(q) &= 0\\
%    \Leftrightarrow dh_q \Minv(q)e_1 p_u + dh_q \Minv(q) \begin{bmatrix}
%    0 & \cdots & 0 \\
%    & I_{n-1} & \\
%    \end{bmatrix} p_a &= -df_{p_u} \partial_{q_u}V(q) \\
%\end{align*}
%\begin{align*}
%    \Leftrightarrow dh_q \Minv(q) \begin{bmatrix}
%    0 & \cdots & 0 \\
%    & I_{n-1} & \\
%    \end{bmatrix} p_a = -\left(df_{p_u}\partial_{q_u}V(q) + dh_q \Minv(q)e_1 p_u\right)
%\end{align*}
%One can linearly solve for \(p_a\) if and only if the matrix in front of it is invertible.
%
%This leads us to a natural definition.
%\begin{defn}\label{defn:solvability}
%A VNHC \(h(q,p)\) is \textbf{solvable} (NOTE: actionable? what's a good name?) if
%\[
%\text{rank}\left(dh_q \Minv(q) \begin{bmatrix}
%    0 & \cdots & 0 \\
%    & I_{n-1} & \\
%    \end{bmatrix}\right) = n - 1
%\]
%\end{defn}
%
%With this analysis and our new definition in hand, we can solve for the dynamics on the constraint manifold.
%
%\begin{thm}\label{thm:equation_for_pa}
%Suppose assumptions \ref{assm:vnhc_is_on_qu_pu} and \ref{assm:M_is_Mqa} hold.
%If a regular VNHC \(h(q,p) = q_a - f(q_u,p_u)\) is solvable, then the
%parameterization for \(p_a\) on the constraint manifold is given by
%\begin{align*}
%p_a &= -\left(dh_q \Minv(q) \begin{bmatrix}
%    0 & \cdots & 0 \\
%    & I_{n-1} & \\
%    \end{bmatrix}\right)^{-1}\left( df_{p_u}\partial_{q_u}V(q) + dh_q \Minv(q)e_1 p_u\right) \\
%    &=: g(q_u,p_u)
%\end{align*}
%and the constrained dynamics on \(\Gamma\) are given by
%\begin{align*}
%    \dot{q_u} &= e_1^T \Minv(q_a) \begin{bmatrix}
%    p_u \\
%    p_a
%    \end{bmatrix}\mid_{q_a = f(q_u,p_u), p_a = g(q_u,p_u)} \\
%    \dot{p_u} &= -\partial_{q_u} V(q_u,q_a) \mid_{q_a = f(q_u,p_u)}
%\end{align*}
%\end{thm}
%
%Theorem \ref{thm:equation_for_pa} guarantees that, on \(\Gamma\), \(q_a\) is a
%parameterized completely by \((q_u,p_u)\). Hence, the zero-dynamics on
%\(\Gamma\) are always two-dimensional regardless of the original dimension
%\(n\).
%
%%---------- Hamiltonian VHCS ---------%
%\subsection{Restriction to Hamiltonian VHCs}
%\textbf{TODO: Why can we not do the standard VHC approach for Hamiltonian? 
%Show how to use the above to make it work}
%
%%/========== /Virtual Nonholonomic Constraints ==========/% 
% vim: set tw=80 ts=4 sw=4 sts=0 et ffs=unix :
