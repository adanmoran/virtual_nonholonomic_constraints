%! TEX root = main.tex

%/========== Virtual Nonholonomic Constraints ==========/%

\chapter{Development of Virtual Nonholonomic Constraints}
\section{Preliminaries on Analytical Mechanics}
In classical mechanics, a mechanical system is comprised of \(N\) point masses.
Each point mass represents the center of mass of a physical body, where the
position of each mass is described using three cartesian coordinates (one for
each spatial axis).  The system as a whole is represented by a vector in
\(\R^{3N}\), and the dynamics of the system are computed by deriving the \(3N\)
\textit{equations of motion} (EOM) produced by Newton's second law \(F = m a\).
While this method works for simple cases, it unfortunately proves impossible to
apply to a majority of mechanical systems since most of the forces are not known
explicitly. 

Most mechanical systems are not solely a collection of point masses which can 
freely move about three-dimensional space; the masses are usually restricted
physically by some equations of constraint (EOC). 
Rather than representing the system by one vector in \(\R^{3N}\) and \(k\) EOC, 
it is often feasible to represent the position of the system using \(n\)
independent scalar-valued variables \(q_1,\ldots,q_n\) called 
\textit{generalized coordinates}, where \(n = 3N - k\) is the number of
\textit{degrees of freedom} (DOF) of the system.

Each generalized coordinate \(q_i\) takes values in \(\Rt{T_i}\), where
\(T_i = \infty\) if \(q_i\) represents a length or \(T_i = 2\pi\) if \(q_i\)
represents an angle.
It is convention to write these as a \textit{configuration} vector 
\(q = (q_1,\ldots,q_n) \in \mathcal{Q}\) 
where the \textit{configuration manifold} \(\mathcal{Q}\) of the system has the
following structure:
\[
    \mathcal{Q} = \Rt{T_1} \times \cdot \times \Rt{T_n}
\] 
The derivative \(\dot{q} = (\dot{q}_1,\ldots,\dot{q}_n)\) of a configuration
is called a \textit{generalized velocity}, and the combined vector
\((q,\dot{q})\) is called a \textit{state} of the system.

The field of analytical mechanics provides a computational method for finding
the EOM of a system in generalized coordinates. The two most common methods for
modelling robotic systems are \textit{Lagrangian} and \textit{Hamiltonian}
mechanics.

% ---------- Lagrangian Mechanics ---------- % 
\subsection{Lagrangian Mechanics}

Lagrangian mechanics uses the kinetic energy \(T(q,\dot{q})\) and potential
energy \(P(q)\) of the system to define the Lagrangian
(\ref{eqn:lagrangian-general})
\begin{equation}\label{eqn:lagrangian-general}
    \mathcal{L}(q,\dot{q}) = T(q,\dot{q}) - P(q)
\end{equation}
In actuated mechanical systems, the EOM are described by \(n\) second-order
ordinary differential equations (ODEs) obtained from the \textit{Euler-Lagrange
equations} with \textit{generalized input forces} \(\tau \in \R^m\) 
(\ref{eqn:el-eqns-general}). 
\begin{equation}\label{eqn:el-eqns-general}
    \diff{}{t}\left\{ \pdiff{\mathcal{L}}{\dot{q}_i} \right\}
    - \pdiff{\mathcal{L}}{q_i} = B_i(q) \tau
\end{equation}
Here, \(B_i: \mathcal{Q} \rightarrow \R^{1\times m}\)
and the matrix  \(B: \mathcal{Q} \rightarrow \R^{n \times m}\) given by
%\[
%    B(q) =\begin{bmatrix}
%        - & B_1(q) & - \\
%          & \vdots & \\
%        - & B_n(q) & -
%    \end{bmatrix}
%\]
\[
    B(q) = [B_i(q)]
\]
is the \textit{input matrix} for the system. If \(m < n\), we say the system is
\textit{underactuated} with degree of underactuation \(n - m\).

Many actuated mechanical systems have quadratic kinetic energies, so that the
Lagrangian can be written explicitly as
\begin{equation}\label{eqn:lagrangian}
    \mathcal{L}(q,\dot{q}) = \frac{1}{2} \dot{q}\tpose D(q) \dot{q} - P(q)
\end{equation}
where the \textit{inertia matrix} \(D: \mathcal{Q} \rightarrow \R^{n\times n}\) 
is a symmetric, positive definite matrix for all \(q \in \mathcal{Q}\) and the
potential function \(P : \mathcal{Q} \rightarrow \R\) is smooth. 
If this is the case, the Euler-Lagrange equations
(\ref{eqn:el-eqns-general}) reduce to (\ref{eqn:el-eqns}),
\begin{equation}\label{eqn:el-eqns}
    D(q)\ddot{q} + C(q,\dot{q})\dot{q} + \nabla P(q) = B(q)\tau
\end{equation}
where the \textit{Coriolis matrix} \(C(q,\dot{q})\) is of the form
\[
    [C]_{i,j} = \frac{1}{2}\sum\limits_{k = 1}^n 
    \left(\pdiff{D_{i,j}}{q_k}  +
     \pdiff{D_{i,k}}{q_j} -
     \pdiff{D_{k,j}}{q_i}\right)\dot{q}_k
\]

% ---------- Hamiltonian Mechanics ---------- %
\subsection{Hamiltonian Mechanics}
% TODO: Describe how to get the Hamiltonian from the lagrangian, and show that
% you can write the hamiltonian directly in certain conditions of the system
Hamiltonian mechanics provides an equivalent representation of the EOM
by converting the \(n\) second-order ODEs generated by Lagrangian mechanics into
\(2n\) first-order ODEs.

To do this, we first define the \textit{conjugate of momentum to \(q_i\)} by
\begin{equation}\label{eqn:p-i}
    p_i(q,\dot{q}) = \pdiff{\mathcal{L}}{\dot{q}_i}(q,\dot{q})
\end{equation}
To ease notation, we will write \(p = (p_1, \ldots, p_n)\) and say that \(p\) is
the \textit{conjugate of momenta to \(q\)} Note that each \(p_i\) is a linear
function of \(\dot{q}\), so one can typically solve for \(\dot{q}(q,p)\) by
inverting the expressions from (\ref{eqn:p-i}). The combined vector \((q,p)\) is
called the \textit{phase} of the system.

Next we define the \textit{Hamiltonian} (\ref{eqn:hamiltonian-general}) 
as the Legendre transform of the Lagrangian \textbf{SOURCE}:
\begin{equation}\label{eqn:hamiltonian-general}
    \mathcal{H}(q,p) = p\tpose \dot{q}(q,p) - \mathcal{L}(q,\dot{q}(q,p))
\end{equation}
The EOM in this framework can be shown to be the \(2n\)
first-order equations called \textit{Hamilton's equations}
\begin{equation}\label{eqn:hamiltons-eqns}
    \begin{cases}
        \dot{q} = \nabla_p\mathcal{H} \\
        \dot{p} = -\nabla_q\mathcal{H} + B(q)\tau \\
    \end{cases}
\end{equation}
where \(B(q) \in \R^{n\times m}\) is the input matrix and 
\(\tau \in \R^m\) is the vector of generalized input forces. A pair of
coordinates \((q,p)\) which satisfy Hamilton's equations are said to be 
\textit{canonical coordinates for} \(\mathcal{H}\).

If the Lagrangian is given as in (\ref{eqn:lagrangian}), the conjugate of
momenta is computed explicitly by
\[
    p = D(q)\dot{q}
\]
which gives the Hamiltonian (\ref{eqn:hamiltonian})
\begin{equation}\label{eqn:hamiltonian}
    \mathcal{H}(q,p) = \frac{1}{2} p\tpose D^{-1}(q) p + P(q)
\end{equation}
and the EOM from (\ref{eqn:hamiltons-eqns}) reduce to 
(\ref{eqn:hamiltonian-eom}).
\begin{equation}\label{eqn:hamiltonian-eom}
    \begin{cases}
        \dot{q} = D^{-1}(q)p \\
        \dot{p} = -\frac{1}{2} \left[p\tpose \pdiff{D^{-1}}{q_i}(q) p\right] 
        - \nabla_q P(q) + B(q) \tau \\
    \end{cases}
\end{equation}

\section{Simply Actuated Hamiltonian Systems}
% TODO: Describe the change of coordinates to get into q_u/q_a mode and show
% that the actuator directly affects pa but not pu. Use M for simply actuated
% inertia, D for normal coordinates
Given a mechanical system with Lagrangian (\ref{eqn:lagrangian}) yielding 
Hamiltonian (\ref{eqn:hamiltonian}) and canonical coordinates \((q,p)\), 
it is not in general obvious how the input forces
\(\tau\) are affecting the conjugate of momenta since they are scaled by
\(B(q)\) (which may be quite complicated). Even if \(B(q)\) is constant, multiple
forces may affect each \(p_i\). This section will demonstrate a useful change of
coordinates which results in a new Hamiltonian mechanical system with simpler
input matrix.

Throughout the rest of this thesis, we will make the following assumptions.
\begin{assm}\label{assm:B-const}
    The input matrix \(B : \mathcal{Q} \rightarrow \R^{n\times k}\) is constant
    and full rank.
\end{assm}
\begin{assm}\label{assm:B-orthogonal}
    The input matrix is left-semi-orthogonal, whereby \(B\tpose B = \Id{k}\).
\end{assm}
\begin{assm}\label{assm:B-perp}
    There exists a right-semi-orthogonal matrix 
    \(B^\perp \in \R^{(n-k)\times n}\) for \(B\)
    which is a left-annihilator for \(B\). That is,
    \(B^\perp(B^\perp)\tpose = \Id{(n-k)}\) and
    \(B^\perp B = \Zmat{(n-k) \times k}\) 
\end{assm}

Observe that Assumption \ref{assm:B-orthogonal} can be taken without loss of
generality (WLOG). If \(B\) is a constant full-rank matrix, 
it has a singular-value decomposition 
\(B = U \Sigma V\tpose\) where \(U^{-1} = U\tpose \in \R^{n \times n}\), 
\(V^{-1} = V\tpose \in \R^{n \times k}\), and \(\Sigma \in \R^{n \times k}\) is
defined by
\[
    \Sigma = \begin{bmatrix}
        \sigma_1 & 0 & \cdots & 0 & 0 \\
        0 & \sigma_2 & \cdots & 0 & 0 \\
        \vdots & & \ddots &  & \vdots \\
        0 & 0 & \cdots & 0 & \sigma_k \\
          & - & \Zmat{(n-k)\times k} & -  &\\
    \end{bmatrix}
\]
where \(\sigma_i \neq 0\).
\textbf{SOURCE}
Defining \(T \in \R^{k \times k}\) by
\[
    T = \begin{bmatrix}
        \frac{1}{\sigma_1} & 0 & \cdots & 0 \\
        0 & \frac{1}{\sigma_2} & \cdots & 0 \\
    \vdots & & \ddots & \vdots \\
    0 & 0 & \cdots & \frac{1}{\sigma_k} \\
    \end{bmatrix}
\]
and assigning the input forces to \(\tau = V T \hat{\tau}\), we get a new input
matrix for \(\hat{\tau}\) given by \(\hat{B} = B V T = U \Sigma T\) 
which is still constant and full-rank. In particular, 
\(\hat{B}\tpose \hat{B} = T\tpose \Sigma\tpose \Sigma\tpose T = \Id{k}\).

If \(k = (n-1)\) then if there is a left annihilator \(A\) for
\(B\), then \(B^\perp = A/\norm{A}\) satisfies Assumption
\ref{assm:B-perp}.

To perform our coordinate transformation, we define a transformation matrix
\[
    \mathbf{B} = 
    \begin{bmatrix}
        B^\perp \\
        B\tpose \\
    \end{bmatrix}
\]
Since both \(B^\perp\) and \(B\tpose\) are right-semi-orthogonal and
\(\mathbf{B} \in \R^{n \times n}\) is full rank, \(\mathbf{B}\) is orthogonal. 
That is, \(\mathbf{B}\mathbf{B}\tpose = \Id{n}\) which means 
\(\mathbf{B}^{-1} = \mathbf{B}\tpose\).

\begin{defn}
    Define \(q_u = B^\perp q \in \R^{(n-k)}\) and 
    \(q_a = B\tpose q \in \R^k\). We say \(q_u\) are the \textit{unactuated
    coordinates} while \(q_a\) are the \textit{actuated coordinates} of the
    system. Their corresponding conjugate of momenta are labelled \(p_u\) and
    \(p_a\).
\end{defn}

In these new coordinates the input matrix will be modified such
that \(\tau_i\) only affects \(p_{a,i}\), and all \(p_u\) are unaffected by the
inputs.

To derive this result, observe that the coordinates 
\(\hat{q} = (q_u,q_a)\) can be written 
as the transformation (\ref{eqn:q-transformation}).
\begin{equation}\label{eqn:q-transformation}
    \hat{q} = \mathbf{B}q
\end{equation}
Since \(\mathbf{B}\) is
an orthogonal invertible matrix, we can equivalently write this as
\(q = \mathbf{B}\tpose \hat{q}\). Applying this change of coordinates to the
Lagrangian (\ref{eqn:lagrangian}) gives the new Lagrangian
\[
    \hat{\mathcal{L}}(\hat{q},\dot{\hat{q}}) = 
    \frac{1}{2}\dot{\hat{q}}\tpose \mathbf{B} 
    D(\mathbf{B}\tpose\hat{q}) 
    \mathbf{B}\tpose\dot{\hat{q}} - P(\mathbf{B}\hat{q})
\]
From here it is straightforward to show that the 
conjugate of momenta \(\hat{p} = (p_u,p_a)\) to \(\hat{q}\) is given by 
(\ref{eqn:p-transformation}).
\begin{equation}\label{eqn:p-transformation}
    \hat{p} = \pdiff{\hat{\mathcal{L}}}{\dot{\hat{q}}}
            = \mathbf{B}D(q)\dot{q} = \mathbf{B}p
\end{equation}

For ease of notation, we define
\(M(\hat{q}) := \mathbf{B}D\left(\mathbf{B}\tpose\hat{q}\right)\mathbf{B}\tpose\)
and 
\(V(\hat{q}) := P\left(\mathbf{B}\tpose \hat{q}\right)\)
Then the Hamiltonian (\ref{eqn:hamiltonian}) is given in
\((\hat{q},\hat{p})\) coordinates by
\begin{align*}
    \hat{\mathcal{H}}(\hat{q},\hat{p})
    &=
    \frac{1}{2}\hat{p}\tpose\mathbf{B}
    D^{-1}\left(\mathbf{B}\tpose\hat{q}\right)
    \mathbf{B}\tpose\hat{p}
    + P\left(\mathbf{B}\hat{q}\right) \\
    &= \hat{p}\tpose \Minv(\hat{q})\hat{p} + V(\hat{q})
\end{align*}

Now we will show \((\hat{q},\hat{p})\) are canonical coordinates
for \(\hat{\mathcal{H}}\). From the definition of \(\hat{q}\) in
(\ref{eqn:q-transformation}), we obtain
\begin{align*}
    \dot{\hat{q}} &= \mathbf{B}\dot{q} = \mathbf{B}D^{-1}(q)p \\
        &= \mathbf{B}D^{-1}(\mathbf{B}\tpose \hat{q})\mathbf{B}\tpose \hat{p} \\
        &= \Minv(\hat{q})\hat{p} \\
        &= \nabla_{\hat{p}}\hat{\mathcal{H}}
\end{align*}

The dynamics of \(\hat{p}\) are given by
\[
    \dot{\hat{p}}_i = \mathbf{B}\dot{p}
                    = -\frac{1}{2}\mathbf{B}
                    \left[p\tpose \pdiff{D^{-1}}{q_i}p\right]
                    -\mathbf{B} \nabla_q P(q) + \mathbf{B}B \tau
\]
where 
\[
    \mathbf{B} \nabla_q P(q)
    = \nabla_{\hat{q}}(\mathbf{B}\tpose \hat{q})\nabla_q P(q)
    = \nabla_{\hat{q}}q \nabla_q P(q)
    = \nabla_{\hat{q}}P(\mathbf{B}\tpose\hat{q}) 
    = \nabla_{\hat{q}} V(\hat{q}) 
\]

\textbf{LOOK INTO SYMPLECTIc jacobian method for canonical transformations to
show this is a canonical transformation}


\section{Virtual Nonholonomic Constraints}
% TODO: Perform the full development of VNHCs, including its stabilizing
% controller.
%TODO: Special case: when dM/dqu = 0, show that we have a nice form and that we
%can solve for p_a and the closed-loop dynamics (qu,pu)_dot



























%\section{Virtual Nonholonomic Constraints}\label{sec:vnhcs}
%
%%----- Motivation -----%
%\subsection{Motivation}
%\textbf{TODO: Why do we bother with Hamiltonian? Why can't we do virtual nonholonomic 
%constraints in Lagrangian? What are some use-cases where VHCs don't work?}
%
%%---------- Hamiltonian from Lagrangian ----------%
%\subsection{Review: Hamiltonian Systems}
%\textbf{TODO: What is a Hamiltonian system and why does it matter?}
%
%One can compute the Hamiltonian of a system by performing a Legendre transform
%on its Lagrangian \textbf{TODO: citation for legendre transform}.
%First, define the conjugate of momenta for \(q\) by 
%\begin{equation*}
%p = \frac{\partial \mathcal{L}}{\partial \dot{q}} \in \mathbb{R}^n
%\end{equation*}
%Then, the Legendre transform is performed by taking
%\begin{equation*}
%\mathcal{H}(q,p) = p^T \dot{q} - \mathcal{L}(q,\dot{q})
%\end{equation*}
%
%For a mechanical system (\ref{eqn:lagrangian}), the conjugate of momenta for
%\(q\) is given by 
%\begin{equation*}
%    p = M(q)\dot{q}
%\end{equation*}
%which means the Hamiltonian of the system is given by
%the total mechanical energy \(E\) (\ref{eqn:hamiltonian} in 
%\((q,p)\) coordinates.
%\begin{equation}\label{eqn:hamiltonian}
%\mathcal{H}(q,p) = E(q,p) = \frac{1}{2} p^T \Minv(q) p + V(q)
%\end{equation}
%Note that \(M(q)\) is the inertia matrix and \(V(q)\) is the potential for
%the Hamiltonian system. This is simply a trick to distinguish them notationally
%from the Lagrangian \(D(q)\) and \(P(q)\); they are, in fact, identical in their
%contents.
%
%The equations of motion for the system in Hamiltonian coordinates is given by
%\begin{align}\label{eqn:hamiltionian_eom}
%\begin{split}
%\dot{q} &= \frac{\partial \mathcal{H}}{\partial p} = \Minv(q) p \\
%\dot{p} &= -\frac{\partial \mathcal{H}}{\partial q} + B \tau
%\end{split}
%\end{align}
%
%\textbf{Why bother looking at Hamiltonian systems? What is the intuition behind
%these equations of motion?}
%
%%---------- Hamiltonian VNHCs ----------%
%\subsection{Hamiltonian Virtual Nonholonomic Constraints}
%While VHCs are still possible in the Hamiltonian framework, the assumptions
%required to make this work are slightly different. Rather than deriving
%Hamiltonian VHCs directly, we will produce results for nonholonomic constraints
%first as VHCs are a special case of this new framework.
%
%Suppose the mechanical system has degree of underactuation one, so that coordinates of the system can be split into an unactuated component \(q_u \in [\mathbb{R}]_T, \, T \in \mathbb{R}_{>0}\) which is not influence by control, along with an actuated component \(q_a\); that is, suppose \(B\) is of the form \(B(q) = [0_m, B_1^T(q), \ldots, B_n^T(q)]^T, \, B_i^T(q) \in \mathbb{R}^{n-1}\) and \(\tau \in \mathbb{R}^{n-1}\). In this case, \(q = (q_u, q_a)^T\) and the equations of motion become
%\begin{align}\label{eqn:unactuated_actuated_eom}
%\begin{split}
%\dot{q_u} &= e_1^T \Minv(q) p \\
%\dot{p_u} &= -p^T\frac{\partial M}{\partial q_u} p - \partial_{q_u}V(q) \\
%\dot{q_a} &= 
%\begin{bmatrix}
%0 \cdots 0 \\
%I_{n-1} \\
%\end{bmatrix} \Minv(q) p \\
%\dot{p_a} &= -p^T\frac{\partial M}{\partial q_a} p - \nabla{q_a}V(q) + 
%\begin{bmatrix}
%B_1(q) \\
%\vdots \\
%B_n(q)
%\end{bmatrix} \tau
%\end{split}
%\end{align}
%
%Now we can begin to talk about Virtual Nonholonomic Constraints. In a similar fashion to what was defined for VHCs, let us first define the goal of these new virtual constraints.
%
%\begin{defn}
%A relation \(h \in C^2\left(\mathcal{Q}\times \mathbb{R}^n ; \mathbb{R}^k\right)\) with \(h(q,p) = 0\) is a \textbf{virtual nonholonomic constraint (VNHC) of order k} if there exists a feedback control \(\tau(q,p)\) which stabilizes the constraint manifold
%\[
%\Gamma = \left\{(q,p) | h(q,p) = 0, dh_q \dot{q} + dh_p \dot{p} = 0\right\}
%\]
%\end{defn}
%Define the output of the system to be \(e = h(q,p)\). We would like to find \(\tau(q,p)\) which drives \(e\) to zero to stabilize our constraint manifold \(\Gamma\). To accomplish this, we will input-output linearize (\ref{eqn:unactuated_actuated_eom}) to find \(\ddot{e} = -k_p e - k_d \dot{e}\) with \(k_p, k_d \in \mathbb{R}_{> 0}\).
%
%To characterize a certain class of VNHCs, let us make the following assumption.
%\begin{assm}\label{assm:vnhc_is_on_qu_pu}
%We assume our relation \(h\) is of the form \(h(q,p) = q_a - f(q_u,p_u)\) for some \(f \in C^2\left([\mathbb{R}]_T \times \mathbb{R} ; \mathbb{R}^{n - 1}\right)\).
%\end{assm}
%
%Now we solve for \(\tau\) by finding \(\ddot{e}\).
%\begin{align*}
%    e &= h(q,p) = qa - f(q_u,p_u)\\
%    \Rightarrow \dot{e} &= \dot{q_a} - df_{q_u}\dot{q_u} -df_{p_u}\dot{p_u} \\
%    &= [-df_{q_u} I_{n-1}]\dot{q} - df_{p_u} \dot{p_u} \\
%    &= dh_q \Minv(q) p - df_{p_u}\left( -\frac{1}{2}p^T \frac{\partial \Minv(q)}{\partial q_u} p - \partial_{q_u}V(q) \right) \\
%    &= dh_q \Minv(q) p + \frac{1}{2}df_{p_u} p^T \frac{\partial \Minv(q)}{\partial q_u} p + df_{p_u}\partial_{q_u}V(q)
%\end{align*}
%The control input \(\tau\) only appears in \(\dot{p_a}\) (see (\ref{eqn:unactuated_actuated_eom})). To simplify the analysis, terms in \(\ddot{e}\) which do not depend on \(\dot{p_a}\) explicitly are lumped together under the symbol \((*)\):
%\begin{align*}
%    \ddot{e} &= dh_q \Minv(q) \dot{p} + df_{p_u} p^T \frac{\partial \Minv(q)}{\partial q_u} \dot{p} + (*) \\
%    &= (dh_q \Minv(q) + df_{p_u} p^T \frac{\partial \Minv(q)}{\partial q_u})B\tau + (*) \\
%    &= (dh_q \Minv(q) + dh_{p_u} p^T \frac{\partial \Minv(q)}{\partial q_u})B\tau + (*)
%\end{align*}
%
%From the derivations above, one can solve for \(\tau\) iff the matrix on the left of \(\tau\) is full rank. Thus, for systems with degree of underactuation one we give the following definition.
%\begin{defn}
%A VNHC \(h(q,p) = 0\) of order \(n - 1\) is \textbf{regular} if \(dh_{p_a} = 0\), \(dh_{q_a} = (1 \ldots 1)^T\), and 
%\[
%\text{rank}\left\{ (dh_q \Minv(q) + dh_{p_u} p^T \frac{\partial \Minv(q)}{\partial q_u})B\right\} = n - 1
%\]
%everywhere on the constraint manifold \(\Gamma\). Equivalently, a VNHC \(h\) of order \(n - 1\) is regular if it satisfies Assumption \ref{assm:vnhc_is_on_qu_pu} and system (\ref{eqn:unactuated_actuated_eom}) with output \(e = h(q,p)\) is of relative degree \(\{2,2,\ldots,2\}\) everywhere on \(\Gamma\).
%\end{defn}
%
%In general, \(\dot{e}\) is a function of \(q_u\) and \(p = (p_u,p_a)^T\). Since the purpose of a regular VNHC is to fully parameterize \(\Gamma\) by \((q_u,p_u)\), it is essential that one can solve for \(p_a = p_a(q_u,p_u)\). Unfortunately this often cannot be done, since \(\dot{e}\) contains the quadratic term
%\[
%\frac{1}{2} df_{p_u} p^T \frac{\partial \Minv(q)}{\partial q_u} p 
%\]
%We can solve for \(p_a\) if this quadratic term does not exist.
%\begin{assm}\label{assm:M_is_Mqa}
%Assume \(\partial M(q) / \partial q_u = 0 \Leftrightarrow \partial \Minv(q) / \partial q_u = 0\)
%\end{assm}
%
%Under Assumption \ref{assm:M_is_Mqa}, we get that the rank condition for \(h(q,p)\) to be a regular VNHC reduces to \(\text{rank}\left(dh_q \Minv B\right) = n - 1\). This is the same rank condition as required for Virtual Holonomic Constraints.
%
%Now we solve for \(p_a\) on the constraint manifold (when \(e = \dot{e} = 0\)):
%\begin{align*}
%    \dot{e} = dh_q \Minv(q)p + df_{p_u} \partial_{q_u}V(q) &= 0\\
%    \Leftrightarrow dh_q \Minv(q)e_1 p_u + dh_q \Minv(q) \begin{bmatrix}
%    0 & \cdots & 0 \\
%    & I_{n-1} & \\
%    \end{bmatrix} p_a &= -df_{p_u} \partial_{q_u}V(q) \\
%\end{align*}
%\begin{align*}
%    \Leftrightarrow dh_q \Minv(q) \begin{bmatrix}
%    0 & \cdots & 0 \\
%    & I_{n-1} & \\
%    \end{bmatrix} p_a = -\left(df_{p_u}\partial_{q_u}V(q) + dh_q \Minv(q)e_1 p_u\right)
%\end{align*}
%One can linearly solve for \(p_a\) if and only if the matrix in front of it is invertible.
%
%This leads us to a natural definition.
%\begin{defn}\label{defn:solvability}
%A VNHC \(h(q,p)\) is \textbf{solvable} (NOTE: actionable? what's a good name?) if
%\[
%\text{rank}\left(dh_q \Minv(q) \begin{bmatrix}
%    0 & \cdots & 0 \\
%    & I_{n-1} & \\
%    \end{bmatrix}\right) = n - 1
%\]
%\end{defn}
%
%With this analysis and our new definition in hand, we can solve for the dynamics on the constraint manifold.
%
%\begin{thm}\label{thm:equation_for_pa}
%Suppose assumptions \ref{assm:vnhc_is_on_qu_pu} and \ref{assm:M_is_Mqa} hold.
%If a regular VNHC \(h(q,p) = q_a - f(q_u,p_u)\) is solvable, then the
%parameterization for \(p_a\) on the constraint manifold is given by
%\begin{align*}
%p_a &= -\left(dh_q \Minv(q) \begin{bmatrix}
%    0 & \cdots & 0 \\
%    & I_{n-1} & \\
%    \end{bmatrix}\right)^{-1}\left( df_{p_u}\partial_{q_u}V(q) + dh_q \Minv(q)e_1 p_u\right) \\
%    &=: g(q_u,p_u)
%\end{align*}
%and the constrained dynamics on \(\Gamma\) are given by
%\begin{align*}
%    \dot{q_u} &= e_1^T \Minv(q_a) \begin{bmatrix}
%    p_u \\
%    p_a
%    \end{bmatrix}\mid_{q_a = f(q_u,p_u), p_a = g(q_u,p_u)} \\
%    \dot{p_u} &= -\partial_{q_u} V(q_u,q_a) \mid_{q_a = f(q_u,p_u)}
%\end{align*}
%\end{thm}
%
%Theorem \ref{thm:equation_for_pa} guarantees that, on \(\Gamma\), \(q_a\) is a
%parameterized completely by \((q_u,p_u)\). Hence, the zero-dynamics on
%\(\Gamma\) are always two-dimensional regardless of the original dimension
%\(n\).
%
%%---------- Hamiltonian VHCS ---------%
%\subsection{Restriction to Hamiltonian VHCs}
%\textbf{TODO: Why can we not do the standard VHC approach for Hamiltonian? 
%Show how to use the above to make it work}
%
%%/========== /Virtual Nonholonomic Constraints ==========/% 
% vim: set tw=80 ts=4 sw=4 sts=0 et ffs=unix :
