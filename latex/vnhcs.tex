%! TEX root = main.tex

%/========== Virtual Nonholonomic Constraints ==========/%

\chapter{Development of Virtual Nonholonomic Constraints}
% TODO: Brief introduction before preliminaries

\section{Preliminaries on Analytical Mechanics}
A mechanical system can be represented by \(N\) point masses where each point
represents the center of mass of a physical body, along with \(r\)
\textit{equations of constraint} (EOC) which model the physical restrictions
between these masses.
The position of each point mass is described using three cartesian coordinates (one
for each spatial axis), so the system as a whole can be described by a vector in
\(\R^{3N}\) with \(r\) EOC. 
The dynamics of the system are computed by deriving the \(3N\)
\textit{equations of motion} (EOM) produced by Newton's second law \(F = m a\).
While this technique works for simple systems, it is tedius and becomes
impossible to apply to complex mechanical systems where the forces are not
explicitly known. 

Rather than modeling a mechanical system by cartesian positions and constraints,
it is often feasible to represent the position of the system using \(n\)
independent scalar-valued variables \(q_1,\ldots,q_n\) called 
\textit{generalized coordinates}, where \(n = 3N - r\) is the number of
\textit{degrees of freedom} (DOF) of the system \cite{greenwood_dynamics}. For
instance, Figure \ref{fig:barbell} shows a barbell on a \(2D\)-plane which can
rotate freely on that plane. 
The barbell has \(n = 3\) DOF, so it can be described by three
independent generalized coordinates with no equations of constraint.

\begin{figure}
   \centering
   \begin{subfigure}[]{0.75\textwidth}
      \includestandalone[width=\textwidth]{images/newtonian_barbell}
      \caption{The Newtonian representation of the barbell 
      requires all six cartesian positions and the corresponding EOC.}
   \end{subfigure}
   \begin{subfigure}[]{0.75\textwidth}
      \includestandalone[width=\textwidth]{images/gc_barbell}
      \caption{One possible set of three generalized coordinates is
       \((x_0,y_0,\theta)\), which represent the position of the 
       center of the bar and the angle of the barbell in the \(xy\)-plane.}
   \end{subfigure}
   \caption{A mechanical system with \(N = 2\) points at \((x_1,y_1,z_1)\)
   and \((x_2,y_2,z_2)\) separated by a bar with \(r = 3\) EOC given by 
   \(c = \norm{x_1 - x_2}\), \(z_1 = 0\), and \(z_2 = 0\). This system has \(n =
   3\) degrees of freedom.}
   \label{fig:barbell}
\end{figure}

For the robotic systems of interest in this thesis, we assume that
each generalized coordinate \(q_i\) represents either the distance or the angle
between two parts of the system.
Mathematically, each \(q_i\) takes values in \(\Rt{T_i}\), where
\(T_i = \infty\) if \(q_i\) represents a length or \(T_i = 2\pi\) if \(q_i\)
represents an angle.
It is convention to collect the coordinates into a \textit{configuration} 
\(q = (q_1,\ldots,q_n) \in \mathcal{Q}\) 
where the \textit{configuration manifold} \(\mathcal{Q}\) of the system is a
so-called \textit{generalized cylinder}:
\[
    \mathcal{Q} = \Rt{T_1} \times \cdot \times \Rt{T_n}
\] 
The derivative \(\dot{q} = (\dot{q}_1,\ldots,\dot{q}_n)\) of a configuration
is called a \textit{generalized velocity} of the system. For arbitrary systems,
the space of allowable velocities depends on the current configuration of the
system.  However, since \(\mathcal{Q}\) is a generalized cylinder, we find that 
\(\dot{q} \in \R^n\).
The combined vector \((q,\dot{q}) \in \mathcal{Q}\times\R^n\) is called a 
\textit{state} of the system.

The field of analytical mechanics provides a computational method for finding
the EOM of a system in generalized coordinates. The two most common analytical
methods for modelling robotic systems are \textit{Lagrangian} and
\textit{Hamiltonian} mechanics.

% ---------- Lagrangian Mechanics ---------- % 
\subsection{Lagrangian Mechanics}\label{sec:lagrangian-mechanics}

Lagrangian mechanics uses the kinetic energy \(T(q,\dot{q})\) and potential
energy \(P(q)\) of the system to define the Lagrangian 
\(\mathcal{L} : \mathcal{Q}\times\R^n \rightarrow \R\) defined by
(\ref{eqn:lagrangian-general}) \cite{greenwood_dynamics}.
\begin{equation}\label{eqn:lagrangian-general}
    \mathcal{L}(q,\dot{q}) = T(q,\dot{q}) - P(q)
\end{equation}
When the mechanical system is actuated, the EOM are described by \(n\) second-order
ordinary differential equations (ODEs) obtained from the \textit{Euler-Lagrange
equations} (\ref{eqn:el-eqns-general}) with \textit{generalized input forces} 
\(\tau \in \R^k\) . 
\begin{equation}\label{eqn:el-eqns-general}
    \diff{}{t}\left\{ \pdiff{\mathcal{L}}{\dot{q}_i} \right\}
    - \pdiff{\mathcal{L}}{q_i} = B_i\tpose(q) \tau
\end{equation}
The vector \(B_i\tpose: \mathcal{Q} \rightarrow \R^{1\times k}\) describes how
the input forces shape the dynamics of \(q_i\).
The matrix  \(B: \mathcal{Q} \rightarrow \R^{n \times k}\) with
\[
    B(q) = \begin{bmatrix}
        - & B_1\tpose(q) & - \\
          & \vdots & \\
        - & B_n\tpose(q) & - \\
    \end{bmatrix}
\]
is called the \textit{input matrix} for the system.
If \(k < n\), we say the system is \textit{underactuated} with degree of
underactuation \((n - k)\).

Many actuated mechanical systems have quadratic kinetic energies, so that the
Lagrangian can be written explicitly as
\begin{equation}\label{eqn:lagrangian}
    \mathcal{L}(q,\dot{q}) = \frac{1}{2} \dot{q}\tpose D(q) \dot{q} - P(q)
\end{equation}
where the \textit{inertia matrix} \(D: \mathcal{Q} \rightarrow \R^{n\times n}\) 
is a symmetric, positive definite matrix for all \(q \in \mathcal{Q}\) and the
potential function \(P : \mathcal{Q} \rightarrow \R\) is smooth. 
%If this is the case, the Euler-Lagrange equations reduce to (\ref{eqn:el-eqns}),
%\begin{equation}\label{eqn:el-eqns}
%    D(q)\ddot{q} + C(q,\dot{q})\dot{q} + \nabla P(q) = B(q)\tau
%\end{equation}
%where the \textit{Coriolis matrix} \(C(q,\dot{q})\) is of the form
%\[
%    [C]_{i,j} = \frac{1}{2}\sum\limits_{k = 1}^n 
%    \left(\pdiff{D_{i,j}}{q_k}  +
%     \pdiff{D_{i,k}}{q_j} -
%     \pdiff{D_{k,j}}{q_i}\right)\dot{q}_k
%\]

% ---------- Hamiltonian Mechanics ---------- %
\subsection{Hamiltonian Mechanics}
Hamiltonian mechanics converts the \(n\) second-order ODEs generated by
Lagrangian mechanics into an equivalent set of \(2n\) first-order ODEs.

To do this, we first define the \textit{conjugate of momentum \(p_i\) to \(q_i\)} by
\begin{equation}\label{eqn:p-i}
    p_i(q,\dot{q}) := \pdiff{\mathcal{L}}{\dot{q}_i}(q,\dot{q})
\end{equation}
To ease notation, we write \(p = (p_1, \ldots, p_n) \in \R^n\) and call
\(p\) the \textit{conjugate of momenta to \(q\)}.
Note that each \(p_i\) is a linear function of \(\dot{q}\), and one can typically
solve for \(\dot{q}(q,p)\) by inverting all the expressions from (\ref{eqn:p-i}).
The combined vector \((q,p) \in \mathcal{Q}\times\R^n\) is called a 
\textit{phase} of the system. 

The \textit{Hamiltonian} of the system in
\(\{q,p\}\) coordinates is the ``Legendre transform"
(\ref{eqn:hamiltonian-legendre}) of the Lagrangian \cite{landau_mechanics}.
\begin{equation}\label{eqn:hamiltonian-legendre}
    \mathcal{H}(q,p) := p\tpose \dot{q}(q,p) - \mathcal{L}(q,\dot{q}(q,p))
\end{equation}
The EOM in the Hamiltonian framework are the \(2n\)
first-order equations called \textit{Hamilton's equations}. They are given by:
\begin{equation}\label{eqn:hamiltons-eqns}
    \begin{cases}
        \dot{q} = \nabla_p\mathcal{H} \\
        \dot{p} = -\nabla_q\mathcal{H} + B(q)\tau \\
    \end{cases}
\end{equation}
Here, \(B(q) \in \R^{n\times k}\) is the same input matrix used by the
Lagrangian framework, with \(\tau \in \R^k\) the same vector of generalized
input forces.

If the kinetic energy of the system is quadratic as in (\ref{eqn:lagrangian}), 
the conjugate of momenta becomes \(p = D(q)\dot{q}\). Since \(D(q)\) is symmetric
and positive definite, it is invertible at each \(q \in \mathcal{Q}\).
The resulting Hamiltonian system reduces to (\ref{eqn:hamiltonian}).
\begin{align}\label{eqn:hamiltonian}
    \mathcal{H}(q,p) &= \frac{1}{2} p\tpose D^{-1}(q) p + P(q) \\
                     &\begin{cases}
        \dot{q} = D\inv(q)p \\
        \dot{p} = -\frac{1}{2} (\Id{n} \otimes p\tpose) \nabla_q D\inv(q) p
        - \nabla_q P(q) + B(q) \tau \\
    \end{cases} 
\end{align}

Any set of coordinates \(\{q,p\}\) which satisfy Hamilton's equations 
under the Hamiltonian \(\mathcal{H}\) are
said to be \textit{canonical coordinates} for the system. A change of
coordinates \((q,p) \rightarrow (Q,P)\) is a \textit{canonical
transformation} if \(\{Q,P\}\) preserve the Hamiltonian structure; that is, if
they are canonical coordinates under the Hamiltonian
\(\mathcal{H}\left(q(Q,P), p(Q,P)\right)\).

\section{Simply Actuated Hamiltonian Systems}
\textbf{Manfredi: What term should we use instead of ``simply actuated"?}

% Describe the change of coordinates to get into q_u/q_a mode and show
% that the actuator directly affects pa but not pu. Use M for simply actuated
% inertia, D for normal coordinates
Suppose we are given a Hamiltonian mechanical system (\ref{eqn:hamiltonian}).
Because \(\tau\) is transformed by the input matrix \(B(q)\) before
entering the EOM, it is not in general clear how any particular input force \(\tau_i\)
will affect the dynamics of the system. 
In this section, we define a new class of Hamiltonian systems where the effect
of the input forces is made obvious. This class of
systems will form the backbone for the rest of the theory developed in this
thesis.

\begin{defn}
    Let \(\mathcal{H}\) be an \(n\)-DOF Hamiltonian system 
    with \(k \leq n\) actuators. 
    A set of canonical coordinates \(\{q,p\}\) for this system
    are said to be \textit{simply actuated coordinates} if the
    input matrix \(B(q) \in \R^{n \times k}\) is of the form
    \[
        B(q) = \simpleB    
    \]
    The first \((n-k)\) coordinates, labelled \(q_u\), are called the
    \textit{unactuated coordinates}. The remaining \(k\) coordinates, labelled
    \(q_a\), are called the \textit{actuated coordinates}. When grouping them
    together, we will always put them in the order \((q_u, q_a)\) to fit with 
    the definition. 
    The corresponding \((p_u, p_a)\) are called the unactuated and actuated
    momenta, respectively.
\end{defn}
\begin{defn}
    A Hamiltonian system is said to be \textit{simply actuated} if there exists
    a canonical transformation from any canonical coordinates 
    \(\{q,p\}\) into simply actuated coordinates.
\end{defn}

Under the following assumptions on the input matrix, we will show that the
Hamiltonian system (\ref{eqn:hamiltonian}) is simply actuated.

\begin{assm}\label{assm:B-const}
    The input matrix \(B(q) \equiv B \in \R^{n\times k}\) is constant,
    full rank, and \(k < n\).
\end{assm}
\begin{assm}\label{assm:B-perp}
    There exists a matrix 
    \(B^\perp \in \R^{(n-k)\times n}\)
    which is right semi-orthogonal 
    \(\left(B^\perp(B^\perp)\tpose = \Id{(n-k)}\right)\)
    and which is a left-annihilator for \(B\). 
    That is, \(B^\perp B = \Zmat{(n-k) \times k}\).
\end{assm}

Note that if \(k = (n-1)\), the existence of any left annihilator 
\(A^0 \in \R^{1\times n}\) implies the left annihilator 
\(B^\perp := A^0/\norm{A^0}\) satisfies Assumption \ref{assm:B-perp}.

\begin{assm}\label{assm:B-orthogonal}
    Assume \textbf{without loss of generality} that the input matrix \(B\) is
    left semi-orthogonal. That is, \(B\tpose B = \Id{k}\).
\end{assm}
\begin{proof}
Since \(B\) is a constant matrix, 
it has a singular-value decomposition 
\(B = U \Sigma V\tpose\) where \(U^{-1} = U\tpose \in \R^{n \times n}\), 
\(V^{-1} = V\tpose \in \R^{k \times k}\), and \(\Sigma \in \R^{n \times k}\) is
defined by
\[
    \Sigma = \begin{bmatrix}
        \sigma_1 & 0 & \cdots & 0 \\
        0 & \sigma_2 & \cdots & 0 \\
        \vdots & & \ddots & \vdots \\
        0 & 0 & \cdots & \sigma_k \\
        - &   & \Zmat{(n-k)\times k} & -  \\
    \end{bmatrix}
\]
where \(\sigma_i \neq 0\) because \(B\) is full-rank \cite{calculating_svd}.
Defining \(T \in \R^{k \times k}\) by
\[
    T = \begin{bmatrix}
        \frac{1}{\sigma_1} & 0 & \cdots & 0 \\
        0 & \frac{1}{\sigma_2} & \cdots & 0 \\
    \vdots & & \ddots & \vdots \\
    0 & 0 & \cdots & \frac{1}{\sigma_k} \\
    \end{bmatrix}
\]
and assigning the input forces to \(\tau = V T \hat{\tau}\), we get a new input
matrix for \(\hat{\tau} \in \R^k\) given by \(\hat{B} = B V T = U \Sigma T\) 
which is still constant and full-rank. In particular, 
\(\hat{B}\tpose \hat{B} = T\tpose \Sigma\tpose \Sigma\tpose T = \Id{k}\).
\end{proof}

Let \(\mathbf{B} \in \R^{n\times n}\) be the following matrix:
\[
    \mathbf{B} = 
    \begin{bmatrix}
        B^\perp \\
        B\tpose \\
    \end{bmatrix}
\]
Since \(B^\perp\) is a left annihilator of \(B\) and both \(B^\perp\) and
\(B\tpose\) are right semi-orthogonal, it is easy to show that \(\mathbf{B}\) is
orthogonal:
\[
    \mathbf{B}\mathbf{B}\tpose = 
    \begin{bmatrix}
        B^\perp (B^\perp)\tpose & B^\perp B \\
        (B^\perp B)\tpose & B\tpose B
    \end{bmatrix} = \Id{n}
\]
Hence, \(\mathbf{B}\) is invertible with \(\mathbf{B}\inv = \mathbf{B}\tpose\).

The following theorem shows that \(\mathbf{B}\) provides a canonical
transformation into simply actuated coordinates, so that only the actuated momenta
are affected by the input forces.

\begin{thm}\label{thm:simply-actuated}
    Under Assumptions \ref{assm:B-const},\ref{assm:B-perp}, and
    \ref{assm:B-orthogonal}, the Hamiltonian system (\ref{eqn:hamiltonian}) is
    simply actuated with simply actuated coordinates 
    \(\{Q = \mathbf{B}q, P = \mathbf{B}p\}\). The resulting dynamics are 
    given by (\ref{eqn:simple-hamiltonian}),
    \begin{align}\label{eqn:simple-hamiltonian}
        \mathcal{H}(Q,P) &= 
        \frac{1}{2} P\tpose \Minv(Q) P + V(Q) \\
       &\begin{cases}
            \dot{Q} = \Minv(Q)P \\
            \dot{P} = -\frac{1}{2} (\Id{n} \otimes P\tpose) \nabla_Q \Minv(Q) P
                - \nabla_Q V(Q) + \simpleB \tau
        \end{cases} \nonumber
    \end{align}
    where
    \begin{align*}
        \Minv(Q) &:= \mathbf{B}D^{-1}(\mathbf{B}\tpose Q)\mathbf{B}\tpose \\
        V(Q) &:= P(\mathbf{B}\tpose Q) \\
    \end{align*}
\end{thm}
\vspace{-4.5em} % For some reason not having this puts a large space before proof
\begin{proof}
    The Poisson bracket between the functions \(f(q,p)\) and \(g(q,p)\) is defined by
    \cite{landau_mechanics} as follows:
    \[
        [f,g] := \sum \limits_{i=1}^n \pdiff{f}{p_i}\pdiff{g}{q_i} - 
                \pdiff{f}{q_i}\pdiff{g}{p_i}
    \]
    For any constant matrix \(A\), the transformation
    \(\{Q = Aq, P = Ap\}\) satisfies
    \(\pdiff{Q_i}{p_m} = \pdiff{P_i}{q_m} = 0\) for all 
    \(i,m \in \n\).
    Hence, for our new coordinates \(\{Q = \mathbf{B}q, P = \mathbf{B}p\}\),
    \begin{align*}
        [Q_i,Q_j] &:= \sum\limits_{m = 1}^n \pdiff{Q_i}{p_m}\pdiff{Q_j}{q_m} - 
        \pdiff{Q_i}{q_m}\pdiff{Q_j}{p_m} = 0 \\
        [P_i,P_j] &:= \sum\limits_{m=1}^n \pdiff{P_i}{p_m}\pdiff{P_j}{q_m} -
        \pdiff{P_i}{q_m}\pdiff{P_j}{p_m} = 0
    \end{align*}
    Since the matrix \(\mathbf{B}\) is orthogonal,
    \((\mathbf{B}_i)\tpose (\mathbf{B}\tpose)_j = (\mathbf{B}_i)\tpose (\mathbf{B}^{-1})_j = \delta_{i,j}\). Using this
    fact we see that the Poisson brackets between \(P_i\) and \(Q_j\) are given by:
    \begin{align*}
        [P_i,Q_j] &= \sum\limits_{m=1}^n\pdiff{P_i}{p_m}\pdiff{Q_j}{q_m}
        - \pdiff{P_i}{q_m}\pdiff{Q_j}{p_m} \\
                  &= \sum\limits_{m=1}^n \mathbf{B}_{i,m}\mathbf{B}_{j,m} - 0 \\
                  &= \sum\limits_{m=1}^n \mathbf{B}_{i,m}\mathbf{B}\tpose_{m,j} \\
                  &= (\mathbf{B}_i)\tpose \mathbf{B}\tpose_j \\
                  &= \delta_{i,j}
    \end{align*}

    Therefore, by (45.10) in \cite{landau_mechanics}, the coordinate change 
    \((Q = \mathbf{B}q, P = \mathbf{B}p)\) is a canonical transformation.
    Furthermore, since \(\dot{P} = \mathbf{B}\dot{p}\), the new input matrix is
    given by 
    \[
        \mathbf{B}B = \begin{bmatrix}
            B^\perp B \\
            B\tpose B \\
        \end{bmatrix} = 
        \begin{bmatrix}
            \Zmat{(n-k)\times k} \\
            \Id{k}
        \end{bmatrix}
    \]
    so the coordinates \(\{Q = (q_u,q_a), P = (p_u,p_a)\}\) are simply actuated coordinates for 
    \(\mathcal{H}\) as desired.
\end{proof}

\section{Virtual Nonholonomic Constraints}
% Motivation for VNHCs, they are an extension of VHCs, etc.
Let us imagine a child on a swing, who wants to go as high as possible. The
child will push off the ground and start swinging with small oscillations,
then extend and retract their feet appropriately to gain energy. 
If a roboticist were designing a machine to do this task, they might design a
control mechanism which makes the legs track a trajectory over time. For
example, they might tell the robot to extend and retract its legs every two
seconds. In ideal situtations, this technique would work perfectly because the
leg motion is synchronized with the swing to gain energy as fast as possible.

% TODO: Get some sources here. Show with research that trajectory control is
% unsafe, and that path tracking is better.
Most children have an adult pushing the swing to help them go higher, or perhaps
they are swinging on a windy day. In either case, they adjust their leg
motion accordingly when presented with these external disturbances, without
keeping track of the time it takes to kick their legs. The
standard control technique of tracking a function of time, known as
\textit{trajectory tracking}, does not work as well as human motion 
because the disturbance affecting the swing will desynchronize the leg motion
and the swing, thereby stopping the energy-gaining effects. 
As has been observed in the field of mobile robotics,
a trajectory tracking controller which has lagged behind due to a disturbance
will do whatever is required to ``catch up" to
the trajectory. This can cause major safety issues, which is why mobile
roboticists prefer to design controllers which drive a vehicle towards a path in
space rather than a trajectory in time.

This path tracking approach has been extended to biologically-inspired robotics
in a method known as virtual holonomic constraints (VHCs).
Instead of a robot's actuators tracking
a trajectory over time, VHCs use the actuators to enforce a relation
\(h(q) = 0\) of the configuration \cite{vhcs_for_el_systems}. 
This method has provided incredible results in the development of 
walking robots \cite{vhc_robotic_walking, vhc_stable_walking}, 
vehicle motion \cite{vhc_bicycle, vhc_helicopter}, 
and has even been used to design a snake-like swimming robot
\cite{vhc_snake}.

The downside to VHCs is that they do not use the additional information imparted
by the generalized velocities of a system.
For the child on a swing, whether they extend or retract their legs
depends on their direction of motion. This inherently requires knowledge of
their velocity, which precludes the usage of VHCs. 
Many authors have attempted to extend the theory of VHCs to enforce relations 
\(h(q,\dot{q}) = 0\) of the full state to account for this drawback. 
Since these relations use actuators to
restrict both the configuration and generalized velocity, they are called
virtual \textit{nonholonomic} constraints. This idea has been used for
human-robot interaction
\cite{vnhc_human_robot_cooperation,psd_based_vnhc_redundant_manipulator,haptic_vnhc},
error-reduction on time-delayed systems \cite{vnhc_time_delay_teleop},
and has shown marked improvements to the field of bipedal locomotion 
\cite{nhvc_dynamic_walking,
hybrid_zero_dynamics_bipedal_nhvcs,output_nhvc_bipedal_control}.
Most interestingly, the nonholonomic approach has been shown to be more robust
than standard VHCs when applied to bipedal robotics \cite{nhvc_incline_walking}.
This suggests that virtual nonholonomic constraints are more capable of
stabilizing specific energy levels while rejecting disturbances, all while
producing realistic biological motion. It is for this reason we choose to study
nonholonomic constraints in this thesis.

Unlike the theory of VHCs, there does not appear to be a standard definition of
virtual nonholonomic constraints: 
all the applications listed above use their own definition of a virtual
nonholonomic constraint, which makes it difficult to compare and analyze their
work. 

This section will provide a new characterization of virtual nonholonomic
constraints in the Hamiltonian framework. 
The goal is to provide a consistent, rigorous foundation for
designing constraints on a general class of systems.

% Perform the full development of VNHCs, including its stabilizing
% controller.
\begin{defn}
    A \textit{virtual nonholonomic constraint} (VNHC) \textit{of order \(k\)} is a
    relation \(h(q,p) = 0\) where \(h : \mathcal{Q}\times\R^n \rightarrow \R^k\) is
    smooth, \(\rank{\left[ dh_q,\, dh_p \right]} = k\) for all 
    \((q,p) \in h\inv(0)\), and there exists a feedback controller \(\tau(q,p)\)
    stabilizing the set
    \[
        \Gamma = \left\{(q,p) \mid h(q,p) = 0, dh_q \dot{q} + dh_p \dot{p} = 0\right\}
    \]
    which is called the \textit{constraint manifold}.
\end{defn}

If we define the error term  \(e = h(q,p)\), stabilizing \(\Gamma\) is
equivalent to solving for \(\tau(q,p)\) which drives \(e \rightarrow 0\) and
\(\dot{e} \rightarrow 0\).
Let us imagine that we have no further structure on the VNHC. 
Then \(\dot{e} = dh_q \dot{q} + dh_p \dot{p}\), where \(\tau\) appears inside
\(\dot{p}\). To solve for \(\tau\) explicitly, we must have that \(dh_p\) is
invertible for all \((q,p)\).
%which is the same as requiring that \(h(q,p)\) is strictly monotonic in \(p\).

This is a very restrictive condition, which many VNHCs will not satisfy. For
this reason, it is preferable to have the torque \(\tau\) appear after two 
derivatives of \(e\) so that there is more freedom in the types of constraints
one can use.
Mathematically, if each \(\tau_i\) appears only after two derivatives, one says 
that \(e\) is of \textit{relative degree} \(\{2,2,\ldots,2\}\). 
We thus define a special type of VNHC which satisfies this
property.

\begin{defn}
    A VNHC \(h(q,p) = 0\) of order \(k\) is \textit{regular} if the output 
    \(e = h(q,p)\) is of relative degree \(\{2,2.\ldots,2\}\) everywhere on the
    constraint manifold \(\Gamma\).
\end{defn}

The authors of
\cite{nhvc_dynamic_walking,hybrid_zero_dynamics_bipedal_nhvcs,nhvc_incline_walking}
observed that a relation which uses only the unactuated conjugate of momentum
\(p_u\) cannot have \(\tau\) appearing after only one derivative. Of course,
they performed their research in Lagrangian form; 
for us to be able to use \(p_u\) in its Hamiltonian form, 
we will continue with the following assumption for the rest of the chapter.

\begin{assm}\label{assm:H-is-simply-actuated}
    The mechanical system under consideration is a
    Hamiltonian system with \(n\) degrees of freedom and 
    \(k < n\) actuators. It is described in simply
    actuated coordinates \(\{q = (q_u,q_a), p = (p_u, p_a)\}\) and has the
    following dynamics:
    \begin{align*}
        \mathcal{H}(q,p) &= p\tpose \Minv(q) p + V(q) \\
         &\begin{cases}
            \dot{q} = \Minv p \\
            \dot{p} = -\frac{1}{2} \pdmat - \nabla_q V(q) + \simpleB \tau \\
        \end{cases}
    \end{align*}
\end{assm}
\begin{notation}
    We will write \(q_u \in \mathcal{Q}_u\), \(q_a \in \mathcal{Q}_a\) where
    \(\mathcal{Q}_u \times \mathcal{Q}_a = \mathcal{Q}\). 
    We also write
    \(p_u \in \mathcal{P}_u := \R^{n-k}\) and 
    \(p_a \in \mathcal{P}_a := \R^k\), so that 
    \(p \in \mathcal{P} := \mathcal{P}_u \times \mathcal{P}_a = \R^n\). 
    In this manner, the phase space of our system can be written as
    \(\mathcal{Q} \times \mathcal{P}\).
\end{notation}

\begin{thm}\label{thm:vnhc-regularity}
    A VNHC \(h(q,p) = 0\) of order \(k\) is regular if and only if \(dh_{p_a} = 0\) 
    and
    \[
        \rank{\left(dh_q \Minv(q) - 
          dh_{p_u} (\Id{n-k} \otimes p\tpose)\nabla_{q_u}\Minv(q) 
         \right)\simpleB} = k
    \]
    everywhere on the constraint manifold \(\Gamma\).
    That is, a regular VNHC is a function with domain 
    \(\mathcal{Q} \times \mathcal{P}_u\) which satisfies the rank condition.
\end{thm}
\begin{proof}
    Let \(e = h(q,p) \in \R^k\). Then 
    \begin{align*}
        \dot{e} &= dh_q \dot{q} + dh_p \dot{p} \\
                &= dh_q \Minv(q)p +{}  \\
            & \begin{bmatrix} dh_{p_u} & dh_{p_a} \end{bmatrix}
        \left( -\frac{1}{2} \begin{bmatrix}
            (\Id{n-k} \otimes p\tpose) \nabla_{q_u}\Minv(q)p \\
            (\Id{k} \otimes p\tpose) \nabla_{q_a}\Minv(q)p
            \end{bmatrix} - \begin{bmatrix}
            \nabla_{q_u}V(q) \\
            \nabla_{q_a}V(q)
        \end{bmatrix} + \simpleB \tau\right)
    \end{align*}
    If \(dh_{p_a} \neq \Zmat{k \times k}\) for some \((q,p)\) on \(\Gamma\), 
    then \(\tau\) appears in \(\dot{e}\) and the VNHC is not of relative degree
    \(\{2,2,\ldots,2\}\).
    Hence, we must have that \(dh_{p_a} = \Zmat{k \times k}\).
    Proceeding with this assumption, we now find that
    \(h : \mathcal{Q} \times \mathcal{P}_u \rightarrow \R^k\), which means that
    \[
        \dot{e} = dh_q \Minv(q)p - 
        dh_{p_u} \left(\frac{1}{2} \pudmat + \nabla_{q_u}V(q)\right)
    \]
    Taking one further derivative provides
    \begin{align*}
        \ddot{e} &= \diff{}{t}\left\{dh_q\right\}\Minv(q) p + 
        dh_q \left(\sum\limits_{i=1}^n \pdiff{\Minv}{q_i}(q)\dot{q_i}\right)p + 
        dh_q \Minv(q) \dot{p} - \\
         & \diff{}{t}\left\{dh_{p_u}\right\}
         \left(\frac{1}{2}\pudmat + \nabla_{q_u}V(q)\right) - \\
         & dh_{p_u}\left(\frac{1}{2}\diff{}{t}\left\{\pudmat\right\} + 
         \diff{}{t}\left\{\nabla_{q_u}V(q)\right\} \right)
    \end{align*}
    We will compute the explicit expression of \(\ddot{e}\) in pieces to find
    out where \(\tau\) appears. 
% d/dt(dh_q)
    We begin with \(\diff{}{t}\left\{dh_q\right\}\). 
    Let \(h = (h^1,\ldots,h^k)\) with
    \(h^i : \mathcal{Q} \times \mathcal{P}_u \rightarrow \R\). By definition of
    the total differential, we have 
    \[
        dh_q = \begin{bmatrix}
            dh^1_q \\
            \vdots \\
            dh^k_q
        \end{bmatrix} = \begin{bmatrix}
        \pdiff{h^1}{q_1} & \cdots & \pdiff{h^1}{q_n} \\
        \vdots & \ddots & \vdots \\
        \pdiff{h^k}{q_1} & \cdots & \pdiff{h^k}{q_n}
        \end{bmatrix}
    \]
    The time derivative is taken element-wise, yielding
    \[
        \diff{}{t}\left\{dh_q\right\} = \begin{bmatrix}
            \sum_{j=1}^n \ppdiff{h^1}{q_1}{q_j}\dot{q}_j & \cdots & \sum_{j=1}^n
            \ppdiff{h^1}{q_n}{q_j}\dot{q}_j \\
            \vdots & \ddots & \vdots \\
            \sum_{j=1}^n \ppdiff{h^k}{q_1}{q_j}\dot{q}_j & \cdots & \sum_{j=1}^n
            \ppdiff{h^k}{q_n}{q_j}\dot{q}_j \\
        \end{bmatrix} + \begin{bmatrix}
            \sum_{l=1}^{n-k}\ppdiff{h^1}{q_1}{p_{u_l}}\dot{p}_{u_l} & \cdots &
            \sum_{l=1}^{n-k}\ppdiff{h^1}{q_n}{p_{u_l}}\dot{p}_{u_l} \\
            \vdots & \ddots & \vdots \\
            \sum_{l=1}^{n-k}\ppdiff{h^k}{q_1}{p_{u_l}}\dot{p}_{u_l} & \cdots & 
            \sum_{l=1}^{n-k}\ppdiff{h^k}{q_n}{p_{u_l}}\dot{p}_{u_l} \\
        \end{bmatrix}
    \]
    It is straightforward computation to confirm that each row of this
    derivative can be written in vector form as follows:
    \begin{align*}
        \diff{}{t}\left\{dh^i_q\right\} &= \begin{bmatrix}
        \sum_{j=1}^n \ppdiff{h^i}{q_1}{q_j}\dot{q}_j & \cdots & 
        \sum_{j=1}^n \ppdiff{h^i}{q_n}{q_j}\dot{q}_j 
        \end{bmatrix} + \begin{bmatrix}
        \sum_{l=1}^{n-k}\ppdiff{h^i}{q_1}{p_{u_l}}\dot{p}_{u_l} & \cdots &
        \sum_{l=1}^{n-k}\ppdiff{h^i}{q_n}{p_{u_l}}\dot{p}_{u_l}
        \end{bmatrix} \\
       &= \dot{q}\tpose \Hess_q\left\{h^i\right\}\tpose + 
       \dot{p}_u\tpose \partial_{p_u}\partial_{q}h^i \\
       &= p\tpose \Minv(q)\Hess_q\left\{h^i\right\}\tpose -
       \left(\frac{1}{2}p\tpose d\Minv_{q_u}(q)(\Id{n-k}\otimes p) + 
       dV_{q_u}(q)\right)\partial_{p_u}\partial_{q}h^i
    \end{align*}
    This means that the entire matrix is given by
    \begin{align*}
        \diff{}{t}\left\{dh_q\right\} &= 
        \left(\Id{n} \otimes 
        \left(p\tpose \Minv(q)\right)\right)
        \Hess_q\left\{h\right\}\tpose -{} \\
      & \left(\frac{1}{2}\Id{n} \otimes 
          \left(p\tpose d\Minv_{q_u}(q)(\Id{n-k}\otimes p)\right)
        + \Id{n}\otimes dV_{q_u}(q)\right)\partial_{p_u}\partial_{q}h
    \end{align*}
% C_1(q,p)
    For the next segment, let \(1_j \in \R^n\) be the vector of all zeros,
    except for a single \(1\) at position \(j\). We define
    \(C_1 : \mathcal{Q} \times \mathcal{P} \rightarrow \R^{n\times n}\) by
    \[
        C_1(q,p) := \sum\limits_{j=1}^n \pdiff{\Minv}{q_j}(q)\dot{q}_j 
        = \sum\limits_{j=1}^n \pdiff{\Minv}{q_j}(q)\left(1_j\tpose\Minv(q)p\right)
    \]
% d/dt(dh_pu)
    Moving on to \(\diff{}{t}\left\{dh_{p_u}\right\}\), a similar approach as
    \(\diff{}{t}\left\{dh_q\right\}\) yields the derivative in matrix form.
    It is given by:
    \begin{align*}
        \diff{}{t}\left\{dh_{p_u}\right\} &= 
        \left( \Id{n} \otimes \left(p\tpose \Minv(q)\right)\right) 
        \partial_q\partial_{p_u}h -{} \\
      & \left(\frac{1}{2}\Id{n} \otimes \left(\pudmat\right) + 
      \Id{n} \otimes \nabla_{q_u}V(q)\right)\Hess_{p_u}\left\{h\right\}\tpose
    \end{align*}
% d/dt(1/2 * pudmat)
    For the next piece, observe that the \(i^\text{th}\) row of
    \(\frac{1}{2} \diff{}{t} \left\{ \pudmat \right\}\)
    is given by
    \begin{align*}
        \frac{1}{2} \diff{}{t}\left\{p\tpose \pdiff{\Minv}{q_{u_i}}(q)p\right\}
        = p\tpose \pdiff{\Minv}{q_{u_i}}(q) \dot{p} + 
        \frac{1}{2} p\tpose \left( \sum_{j=1}^n \ppdiff{\Minv}{q_{u_i}}{q_j}
        \dot{q}_j \right) p
    \end{align*}
    Define \(C_2 : \mathcal{Q} \times \mathcal{P} \rightarrow \R^{n(n-k)\times n}\) 
    by
    \[
        C_2(q,p) := \begin{bmatrix}
            \sum_{j=1}^n \ppdiff{\Minv}{q_{u_1}}{q_j} (1_j\tpose \Minv(q)p) \\
            \vdots \\
            \sum_{j=1}^n \ppdiff{\Minv}{q_{u_{n-k}}}{q_j} (1_j\tpose \Minv(q)p) \\
        \end{bmatrix}
    \]
    In matrix form, this becomes:
    \begin{align*}
        & \frac{1}{2} \diff{}{t} \left\{ \pudmat \right\} ={} \\
        & (\Id{n-k} \otimes p\tpose)\nabla_{q_u}\Minv(q)
        \left(-\frac{1}{2}\pudmat - \nabla_{q_u}V(q) + \simpleB \tau \right) +{}
        \\
        & \frac{1}{2}\left(\Id{n-k}\otimes p\tpose\right) C_2(q,p) p
    \end{align*}
% d/dt(nabla_qu V)
    Finally, we compute the derivative of \(\nabla_{q_u}V(q)\):
    \begin{align*}
        \diff{}{t}\left\{\nabla_{q_u}V(q)\right\} = 
        \begin{bmatrix}
            \sum_{j=1}^n \ppdiff{V}{q_{u_1}}{q_j}\dot{q}_j \\
            \vdots \\
            \sum_{j=1}^n \ppdiff{V}{q_{u_{n-k}}}{q_j}\dot{q}_j
        \end{bmatrix} = \partial_{q_u} \partial_q V(q) \dot{q} 
        = \partial_{q_u} \partial_q V(q) \Minv(q)p
    \end{align*}
% Full ddot(e)
    Putting this all together, we can find the explicit form for \(\ddot{e}\):
    \begin{align*}
        \ddot{e} &= \diff{}{t}\left\{dh_q\right\}\Minv(q)p + dh_q C_1(q,p)p -{}\\
     & \frac{1}{2}dh_q\Minv(q)(\pdmat) + 
        dh_q\Minv(q)\nabla_qV(q) -{}\\
     & \diff{}{t}\left\{dh_{p_u}\right\}
     \left(\frac{1}{2}\pudmat + \nabla_{q_u}V(q)\right) +{} \\
     & dh_{p_u}(\Id{n-k}\otimes p\tpose)\nabla_{q_u}\Minv(q)\left(
     \frac{1}{2}\pudmat + \nabla_{q_u}V(q)\right) -{}\\
     & \frac{1}{2}dh_{p_u}(\Id{n-k}\otimes p\tpose)C_2(q,p)p - 
     dh_{p_u} \partial_{q_u}\partial_q V(q)\Minv(q)p +{} \\
     & \left(dh_q \Minv(q) - dh_{p_u}(\Id{n-k} \otimes
     p\tpose)\nabla_{q_u}\Minv(q) \right) \simpleB \tau
    \end{align*}
% End of Proof
    For shorthand, we'll write \(\ddot{e} = E(q,p) + H(q,p)\tau\) where \(E\)
    and \(H\) are defined appropriately.
    From the definition of regularity, the VNHC \(h\) is regular 
    when \(e\) is of relative degree \(\{2,\ldots,2\}\), which is true 
    if and only if
    %\(\tau\) appears in each row of \(\ddot{e}\) on \(\Gamma\). 
    one can solve for \(\tau\) when \(\ddot{e} = 0\).
    This is equivalent to requiring that the matrix \(H\) be
    %full rank everywhere on the constraint manifold, 
    invertible,
    proving the theorem.
\end{proof}

Using the expression \(\ddot{e} = E(q,p) + H(q,p)\tau\) from the proof of 
Theorem \ref{thm:vnhc-regularity}, a regular VNHC of order \(k\) can be
stabilized by the output-linearizing phase-feedback controller
(\ref{eqn:vnhc-torque-controller}).
\begin{equation}\label{eqn:vnhc-torque-controller}
    \tau(q,p) = -H\inv(q,p)\left(E(q,p) + k_p e + k_d \dot{e}\right)
\end{equation}
where \(k_p, k_d \in \R_{>0}\) are control parameters which can be tuned on the
resulting linear system \(\ddot{e} = -k_p e - k_d\dot{e}\). 

Note that one generally cannot measure conjugate of momenta directly, as sensors
on mechanical systems will only measure the state \((q,\dot{q})\). To
implement this controller in practice, one must compute \(p = M(q)\dot{q}\) at
every iteration. In other words, this controller requires knowledge of the full
state of the system.

% Special case: when dM/dqu = 0, show that we have a nice form and that we
%can solve for p_a and the closed-loop dynamics (qu,pu)_dot

Now that we have found a controller to enforce a regular VNHC of order \(k\), we
would like to solve for the closed-loop dynamics. Intuitively, these dynamics
should be parameterized by \((q_u, p_u)\) since \(q_a\) is a function of these
as specified by \(h(q,p_u) = 0\).
Unfortunately, \(\dot{q}_u\) depends on \(p_a\), and for general systems one
cannot solve explicitly for \(p_a\) in terms of \((q_u,p_u)\). This is because
the \(\dot{p}\) dynamics contains the coupling term 
\((\Id{n} \otimes p\tpose)\nabla_{q_u}M(q)p\). 

We now introduce a class of systems where explicitly solving for the closed-loop
dynamics is feasible.

% TODO: Come up with a name for this type of system
\begin{defn}
    A mechanical system is \textbf{TODO: name this} ------ if 
    \(\nabla_{q_u}M(q) = \Zmat{n(n-k) \times n}\).
\end{defn}

% TODO: Do we need the assumption that one can solve for qa linearly?
\begin{thm}\label{thm:zero-dynamics}
    Let \(\mathcal{H}\) be a ----- mechanical system satisfying Assumption
    \ref{assm:H-is-simply-actuated}. Let
    \(h(q,p_u) = 0\) be a regular VNHC of order \(k\) 
    with constraint manifold \(\Gamma\). Suppose that on \(\Gamma\) one can
    solve linearly for \(q_a\) as a function of \((q_u,p_u)\).
    Then the closed-loop dynamics are given by
    \begin{equation}\label{eqn:qpu-dynamics}
        \left.\begin{aligned}
            \dot{q}_u &= M(q)p \\
            \dot{p}_u &= -\nabla_{q_u}V(q) \\
            \end{aligned}\right|_{\begin{array}{c}
                h(q,p_u) = 0 \\ 
                pa = g(q_u,p_u) \\
            \end{array}}
    \end{equation}
    where
    \begin{equation}\label{eqn:g-qpu}
    \begin{aligned}
        g(q,p_u) &:= \\
           &\left.\left(dh_q \Minv(q) \simpleB \right)\inv 
        \left(dh_{p_u} \nabla_{q_u}V(q) - dh_q \Minv(q)
        \begin{bmatrix}
            I_{n-k} \\
            \Zmat{k \times (n-k)} \\
        \end{bmatrix} p_u\right)\right|_{h(q,p_u) = 0}
    \end{aligned}
    \end{equation}
\end{thm}
\begin{proof}
    Setting \(e = h(q,p_u)\) and using the fact that 
    \(\nabla_{q_u}\Minv(q) = 0\), we find that
    \[
        \dot{e} = dh_q\Minv(q)p - dh_{p_u}\nabla_{q_u}V(q)
    \]
    Observe that
    \begin{align*}
        dh_q\Minv(q)p &= dh_q\Minv(q) \begin{bmatrix}p_u \\ p_a \end{bmatrix} \\
          &= dh_q\Minv(q) \begin{bmatrix}
              \Id{n-k} & \Zmat{(n-k)\times k} \\
              \Zmat{k \times(n-k)} & \Id{k} 
              \end{bmatrix} \begin{bmatrix} p_u \\ p_a \end{bmatrix} \\
        &= dh_q\Minv(q)\begin{bmatrix} \Id{n-k} \\ \Zmat{k \times
        (n-k)}\end{bmatrix} p_u + dh_q \Minv(q)\simpleB p_a
    \end{align*}
    On the constraint manifold, we have \(e = \dot{e} = 0\), which means
    \[
        dh_q\Minv(q)\simpleB p_a = dh_{p_u}\nabla_{q_u}V(q) -
            dh_q\Minv(q)\begin{bmatrix} 
            \Id{n-k} \\ \Zmat{k \times (n-k)}\end{bmatrix} p_u
    \]
    Since \(h\) is regular and \(\nabla_{q_u}\Minv(q) = 0\), we have that 
    \[
        \rank{dh_q\Minv(q)\simpleB} = k
    \]
    Solving for \(p_a\) gives
    \[
        p_a(q,p_u) = \left(dh_q\Minv(q)\simpleB\right)\inv
        \left(dh_{p_u}\nabla_{q_u}V(q) - 
            dh_q\Minv(q)\begin{bmatrix} 
        \Id{n-k} \\ \Zmat{k \times (n-k)}\end{bmatrix} p_u \right)
    \]
    This yields a function \(p_a(q,p_u)\). However, on \(\Gamma\) we have 
    \(h(q,p_u) = 0\) and can solve for \(q_a\) in terms of \((q_u,p_u)\);
    hence, we can solve for \(p_a = g(q_u,p_u)\). Since \(q_a\) and
    \(p_a\) can be computed directly from \((q_u,p_u)\), the dynamics on
    \(\Gamma\) are parameterized only by \((\dot{q}_u,\dot{p}_u)\).

    \textbf{MANFREDI: Do we need this assumption that \(q_a\) can be solved in
        terms of \((q_u,p_u)\) or is there some other way of parameterizing it
    like you did with \(s\) for VHCs?}
\end{proof}

Theorem \ref{thm:zero-dynamics} shows that, for a particular class of systems
and constraints, the dynamics on \(\Gamma\) are entirely described by the \(2(n-k)\)
unactuated coordinates. 
This is true regardless of the number of degrees of freedom of the system, which
means that \(\Gamma\) is a \(2(n-k)\)-dimensional surface inside 
\(\mathcal{Q}\times \mathcal{P}\). 

The following corollay applies Theorem \ref{thm:zero-dynamics} to
systems with only one unactuated coordinate.

\begin{cor}\label{cor:2d-zero-dynamics}
    Suppose \(\mathcal{H}\) is a ---- mechanical system satisfying Assumption
    \ref{assm:H-is-simply-actuated} which has degree of underactuation one.
    Let \(h(q,p_u) = 0\) be a regular VNHC of order \((n-1)\) of the form
    \(h(q,p_u) = q_a - f(q_u,p_u)\)
    where \(f\) is a suitably-defined smooth function.
    Then \(dh_q = \begin{bmatrix} -\partial f_{q_u} & \Id{(n-1)}
    \end{bmatrix}\).
    Defining \(e_1 := (1,0,\ldots,0) \in \R^n\), the actuated momentum is given
    by (\ref{eqn:g-qupu}):
    \begin{equation}\label{eqn:g-qupu}
        p_a = -\left.\left(dh_q \Minv(q)
        \begin{bmatrix}
            \Zmat{1\times (n-1)}\\
            I_{(n-1)}
        \end{bmatrix}\right)\inv 
        \left(\partial_{p_u}f \partial_{q_u}V + dh_q \Minv(q)e_1 p_u\right) 
            \right|_{q_a =f(q_u,p_u)}
    \end{equation}
    Since \(q_u \in \Rt{T}\) for some \(T \in ]0,\infty]\) and \(p_u \in \R\),
    the orbit \((q_u(t),p_u(t))\) traces out
    a curve on the 2D-plane \(\Rt{T} \times \R\) which we call the \((q,p)\)-plane.
\end{cor}

\section{Summary of Results}
%TODO: summarize the assumptions and results 
\textbf{TODO: summarize the assumptions and results }















%\section{Virtual Nonholonomic Constraints}\label{sec:vnhcs}
%
%%----- Motivation -----%
%\subsection{Motivation}
%\textbf{TODO: Why do we bother with Hamiltonian? Why can't we do virtual nonholonomic 
%constraints in Lagrangian? What are some use-cases where VHCs don't work?}
%
%%---------- Hamiltonian from Lagrangian ----------%
%\subsection{Review: Hamiltonian Systems}
%\textbf{TODO: What is a Hamiltonian system and why does it matter?}
%
%One can compute the Hamiltonian of a system by performing a Legendre transform
%on its Lagrangian \textbf{TODO: citation for legendre transform}.
%First, define the conjugate of momenta for \(q\) by 
%\begin{equation*}
%p = \frac{\partial \mathcal{L}}{\partial \dot{q}} \in \mathbb{R}^n
%\end{equation*}
%Then, the Legendre transform is performed by taking
%\begin{equation*}
%\mathcal{H}(q,p) = p^T \dot{q} - \mathcal{L}(q,\dot{q})
%\end{equation*}
%
%For a mechanical system (\ref{eqn:lagrangian}), the conjugate of momenta for
%\(q\) is given by 
%\begin{equation*}
%    p = M(q)\dot{q}
%\end{equation*}
%which means the Hamiltonian of the system is given by
%the total mechanical energy \(E\) (\ref{eqn:hamiltonian} in 
%\((q,p)\) coordinates.
%\begin{equation}\label{eqn:hamiltonian}
%\mathcal{H}(q,p) = E(q,p) = \frac{1}{2} p^T \Minv(q) p + V(q)
%\end{equation}
%Note that \(M(q)\) is the inertia matrix and \(V(q)\) is the potential for
%the Hamiltonian system. This is simply a trick to distinguish them notationally
%from the Lagrangian \(D(q)\) and \(P(q)\); they are, in fact, identical in their
%contents.
%
%The equations of motion for the system in Hamiltonian coordinates is given by
%\begin{align}\label{eqn:hamiltionian_eom}
%\begin{split}
%\dot{q} &= \frac{\partial \mathcal{H}}{\partial p} = \Minv(q) p \\
%\dot{p} &= -\frac{\partial \mathcal{H}}{\partial q} + B \tau
%\end{split}
%\end{align}
%
%\textbf{Why bother looking at Hamiltonian systems? What is the intuition behind
%these equations of motion?}
%
%%---------- Hamiltonian VNHCs ----------%
%\subsection{Hamiltonian Virtual Nonholonomic Constraints}
%While VHCs are still possible in the Hamiltonian framework, the assumptions
%required to make this work are slightly different. Rather than deriving
%Hamiltonian VHCs directly, we will produce results for nonholonomic constraints
%first as VHCs are a special case of this new framework.
%
%Suppose the mechanical system has degree of underactuation one, so that coordinates of the system can be split into an unactuated component \(q_u \in [\mathbb{R}]_T, \, T \in \mathbb{R}_{>0}\) which is not influence by control, along with an actuated component \(q_a\); that is, suppose \(B\) is of the form \(B(q) = [0_m, B_1^T(q), \ldots, B_n^T(q)]^T, \, B_i^T(q) \in \mathbb{R}^{n-1}\) and \(\tau \in \mathbb{R}^{n-1}\). In this case, \(q = (q_u, q_a)^T\) and the equations of motion become
%\begin{align}\label{eqn:unactuated_actuated_eom}
%\begin{split}
%\dot{q_u} &= e_1^T \Minv(q) p \\
%\dot{p_u} &= -p^T\frac{\partial M}{\partial q_u} p - \partial_{q_u}V(q) \\
%\dot{q_a} &= 
%\begin{bmatrix}
%0 \cdots 0 \\
%I_{n-1} \\
%\end{bmatrix} \Minv(q) p \\
%\dot{p_a} &= -p^T\frac{\partial M}{\partial q_a} p - \nabla{q_a}V(q) + 
%\begin{bmatrix}
%B_1(q) \\
%\vdots \\
%B_n(q)
%\end{bmatrix} \tau
%\end{split}
%\end{align}
%
%Now we can begin to talk about Virtual Nonholonomic Constraints. In a similar fashion to what was defined for VHCs, let us first define the goal of these new virtual constraints.
%
%\begin{defn}
%A relation \(h \in C^2\left(\mathcal{Q}\times \mathbb{R}^n ; \mathbb{R}^k\right)\) with \(h(q,p) = 0\) is a \textbf{virtual nonholonomic constraint (VNHC) of order k} if there exists a feedback control \(\tau(q,p)\) which stabilizes the constraint manifold
%\[
%\Gamma = \left\{(q,p) | h(q,p) = 0, dh_q \dot{q} + dh_p \dot{p} = 0\right\}
%\]
%\end{defn}
%Define the output of the system to be \(e = h(q,p)\). We would like to find \(\tau(q,p)\) which drives \(e\) to zero to stabilize our constraint manifold \(\Gamma\). To accomplish this, we will input-output linearize (\ref{eqn:unactuated_actuated_eom}) to find \(\ddot{e} = -k_p e - k_d \dot{e}\) with \(k_p, k_d \in \mathbb{R}_{> 0}\).
%
%To characterize a certain class of VNHCs, let us make the following assumption.
%\begin{assm}\label{assm:vnhc_is_on_qu_pu}
%We assume our relation \(h\) is of the form \(h(q,p) = q_a - f(q_u,p_u)\) for some \(f \in C^2\left([\mathbb{R}]_T \times \mathbb{R} ; \mathbb{R}^{n - 1}\right)\).
%\end{assm}
%
%Now we solve for \(\tau\) by finding \(\ddot{e}\).
%\begin{align*}
%    e &= h(q,p) = qa - f(q_u,p_u)\\
%    \Rightarrow \dot{e} &= \dot{q_a} - df_{q_u}\dot{q_u} -df_{p_u}\dot{p_u} \\
%    &= [-df_{q_u} I_{n-1}]\dot{q} - df_{p_u} \dot{p_u} \\
%    &= dh_q \Minv(q) p - df_{p_u}\left( -\frac{1}{2}p^T \frac{\partial \Minv(q)}{\partial q_u} p - \partial_{q_u}V(q) \right) \\
%    &= dh_q \Minv(q) p + \frac{1}{2}df_{p_u} p^T \frac{\partial \Minv(q)}{\partial q_u} p + df_{p_u}\partial_{q_u}V(q)
%\end{align*}
%The control input \(\tau\) only appears in \(\dot{p_a}\) (see (\ref{eqn:unactuated_actuated_eom})). To simplify the analysis, terms in \(\ddot{e}\) which do not depend on \(\dot{p_a}\) explicitly are lumped together under the symbol \((*)\):
%\begin{align*}
%    \ddot{e} &= dh_q \Minv(q) \dot{p} + df_{p_u} p^T \frac{\partial \Minv(q)}{\partial q_u} \dot{p} + (*) \\
%    &= (dh_q \Minv(q) + df_{p_u} p^T \frac{\partial \Minv(q)}{\partial q_u})B\tau + (*) \\
%    &= (dh_q \Minv(q) + dh_{p_u} p^T \frac{\partial \Minv(q)}{\partial q_u})B\tau + (*)
%\end{align*}
%
%From the derivations above, one can solve for \(\tau\) iff the matrix on the left of \(\tau\) is full rank. Thus, for systems with degree of underactuation one we give the following definition.
%\begin{defn}
%A VNHC \(h(q,p) = 0\) of order \(n - 1\) is \textbf{regular} if \(dh_{p_a} = 0\), \(dh_{q_a} = (1 \ldots 1)^T\), and 
%\[
%\text{rank}\left\{ (dh_q \Minv(q) + dh_{p_u} p^T \frac{\partial \Minv(q)}{\partial q_u})B\right\} = n - 1
%\]
%everywhere on the constraint manifold \(\Gamma\). Equivalently, a VNHC \(h\) of order \(n - 1\) is regular if it satisfies Assumption \ref{assm:vnhc_is_on_qu_pu} and system (\ref{eqn:unactuated_actuated_eom}) with output \(e = h(q,p)\) is of relative degree \(\{2,2,\ldots,2\}\) everywhere on \(\Gamma\).
%\end{defn}
%
%In general, \(\dot{e}\) is a function of \(q_u\) and \(p = (p_u,p_a)^T\). Since the purpose of a regular VNHC is to fully parameterize \(\Gamma\) by \((q_u,p_u)\), it is essential that one can solve for \(p_a = p_a(q_u,p_u)\). Unfortunately this often cannot be done, since \(\dot{e}\) contains the quadratic term
%\[
%\frac{1}{2} df_{p_u} p^T \frac{\partial \Minv(q)}{\partial q_u} p 
%\]
%We can solve for \(p_a\) if this quadratic term does not exist.
%\begin{assm}\label{assm:M_is_Mqa}
%Assume \(\partial M(q) / \partial q_u = 0 \Leftrightarrow \partial \Minv(q) / \partial q_u = 0\)
%\end{assm}
%
%Under Assumption \ref{assm:M_is_Mqa}, we get that the rank condition for \(h(q,p)\) to be a regular VNHC reduces to \(\text{rank}\left(dh_q \Minv B\right) = n - 1\). This is the same rank condition as required for Virtual Holonomic Constraints.
%
%Now we solve for \(p_a\) on the constraint manifold (when \(e = \dot{e} = 0\)):
%\begin{align*}
%    \dot{e} = dh_q \Minv(q)p + df_{p_u} \partial_{q_u}V(q) &= 0\\
%    \Leftrightarrow dh_q \Minv(q)e_1 p_u + dh_q \Minv(q) \begin{bmatrix}
%    0 & \cdots & 0 \\
%    & I_{n-1} & \\
%    \end{bmatrix} p_a &= -df_{p_u} \partial_{q_u}V(q) \\
%\end{align*}
%\begin{align*}
%    \Leftrightarrow dh_q \Minv(q) \begin{bmatrix}
%    0 & \cdots & 0 \\
%    & I_{n-1} & \\
%    \end{bmatrix} p_a = -\left(df_{p_u}\partial_{q_u}V(q) + dh_q \Minv(q)e_1 p_u\right)
%\end{align*}
%One can linearly solve for \(p_a\) if and only if the matrix in front of it is invertible.
%
%This leads us to a natural definition.
%\begin{defn}\label{defn:solvability}
%A VNHC \(h(q,p)\) is \textbf{solvable} (NOTE: actionable? what's a good name?) if
%\[
%\text{rank}\left(dh_q \Minv(q) \begin{bmatrix}
%    0 & \cdots & 0 \\
%    & I_{n-1} & \\
%    \end{bmatrix}\right) = n - 1
%\]
%\end{defn}
%
%With this analysis and our new definition in hand, we can solve for the dynamics on the constraint manifold.
%
%\begin{thm}\label{thm:equation_for_pa}
%Suppose assumptions \ref{assm:vnhc_is_on_qu_pu} and \ref{assm:M_is_Mqa} hold.
%If a regular VNHC \(h(q,p) = q_a - f(q_u,p_u)\) is solvable, then the
%parameterization for \(p_a\) on the constraint manifold is given by
%\begin{align*}
%p_a &= -\left(dh_q \Minv(q) \begin{bmatrix}
%    0 & \cdots & 0 \\
%    & I_{n-1} & \\
%    \end{bmatrix}\right)^{-1}\left( df_{p_u}\partial_{q_u}V(q) + dh_q \Minv(q)e_1 p_u\right) \\
%    &=: g(q_u,p_u)
%\end{align*}
%and the constrained dynamics on \(\Gamma\) are given by
%\begin{align*}
%    \dot{q_u} &= e_1^T \Minv(q_a) \begin{bmatrix}
%    p_u \\
%    p_a
%    \end{bmatrix}\mid_{q_a = f(q_u,p_u), p_a = g(q_u,p_u)} \\
%    \dot{p_u} &= -\partial_{q_u} V(q_u,q_a) \mid_{q_a = f(q_u,p_u)}
%\end{align*}
%\end{thm}
%
%Theorem \ref{thm:equation_for_pa} guarantees that, on \(\Gamma\), \(q_a\) is a
%parameterized completely by \((q_u,p_u)\). Hence, the zero-dynamics on
%\(\Gamma\) are always two-dimensional regardless of the original dimension
%\(n\).
%
%%---------- Hamiltonian VHCS ---------%
%\subsection{Restriction to Hamiltonian VHCs}
%\textbf{TODO: Why can we not do the standard VHC approach for Hamiltonian? 
%Show how to use the above to make it work}
%
%%/========== /Virtual Nonholonomic Constraints ==========/% 
% vim: set tw=80 ts=4 sw=4 sts=0 et ffs=unix :
