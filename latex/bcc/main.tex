%! TEX program = lualatex

\documentclass[12pt]{article}
\usepackage[utf8]{inputenc}

%----------- Choose Margins ----------%
\usepackage[margin=1in]{geometry}

%---------- Bibliography Style ----------%
%\usepackage[style=ieee,backend=biber]{biblatex}
%\addbibresource{bib.bib}

%/========== Preamble ==========/%

%----------- Math Packages -----------%
\usepackage{amsfonts}
\usepackage{mathtools}

%---------- Remove indenting from paragraphs ----------%
% Also makes each new paragraph start with an vertical space.
\usepackage[parfill]{parskip}

%/========== /Preamble ==========/%

%/========== Main Document ==========/%
\begin{document}
%---------- Title ----------%
\title{Energy Injection for Mechanical Systems via Virtual Nonholonomic Constraints}
\author{Adan Moran-MacDonald}
\maketitle

%---------- Motivation ----------%
When a gymnast is hanging on a bar, they do not attempt to track a trajectory
over time to perform their swing-up routine (a so-called ``giant");
likewise, a child on a standing swing does not swing higher by standing or
squatting at specific time intervals. 
Rather, these natural human motions involve moving one's legs based on the
current direction and speed of motion. 
In doing so, the gymnast will gain enough energy to start rotating on the bar,
and the child will reach greater heights on their swing. 
This style of realistic, energy-increasing motion can be achieved when
controlling mechanical systems by using the method of virtual nonholonomic
constraints.

%---------- Summary of Ideas ----------%
A \textit{virtual nonholonomic constraint} (VNHC) is a smooth relation 
\(
    h : T^\star \mathcal{Q} \rightarrow \mathbb{R}^{n-1}
\) 
between the configuration variables \(q \in \mathcal{Q}\) and conjugate of
momenta \(p \in T_q^\star\mathcal{Q}\) of an \(n\)-degree-of-freedom mechanical
system with \(n-1\) control inputs, such that \(h(q,p) = 0\).
Enforcing these types of constraints via feedback control provides
time-independent controllers which can inject or remove energy from the system.

%---------- Results ----------%
This talk will give sufficient conditions under which a VNHC can be enforced by
feedback control.
Then, using the motivations of the child on the standing swing and the gymnast
on a bar, we apply the theory to develop energy-injecting VNHCs for two
classical systems: the variable-length pendulum and the two-link acrobot. 

For the variable-length pendulum we design and enforce a VNHC which achieves a
time-optimal swingup motion, without tracking a signal over time. 
We then prove that any pendulum constrained by this VNHC will gain
enough energy to start rotating completely around the swinging axis.

For the acrobot, we design several VNHCs (modelled from the leg motions of human
gymnasts) which achieve human-like ``giant" motion.
We then apply one of these VNHCs to a real acrobot and, through experiments,
demonstrate the energy-injection properties of the VNHC as the robot kicks its
legs until it begins doing backflips on the bar. 

%---------- Conclusion ----------%
In summary, this talk introduces the notion of virtual nonholonomic constraints
and demonstrates their potential for generating realistic, biologically-inspired
motions which inject energy into mechanical systems while forgoing any
dependence on time in the control architecture.

%---------- Bibliography ----------%
%\newpage
%\printbibliography
\end{document}
%/========== /Main Document ==========/%

% vim: set tw=80 ts=4 sw=4 sts=0 et ffs=unix :
