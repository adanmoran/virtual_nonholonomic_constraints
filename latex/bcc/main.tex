%! TEX program = lualatex

\documentclass[12pt]{article}
\usepackage[utf8]{inputenc}

%----------- Choose Margins ----------%
\usepackage[margin=1in]{geometry}

%---------- Bibliography Style ----------%
\usepackage[style=ieee,backend=biber]{biblatex}
\addbibresource{bib.bib}

%/========== Preamble ==========/%

%----------- Math Packages -----------%
\usepackage{amsfonts}
\usepackage{mathtools}
\usepackage{amsthm} % allow theorems and definitions as a block

%---------- Remove indenting from paragraphs ----------%
% Also makes each new paragraph start with an vertical space.
\usepackage[parfill]{parskip}

%/========== /Preamble ==========/%

%/========== Main Document ==========/%
\begin{document}
%---------- Title ----------%
\title{Energy Injection for Mechanical Systems via Virtual Nonholonomic Constraints}
\author{Adan Moran-MacDonald}
\maketitle

%---------- Content ----------%
When a gymnast hanging on a bar performs a swing-up routine (a so-called ``giant"),
they do not attempt to track a trajectory over time to start doing backflips.
Likewise, a child on a swing does not swing higher by kicking their legs at
specific time intervals; rather, these natural human motions change one's leg
angle based on the current direction and speed of motion. In doing so, the
gymnast will gain enough energy to start rotating on the bar, 
and the child will reach greater heights on their swing.
This type of motion can be adapted to the control of mechanical systems 
using the method of virtual nonholonomic constraints.

A \textit{virtual nonholonomic constraint} (VNHC) is a smooth relation 
\[
    h : T^\star \mathcal{Q} \rightarrow \mathbb{R}^{n-1}
\]
between the \(n\) configuration variables \(q \in \mathcal{Q}\) and conjugate of
momenta \(p\) of a mechanical system with \(n-1\) control inputs.
Enforcing these types of constraints via feedback control provides
time-independent controllers which can inject or remove energy from the system.

This talk will give sufficient conditions under which a VNHC 
can be enforced by feedback control.
Then, using the motivations of the child on the swing and the gymnast,
we apply the theory to create VNHCs for two classical mechanical systems:
the variable-length pendulum and the two-link acrobot. 

For the variable-length pendulum we design and enforce a VNHC which achieves a
time-optimal swingup motion, without tracking a time signal. 
Then, we prove that a system constrained by this VNHC will gain energy and start
rotating.

For the acrobot, we design several VNHCs which achieve human-like motion by
observing the leg motions of human gymnasts. We then apply one of these
VNHCs to a real acrobot and, through experiments, demonstrate the
energy-injection properties of the VNHC. 

This talk demonstrates the potential of VNHCs in generating realistic,
biologically-inspired motions which can inject energy into a mechanical system
while forgoing any dependence on time in the control architecture.

%---------- Bibliography ----------%
\newpage
\printbibliography
\end{document}
%/========== /Main Document ==========/%

% vim: set tw=80 ts=4 sw=4 sts=0 et ffs=unix :
