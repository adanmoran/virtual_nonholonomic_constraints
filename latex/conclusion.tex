%! TEX root = main.tex

%/========== Conclusion ==========/%

\chapter{Conclusion}\label{ch:conclusion}

This thesis has shown the utility of VNHCs as a means of injecting energy into
mechanical systems.
In Chapter \ref{ch:vnhcs} we developed the framework of VNHCs for underactuated
mechanical systems, with a focus on VNHCs for simply actuated Hamiltonian
systems.
We then applied this framework to two benchmark systems: the variable-length
pendulum and the acrobot.
In Chapter \ref{ch:vlp} we proved that a certain class of VNHCs will always
inject energy into the VLP, while in Chapter \ref{ch:acrobot} we designed a
constraint inspired by gymnastics and proved it injects energy into the
acrobot.
In the end, we demonstrated that virtual nonholonomic constraints are capable of
injecting and dissipating energy in a robust manner, all while producing
realistic biological motion.

\section{Limitations and Future Research}
Our VNHC framework relies on the following assumptions:
\begin{enumerate}
    \item The input matrix \(B(q) \equiv B \in \R^{n \times k}\) is constant and
        full rank.
    \item The input matrix has a left-annihilator 
        \(B^\perp \in \R^{(n-k)\times n}\). 
    \item The annihilator matrix \(B^\perp\) is right semi-orthogonal.
    \item The inertia matrix of the system satisfies 
        \(\nabla_{q_u}M(q) = \Zmat{n(n-k) \times n}\).
    \item On the constraint manifold \(\Gamma\), one can solve for \(q_a\) as a
        function of \((q_u,p_u)\).
\end{enumerate}
If any of these assumptions are not satisfied, one may not be able to find the
constrained dynamics in simply actuated coordinates, and the results of this
thesis may not hold.
In particular, the theoretical guarantees of this thesis do not apply to systems
with physical nonholonomic constraints (\eg friction), nor do they apply to vehicles
with wheels.

These limitations guide us toward the following research directions. 
First, one might relax the assumptions on the input and inertia
matrices, thereby widening the class of systems to which VNHCs can be applied.
Second, one might forgo the assumption that \(q_a\) is solvable on the
constraint manifold; instead, one could borrow from the VHC literature
and parameterize the constraint manifold with a different set of
coordinates than \((q_u,p_u)\).
Third, one might consider more general mechanical systems with dynamics given by
\[
    \begin{cases}
        \dot{q} = \Minv(q) p
        , \\
        \dot{p} = -\pdmat - \nabla_q V(q) + Q_{nh}(q,p) + B(q)\tau
        ,
    \end{cases}
\]
where \(Q_{nh}(q,p) \in \R^n\) are terms corresponding to physical nonholonomic
constraints like friction.
Fourth, one can investigate a smoother mechanism for energy stabilization, where
a VNHC is explicitly designed to inject energy below some energy level, and
dissipate energy above it.
For example, this might involve a mechanism for smoothly transferring between 
VNHCs while maintaining certain safety requirements.

Finally, the proof of Theorem \ref{thm:acrobot-energy-stabilization} 
in Chapter \ref{sec:acrobot-proof} opens a door to the possibility of VNHC generation. 
One might search for suitable conditions on VNHCs where the Poincar\'{e}
sections for oscillations and rotations are increasing.
Then, one can extend Otsason's ``virtual constraint
generator" \cite{vcg} to generate regular VNHCs which satisfy
these energy injection/dissipation conditions.
In this way, one could automatically generate regular VNHCs which stabilize a
desired energy level.

%/========== /Conclusion ==========/%
% vim: set tw=80 ts=4 sw=4 sts=0 et ffs=unix :
