%! TEX root = main.tex

%/========== Introduction ==========/%

\chapter{Introduction}

Consider a child on a standing swing.
Naturally, they begin by pushing off the ground and proceed to alternate between
standing and squatting. 
This allows them to swing higher and faster over time, until they have reached
their express goal of going as high and fast as possible.

Now imagine that the child is a robot, and their creator is teaching them to
behave like a real boy.
If the roboticist had studied classical control theory, they would plot the leg
motion of a human child as a trajectory over time and tell the robot to
synchronize with this trajectory.
In an ideal world, this technique would work perfectly; unfortunately, this
control approach is entirely unnatural. 
Humans on swings do not have an internal stopwatch.
Rather, they adjust how much they bend their knees based on the current angle of
the swing along with their direction of motion.
This position-velocity adjustment allows humans to correct for external
disturbances (like strong winds or overly enthusiastic parents).
Figure \ref{fig:swing-pos-vel} shows this usual adjustment:
a person will squat at the peak of the swing and stand as they reach the fastest
point at the bottom \cite{pumping_swing_standing_squatting}.
Even if the robot could perfectly track a time-based trajectory, an external
disturbance may swap the order of squatting and standing to the point that
the swing slows down rather than speeding up.

\begin{figure}
    \centering
    \includegraphics[width=0.75\textwidth]{images/swing_pos_vel.png}
    \caption{A person on a standing swing will squat as they approach the
        bottom of the swing path, and stand after they pass it until they hit
        the peak of their swing. Image taken and modified from
    \cite{pumping_swing_standing_squatting}.}
    \label{fig:swing-pos-vel}
\end{figure}

One recent control technique known as \textit{virtual holonomic constraints}
\cite{vhcs_for_el_systems} forces a robot's actuators to track a function of
position rather than time.
Sadly, this will not work for the swing-bot: one needs to know their velocity to
determine when they have reached the peak of their swing.
In other words, the robot needs to move according to a function of both position
and velocity if it has any hope of behaving like a real person.
These position-velocity controllers are known as 
\textit{virtual nonholonomic constraints}, and they are the focus of this
thesis.

Throughout this thesis we will refer to two examples to build intuition.
The first is the variable-length pendulum, which models the child on a swing.
The second is the acrobot, which models a gymnast hanging on a horizontal bar.
For each of these mechanical systems, we will develop virtual nonholonomic
constraints which replicate human behaviours. 
Finally, we will rigorously prove that these constraints inject energy into
their respective robots.

\section{Literature Review}
Here we review some of the previous work on energy injection and virtual
nonholonomic constraints.

Energy injection for mechanical systems is often performed by passivity-based
control and energy shaping --- one views the robot and its inputs as energy
transformations which are moulded to force the system to behave as desired.
Much of the theoretical work in this field has been performed by Ortega, van der
Schaft, and others (see for example \cite{ida_pbc_underactuation_one,
ida_pbc_acrobot_example,energy_shaping_revisited}).
Energy shaping has been applied to both the variable-length
pendulum \cite{vlp_energy_shaping} and the acrobot
\cite{swingup_acrobot_energy,swingup_giant_acrobot}.
This technique is useful, but it does not necessarily produce the structured
human-like motion we wish to generate using virtual nonholonomic constraints.

The modern concept of virtual nonholonomic constraints was described by
\citet{nhvc_dynamic_walking} in 2015, though there are references to more
primitive versions going back as early as the the year 2000
\cite{vnhc_human_robot_cooperation}.
Virtual nonholonomic constraints have been of most notable use in the area of
bipedal locomotion \cite{nhvc_incline_walking,output_nhvc_bipedal_conrol},
where they have shown marked improvements in the robustness of walking gaits
when compared to previous control technqiues \cite{nhvc_incline_walking}.
They have also been used for error-reduction in time-delayed teleoperation
\cite{vnhc_time_delay_teleop} and human-robot interaction
\cite{psd_based_vnhc_redundant_manipulator,haptic_vnhc}.
All these fields use different formulations of virtual nonholonomic constraints;
\citet{hybrid_zero_dynamics_bipedal_nhvcs} have derived the constrained dynamics
of virtual nonholonomic constraints for hybrid mechanical systems,
but to the best of our knowledge no one has attempted to generalize the
concept to abstract mechanical systems.

Many of our contributions in Chapter \ref{ch:vnhcs} extend
concepts from the field of virtual \textit{holonomic} constraints. 
Of note is the theoretical framework devised by Mohammadi, Maggiore, and
Consolini, among others
\cite{vhcs_for_el_systems,dynamic_vhcs_stabilize_closed_orbits,lagrangian_structure_reduced_dynamics_vhcs,xingbo_thesis}.

\section{Statement of Contributions}
Here are the contributions of this thesis.
\begin{itemize}[label={}]
   \item \textbf{Chapter \ref{ch:vnhcs}} The development of the framework of
      virtual nonholonomic constraints.
      This includes the definition of simply actuated mechanical systems, virtual
      nonholonomic constraints, regular constraints, and energy injection.
      Theorem \ref{thm:vnhc-regularity} yields a computational characterization
      of regularity, while Theorem \ref{thm:zero-dynamics} explicitly finds the
      constrained dynamics for a certain class of systems.
   \item \textbf{Chapter \ref{ch:vlp}} An application of virtual nonholonomic
      constraints to the variable-length pendulum, based on a pumping technique
      used by children on standing swings.
      The chapter culminates in Theorem \ref{thm:vlp-energy-stabilization},
      which guarantees that a certain class of VNHCs will inject energy into
      this system.
   \item \textbf{Chapter \ref{ch:acrobot}} An application of virtual
      nonholonomic constraints to the acrobot, based on the motion performed by
      human gymnasts.
      The chapter ends with Theorem \ref{thm:acrobot-energy-stabilization},
      which proves that the constraint we design does in fact inject energy into
      the acrobot.
\end{itemize}

\section{Organization of the Thesis}
The thesis is laid out as follows: 
in Chapter \ref{ch:vnhcs} we cover the requisite background on analytical 
mechanics, after which we develop the main theory of virtual nonholonomic
constraints;
in Chapter \ref{ch:vlp} we reformulate a time-optimal energy-injection strategy
for the variable-length pendulum as a virtual nonholonomic constraint, and
prove that the pendulum will gain enough energy to rotate around the bar with
arbitrary speed;
and in Chapter \ref{ch:acrobot} we find a virtual nonholonomic constraint 
which enables the acrobot to kick its legs like a gymnast until it is
performing backflips on a horizontal bar.
Finally, we show experimental results on a physical acrobot which confirm
that the theory works in the real world.

%/========== /Introduction ==========/%
% vim: set tw=80 ts=4 sw=4 sts=0 et ffs=unix :
