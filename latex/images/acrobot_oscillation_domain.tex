%! TEX program = lualatex

% Use :VimtexToggleMain to compile this file alone with Vimtex
\documentclass[tikz]{standalone}

\usepackage{sty/adantikz}
%TODO: Add a figure with one oscillation mark at a distance mu from
    % the origin to the intersection on the q-axis. We also draw the boundary of
    % mathcalO and shade the interior, and draw a line to the oscillation with
    % clockwise angle alpha

\begin{document}
\begin{tikzpicture}
    % Define the value of m2gl3 making sqrt(60m2gl3)=15
    \pgfmathsetmacro{\mgl}{15^2/60};

% Define the figure
\begin{axis}[
 axis line style = thick,
 % Set the axes to be centered nicely
 axis lines  = center,
 % Turn off arrows
 axis line style={-},
 % Set the axes
 xlabel = $q_u$, 
 x label style={right},
 xmin = -3.5, xmax = 3.5,
 ylabel = $p_u$,
 y label style={above},
 ymin = -22, ymax = 22,
 % Custom tick labels at specified positions
 xtick = {-3.1415, 3.1415},
 xticklabels = {$-\pi$, $\pi$},
 xticklabel style={xshift={(\ticknum==0)*(-0.3cm) + (\ticknum==1)*(0.25cm)}},
 ytick = {-15, 15},
 yticklabels = {},
 % Move the label so it looks good
 yticklabel style={yshift={(\ticknum == 0)*(-0.3cm) + (\ticknum == 1)*(0.25cm)}}
]
    % Colour in the region for oscillations
    \addplot[black,->,thick,name path=upper,samples=100,domain=-pi:pi]
        {sqrt(30*\mgl*(cos(deg(x))+1))};
    \addplot[black,<-,thick,name path=lower, samples=100,domain=-pi:pi]
        {-sqrt(30*\mgl*(cos(deg(x))+1))};

    % Colour in the region by filling between homoclinic orbit
    \addplot[fill=blue,opacity=0.2] 
        fill between[of=upper and lower];

    % Add a label
    \node at (-pi/5,17) {$E(\pi,0)$};

    % Draw a spiraling orbit heading towards the homoclinic orbit
    \spiral[blue,thick]{center={(0,0)},
        start angle=0,end angle=90,
        start radius=0cm, end radius=1.95cm,
        revolutions=4,
        clockwise};

    % Draw vertical dashed lines at +-pi
    \draw[gray, dashed] (-pi,-22) -- (-pi,22);
    \draw[gray, dashed] (pi,-22) -- (pi,22);

\end{axis}
\end{tikzpicture}
\end{document}
% vim: set tw=80 ts=4 sw=4 sts=0 et ffs=unix :
