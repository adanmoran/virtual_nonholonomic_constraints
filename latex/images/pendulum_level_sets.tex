
%! TEX program = lualatex

% Use :VimtexToggleMain to compile this file alone with Vimtex
\documentclass[tikz]{standalone}

\usepackage{sty/adantikz}
%TODO: Add a figure with one oscillation mark at a distance mu from
    % the origin to the intersection on the q-axis. We also draw the boundary of
    % mathcalO and shade the interior, and draw a line to the oscillation with
    % clockwise angle alpha

\begin{document}
\begin{tikzpicture}
    % Set homoclinic orbit x and y intersection
    \pgfmathsetmacro{\homoy}{15};
    % Set oscillation mu 
    \pgfmathsetmacro{\mu}{pi/2};
    % Define the value of m2gl3 making sqrt(60m2gl3)=\homoy
    \pgfmathsetmacro{\mgl}{\homoy^2/60};
    % Set rotation y intersection (positive)
    \pgfmathsetmacro{\roty}{20};
    % Set y axis dimension
    \pgfmathsetmacro{\ymax}{22};
% Define the figure
\begin{axis}[
 axis line style = thick,
 % Set the axes to be centered nicely
 axis lines  = center,
 % Turn off arrows
 axis line style={-},
 % Set the axes
 xlabel = $q_u$, 
 x label style={right},
 xmin = -3.5, xmax = 3.5,
 ylabel = $p_u$,
 y label style={above},
 ymin = -\ymax, ymax = \ymax,
 % Custom tick labels at specified positions
 xtick = {-3.1415, 3.1415},
 xticklabels = {$-\pi$, $\pi$},
 xticklabel style={xshift={(\ticknum==0)*(-0.3cm) + (\ticknum==1)*(0.25cm)}},
 ytick = {-15, 15},
 yticklabels = {},
 % Move the label so it looks good
 yticklabel style={yshift={(\ticknum == 0)*(-0.3cm) + (\ticknum == 1)*(0.25cm)}}
]
    % Draw the homoclinic orbit with arrows pointing in the direction of flow
%    \draw[black, thick,->]  (-pi,0) arc 
%        [start angle=180, end angle = 0, x radius=pi, y radius=\homoy];
%    \draw[black, thick,->]  (pi,0) arc 
%        [start angle=0, end angle = -180, x radius=pi, y radius=\homoy];
    \addplot[black,->,thick,samples=100,domain=-pi:pi]
        {sqrt(30*\mgl*(cos(deg(x))+1))};
    \addplot[black,<-,thick,samples=100,domain=-pi:pi]
        {-sqrt(30*\mgl*(cos(deg(x))+1))};

    % Draw vertical dashed lines at +-pi
    \draw[gray, dashed] (-pi,-\ymax) -- (-pi,\ymax);
    \draw[gray, dashed] (pi,-\ymax) -- (pi,\ymax);

    % Draw an oscillation contained entirely in the oscillation region
    \addplot[blue,->,thick,samples=100,domain=-\mu:\mu]
        {sqrt(30*\mgl*(cos(deg(x))-cos(deg(\mu))))};
    \addplot[blue,<-,thick,samples=100,domain=-\mu:\mu]
        {-sqrt(30*\mgl*(cos(deg(x))-cos(deg(\mu))))};

    % Draw a rotation contained outside the oscillation region on the top and bottom
    % The rotation has pu = sign(rho)sqrt(rhob^2+30m^2gl^3(cos(q_u)-1)
    \addplot[red,->,name path=upper,thick,samples=100,domain=-pi:pi]
        {sqrt(\roty^2+30*\mgl*(cos(deg(x))-1))};
    \addplot[red,<-,name path=upper,thick,samples=100,domain=-pi:pi]
        {-sqrt(\roty^2+30*\mgl*(cos(deg(x))-1))};

\end{axis}
\end{tikzpicture}
\end{document}
% vim: set tw=80 ts=4 sw=4 sts=0 et ffs=unix :
