
%! TEX program = lualatex

% Use :VimtexToggleMain to compile this file alone with Vimtex
\documentclass[tikz]{standalone}

\usepackage{sty/adantikz}

\begin{document}
\begin{tikzpicture}
    % Define coordinates
    \coordinate (origin) at (0,0);
    % Define our lengths, etc
    \pgfmathsetmacro{\len}{3};
    \pgfmathsetmacro{\qu}{60};
    \pgfmathsetmacro{\qa}{160};
    \pgfmathsetmacro{\quc}{\len*cos(\qu)};
    \pgfmathsetmacro{\qus}{\len*sin(\qu)};
    \pgfmathsetmacro{\qac}{\len*cos(\qa)};
    \pgfmathsetmacro{\qas}{\len*sin(\qa)};
    % The base of the pivot
    \draw[thick] (-2,0) coordinate (leftwall) -- (2+\len,0) coordinate (rightwall);
    \fill[black] (origin) circle (0.05);
    % Fill a cross-hatched wall above the pivot
    \fill[pattern=north east lines] (leftwall) rectangle ($(rightwall) + (0,0.1)$);
    % The gravity arrow
    \draw[draw=gray, ->] ($(leftwall) + (0.5,-0.5)$) -- ($(leftwall) +(0.5,-1)$) node [right,black] {$g$};

    % The dashed line at the vertical of the origin
    \draw[draw=gray, dashed] (origin) -- ++(0,-2) coordinate (vert);

    % The first link as a thick rod
    \draw[draw=black, ultra thick] (origin) -- 
        ($(origin) + (\quc,-\qus)$) coordinate (m1)
	node[midway,above,sloped] {$l$};
    \fill[black] (m1) circle (0.05);
    % The angle q_u
    \pic [draw, ->, "$q_u$", angle eccentricity=1.5] {angle=vert--origin--m1};

    % The dashed gray line extending out of the first rod
    \draw[draw=gray,dashed] (m1) -- 
        ($(m1) + 0.4*(m1)$) coordinate (gray);

    % The second link
    \draw[draw=black, ultra thick] (m1) -- 
        ($(m1) + (-\qac,-\qas)$) coordinate (m2)
	node[midway,above,sloped] {$l$};

    % The angle q_a
    \pic [draw,->, "$q_a$", angle eccentricity=1.5] {angle=gray--m1--m2};

    % The first center of mass
    \node[draw,fill=red!80,circle,minimum size = 4mm,inner sep = 0pt] at (m1) {$m$};
    % The second center of mass
    \node[draw,fill=red!80,circle,minimum size = 4mm,inner sep = 0pt] at (m2) {$m$};

\end{tikzpicture}
\end{document}
% vim: set tw=80 ts=4 sw=4 sts=0 et ffs=unix :
