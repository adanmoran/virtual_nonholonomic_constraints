%! TEX program = lualatex

\documentclass[12pt]{ut-thesis}
%/========== Preamble ==========/%
\usepackage{adanstyle}

%---------- Bibliography Style ----------%
\usepackage[numbers,sort&compress]{natbib}
\bibliographystyle{IEEEtranN}

%---------- Ut-Thesis Recommendations ----------%
%% List only down to subsections in the table of contents;
%% 0=chapter, 1=section, 2=subsection, 3=subsubsection, etc.
\setcounter{tocdepth}{2}

%% Make each page fill up the entire page.
\flushbottom

%/========== /Preamble ==========/%

%/========== Main Document ==========/%
\begin{document}

%---------- Preliminary Pages -----------%
\begin{preliminary}

% Generate the title page
%! TEX root = main.tex

%/========== Title Page Info ==========/%
\title{Energy injection for mechanical systems through the method of Virtual Nonholonomic Constraints}
\author{Adan Moran-MacDonald }
\degree{Master of Applied Science}
\department{Electrical and Computer Engineering}
\gradyear{2020}
\date{\today}

\maketitle
%/========== Title Page Info ==========/%

% vim: set ts=4 sw=4 sts=0 et ffs=unix :


%% There should be NOTHING between the title page and abstract.
%% However, if your document is two-sided and you want the abstract
%% _not_ to appear on the back of the title page, then uncomment the
%% following line.
\cleardoublepage

% Generate the abstract page, with line spacing adjusted according to SGS
% guidelines
%! TEX root = main.tex

%/========== Abstract ==========/%
%% (At most 150 words for M.Sc. or 350 words for Ph.D.)
\begin{abstract}
   This thesis develops a theoretical foundation for the development of virtual
   nonholonomic constraints, which are relations between the generalized
   coordinates and generalized momenta of a mechanical system that can be
   enforced via feedback control.
   The theory is applied towards energy injection for two standard underactuated
   systems: the variable-length pendulum and the acrobot. 
   Virtual nonholonomic constraints are designed for each system by examining
   human motion, and energy injection properties of these constraints are proven
   rigorously.
   The acrobot constraint is tested on a real-world acrobot, demonstrating
   highly effective energy-injection properties and robustness to a variety of
   external disturbances.
\end{abstract}

%/========== /Abstract ==========/%
% vim: set tw=80 ts=4 sw=4 sts=0 et ffs=unix :


%% Anything placed between the abstract and table of contents will
%% appear on a separate page since the abstract ends with \newpage and
%% the table of contents starts with \clearpage.  Use \cleardoublepage
%% for anything that you want to appear on a right-hand page.

%% This generates the 'dedication' page, which will appear on a right-hand page
\cleardoublepage
%! TEX root = main.tex

%/========== Dedication ==========/%

\begin{dedication}
	\textbf{TODO}: Fill in the dedication
\end{dedication}

%/========== /Dedication ==========/%
% vim: set ts=3 sw=3 sts=0 et ffs=unix :


%% The `dedication' and `acknowledgements' sections do not create new
%% pages so we need to separate them
\newpage  % separate pages for dedication and acknowledgements

%% This generates the "acknowledgements" section
%! TEX root = main.tex

%/========== Acknowledgements ==========/%

\begin{acknowledgements}
    Manfredi, you are \textit{without a doubt} the greatest supervisor, the
    best mentor, and one of the most wonderful people in the world.
    Knowing you, you'll likely respond to that by saying 
    ``thank you Adan, you're too kind", and then proceed to laugh uncomfortably before
    changing the topic.
    But it's true, and I don't know how else to tell you how lucky and grateful I am to
    have you in my life.

    These two years of doing research together were the most fun I've ever
    had; yet they've also been fraught with death, sorrow, and the most
    devastating heartbreak I've ever experienced.
    Through it all you've been nothing but kind, honest, generous,
    compassionate, understanding, and so incredibly humble.
    You work unbelievably hard in the background to ensure your students succeed, and you
    ask for so little in return.
    My name is on the title page of this thesis, but if not for the required
    academic formality I would change the authorship to read
    ``Adan Moran-MacDonald, who couldn't have done any of this without the love
    and support of his supervisor and dear friend, Manfredi Maggiore."

    Manfredi: you are my hero. 

    % TODO: do the rest of the acknowledgements: parents, SCG friends, other
    % friends, committee

\end{acknowledgements}

%/========== /Acknowledgements ==========/%
% vim: set tw=80 ts=4 sw=4 sts=0 et ffs=unix :


%% This generates the Table of Contents (on a separate page).
\tableofcontents

%% This generates the List of Tables (on a separate page), if needed
%% (uncomment to have it appear in the document).
%\listoftables

%% This generates the List of Figures (on a separate page), if needed
%% (uncomment to have it appear in the document).
%\listoffigures

%% You can add commands here to generate any other material that belongs
%% in the head matter (for example, List of Plates, Index of Symbols, or
%% List of Appendices).

% TODO: Index of Symbols and Notation
%! TEX root = main.tex

%/========== Symbols ==========/%

% Set the heading so it appears in the Table of Contents
\addcontentsline{toc}{chapter}{List of Symbols}
\chapter*{List of Symbols}

% Define a table for the symbols using the booktabs convention
\noindent
\begin{table}[h!]
\resizebox{\textwidth}{!}{%
\centering
\begin{tabular}{ll}
   \toprule
   Symbol & Definition \\
   \midrule
   \(\n\) & The index set \(\{1,\ldots,n\}\) of natural numbers up to \(n\). \\
   \(\R^n\) & Real numbers in \(n\) dimensions. \\
   \(\Rt{T}\) & Real numbers modulo \(T > 0\), with \(\Rt{\infty} = \R\). \\
   \(\mathbb{S}^1\) & The unit circle, equivalent to \(\Rt{2\pi}\). \\
   \(\mathcal{Q}\) & The configuration manifold of a system. \\
   \(\R^{n \times m}\) & The space of real-valued matrices with \(n\) rows and \(m\) columns. \\
   \(\Id{n}\) & The \(n \times n\) identity matrix. \\
   \(\Zmat{n \times m}\) & The \(n \times m\) matrix of all zeros. \\
   \(M_i\) & If \(M\) is a vector, the \(i\)th element of \(M\). \\
         & If \(M\) is a matrix, the \(i\)th column of \(M\). \\
   \(M_{i,j}\) & The value of row \(i\), column \(j\) for the matrix \(M\). \\
   \(\dot{x}\) & Derivative of \(x\) with respect to time \(t\). \\
   \(\nabla_v F\) & If \(F\) is \(\R\)-valued, the gradient of \(F\) with respect to \(v\). \\
            & If \(F : \R^m \rightarrow \R^{n\times n}\), the block matrix gradient \((\pdiff{F}{v_1},\ldots,\pdiff{F}{v_m}) \in \R^{nm \times n}\). \\
   \(dF_v\) & Total differential (Jacobian) of \(F\), equivalent to \((\nabla_v F)\tpose\). \\
   \(\Hess F\) & If \(F : \R^n \rightarrow \R\), the \(n \times n\) Hessian matrix of double derivatives of \(F\). \\
               & If \(F : \R^n \rightarrow \R^k\), the block matrix \((\Hess F_1, \ldots, \Hess F_k) \in \R^{n \times nk}\). \\
   \(\partial_v \partial_w F\) & Derivative matrix of \(F : \R^n \times \R^m \rightarrow \R\), with \((i,j)\) element \(\ppdiff{F}{v_i}{w_j}\). \\
   \(\delta_{i,j}\) & The Kronecker delta: 1 if \(i = j\) and 0 otherwise. \\
   \(\otimes\) & The matrix kronecker product (see Appendix \ref{appendix:matrix-kron-prod}). \\
   \bottomrule
\end{tabular}

}% resizebox
\end{table}

%---------- Nomenclature ----------%
%\chapter*{Nomenclature}
%
%% Define a table for the symbols using the booktabs convention
%\noindent
%\begin{table}[h!]
%\resizebox{\textwidth}{!}{%
%\centering
%\begin{tabular}{ll}
%   \toprule
%   Acronym & Meaning \\
%   \midrule
%   DOF & Degrees of freedom
%   EOC & Equation of constraint \\
%   EOM & Equation of motion \\
%   ODE & Ordinary differential equation \\
%   VNHC & Virtual nonholonomic constraint \\
%   WLOG & Without loss of generality \\
%   \bottomrule
%\end{tabular}
%
%}% resizebox
%\end{table}
%/========== /Symbols ==========/%
% vim: set ts=3 sw=3 sts=0 et ffs=unix :


%% End of the preliminary sections: reset page style and numbering.
\end{preliminary}

%---------- Main Content ----------%
TEST CITATIONS:

\cite{constrained_hamiltonian_systems}
\cite{control_giant_two_link_gymnastic_robot}
\cite{dynamical_servo_acrobot_vc}
\cite{dynamic_vhcs_stabilize_closed_orbits}
\cite{energy_pumping_robotic_swinging}
\cite{hamiltonian_nonholonomic_systems}
\cite{hamiltonization_of_nh_systems}
\cite{how_to_pump_a_swing}
\cite{hybrid_zero_dynamics_bipedal_nhvcs}
\cite{integrating_factors}
\cite{lagrangian_structure_reduced_dynamics_vhcs} 
\cite{multiple_lyapunov_functions}
\cite{multiple_lyapunov_functions_switched_systems}
\cite{nonholonomic_dynamics}
\cite{pendulum_length_giant_gymnastics}
\cite{pumping_swing_standing_squatting}
\cite{swingup_acrobot_pendulum}
\cite{swingup_giant_acrobot}
\cite{vhcs_for_el_systems}
\cite{vlp_energy_shaping}
\cite{xingbo_thesis}
\cite{khalil_nonlinear}
\cite{greenwood_dynamics}
\cite{landau_mechanics}
\cite{ida_pbc_underactuation_one}
\cite{ida_pbc_acrobot_example}
\cite{vhc_robotic_walking}
\cite{vhc_stable_walking}
\cite{vhc_snake}
\cite{vhc_bicycle}
\cite{vhc_helicopter}
\cite{vnhc_human_robot_cooperation}
\cite{psd_based_vnhc_redundant_manipulator}
\cite{haptic_vnhc}
\cite{vnhc_time_delay_teleop}
\cite{nhvc_dynamic_walking}
\cite{nhvc_incline_walking}
\cite{output_nhvc_bipedal_control}
\cite{dynamics_periodic_vlp}
\cite{oscillation_suppression_vlp}
\cite{vlp_tuned_mass_damper}
\cite{lyapunov}
\cite{nonlinear_controllers_nonintegrable_acrobot}
\cite{toward_framework_biped_locomotion}
\cite{swingup_problem_acrobot}
\cite{swingup_acrobot_energy}
\cite{motion_control_gymnastic_skill}
\cite{usagym_giant}
\cite{calculating_svd}
\cite{krasovskii_lasalle}
\cite{kronprod}
\cite{poincare_bendixson}
\cite{stable_manifold}
% When using external files, we use \input instead of \include because \input
% allows you to nest the inclusion. This way, we can separate the .tikz files
% for our images and still input them into the code.

% Introduce the thesis
%! TEX root = main.tex

%/========== Introduction ==========/%

\chapter{Introduction}
% TODO: Introduce the thesis with a short, sweet, intuitive approach.
% No math, just whet the reader's appetite for what we are doing. What are the motivating examples? Why do we study what we study? Why choose VNHCs over other methods?
Consider a child who is standing on a swing.
Naturally, the child pushes off the ground and starts bending and extending
their knees; they swing higher and faster over time, until they 
have reached their express goal of going as high and fast as possible.

Now imagine that the child is a robot, and their creator is teaching them to
swing like a real boy.
The first thing the roboticist might do is plot the leg motion of a human child
as a function of time and have the robot synchronize with this plot.
In an ideal world, this technique would work perfectly and the robot boy would
swing high, experiencing the same soaring feeling as his human counterpart.

Unfortunately, this behaviour is entirely unnatural. 
Humans on swings do not bend or extend their knees at specific times; rather,
they adjust their leg position based on the current angle of the swing along
with their direction of motion, as in Figure \ref{fig:swing-pos-vel}.
This position-velocity adjustment allows humans to correct for external
disturbances (like strong winds or overly enthusiastic parents).
Even if the robot could perfectly track a time-based trajectory, an external
disturbance may desynchronize the leg motion with the swing to the point that
the swing will slow down rather than speed up.

\begin{figure}
    \centering
    %\includegraphis[width=0.5\textwidth]{images/swing_pos_vel.png}
    \caption{A person on a standing swing will squat as they approach the
        bottom of the swing path, and stand after they pass it until they hit
        the peak of their swing. Image modified from one in
    \cite{pumping_swing_standing_squatting}.}
    \label{fig:swing-pos-vel}
\end{figure}

One recent control technique known as \textit{virtual holonomic constraints}
forces a robot's actuators to track a function of position rather than time.
Sadly, this still won't do for our robot boy because a real child also bends
their knees according to their direction of motion.
In other words, the robot's legs need to move as a function not only of
position, but also of velocity.

These are the types of controllers we study in this thesis: they are called
\textit{virtual nonholonomic constraints}.

\section{Literature Review}

% TODO: write a brief lit review on energy injection and VNHCs.

\section{Statement of Contributions}
Here are the contributions of this thesis.
\begin{itemize}[label={}]
   \item \textbf{Chapter \ref{ch:vnhcs}} The development of the framework of
      virtual nonholonomic constraints.
      This includes the definition of simply actuated mechanical systems, virtual
      nonholonomic constraints, regular constraints, and energy injection.
      Theorem \ref{thm:vnhc-regularity} yields a characterization of regularity, 
      while Theorem \ref{thm:zero-dynamics} provides the constrained dynamics
      for a certain class of systems.
   \item \textbf{Chapter \ref{ch:vlp}} An application of virtual nonholonomic
      constraints to the variable-length pendulum, based on a pumping technique
      used by children on standing swings.
      The chapter culminates in Theorem \ref{thm:vlp-energy-stabilization},
      which guarantees that a certain class of VNHCs will inject energy into
      this system.
   \item \textbf{Chapter \label{ch:acrobot}} An application of virtual
      nonholonomic constraints to the acrobot, where we develop a constraint
      by examining the intuitive motion performed by human gymnasts.
      The chapter ends with Theorem \ref{thm:acrobot-energy-stabilization},
      which proves that the constraint injects energy into the acrobot.
\end{itemize}

\section{Organization of the Thesis}
The thesis is laid out as such: 
in Chapter \ref{ch:vnhcs} we cover the requisite background on Hamiltonian
mechanics, after which we develop the main theory of virtual nonholonomic
constraints;
in Chapter \ref{ch:vlp} we reformulate a time-optimal energy-injection strategy
for the variable-length pendulum as a virtual nonholonomic constraint, and
prove that the pendulum will gain enough energy to rotate around the bar with
arbitrary speed;
and in Chapter \ref{ch:acrobot} we find a virtual nonholonomic constraint 
which enables the acrobot to kick its legs like a gymnast until it is
performing backflips on a horizontal bar.
Finally, we show experimental results for the acrobot which confirm that the
theory works in the real world.

%/========== /Introduction ==========/%
% vim: set tw=80 ts=4 sw=4 sts=0 et ffs=unix :


% Develop the formulation of VNHCs
%! TEX root = main.tex

%/========== Virtual Nonholonomic Constraints ==========/%

\chapter{Development of Virtual Nonholonomic Constraints}
% TODO: Brief introduction before preliminaries

\section{Preliminaries on Analytical Mechanics}
A mechanical system can be represented by \(N\) point masses where each point
represents the center of mass of a physical body, along with \(r\)
\textit{equations of constraint} (EOC) which model the physical restrictions
between these masses.
The position of each point mass is described using three cartesian coordinates (one
for each spatial axis), so the system as a whole can be described by a vector in
\(\R^{3N}\) with \(r\) EOC. 
The dynamics of the system are computed by deriving the \(3N\)
\textit{equations of motion} (EOM) produced by Newton's second law \(F = m a\).
While this technique works for simple systems, it is impossible to
apply to a majority of mechanical systems since most of the forces 
on the system are not explicitly known. 

Rather than modeling a mechanical system with point masses and constraints,
it is often feasible to represent the position of the system using \(n\)
independent scalar-valued variables \(q_1,\ldots,q_n\) called 
\textit{generalized coordinates}, where \(n = 3N - r\) is the number of
\textit{degrees of freedom} (DOF) of the system \cite{greenwood_dynamics}.

For the robotic systems of interest in this thesis, we assume that
each generalized coordinate \(q_i\) represents either the distance or the angle
between two parts of the system.
Mathematically, each \(q_i\) takes values in \(\Rt{T_i}\), where
\(T_i = \infty\) if \(q_i\) represents a length or \(T_i = 2\pi\) if \(q_i\)
represents an angle.
It is convention to collect the coordinates into a \textit{configuration} 
\(q = (q_1,\ldots,q_n) \in \mathcal{Q}\) 
where the \textit{configuration manifold} \(\mathcal{Q}\) of the system is a
so-called \textit{generalized cylinder}:
\[
    \mathcal{Q} = \Rt{T_1} \times \cdot \times \Rt{T_n}
\] 
The derivative \(\dot{q} = (\dot{q}_1,\ldots,\dot{q}_n)\) of a configuration
is called a \textit{generalized velocity} of the system. For arbitrary systems,
the space of allowable velocities depends on the current configuration of the
system.  However, since \(\mathcal{Q}\) is a generalized cylinder, we find that 
\(\dot{q} \in \R^n\).
The combined vector \((q,\dot{q}) \in \mathcal{Q}\times\R^n\) is called a 
\textit{state} of the system.

The field of analytical mechanics provides a computational method for finding
the EOM of a system in generalized coordinates. The two most common analytical
methods for modelling robotic systems are \textit{Lagrangian} and
\textit{Hamiltonian} mechanics.

% ---------- Lagrangian Mechanics ---------- % 
\subsection{Lagrangian Mechanics}

Lagrangian mechanics uses the kinetic energy \(T(q,\dot{q})\) and potential
energy \(P(q)\) of the system to define the Lagrangian 
\(\mathcal{L} : \mathcal{Q}\times\R^n \rightarrow \R\) defined by
(\ref{eqn:lagrangian-general}) \cite{greenwood_dynamics}.
\begin{equation}\label{eqn:lagrangian-general}
    \mathcal{L}(q,\dot{q}) = T(q,\dot{q}) - P(q)
\end{equation}
When the mechanical system is actuated, the EOM are described by \(n\) second-order
ordinary differential equations (ODEs) obtained from the \textit{Euler-Lagrange
equations} with \textit{generalized input forces} \(\tau \in \R^k\) 
(\ref{eqn:el-eqns-general}). 
\begin{equation}\label{eqn:el-eqns-general}
    \diff{}{t}\left\{ \pdiff{\mathcal{L}}{\dot{q}_i} \right\}
    - \pdiff{\mathcal{L}}{q_i} = B_i\tpose(q) \tau
\end{equation}
The vector \(B_i\tpose: \mathcal{Q} \rightarrow \R^{1\times k}\) describes how
the input forces shape the dynamics of \(q_i\).
The matrix  \(B: \mathcal{Q} \rightarrow \R^{n \times k}\) with
\[
    B(q) = \begin{bmatrix}
        - & B_1\tpose(q) & - \\
          & \vdots & \\
        - & B_n\tpose(q) & - \\
    \end{bmatrix}
\]
is called the \textit{input matrix} for the system.
If \(k < n\), we say the system is \textit{underactuated} with degree of
underactuation \(n - k\).

Many actuated mechanical systems have quadratic kinetic energies, so that the
Lagrangian can be written explicitly as
\begin{equation}\label{eqn:lagrangian}
    \mathcal{L}(q,\dot{q}) = \frac{1}{2} \dot{q}\tpose D(q) \dot{q} - P(q)
\end{equation}
where the \textit{inertia matrix} \(D: \mathcal{Q} \rightarrow \R^{n\times n}\) 
is a symmetric, positive definite matrix for all \(q \in \mathcal{Q}\) and the
potential function \(P : \mathcal{Q} \rightarrow \R\) is smooth. 
%If this is the case, the Euler-Lagrange equations reduce to (\ref{eqn:el-eqns}),
%\begin{equation}\label{eqn:el-eqns}
%    D(q)\ddot{q} + C(q,\dot{q})\dot{q} + \nabla P(q) = B(q)\tau
%\end{equation}
%where the \textit{Coriolis matrix} \(C(q,\dot{q})\) is of the form
%\[
%    [C]_{i,j} = \frac{1}{2}\sum\limits_{k = 1}^n 
%    \left(\pdiff{D_{i,j}}{q_k}  +
%     \pdiff{D_{i,k}}{q_j} -
%     \pdiff{D_{k,j}}{q_i}\right)\dot{q}_k
%\]

% ---------- Hamiltonian Mechanics ---------- %
\subsection{Hamiltonian Mechanics}
Hamiltonian mechanics provides an equivalent representation of the EOM
by converting the \(n\) second-order ODEs generated by Lagrangian mechanics into
\(2n\) first-order ODEs.

To do this, we first define the \textit{conjugate of momentum \(p_i\) to \(q_i\)} by
\begin{equation}\label{eqn:p-i}
    p_i(q,\dot{q}) = \pdiff{\mathcal{L}}{\dot{q}_i}(q,\dot{q})
\end{equation}
To ease notation, we will write \(p = (p_1, \ldots, p_n) \in \R^n\) and say that
\(p\) is the \textit{conjugate of momenta to \(q\)}.
Note that each \(p_i\) is a linear function of \(\dot{q}\), so one can typically
solve for \(\dot{q}(q,p)\) by inverting the expressions from (\ref{eqn:p-i}).
The combined vector \((q,p) \in \mathcal{Q}\times\R^n\) is called a 
\textit{phase} of the system. 

The \textit{Hamiltonian} of the system in
\(\{q,p\}\) coordinates is defined as the Legendre transform 
(\ref{eqn:hamiltonian-legendre}) of the Lagrangian \cite{landau_mechanics}.
\begin{equation}\label{eqn:hamiltonian-legendre}
    \mathcal{H}(q,p) = p\tpose \dot{q}(q,p) - \mathcal{L}(q,\dot{q}(q,p))
\end{equation}
The EOM in this framework can be shown to be the \(2n\)
first-order equations called \textit{Hamilton's equations}
\begin{equation}\label{eqn:hamiltons-eqns}
    \begin{cases}
        \dot{q} = \nabla_p\mathcal{H} \\
        \dot{p} = -\nabla_q\mathcal{H} + B(q)\tau \\
    \end{cases}
\end{equation}
where \(B(q) \in \R^{n\times k}\) is the same input matrix used by the
Lagrangian and \(\tau \in \R^k\) is the vector of generalized input forces.

If the Lagrangian has quadratic kinetic energy as in (\ref{eqn:lagrangian}), 
the conjugate of
momenta to \(q\) can be computed explicitly by \(p = D(q)\dot{q}\).

% TODO: Replace [p D_i p] here with our new notation
The resulting Hamiltonian system reduces to (\ref{eqn:hamiltonian}).
\begin{align}\label{eqn:hamiltonian}
    \mathcal{H}(q,p) &= \frac{1}{2} p\tpose D^{-1}(q) p + P(q) \\
                     &\begin{cases}
        \dot{q} = D\inv(q)p \\
        \dot{p} = -\frac{1}{2} (\Id{n} \otimes p\tpose) \nabla_q D\inv(q) p
        - \nabla_q P(q) + B(q) \tau \\
    \end{cases} 
\end{align}

A set of coordinates \(\{q,p\}\) which satisfy Hamilton's equations 
under the Hamiltonian \(\mathcal{H}\) are
said to be \textit{canonical coordinates} for the system. A change of
coordinates \((q,p) \rightarrow (Q,P)\) is a \textit{canonical
transformation} if \(\{Q,P\}\) are canonical coordinates with Hamiltonian
\(H(Q,P) = \mathcal{H}\left(q(Q,P), p(Q,P)\right)\).

\section{Simply Actuated Hamiltonian Systems}
% TODO: Describe the change of coordinates to get into q_u/q_a mode and show
% that the actuator directly affects pa but not pu. Use M for simply actuated
% inertia, D for normal coordinates
Given a Hamiltonian mechanical system (\ref{eqn:hamiltonian}), it is not obvious
how the input forces \(\tau\) affect the conjugate of momenta \(p_i\). 
This is because \(\tau\) is transformed by the input matrix \(B(q)\), which may
be quite complicated. 

We will define a new class of Hamiltonian systems where the effect of the input
forces on the conjugate of momenta is made obvious. This class of systems will
form the backbone for the rest of the theory developed in this thesis.

\begin{defn}
    Let \(\mathcal{H}\) be an underactuated Hamiltonian system 
    \(\mathcal{H}\) with degree of underactuation \((n-k)\geq 0\). 
    A pair of canonical coordinates \(\{q,p\}\) for this system
    are said to be \textit{simply actuated coordinates} if the
    input matrix \(B(q) \in \R^{n \times k}\) is of the form
    \[
        B(q) = \simpleB    
    \]
    The first \((n-k)\) coordinates \(q_u\) are called the \textit{unactuated
    coordinates}, while the remaining \(k\) coordinates \(q_a\) are called the
    \textit{actuated coordinates}. We write \(q = (q_u, q_a)\) and the
    corresponding conjugate of momenta \(p = (p_u, p_a)\) are the called the
    unactuated and actuated momenta (respectively).
\end{defn}
\begin{defn}
    A Hamiltonian system is said to be \textit{simply actuated} if there exists
    a canonical transformation from any set of canonical coordinates 
    \(\{q,p\}\) into simply actuated coordinates for \(\mathcal{H}\).
\end{defn}

Under the following assumptions on the input matrix, we will show that the
Hamiltonian system (\ref{eqn:hamiltonian}) is simply actuated.

\begin{assm}\label{assm:B-const}
    The input matrix \(B(q) \equiv B \in \R^{n\times k}\) is constant,
    full rank, and \(k \leq n\).
\end{assm}
\begin{assm}\label{assm:B-perp}
    There exists a matrix 
    \(B^\perp \in \R^{(n-k)\times n}\)
    which is right semi-orthogonal 
    (\ie \(B^\perp(B^\perp)\tpose = \Id{(n-k)}\))
    and which is a left-annihilator for \(B\)
    (\ie \(B^\perp B = \Zmat{(n-k) \times k}\)).
\end{assm}

Note that if \(k = (n-1)\), the existence of any left annihilator \(A^0\) for
\(B\) implies the left annihilator \(B^\perp := A^0/\norm{A^0}\) satisfies Assumption
\ref{assm:B-perp}.

\begin{assm}\label{assm:B-orthogonal}
    Assume without loss of generality that the input matrix \(B\) is left
    semi-orthogonal.
    That is, assume \(B\tpose B = \Id{k}\).
\end{assm}
\begin{proof}
Since \(B\) is a constant matrix, 
it has a singular-value decomposition 
\(B = U \Sigma V\tpose\) where \(U^{-1} = U\tpose \in \R^{n \times n}\), 
\(V^{-1} = V\tpose \in \R^{k \times k}\), and \(\Sigma \in \R^{n \times k}\) is
defined by
\[
    \Sigma = \begin{bmatrix}
        \sigma_1 & 0 & \cdots & 0 \\
        0 & \sigma_2 & \cdots & 0 \\
        \vdots & & \ddots & \vdots \\
        0 & 0 & \cdots & \sigma_k \\
        - &   & \Zmat{(n-k)\times k} & -  \\
    \end{bmatrix}
\]
where \(\sigma_i \neq 0\) because \(B\) is full-rank \textbf{SOURCE?}.
Defining \(T \in \R^{k \times k}\) by
\[
    T = \begin{bmatrix}
        \frac{1}{\sigma_1} & 0 & \cdots & 0 \\
        0 & \frac{1}{\sigma_2} & \cdots & 0 \\
    \vdots & & \ddots & \vdots \\
    0 & 0 & \cdots & \frac{1}{\sigma_k} \\
    \end{bmatrix}
\]
and assigning the input forces to \(\tau = V T \hat{\tau}\), we get a new input
matrix for \(\hat{\tau} \in \R^k\) given by \(\hat{B} = B V T = U \Sigma T\) 
which is still constant and full-rank. In particular, 
\(\hat{B}\tpose \hat{B} = T\tpose \Sigma\tpose \Sigma\tpose T = \Id{k}\).
\end{proof}

Let \(\mathbf{B} \in \R^{n\times n}\) be the following matrix:
\[
    \mathbf{B} = 
    \begin{bmatrix}
        B^\perp \\
        B\tpose \\
    \end{bmatrix}
\]
Since \(B^\perp\) is a left annihilator of \(B\) and both \(B^\perp\) and
\(B\tpose\) are right semi-orthogonal, it is easy to show that \(\mathbf{B}\) is
orthogonal:
\[
    \mathbf{B}\mathbf{B}\tpose = 
    \begin{bmatrix}
        B^\perp (B^\perp)\tpose & B^\perp B \\
        (B^\perp B)\tpose & B\tpose B
    \end{bmatrix} = \Id{n}
    \Rightarrow
    \mathbf{B}^{-1} = \mathbf{B}\tpose
\]

The following theorem shows that \(\mathbf{B}\) provides a canonical
transformation into simply actuated coordinates, so that all unactuated momenta
are unaffected by the input forces.

% TODO: Replace the [P M_i P] with new notation
\begin{thm}
    Under Assumptions \ref{assm:B-const},\ref{assm:B-perp}, and
    \ref{assm:B-orthogonal}, the Hamiltonian system (\ref{eqn:hamiltonian}) is
    simply actuated with simply actuated coordinates 
    \(\{Q = \mathbf{B}q, P = \mathbf{B}p\}\). The resulting dynamics are 
    given by (\ref{eqn:simple-hamiltonian}),
    \begin{align}\label{eqn:simple-hamiltonian}
        \hat{H}(Q,P) &= 
        \frac{1}{2} P\tpose \Minv(Q) P + V(Q) \\
       &\begin{cases}
            \dot{Q} = \Minv(Q)P \\
            \dot{P} = -\frac{1}{2} (\Id{n} \otimes P\tpose) \nabla_Q \Minv(Q) P
                - \nabla_Q V(Q) + \simpleB \tau
        \end{cases} \nonumber
    \end{align}
    where
    \begin{align*}
        \Minv(Q) &:= \mathbf{B}D^{-1}(\mathbf{B}\tpose Q)\mathbf{B}\tpose \\
        V(Q) &:= P(\mathbf{B}\tpose Q) \\
    \end{align*}
\end{thm}
\begin{proof}
    The Poisson bracket between the functions \(f(q,p)\) and \(g(q,p)\) is defined by
    \cite{landau_mechanics} as follows:
    \[
        [f,g] := \sum \limits_{i=1}^n \pdiff{f}{p_i}\pdiff{g}{q_i} - 
                \pdiff{f}{q_i}\pdiff{g}{p_i}
    \]
    For any constant matrix \(A\), the transformation
    \(\{Q = Aq, P = Ap\}\) satisfies
    \(\pdiff{Q_i}{p_m} = \pdiff{P_i}{q_m} = 0\) for all 
    \(i,m \in \n\).
    Hence, for our new coordinates \(\{Q = \mathbf{B}q, P = \mathbf{B}p\}\),
    \begin{align*}
        [Q_i,Q_j] &:= \sum\limits_{m = 1}^n \pdiff{Q_i}{p_m}\pdiff{Q_j}{q_m} - 
        \pdiff{Q_i}{q_m}\pdiff{Q_j}{p_m} = 0 \\
        [P_i,P_j] &:= \sum\limits_{m=1}^n \pdiff{P_i}{p_m}\pdiff{P_j}{q_m} -
        \pdiff{P_i}{q_m}\pdiff{P_j}{p_m} = 0
    \end{align*}
    Since the matrix \(\mathbf{B}\) is orthogonal,
    \((\mathbf{B}_i)\tpose \mathbf{B}\tpose_j = (\mathbf{B}_i)\tpose (\mathbf{B}^{-1})_j = \delta_{i,j}\). Using this
    fact we see that the Poisson brackets between \(P\) and \(Q\) are given by:
    \begin{align*}
        [P_i,Q_j] &= \sum\limits_{m=1}^n\pdiff{P_i}{p_m}\pdiff{Q_j}{q_m}
        - \pdiff{P_i}{q_m}\pdiff{Q_j}{p_m} \\
                  &= \sum\limits_{m=1}^n \mathbf{B}_{i,m}\mathbf{B}_{j,m} - 0 \\
                  &= \sum\limits_{m=1}^n \mathbf{B}_{i,m}\mathbf{B}\tpose_{m,j} \\
                  &= (\mathbf{B}_i)\tpose \mathbf{B}\tpose_j \\
                  &= \delta_{i,j}
    \end{align*}

    Therefore, by (45.10) in \cite{landau_mechanics}, the coordinate change 
    \((Q = \mathbf{B}q, P = \mathbf{B}p)\) is a canonical transformation with
    new Hamiltonian 
    \(\hat{H}(Q,P) = \mathcal{H}(\mathbf{B}\tpose Q,\mathbf{B}\tpose P)\).

    Furthermore, since \(\dot{P} = \mathbf{B}\dot{p}\), the new input matrix is
    given by 
    \[
        \mathbf{B}B = \begin{bmatrix}
            B^\perp B \\
            B\tpose B \\
        \end{bmatrix} = 
        \begin{bmatrix}
            \Zmat{(n-k)\times k} \\
            \Id{k}
        \end{bmatrix}
    \]
    so the coordinates \(\{Q = (q_u,q_a), P = (p_u,p_a)\}\) are simply actuated coordinates for 
    \(\hat{H}\) as desired.
\end{proof}

\section{Virtual Nonholonomic Constraints}
% TODO: Motivation for VNHCs, they are an extension of VHCs, etc.
Let us imagine a child on a swing, who wants to go as high as possible. The
child will push off the ground and start swinging with small oscillations,
then extend and retract their feet appropriately to gain energy. 
If a roboticist were designing a machine to do this task, they might design a
control mechanism which makes the legs track a trajectory over time. For
example, they might tell the robot to extend and retract its legs every two
seconds. In ideal situtations, this technique would work perfectly because the
leg motion is synchronized with the swing to gain energy as fast as possible.

Most children have an adult pushing the swing to help them go higher; or perhaps
the child is swinging on a windy day. In either case, they adjust their leg
motion accordingly and do not keep track of time when kicking their legs. Te
standard control technique of tracking a function of time, known as
\textit{trajectory tracking}, would not work because the disturbance affecting
the swing will desynchronize the motion of the legs with the swing and stop the
energy-gaining effects.

This is a common issue in the control of many biologically-inspired systems. 
One method which can provide more realistic, robust motion is the method of
of virtual holonomic constraints (VHCs). Instead of a robot's actuators tracking
a trajectory over time, VHCs use the actuators to enforce a relation
\(h(q) = 0\) of the configuration \cite{vhcs_for_el_systems}. 
This method has provided incredible results in the development of 
walking robots \cite{vhc_robotic_walking, vhc_stable_walking}, 
vehicle motion \cite{vhc_bicycle, vhc_helicopter}, 
and has even been used to design a snake-like swimming robot
\cite{vhc_snake}.

Many authors have attempted to extend these results to enforce a relation 
\(h(q,\dot{q}) = 0\) of the full state. Since these relations use actuators to
restrict both the configuration and generalized velocity, they are called
virtual \textit{nonholonomic} constraints. This idea has been used for
human-robot interaction
\cite{vnhc_human_robot_cooperation,psd_based_vnhc_redundant_manipulator,haptic_vnhc},
error-reduction on time-delayed systems \cite{vnhc_time_delay_teleop},
and improving bipedal locomotion \cite{nhvc_dynamic_walking,
hybrid_zero_dynamics_bipedal_nhvcs,nhvc_incline_walking,output_nhvc_bipedal_control}.

Unlike the theory of VHCs, there does not appear to be a standard definition of virtual
nonholonomic constraints. One issue is that Lagrangian dynamics is not
particularly suited to finding controllers which stabilize desired constraints. Because
of this, all the applications listed above use their own definition of a virtual
nonholonomic constraint, which makes it difficult to compare and analyze their
work.

This section will provide a new characterization of virtual nonholonomic
constraints. The goal is to provide a consistent, rigorous foundation for
designing constraints on a general class of systems.

% TODO: Perform the full development of VNHCs, including its stabilizing
% controller.
\begin{defn}
    A \textit{virtual nonholonomic constraint} (VNHC) \textit{of order \(k\)} is a
    relation \(h(q,p) = 0\) where \(h : \mathcal{Q}\times\R^n \rightarrow \R^k\) is
    smooth, \(\rank{\left[ dh_q,\, dh_p \right]} = k\) for all 
    \((q,p) \in h\inv(0)\), and there exists a feedback controller \(\tau(q,p)\)
    stabilizing the set
    \[
        \Gamma = \left\{(q,p) \mid h(q,p) = 0, dh_q \dot{q} + dh_p \dot{p} = 0\right\}
    \]
    which is called the \textit{constraint manifold}.
\end{defn}

If we define the error term  \(e = h(q,p)\), stabilizing \(\Gamma\) is
equivalent to solving for \(\tau(q,p)\) which drives \(e \rightarrow 0\) and
\(\dot{e} \rightarrow 0\).
Let us imagine that we have no further structure on the VNHC. 
Then \(\dot{e} = dh_q \dot{q} + dh_p \dot{p}\), so \(\tau\) appears inside
\(\dot{e}\). To solve for \(\tau\) explicitly, we must have that \(dh_p\) is
invertible for all \(p\), which is the same as requiring that \(h(q,p)\) is
strictly monotonic in \(p\).

This is a very restrictive condition, which many VNHCs will not satisfy. For
this reason, it is preferable to have the torque \(\tau\) appear after two 
derivatives of \(e\) so that there is more freedom in the types of constraints
one can use.
Mathematically, if \(\tau\) appears only after two derivatives, one says 
that \(e\) is of \textit{relative degree} \(\{2,2,\ldots,2\}\). 
We thus define a special type of VNHC which satisfies this
property.

\begin{defn}
    A VNHC of order \(k\) \(h(q,p) = 0\) is \textit{regular} if the output 
    \(e = h(q,p)\) is of relative degree \(\{2,2.\ldots,2\}\) everywhere on the
    constraint manifold \(\Gamma\).
\end{defn}

Modifying the results of
\cite{nhvc_dynamic_walking,hybrid_zero_dynamics_bipedal_nhvcs,nhvc_incline_walking}
into the Hamiltonian framework, one observes that
relations which use only the unactuated momentum \(p_u\) result in \(\tau\)
appearing only after two derivatives of \(e\).
To be able to use \(p_u\), we will continue with the following assumption for
the rest of the chapter.

\begin{assm}\label{assm:H-is-simply-actuated}
    The mechanical system under consideration is a
    Hamiltonian system with \(n\) degrees of freedom and 
    \(k < n\) actuators. It is described in simply
    actuated coordinates \(\{q = (q_u,q_a), p = (p_u, p_a)\}\) and has the
    following dynamics:
    \begin{align*}
        \mathcal{H}(q,p) &= p\tpose \Minv(q) p + V(q) \\
         &\begin{cases}
            \dot{q} = \Minv p \\
            \dot{p} = -\frac{1}{2} \pdmat - \nabla_q V(q) + \simpleB \tau \\
        \end{cases}
    \end{align*}
\end{assm}
\begin{notation}
    We will write \(q_u \in \mathcal{Q}_u\), \(q_a \in \mathcal{Q}_a\) where
    \(\mathcal{Q}_u \times \mathcal{Q}_a = \mathcal{Q}\). 
    We also write
    \(p_u \in \mathcal{P}_u = \R^{n-k}\) and 
    \(p_a \in \mathcal{P}_a = \R^k\), so that 
    \(p \in \mathcal{P} := \mathcal{P}_u \times \mathcal{P}_a\). 
    In this manner, the phase space of our system can be written as
    \(\mathcal{Q} \times \mathcal{P}\).
\end{notation}

\begin{thm}\label{thm:vnhc-regularity}
    A VNHC of order \(k\) is regular if and only if \(dh_{p_a} = 0\) 
    and
    \[
        \rank{\left(dh_q \Minv(q) - 
          dh_{p_u} (\Id{n-k} \otimes p\tpose)\nabla_{q_u}\Minv(q) 
         \right)\simpleB} = k
    \]
    everywhere on the constraint manifold \(\Gamma\).
\end{thm}
\begin{proof}
    % TODO: complete this proof with our notation
    Let \(e = h(q,p) \in \R^k\). Then 
    \begin{align*}
        \dot{e} &= dh_q \dot{q} + dh_p \dot{p} \\
                &= dh_q \Minv(q)p +{}  \\
            & \begin{bmatrix} dh_{p_u} & dh_{p_a} \end{bmatrix}
        \left( -\frac{1}{2} \begin{bmatrix}
            (\Id{n-k} \otimes p\tpose) \nabla_{q_u}\Minv(q)p \\
            (\Id{k} \otimes p\tpose) \nabla_{q_a}\Minv(q)p
            \end{bmatrix} - \begin{bmatrix}
            \nabla_{q_u}V(q) \\
            \nabla_{q_a}V(q)
        \end{bmatrix} + \simpleB \tau\right)
    \end{align*}
    If \(dh_{p_a} \neq \Zmat{k \times k}\) for some \((q,p)\) on \(\Gamma\), 
    then \(\tau\) appears in \(\dot{e}\) and the VNHC is not of relative degree
    \(\{2,2,\ldots,2\}\).
    Hence, we must have that \(dh_{p_a} = \Zmat{k \times k}\) if we want the
    VNHC to be regular. Proceeding with this assumption, we now find that
    \(h : \mathcal{Q} \times \mathcal{P}_u \rightarrow \R^k\), which means
    \[
        \dot{e} = dh_q \Minv(q)p - 
        dh_{p_u} \left(\frac{1}{2} \pudmat + \nabla_{q_u}V(q)\right)
    \]
    Taking one further derivative provides
    \begin{align*}
        \ddot{e} &= \diff{}{t}\left\{dh_q\right\}\Minv(q) p + 
        dh_q \left(\sum\limits_{i=1}^n \pdiff{\Minv}{q_i}(q)\dot{q_i}\right)p + 
        dh_q \Minv(q) \dot{p} - \\
         & \diff{}{t}\left\{dh_{p_u}\right\}
         \left(\frac{1}{2}\pudmat + \nabla_{q_u}V(q)\right) - \\
         & dh_{p_u}\left(\frac{1}{2}\diff{}{t}\left\{\pudmat\right\} + 
         \diff{}{t}\left\{\nabla_{q_u}V(q)\right\} \right)
    \end{align*}
    We will compute the explicit expression of \(\ddot{e}\) in pieces. 
    First we begin with \(\diff{}{t}\left\{dh_q\right\}\). 
    Since \(h = (h^1,\ldots,h^k)\) where 
    \(h^i : \mathcal{Q} \times \mathcal{P}_u \rightarrow \R\), we have that 
    \[
        dh_q = \begin{bmatrix}
            dh^1_q \\
            \vdots \\
            dh^k_q
        \end{bmatrix} = \begin{bmatrix}
        \pdiff{h^1}{q_1} & \cdots & \pdiff{h^1}{q_n} \\
        \vdots & \ddots & \vdots \\
        \pdiff{h^k}{q_1} & \cdots & \pdiff{h^k}{q_n}
        \end{bmatrix}
    \]
    The derivative is taken element-wise in each matrix, yielding
    \[
        \diff{}{t}\left\{dh_q\right\} = \begin{bmatrix}
            \sum_{j=1}^n \ppdiff{h^1}{q_1}{q_j}\dot{q}_j & \cdots & \sum_{j=1}^n
            \ppdiff{h^1}{q_n}{q_j}\dot{q}_j \\
            \vdots & \ddots & \vdots \\
            \sum_{j=1}^n \ppdiff{h^k}{q_1}{q_j}\dot{q}_j & \cdots & \sum_{j=1}^n
            \ppdiff{h^k}{q_n}{q_j}\dot{q}_j \\
        \end{bmatrix} + \begin{bmatrix}
            \sum_{l=1}^{n-k}\ppdiff{h^1}{q_1}{p_{u_l}}\dot{p}_{u_l} & \cdots &
            \sum_{l=1}^{n-k}\ppdiff{h^1}{q_n}{p_{u_l}}\dot{p}_{u_l} \\
            \vdots & \ddots & \vdots \\
            \sum_{l=1}^{n-k}\ppdiff{h^k}{q_1}{p_{u_l}}\dot{p}_{u_l} & \cdots & 
            \sum_{l=1}^{n-k}\ppdiff{h^k}{q_n}{p_{u_l}}\dot{p}_{u_l} \\
        \end{bmatrix}
    \]
    It is straightforward computation to confirm that each row of this
    derivative can be written in vector form as follows:
    \begin{align*}
        \diff{}{t}\left\{dh^i_q\right\} &= \begin{bmatrix}
        \sum_{j=1}^n \ppdiff{h^i}{q_1}{q_j}\dot{q}_j & \cdots & 
        \sum_{j=1}^n \ppdiff{h^i}{q_n}{q_j}\dot{q}_j 
        \end{bmatrix} + \begin{bmatrix}
        \sum_{l=1}^{n-k}\ppdiff{h^i}{q_1}{p_{u_l}}\dot{p}_{u_l} & \cdots &
        \sum_{l=1}^{n-k}\ppdiff{h^i}{q_n}{p_{u_l}}\dot{p}_{u_l}
        \end{bmatrix} \\
       &= \dot{q}\tpose \Hess_q\left\{h^i\right\}\tpose + 
       \dot{p}_u\tpose \partial_{p_u}\partial_{q}h^i \\
       &= p\tpose \Minv(q)\Hess_q\left\{h^i\right\}\tpose -
       \left(\frac{1}{2}p\tpose d\Minv_{q_u}(q)(\Id{n-k}\otimes p) + 
       dV_{q_u}(q)\right)\partial_{p_u}\partial_{q}h^i
    \end{align*}
    This means that the entire matrix is given by
    \begin{align*}
        \diff{}{t}\left\{dh_q\right\} &= 
        \left(\Id{n} \otimes 
        \left(p\tpose \Minv(q)\right)\right)
        \Hess_q\left\{h\right\}\tpose -{} \\
      & \left(\frac{1}{2}\Id{n} \otimes 
          \left(p\tpose d\Minv_{q_u}(q)(\Id{n-k}\otimes p)\right)
        + \Id{n}\otimes dV_{q_u}(q)\right)\partial_{p_u}\partial_{q}h
    \end{align*}
    For the next segment, let \(e_j \in \R^n\) be the vector of all zeros,
    except for a single \(1\) at position \(j\). We define
    \(C_1 : \mathcal{Q} \times \mathcal{P} \rightarrow \R^{n\times n}\) by
    \[
        C_1(q,p) := \sum\limits_{j=1}^n \pdiff{\Minv}{q_j}(q)\dot{q}_j 
        = \sum\limits_{j=1}^n \pdiff{\Minv}{q_j}(q)\left(e_j\tpose\Minv(q)p\right)
    \]
    Moving on, one can use a similar approach to \(\diff{}{t}\left\{dh_q\right\}\)
    to find the derivative of \(dh_{p_u}\) in matrix form. It is given by
    \begin{align*}
        \diff{}{t}\left\{dh_{p_u}\right\} &= 
        \left( \Id{n} \otimes \left(p\tpose \Minv(q)\right)\right) 
        \partial_q\partial_{p_u}h -{} \\
      & \left(\frac{1}{2}\Id{n} \otimes \left(\pudmat\right) + 
      \Id{n} \otimes \nabla_{q_u}V(q)\right)\Hess_{p_u}\left\{h\right\}\tpose
    \end{align*}
    Now observe that the \(i^\text{th}\) row of
    \[
        \frac{1}{2} \diff{}{t} \left\{ \pudmat \right\}
    \]
    is given by
    \begin{align*}
        \frac{1}{2} \diff{}{t}\left\{p\tpose \pdiff{\Minv}{q_{u_i}}(q)p\right\}
        = p\tpose \pdiff{\Minv}{q_{u_i}}(q) \dot{p} + 
        \frac{1}{2} p\tpose \left( \sum_{j=1}^n \ppdiff{\Minv}{q_{u_i}}{q_j}
        \dot{q}_j \right) p
    \end{align*}
    Defining \(C_2 : \mathcal{Q} \times \mathcal{P} \rightarrow \R^{n(n-k)\times n}\)
    allows us to write the entire derivative in matrix form:
    \begin{align*}
        & \frac{1}{2} \diff{}{t} \left\{ \pudmat \right\} ={} \\
        & (\Id{n-k} \otimes p\tpose)\nabla_{q_u}\Minv(q)
        \left(-\frac{1}{2}\pudmat - \nabla_{q_u}V(q) + \simpleB \tau \right) +{}
        \\
        & \frac{1}{2}\left(\Id{n-k}\otimes p\tpose\right) C_2(q,p) p
    \end{align*}
    Finally, we compute the derivative of \(\nabla_{q_u}V(q)\)
    \[
        \diff{}{t}\left\{\nabla_{q_u}V(q)\right\} = 
        \begin{bmatrix}
            \sum_{j=1}^n \ppdiff{V}{q_{u_1}}{q_j}\dot{q}_j \\
            \vdots \\
            \sum_{j=1}^n \ppdiff{V}{q_{u_{n-k}}}{q_j}\dot{q}_j
        \end{bmatrix = \partial_{q_u} \partial_q V(q) \dot{q}
    \]
    and so we get that
    \[
        \diff{}{t}\left\{\nabla_{q_u}V(q)\right\} = 
        \partial_{q_u} \partial_q V(q) \Minv(q)p
    \]
    Putting this all together, we can find the explicit form for \(\ddot{e}\):
    \begin{align*}
        \ddot{e} &= \diff{}{t}\left\{dh_q\right\}\Minv(q)p + dh_q C_1(q,p)p -{}\\
     & \frac{1}{2}dh_q\Minv(q)(\pdmat) + 
        dh_q\Minv(q)\nabla_qV(q) -{}\\
     & \diff{}{t}\left\{dh_{p_u}\right\}
     \left(\frac{1}{2}\pudmat + \nabla_{q_u}V(q)\right) +{} \\
     & dh_{p_u}(\Id{n-k}\otimes p\tpose)\nabla_{q_u}\Minv(q)\left(
     \frac{1}{2}\pudmat + \nabla_{q_u}V(q)\right) -{}\\
     \frac{1}{2}dh_{p_u}(\Id{n-k}\otimes p\tpose)C_2(q,p)p - 
     dh_{p_u} \partial_{q_u}\partial_q V(q)\Minv(q)p +{} \\
     \left(dh_q \Minv(q) - dh_{p_u}(\Id{n-k} \otimes
     p\tpose)\nabla_{q_u}\Minv(q) \right) \simpleB \tau
    \end{align*}
    The VNHC \(h\) is regular iff \(e\) is of relative degree
    \(\{2,\ldots,2\}\), which is true iff \(\tau\) appears in each element of
    \(e\) on the constraint manifold. 
    Letting \(\ddot{e} = E(q,p) + H(q,p)\tau\), this is equivalent to requiring
    that the matrix \(H\) be full rank.
\end{proof}

Using the expression \(\ddot{e} = E(q,p) + H(q,p)\tau\) from the proof of 
Theorem \ref{thm:vnhc-regularity}, a regular VNHC of order \(k\) can be
stabilized by the output-linearizing phase-feedback controller
(\ref{eqn:vnhc-torque-controller}).
\begin{equation}\label{eqn:vnhc-torque-controller}
    \tau(q,p) = -H\inv(q,p)\left(E(q,p) + k_p e + k_d \dot{e}\right)
\end{equation}
where \(k_p, k_d \in \R_{>0}\) are control parameters which can be tuned on the
resulting linear system \(\ddot{e} = -k_p e - k_d\dot{e}\). 

Note that one generally cannot measure conjugate of momenta directly, as sensors
on mechanical systems typically only measure the state \((q,\dot{q})\). To
implement this controller in practice, one must compute \(p = M(q)\dot{q}\) at
every iteration. In other words, this controller requires knowledge of the full
state of the system.

%TODO: Special case: when dM/dqu = 0, show that we have a nice form and that we
%can solve for p_a and the closed-loop dynamics (qu,pu)_dot

Now that we have found a controller to enforce a regular VNHC of order \(k\), we
would like to solve for the closed-loop dynamics. Naturally, these dynamics
should be specified through \((q_u, p_u)\) since \(q_a\) is a function of these
as specified by \(h(q,p_u) = 0\).
Unfortunately, \(\dot{q}_u\) depends on \(p_a\), and for general systems one
cannot solve explicitly for \(p_a\) in terms of \((q_u,p_u)\). This is because
the \(\dot{p}\) dynamics contains the coupling term 
\((\Id{n} \otimes p\tpose)\nabla_{q_u}M(q)p\). 

We now introduce a class of systems where explicitly solving for the closed-loop
dynamics is feasible.

\begin{defn}
    A mechanical system is ------ if \(\nabla_{q_u}M(q) = 0\)
\end{defn}

\begin{thm}\label{thm:zero-dynamics}
    Let \(\mathcal{H}\) be a ----- mechanical system satisfying Assumption
    \ref{assm:H-is-simply-actuated}. Let
    \(h(q,p_u) = 0\) be a regular VNHC of order \(k\) with constraint manifold
    \(\Gamma\).
    Then, on \(\Gamma\), the closed-loop dynamics are given by
    \begin{equation}\label{eqn:qpu-dynamics}
        \left.\begin{aligned}
            \dot{q}_u &= M(q)p \\
            \dot{p}_u &= -\nabla_{q_u}V(q) \\
            \end{aligned}\right|_{\begin{array}{c}
                h(q,p_u) = 0 \\ 
                pa = g(q_u,p_u) \\
            \end{array}}
    \end{equation}
    where
    \begin{equation}\label{eqn:g-qpu}
    \begin{aligned}
        g(q,p_u) &:= \\
           &\left.\left(dh_q \Minv(q) \simpleB \right)\inv 
        \left(dh_{p_u} \nabla_{q_u}V(q) - dh_q \Minv(q)
        \begin{bmatrix}
            I_{n-k} \\
            \Zmat{k \times (n-k)} \\
        \end{bmatrix} p_u\right)\right|_{h(q,p_u) = 0}
    \end{aligned}
    \end{equation}
\end{thm}
\begin{proof}
%TODO: Complete the proof for this theorem
\end{proof}

The most fascinating part of Theorem \ref{thm:zero-dynamics} is that it shows
that the dynamics on \(\Gamma\) are entirely described by the \(2(n-k)\)
unactuated coordinates, regardless of the number of degrees of freedom of the
system. This means that \(\Gamma\) is a \(2(n-k)\)-dimensional surface from
which the system cannot escape. This becomes especially important when analyzing
systems of degree of underactuation one, since it means the constraint is always
a curve which can be plotted on the plane with axes \(q_u\) and \(p_u\)
regardless of the original dimension \(n\) of the system. 

In future sections of this thesis, we will apply a very specific type of
constraint. The following corollay applies Theorem \ref{thm:zero-dynamics} to
these constraints.

\begin{cor}\label{cor:2d-zero-dynamics}
    Suppose \(\mathcal{H}\) is a ---- mechanical system satisfying Assumption
    \ref{assm:H-is-simply-actuated} which has degree of underactuation one.
    Let \(h(q,p_u) = q_a - f(q_u,p_u)\) be a regular VNHC of order \((n-1)\),
    where \(f\) is a suitably-defined smooth function.
    Defining \(e_1 := (1,0,\ldots,0) \in \R^n\), the actuated momentum 
    is given by
    \begin{equation}\label{eqn:g-qupu}
        p_a = -\left.\left(dh_q \Minv(q)
        \begin{bmatrix}
            \Zmat{1\times (n-1)}\\
            I_{(n-1)}
        \end{bmatrix}\right)\inv 
        \left(\partial_{p_u}f \partial_{q_u}V + dh_q \Minv(q)e_1 p_u\right) 
            \right|_{q_a =f(q_u,p_u)}
    \end{equation}
    Since \(q_u \in \Rt{T}\) for some \(T > 0\), \((q_u(t),p_u(t))\) traces out
    a curve on the so-called \((q,p)\)-plane \(\Rt{T} \times \R\).
\end{cor}

\section{Summary of Results}
%TODO: summarize the assumptions and results 















%\section{Virtual Nonholonomic Constraints}\label{sec:vnhcs}
%
%%----- Motivation -----%
%\subsection{Motivation}
%\textbf{TODO: Why do we bother with Hamiltonian? Why can't we do virtual nonholonomic 
%constraints in Lagrangian? What are some use-cases where VHCs don't work?}
%
%%---------- Hamiltonian from Lagrangian ----------%
%\subsection{Review: Hamiltonian Systems}
%\textbf{TODO: What is a Hamiltonian system and why does it matter?}
%
%One can compute the Hamiltonian of a system by performing a Legendre transform
%on its Lagrangian \textbf{TODO: citation for legendre transform}.
%First, define the conjugate of momenta for \(q\) by 
%\begin{equation*}
%p = \frac{\partial \mathcal{L}}{\partial \dot{q}} \in \mathbb{R}^n
%\end{equation*}
%Then, the Legendre transform is performed by taking
%\begin{equation*}
%\mathcal{H}(q,p) = p^T \dot{q} - \mathcal{L}(q,\dot{q})
%\end{equation*}
%
%For a mechanical system (\ref{eqn:lagrangian}), the conjugate of momenta for
%\(q\) is given by 
%\begin{equation*}
%    p = M(q)\dot{q}
%\end{equation*}
%which means the Hamiltonian of the system is given by
%the total mechanical energy \(E\) (\ref{eqn:hamiltonian} in 
%\((q,p)\) coordinates.
%\begin{equation}\label{eqn:hamiltonian}
%\mathcal{H}(q,p) = E(q,p) = \frac{1}{2} p^T \Minv(q) p + V(q)
%\end{equation}
%Note that \(M(q)\) is the inertia matrix and \(V(q)\) is the potential for
%the Hamiltonian system. This is simply a trick to distinguish them notationally
%from the Lagrangian \(D(q)\) and \(P(q)\); they are, in fact, identical in their
%contents.
%
%The equations of motion for the system in Hamiltonian coordinates is given by
%\begin{align}\label{eqn:hamiltionian_eom}
%\begin{split}
%\dot{q} &= \frac{\partial \mathcal{H}}{\partial p} = \Minv(q) p \\
%\dot{p} &= -\frac{\partial \mathcal{H}}{\partial q} + B \tau
%\end{split}
%\end{align}
%
%\textbf{Why bother looking at Hamiltonian systems? What is the intuition behind
%these equations of motion?}
%
%%---------- Hamiltonian VNHCs ----------%
%\subsection{Hamiltonian Virtual Nonholonomic Constraints}
%While VHCs are still possible in the Hamiltonian framework, the assumptions
%required to make this work are slightly different. Rather than deriving
%Hamiltonian VHCs directly, we will produce results for nonholonomic constraints
%first as VHCs are a special case of this new framework.
%
%Suppose the mechanical system has degree of underactuation one, so that coordinates of the system can be split into an unactuated component \(q_u \in [\mathbb{R}]_T, \, T \in \mathbb{R}_{>0}\) which is not influence by control, along with an actuated component \(q_a\); that is, suppose \(B\) is of the form \(B(q) = [0_m, B_1^T(q), \ldots, B_n^T(q)]^T, \, B_i^T(q) \in \mathbb{R}^{n-1}\) and \(\tau \in \mathbb{R}^{n-1}\). In this case, \(q = (q_u, q_a)^T\) and the equations of motion become
%\begin{align}\label{eqn:unactuated_actuated_eom}
%\begin{split}
%\dot{q_u} &= e_1^T \Minv(q) p \\
%\dot{p_u} &= -p^T\frac{\partial M}{\partial q_u} p - \partial_{q_u}V(q) \\
%\dot{q_a} &= 
%\begin{bmatrix}
%0 \cdots 0 \\
%I_{n-1} \\
%\end{bmatrix} \Minv(q) p \\
%\dot{p_a} &= -p^T\frac{\partial M}{\partial q_a} p - \nabla{q_a}V(q) + 
%\begin{bmatrix}
%B_1(q) \\
%\vdots \\
%B_n(q)
%\end{bmatrix} \tau
%\end{split}
%\end{align}
%
%Now we can begin to talk about Virtual Nonholonomic Constraints. In a similar fashion to what was defined for VHCs, let us first define the goal of these new virtual constraints.
%
%\begin{defn}
%A relation \(h \in C^2\left(\mathcal{Q}\times \mathbb{R}^n ; \mathbb{R}^k\right)\) with \(h(q,p) = 0\) is a \textbf{virtual nonholonomic constraint (VNHC) of order k} if there exists a feedback control \(\tau(q,p)\) which stabilizes the constraint manifold
%\[
%\Gamma = \left\{(q,p) | h(q,p) = 0, dh_q \dot{q} + dh_p \dot{p} = 0\right\}
%\]
%\end{defn}
%Define the output of the system to be \(e = h(q,p)\). We would like to find \(\tau(q,p)\) which drives \(e\) to zero to stabilize our constraint manifold \(\Gamma\). To accomplish this, we will input-output linearize (\ref{eqn:unactuated_actuated_eom}) to find \(\ddot{e} = -k_p e - k_d \dot{e}\) with \(k_p, k_d \in \mathbb{R}_{> 0}\).
%
%To characterize a certain class of VNHCs, let us make the following assumption.
%\begin{assm}\label{assm:vnhc_is_on_qu_pu}
%We assume our relation \(h\) is of the form \(h(q,p) = q_a - f(q_u,p_u)\) for some \(f \in C^2\left([\mathbb{R}]_T \times \mathbb{R} ; \mathbb{R}^{n - 1}\right)\).
%\end{assm}
%
%Now we solve for \(\tau\) by finding \(\ddot{e}\).
%\begin{align*}
%    e &= h(q,p) = qa - f(q_u,p_u)\\
%    \Rightarrow \dot{e} &= \dot{q_a} - df_{q_u}\dot{q_u} -df_{p_u}\dot{p_u} \\
%    &= [-df_{q_u} I_{n-1}]\dot{q} - df_{p_u} \dot{p_u} \\
%    &= dh_q \Minv(q) p - df_{p_u}\left( -\frac{1}{2}p^T \frac{\partial \Minv(q)}{\partial q_u} p - \partial_{q_u}V(q) \right) \\
%    &= dh_q \Minv(q) p + \frac{1}{2}df_{p_u} p^T \frac{\partial \Minv(q)}{\partial q_u} p + df_{p_u}\partial_{q_u}V(q)
%\end{align*}
%The control input \(\tau\) only appears in \(\dot{p_a}\) (see (\ref{eqn:unactuated_actuated_eom})). To simplify the analysis, terms in \(\ddot{e}\) which do not depend on \(\dot{p_a}\) explicitly are lumped together under the symbol \((*)\):
%\begin{align*}
%    \ddot{e} &= dh_q \Minv(q) \dot{p} + df_{p_u} p^T \frac{\partial \Minv(q)}{\partial q_u} \dot{p} + (*) \\
%    &= (dh_q \Minv(q) + df_{p_u} p^T \frac{\partial \Minv(q)}{\partial q_u})B\tau + (*) \\
%    &= (dh_q \Minv(q) + dh_{p_u} p^T \frac{\partial \Minv(q)}{\partial q_u})B\tau + (*)
%\end{align*}
%
%From the derivations above, one can solve for \(\tau\) iff the matrix on the left of \(\tau\) is full rank. Thus, for systems with degree of underactuation one we give the following definition.
%\begin{defn}
%A VNHC \(h(q,p) = 0\) of order \(n - 1\) is \textbf{regular} if \(dh_{p_a} = 0\), \(dh_{q_a} = (1 \ldots 1)^T\), and 
%\[
%\text{rank}\left\{ (dh_q \Minv(q) + dh_{p_u} p^T \frac{\partial \Minv(q)}{\partial q_u})B\right\} = n - 1
%\]
%everywhere on the constraint manifold \(\Gamma\). Equivalently, a VNHC \(h\) of order \(n - 1\) is regular if it satisfies Assumption \ref{assm:vnhc_is_on_qu_pu} and system (\ref{eqn:unactuated_actuated_eom}) with output \(e = h(q,p)\) is of relative degree \(\{2,2,\ldots,2\}\) everywhere on \(\Gamma\).
%\end{defn}
%
%In general, \(\dot{e}\) is a function of \(q_u\) and \(p = (p_u,p_a)^T\). Since the purpose of a regular VNHC is to fully parameterize \(\Gamma\) by \((q_u,p_u)\), it is essential that one can solve for \(p_a = p_a(q_u,p_u)\). Unfortunately this often cannot be done, since \(\dot{e}\) contains the quadratic term
%\[
%\frac{1}{2} df_{p_u} p^T \frac{\partial \Minv(q)}{\partial q_u} p 
%\]
%We can solve for \(p_a\) if this quadratic term does not exist.
%\begin{assm}\label{assm:M_is_Mqa}
%Assume \(\partial M(q) / \partial q_u = 0 \Leftrightarrow \partial \Minv(q) / \partial q_u = 0\)
%\end{assm}
%
%Under Assumption \ref{assm:M_is_Mqa}, we get that the rank condition for \(h(q,p)\) to be a regular VNHC reduces to \(\text{rank}\left(dh_q \Minv B\right) = n - 1\). This is the same rank condition as required for Virtual Holonomic Constraints.
%
%Now we solve for \(p_a\) on the constraint manifold (when \(e = \dot{e} = 0\)):
%\begin{align*}
%    \dot{e} = dh_q \Minv(q)p + df_{p_u} \partial_{q_u}V(q) &= 0\\
%    \Leftrightarrow dh_q \Minv(q)e_1 p_u + dh_q \Minv(q) \begin{bmatrix}
%    0 & \cdots & 0 \\
%    & I_{n-1} & \\
%    \end{bmatrix} p_a &= -df_{p_u} \partial_{q_u}V(q) \\
%\end{align*}
%\begin{align*}
%    \Leftrightarrow dh_q \Minv(q) \begin{bmatrix}
%    0 & \cdots & 0 \\
%    & I_{n-1} & \\
%    \end{bmatrix} p_a = -\left(df_{p_u}\partial_{q_u}V(q) + dh_q \Minv(q)e_1 p_u\right)
%\end{align*}
%One can linearly solve for \(p_a\) if and only if the matrix in front of it is invertible.
%
%This leads us to a natural definition.
%\begin{defn}\label{defn:solvability}
%A VNHC \(h(q,p)\) is \textbf{solvable} (NOTE: actionable? what's a good name?) if
%\[
%\text{rank}\left(dh_q \Minv(q) \begin{bmatrix}
%    0 & \cdots & 0 \\
%    & I_{n-1} & \\
%    \end{bmatrix}\right) = n - 1
%\]
%\end{defn}
%
%With this analysis and our new definition in hand, we can solve for the dynamics on the constraint manifold.
%
%\begin{thm}\label{thm:equation_for_pa}
%Suppose assumptions \ref{assm:vnhc_is_on_qu_pu} and \ref{assm:M_is_Mqa} hold.
%If a regular VNHC \(h(q,p) = q_a - f(q_u,p_u)\) is solvable, then the
%parameterization for \(p_a\) on the constraint manifold is given by
%\begin{align*}
%p_a &= -\left(dh_q \Minv(q) \begin{bmatrix}
%    0 & \cdots & 0 \\
%    & I_{n-1} & \\
%    \end{bmatrix}\right)^{-1}\left( df_{p_u}\partial_{q_u}V(q) + dh_q \Minv(q)e_1 p_u\right) \\
%    &=: g(q_u,p_u)
%\end{align*}
%and the constrained dynamics on \(\Gamma\) are given by
%\begin{align*}
%    \dot{q_u} &= e_1^T \Minv(q_a) \begin{bmatrix}
%    p_u \\
%    p_a
%    \end{bmatrix}\mid_{q_a = f(q_u,p_u), p_a = g(q_u,p_u)} \\
%    \dot{p_u} &= -\partial_{q_u} V(q_u,q_a) \mid_{q_a = f(q_u,p_u)}
%\end{align*}
%\end{thm}
%
%Theorem \ref{thm:equation_for_pa} guarantees that, on \(\Gamma\), \(q_a\) is a
%parameterized completely by \((q_u,p_u)\). Hence, the zero-dynamics on
%\(\Gamma\) are always two-dimensional regardless of the original dimension
%\(n\).
%
%%---------- Hamiltonian VHCS ---------%
%\subsection{Restriction to Hamiltonian VHCs}
%\textbf{TODO: Why can we not do the standard VHC approach for Hamiltonian? 
%Show how to use the above to make it work}
%
%%/========== /Virtual Nonholonomic Constraints ==========/% 
% vim: set tw=80 ts=4 sw=4 sts=0 et ffs=unix :


% Discuss the variable length pendulum and prove our VNHC gains energy
%! TEX root = main.tex

%/========== Variable Length Pendulum ==========/%
\chapter{Application of VNHCS: The Variable Length Pendulum}\label{sec:vlp}
\section{Motivation}
% TODO: Why is this a good problem to study? What is the problem? What would a human do on a standing swing?
\section{The VLP Constraint}
% TODO: Go through the development of the constraint and prove it gains energy

\section{Simulation Results}

%/========== /Variable Length Pendulum ==========/%
% vim: set ts=3 sw=3 sts=0 et ffs=unix :


% Give the Acrobot model, its VNHC, and prove it gains energy
%! TEX root = main.tex

%/========== Acrobot ==========/%

\chapter{Application of VNHCs: The Acrobot}\label{ch:acrobot}
\section{Motivation}
The acrobot is a two-link pendulum, actuated at the center joint (as in Figure
\ref{fig:acrobot-model}). 
Since its first description in 1990
\cite{nonlinear_controllers_nonintegrable_acrobot}, the acrobot has become a
benchmark problem in control theory; 
it is an underactuated mechanical system which produces complex nonlinear motion
from an easy-to-describe model.
The acrobot models a gymnast on a bar, since it represents a torso (top link)
and legs (bottom link) with motion generated by the swinging of the legs at the
hips. 
It is also one of the simplest models for a biped walking robot
\cite{toward_framework_biped_locomotion}.

\begin{figure}
    \centering
    \includestandalone[width=0.7\textwidth]{images/acrobot_model}
    \caption{The general acrobot model, represented by two weighted rods
    differing in both length and mass.}%
    \label{fig:acrobot-model}
\end{figure}

Controlling the acrobot is a nontrivial task, since it is not feedback
linearizable \cite{nonlinear_controllers_nonintegrable_acrobot}. 
Several researchers have studied the swing-up problem of driving the acrobot to
its equilibrium point above the bar using partial feedback linearization
\cite{swingup_problem_acrobot}, energy-based control
\cite{swingup_acrobot_pendulum, swingup_acrobot_energy}, and through studying
human motion \cite{swingup_giant_acrobot, motion_control_gymnastic_skill}.

In gymnastics terminology, a ``giant" is the motion a gymnast performs to
achieve full rotations around the bar \cite{usagym_giant}. 
We are interested in using VNHCs to generate giant motion, with the aim of
stabilizing desired energy levels.
The control of giant motion for the acrobot has been studied in
\cite{energy_pumping_robotic_swinging, swingup_giant_acrobot}, 
and some authors have used virtual holonomic constraints to achieve this
behaviour
\cite{dynamical_servo_acrobot_vc, control_giant_two_link_gymnastic_robot,
xingbo_thesis}. 
However, these controllers are neither intuitive nor easy to
design:
\cite{control_giant_two_link_gymnastic_robot} defines a constraint by inverting
a trajectory in time onto the state space; 
\cite{dynamical_servo_acrobot_vc} requires a cascade controller to stabilize
both a constraint and a desired limit cycle in the state space; 
and \cite{xingbo_thesis} enforces the giant by adding an extra state to estimate
velocity, which increases the dimensionality of the problem in a crude
approach to using VNHCs.

In this chapter we will design a physically-intuitive VNHC which generates giant
motion and prove the acrobot gains energy. 
In the process of completing this proof, we will arrive at a promising method
which might one day be useful for generating energy-injecting VNHCs on arbitrary
mechanical systems.

\section{Dynamics of the Acrobot}
Suppose we are given an acrobot as in Figure \ref{fig:acrobot-model} modelling a
gymnast hanging on a horizontal bar, where the ``torso" has moment of
inertia \(J_u\) and the ``leg" has moment of inertia \(J_a\) (each with respect
to their own center of mass).
Let \(q_u \in \mathbb{S}^1\) be the shoulder angle and \(q_a \in \mathbb{S}^1\) 
be the hip angle, where only \(q_a\) is actuated. 
Collecting them together provides the configuration
\(q = (q_u,q_a) \in \mathbb{S}^1 \times \mathbb{S}^1\). 
The acrobot has inertia matrix \(D\), potential function \(P\) (with respect to
the horizontal bar), and input matrix \(B\)  given as
follows \cite{xingbo_thesis}:
\begin{align}\label{eqn:general-acrobot-inertia}
    D(q) &= \begin{bmatrix}
      m_al_u^2 + 2m_a\cos(q_a)l_u l_{c_a} + m_al_{c_a}^2 + m_ul_{c_u}^2 + J_u + J_a &
      m_al_{c_a}^2 + m_al_ul_{c_a}\cos(q_a) + J_a \\
      m_al_{c_a}^2 + m_al_ul_{c_a}\cos(q_a) + J_a &
      m_al_{c_a}^2 + J_a
    \end{bmatrix} 
    , \\
    \label{eqn:general-acrobot-potential}
    P(q) &= g\left(m_al_{c_a}(1 - \cos(q_u+q_a)) + 
        (m_al_u + m_ul_{c_u})(1-\cos(q_u))\right) 
    , \\
    B(q) &= \begin{bmatrix} 0 \\ 1 \end{bmatrix}
    .
\end{align}

While this is the most general representation of an acrobot, the dynamics
become unwieldy.
To make rigorous analysis of these dynamics more tractable, we begin by assuming
the acrobot is comprised of two massless rods of equal length \(l\), with equal
point masses \(m\) at the tips.
We call this a \textit{simple} acrobot, which is displayed in Figure
\ref{fig:simple-acrobot-model}.
We will also ignore any frictional forces at both the hip and shoulder joints. 
Finally, it is important to note that a real gymnast cannot swing their legs in
full circles, though they are usually flexible enough to raise them parallel to
the floor; 
for this reason, we assume that \(q_a \in [-Q_a, Q_a]\) where 
\(Q_a \in [\frac{\pi}{2}, \pi[\). 

\begin{figure}
    \centering
    \includestandalone[width=0.7\textwidth]{images/simple_acrobot_model}
    \caption{A simple acrobot has massless rods of equal length \(l\) and 
    equal masses \(m\) at the tips.}
    \label{fig:simple-acrobot-model}
\end{figure}

Since we are now working with a simple acrobot, 
we have \(l_{c_u} = l_{c_a} = l_u = l_a = l\) and
\(m_u = m_a = m\). 
On top of this, the moments of inertia \(J_u\) and \(J_a\) of the rods vanish.
Reducing
(\ref{eqn:general-acrobot-inertia}-\ref{eqn:general-acrobot-potential})
yields the simplified inertia matrix \(D_s\) and potential function \(P_s\),
where
\begin{align}
    D_s(q) &= \begin{bmatrix}
        ml^2\left(3+2\cos(q_a)\right) & 
        ml^2\left(1+\cos(q_a)\right) \\
        ml^2\left(1+\cos(q_a)\right) &
        ml^2
    \end{bmatrix} 
    , \\
    P_s(q) &= -mgl\left(2\cos(q_u)+\cos(q_u+q_a)\right)
    .
\end{align}
\begin{notation}
    For shorthand, we write \(c_u := \cos(q_u)\), \(c_a := \cos(q_a)\), and 
    \(c_{ua} := \cos(q_u + q_a)\). Likewise, \(s_u := \sin(q_u)\), 
    \(s_a := \sin(q_a)\), and \(s_{ua} := \sin(q_u + q_a)\).
\end{notation}

Defining \(M(q) := D_s(q)\) and \(V(q) := P_s(q)\), we find the conjugate of momenta 
is \(p = (p_u, p_a) = M(q)\dot{q}\).
The dynamics in \((q,p)\) coordinates are given by
\begin{align}\label{eqn:acrobot-hamiltonian}
    \mathcal{H}(q,p) &= \frac{1}{2}p\tpose \Minv(q) p -
    mgl\left(2 c_u + c_{ua}\right)
    , \\
     &\begin{cases}
        \dot{q} = \Minv(q) p \\
        \dot{p}_u = -mgl\left(2s_u + s_{ua}\right) \\
        \dot{p}_a =-\frac{1}{2}p\tpose \nabla_{q_a}\Minv(q) p
        - mgl s_{ua} + \tau,
    \end{cases} \nonumber
\end{align}
where the inverse inertia matrix is
\begin{equation}\label{eqn:Minv}
    \Minv(q) = \frac{1}{ml^2\left(2-c_a^2\right)}
    \begin{bmatrix}
        1 &
        -\left(1+c_a\right) \\
        -\left(1+c_a\right) &
        3+2c_a
    \end{bmatrix}
    .
\end{equation}
The control input is a force \(\tau \in \R\) affecting only the dynamics of
\(p_a\), representing a torque acting on the hip joint.
This means \((q,p)\) are simply actuated coordinates inside the phase space
\(\mathcal{Q} \times \mathcal{P}\) where
\[
    \mathcal{Q} = \mathcal{Q}_u \times \mathcal{Q}_a := 
    \mathbb{S}^1 \times \mathbb{S}^1
    ,
\]
and
\[
    \mathcal{P} = \mathcal{P}_u \times \mathcal{P}_a
    := \R \times \R
    .
\]
This allows us once again to apply the theory of VNHCs from Chapter
\ref{ch:vnhcs}.

Let us define the VNHC \(h(q,p) = q_a - f(q_u,p_u)\) of order 1, where
\(f \in C^2\left(\mathcal{Q}_u \times \mathcal{P}_u; \mathcal{Q}_a\right)\).
Since \(\nabla_{q_u}\Minv(q) = \Zmat{2\times 2}\),
Theorem \ref{thm:vnhc-regularity} tells us that this VNHC will be regular
if the regularity matrix
\[
    dh_q \Minv(q) \begin{bmatrix}0 \\ 1 \end{bmatrix}
    ,
\]
is of full rank on the constraint manifold \(\Gamma\).
Then, since 
\[
    dh_q = \begin{bmatrix}
        -\partial_{q_u} f & 1
    \end{bmatrix}
    ,
\]
the regularity matrix evaluates to the scalar equation
\begin{equation}\label{eqn:regularity-matrix-acrobot}
    \frac{(1+c_a)\partial_{q_u}f(q_u,p_u) + (3+2c_a)}{ml^2(2-c_a^2)}
    .
\end{equation}
This is full rank on \(\Gamma\) if and only if the numerator does not
change sign.

A sufficient condition for regularity is when \(f\) is a function solely of \(p_u\),
because then \(\partial_{q_u}f = 0\) and (\ref{eqn:regularity-matrix-acrobot})
is strictly positive for all values of \(q_a\). 
This will be useful later, as we will not need to check regularity if we design
a function of the unactuated momentum. 

The acrobot is noticeably more complex than the VLP, as
the dynamics of \((q_u,p_u)\) and \((q_a,p_a)\) are coupled through \(\Minv(q)\).
Because of this, the constrained dynamics of an arbitrary VNHC may not be easy
to write out.
In the rest of this chapter, our goal is to design the function \(f(q_u,p_u)\)
based on the natural human motion of a gymnast, with one caveat: 
we must be able to prove the constrained dynamics will inject energy into the
acrobot.

\section{Previous Constraint Approaches}
% TODO: Give Xingbo's intuition, abstract away to qa = sin(theta) constraint
% based on VLP but show it doesn't work well in our framework
% TODO: Describe Xingbo's results and observe that his results were similar to a VNHC
% except they used VHC tools. We will modify his approach to use VNHCs and prove
% rigorously that we can stabilize any energy level set on the acrobot.
Let us examine some of the existing approaches to generating giant motion for
the acrobot, since these may be viable candidates on which to base a VNHC.

One initial approach to controlling the acrobot is to model it as a
variable-length pendulum by collapsing the two rods and masses into one
equivalent center of mass (ECM), as in Figure \ref{fig:acrobot-ecm}.
This seems a reasonable model reduction, since the length from the pivot to the
ECM changes depending on the angle \(q_a\) of the leg.
Indeed, \citet{swingup_giant_acrobot} use this approach to design a trajectory
for the ECM, then determine which leg angles \(q_a(t)\) are required to generate
that trajectory.
Following in their footsteps, we might consider using the results
from Chapter \ref{ch:vlp} to find the leg angles that allow the ECM to gain
energy. 
Then we could apply Theorem \ref{thm:vlp-energy-stabilization} to prove
the acrobot is gaining energy too.
\begin{figure}
    \centering
    \includestandalone[width=0.5\textwidth]{images/acrobot_ecm}
    \caption{A simple acrobot modelled as a VLP with equivalent center of mass \(2m\). 
        The length of the VLP changes according to \(q_a\).}
    \label{fig:acrobot-ecm}
\end{figure}

Unfortunately, the VLP is not a true representation of the acrobot.
The effective length of the ECM is 
\[
    l_e(q_a) := l\sqrt{\frac{5}{4} + c_a}
    ,
\]
and its effective angle is
\[
    q_e := \arctan_2\left(s_u + \frac{1}{2}s_{ua}, -c_u - \frac{1}{2}c_{ua}\right)
    .
\]
There are two important notes to consider based on these equations. 
First, Figure \ref{fig:acrobot-vlp-symmetry} shows that for each pose of the
VLP representation, there are two configurations
of the acrobot which give the same effective length and angle.
This means the acrobot and the VLP are not equivalent representations;
designing a VNHC that injects energy using the ECM may not produce human-like
leg motion on the acrobot.

Second, if we were to compute the conjugate of momenta
\(p_{l_e}\) to \(l_e\) and \(p_e\) to \(q_e\), we would see the torque input
\(\tau\) appearing in both of their dynamic equations.
In the VLP model from Chapter \ref{ch:vlp}, the control input only
affects the dynamics of the length variable.
If we want to design a VNHC for this system, we cannot use any of the results
from Chapter \ref{ch:vlp} because the VLP models don't match.

\begin{figure}
    \centering
    \includestandalone[width=0.3\textwidth]{images/acrobot_vlp_symmetry}
    \caption{The equivalent center of mass of the acrobot generally has two configurations
        which correspond to the same effective length and angle. These
        configurations are symmetric about the line connecting the pivot to the
        ECM.}
    \label{fig:acrobot-vlp-symmetry}
\end{figure}

Since we cannot apply the results of Chapter \ref{ch:vlp} to simplify the proof
of energy injection, and the resulting ECM motion may not even produce
realistic leg motion, this model reduction is ineffective for our purposes. 

Let us turn next to the thesis of \citet{xingbo_thesis}, who designs a VHC to enforce a
so-called ``tap" motion with the purpose of injecting energy into the acrobot. 
First, he defines a compensator variable \(s\) which tracks \(\dot{q}_u\), so
that he can use the theory of VHCs with the extended configuration 
\((q_u,q_a,s)\).
He then finds \(h_1, h_2 \in \R_{>0}\) to define the
normalized radius \(\rho\) and normalized angle \(\xi\) in the
\((q_u, s)\)-plane.
These normalized variables are given by
\begin{align*}
    \rho &:= \sqrt{h_1 q_u^2 + h_2 s^2}
    , \\
    \xi &:= \arctan_2(h_2 s, h_1 q_u)
    . 
\end{align*}
He then sets the VHC to be \(h(q) = q_a - f_\text{rad}(\rho)f_\text{ang}(\xi)\)
with the control parameters \(\bar{q}_u\) and \(\rho_0\), where
\begin{align}
    \label{eqn:xingbo-frad}
    f_\text{rad}(\rho) &:= \tanh^2(\rho/\rho_0)
    , \\
    \label{eqn:xingbo-fang}
    f_\text{ang}(\xi) &:= 
    \begin{cases}
        0 & -\pi < \xi \leq 0 \\
        \bar{q}_u \exp\left(1 - \frac{1}{1-(\frac{4\xi}{\pi} - 1)^2}\right) 
          & 0 < \xi \leq \frac{\pi}{2} \\
        0 & \frac{\pi}{2} < \xi \leq \pi
        .
    \end{cases}
\end{align}

While this constraint shows promising experimental results and it accurately
emulates true human motion, \citeauthor{xingbo_thesis}
does not provide an analytical proof that the acrobot will gain energy.
His lack of analysis is tied to the fact that the constrained
dynamics are incredibly complicated.
In fact, just showing the constraint is regular is a challenging task.
While we could very easily convert his VHC into a VNHC by replacing \(s\) with
\(p_u\), we would run into the same problem. 
Since we want our constraint to \textit{provably} inject energy, we must forgo
this type of constraint in favour of something less complex.

\section{The Acrobot Constraint}
% TODO: Give the constraint, state the theorem, and show the outline of the
% proof (both as a sketch and in a figure)
One may be tempted to design a constraint of the form \(q_a = \bar{q}_a\sin(\theta)\) 
with \(\theta := \arctan_2(p_u,q_u)\), since a similar approach 
was so effective for the VLP in Chapter \ref{ch:vlp}.
Unfortunately, this constraint is not regular, and it is difficult to find any
VNHC of the form \(q_a = f(\theta)\) where it is easy to prove regularity
analytically.
Instead, we will develop a constraint \(h(q,p) = q_a - f(p_u)\) which,
as we saw earlier, is always regular and is likely to inject energy.
 
To design this constraint, let us begin by examining a person on a seated swing.
The person extends their legs when the swing moves forwards, and retracts their
legs when the swing is moving backwards.
As the swing gains speed, the person leans their body back while
extending their legs.
This allows them to bring their legs higher, shortening the distance
from their center of mass to the pivot and adding more energy to the swing.
When the swing moves backward, they sit up and fully retract their legs
underneath them \cite{how_to_pump_a_swing}.

Now imagine the person's torso is affixed to the swing's rope so that they are
always upright. 
Imagine further that the swing has no seat at all, allowing the person to extend
their legs beneath them. 
This position is identical to that of a gymnast on a bar, which is why we can
use the leg motion from seated swinging to design a controller for the acrobot.

The acrobot's legs are rigid rods which cannot retract, so we emulate the person
on a swing by pivoting the legs in the direction of motion. 
To account for how a person leans back at higher speeds, the legs should pivot to an
angle proportional to the swing's speed.
One VNHC which emulates this process is \(q_a = \bar{q}_a\arctan_2( I p_u)\), as
in Figure \ref{fig:qa-arctan}.
Here, \(\bar{q}_a \in ]0,\frac{2 Q_a}{\pi}]\)
and \(I \in \R_{>0}\) is a fixed control parameter.

This constraint does not perfectly recreate a gymnast's motion -- during
a giant the gymnast's legs are almost completely extended, while this constraint
pivots the legs partially during rotations.
However, the motion looks similar enough that this constraint should provide a
decent foundation for injecting energy into the acrobot.
It is for this reason that we choose our acrobot's constraint to be
\begin{equation}\label{eqn:acrobot-constraint}
    h(q,p) = q_a - \bar{q}_a \arctan_2(I p_u)
    .
\end{equation}

\begin{figure}
    \centering
    \includestandalone[width=0.5\textwidth]{images/qa_arctan}
    \caption{The acrobot constraint \(q_a = \bar{q}_a \arctan_2(I p_u)\).}
    \label{fig:qa-arctan}
\end{figure}

Let us now compute the constrained dynamics under
(\ref{eqn:acrobot-constraint}).
Note that \(dh_q = \begin{bmatrix}0 & 1\end{bmatrix}\), while
\[
    dh_{p_u} = \frac{-\bar{q}_a I}{1 + I^2 p_u^2}
    .
\]
Inserting these to (\ref{eqn:g-qupu}), we get the solution for \(p_a\) on the
constraint manifold:
\[
    p_a(q_u,p_u) = \frac{
        (1+c_a)(1+I^2 p_u^2)p_u - m^2gl^3\bar{q}_a I (2-c_a^2)(2s_u + s_{ua})
    }{ml^2(3+2c_a)(1+I^2 p_u^2)}
    .
\]
The dynamics for \(p_u\) do not contain \(p_a\), so they remain unchanged.
The constrained dynamics for \(q_u\) are given by 
\begin{equation*}
    \dot{q}_u = e_1\tpose \Minv(q) \begin{bmatrix}
                    p_u \\ p_a(q_u,p_u)
                \end{bmatrix} %\\
    ,
\end{equation*}
%        &= \frac{1}{ml^2(2-c_a^2)} 
%            \left( p_u - 
%                (1+c_a)\frac{(1+c_a)(1+I^2 p_u^2)p_u - m^2gl^3\bar{q}_a I (2-c_a^2)(2s_u + s_{ua})
%                }{ml^2(3+2c_a)(1+I^2 p_u^2)}
%            \right) 
%        ,
%\end{align*}
which can be simplified to 
\begin{equation*}
    \dot{q_u} = \frac{(1+I^2 p_u^2)p_u - m^2gl^3\bar{q}_a I(2s_u + s_{ua})(1+c_a) }{ml^2(1+I^2 p_u^2)(3+2c_a)}
    .
\end{equation*}
Hence, the constrained dynamics for the acrobot under
(\ref{eqn:acrobot-constraint}) are
\begin{equation}\label{eqn:acrobot-constrained-dynamics}
    \begin{cases}
        \dot{q}_u = \frac{(1+I^2 p_u^2)p_u - m^2gl^3\bar{q}_a I(2s_u + s_{ua})(1+c_a) }
            {ml^2(1+I^2 p_u^2)(3+2c_a)}
        , \\
        \dot{p}_u = - m g l (2s_u + s_{ua})
        .
    \end{cases}
\end{equation}

These dynamics do not always gain energy -- we need additional conditions
on the control parameters to guarantee this is true.

% TODO: This needs modification. The energy injection is provable for any system
% only during oscillations. For rotations, we need to do a numerical check of an
% integral that depends on m,g,l. If this integral is positive, then the system
% is gaining energy up to a point - it sems the integral decays to 0 so that you
% can reach a limiting speed, which means you don't get full energy injection
% everywhere on Gamma.
\begin{thm}\label{thm:acrobot-energy-stabilization}
    Consider the simple acrobot (\ref{eqn:acrobot-hamiltonian}) constrained by
    (\ref{eqn:acrobot-constraint}) with \(\bar{q}_a \in ]0,1]\).
    There exists a small enough\footnotemark \(I^\star \in \R_{>0}\)
    such that the VNHC injects energy whenever \( I \in ]0,I^\star[\). 
    If \(I \in ]-I^\star,0[\), the VNHC dissipates energy.
\end{thm}
\footnotetext{In fact, we can be more precise than this. 
    Letting \(\epsilon > 0\), we need 
    \(I \leq \min\left\{I_1(\epsilon),I_2\right\}\) where
    \(I_1(\epsilon) = \frac{\tan(\epsilon)}{(\pi-\epsilon)}
    \left(\frac{m^2gl^3(\cos(\epsilon)-2\sin(\epsilon))}{\pi-\epsilon}
    \left(\frac{2\bar{q}_a\tan(\epsilon)}{(1+\tan(\epsilon)^2)(\pi-\epsilon)}+5\right) 
    \right)^{-1/2}\)
    and \(I_2 = \sqrt{\frac{3}{10 m^2gl^3\bar{q}_a^2}}\).
    See Apendix \ref{apx:I-bounds} for details.
}

Unfortunately, the energy-based proof used in Chapter \ref{ch:vlp} does not
readily transfer to the acrobot -- there is no obvious Lyapunov function for
this constrained system.
Proving Theorem \ref{thm:acrobot-energy-stabilization} requires an intelligent
change of coordinates and the use of perturbation theory from
\cite{khalil_nonlinear}.
The full proof is given in Chapter \ref{sec:acrobot-proof}, so 
we conclude this section with a high level sketch of the proof.

There is a suitable change of coordinates into a pseudo-radius 
\(\mu \in \R_{>0}\) and pseudo-angle \(\alpha \in \mathbb{S}^1\) on the
\((q_u,p_u)\)-plane where \(\dot{\mu} = 0\) and \(\dot{\alpha} > 0\) when 
we set \(I = 0\).
Perturbation theory shows that, for small enough \(I\), the radius \(\mu\)
increases on average along \(\alpha\).
Hence, the acrobot is gaining energy since orbits in the \((q_u,p_u)\)-plane
spiral away from the origin, and the acrobot always begins performing rotations.
 
\section{Proving the Acrobot Gains Energy}\label{sec:acrobot-proof}
% TODO: Prove this in sections
% TODO: You don't have to use the proof of theta_dot being positive. Stick to
% Manfredi's proof, and when you say "for I small enough" put in a footnote that
% says "in fact, we can be more precise and set bounds for I given as I <
% min{...}. See appendix for more detail."
% Then, we can have an appendix where we derive the bounds on I so that
% theta_dot is always positive. This will help the reader keep track and not
% bother them too much.
Proving Theorem \ref{thm:acrobot-energy-stabilization} is not a simple task.
In an effort to make the proof as clear as possible, we will break it down into
the following segments:
\begin{enumerate}
    \item Background on perturbation theory and averaging.
    \item Perturbation analysis for oscillations.
    \item Perturbation analysis for rotations.
\end{enumerate}

When \(I = 0\), the constrained acrobot behaves like a single
pendulum with masses at a distance \(l\) and \(2l\) from the pivot 
(Figure \ref{fig:acrobot-I0}) whose energy
\begin{equation}\label{eqn:acrobot-nominal-E}
    E(q_u,p_u) = \frac{p_u^2}{10ml^2} + 3(1 - \cos(q_u))
    ,
\end{equation}
is conserved.
Level sets of \(E\) are ellipses when \(E < E(\pi,0)\), which we call
``oscillations"; and they are open curves when \(E > E(\pi,0)\), which we call
``rotations". 
Examples of these can be seen in Figure \ref{fig:pendulum-level-sets}.

\begin{figure}
    \centering
    \begin{subfigure}[t]{0.45\textwidth}
        % TODO: figure of acrobot with I = 0
        %\includestandalone[]{images/acrobot_I0}
        \caption{The simple pendulum with two masses.}
        \label{fig:acrobot-I0}
    \end{subfigure}
    \hfill
    \begin{subfigure}[t]{0.45\textwidth}
        % TODO: figure showing oscillations and rotations as level curves of E
        %\includestandalone[]{images/pendulum_level_sets}
        \caption{Level sets of the pendulum energy function.
            The red ellipse represents oscillations while the purple curve
            represents rotations.}
        \label{fig:pendulum-level-sets}
    \end{subfigure}
    \caption{Our constrained acrobot is a simple pendulum when \(I = 0\).}
\end{figure}

Drawing inspiration from \citet{dynamic_vhcs_stabilize_closed_orbits}, we want
to find a change of coordinates \((q_u,p_u) \to (\alpha, \mu)\) with the
following properties:
\begin{itemize}
    \item All oscillations in \((\alpha,\mu)\) coordinates are represented by
        \(\mu = \text{const}\).
    \item \(\alpha\) is a pseudo-angle living in \(\mathbb{S}^1\) which is
        always increasing.
\end{itemize}
We also want to find a second set of coordinates where the same properties hold for
rotations.
Once we have these coordinates in hand, we can use perturbation theory to prove
that \(\mu\) increases on average. 
If this is true, the acrobot must be gaining energy since the momentum is
increasing over time.

\subsection*{Background on Perturbation Theory}
Perturbation theory allows one to understand how solutions of nonlinear systems
behave by studying a nominal system which can be analyzed analytically.
The solutions of nonlinear systems are approximated by a Taylor-like expansion
around this nominal system.

\citet{khalil_nonlinear} considers a system of the form
\begin{equation}\label{eqn:khalil-setup}
    \begin{cases}
        \dot{x} = f(t,x,\epsilon), \\
        x(t_0) = \eta(\epsilon),
    \end{cases}
\end{equation}
where 
\(f : [t_0,t_1] \times D \times [-\epsilon_0,\epsilon_0] \rightarrow \R^n\) is
sufficiently smooth.
In the context of the acrobot, \(x\) is the phase in whichever coordinates make
our analysis convenient;
\(\epsilon\) is our control parameter \(I\);
the function \(f\) is the constrained dynamics in \(x\)-coordinates;
and \(\eta(\epsilon) \equiv \eta_0\) is a constant initial condition.
To keep a notational reminder that we are applying perturbation theory to
the acrobot, we will henceforth use \(I\) in place of \(\epsilon\).

Letting \(I = 0\) we get the nominal system
\begin{equation}\label{eqn:khalil-perturbation-nominal}
    \begin{cases}
        \dot{x}_0 = f(t,x,0) ,\\
        x_0(t_0) = \eta_0 ,
    \end{cases}
\end{equation}
which we need to solve for the explicit solution \(x_0(t)\) on \([t_0,t_1]\).
One can use \(x_0(t)\) to compute a first-order approximation of
the nonlinear system
\begin{equation}\label{eqn:khalil-perturbation-firstorder}
    \begin{cases}
        \dot{x}_1 = \pdiff{f}{x}(t,x_0(t),0)x_1 + \pdiff{f}{I}(t,x_0(t),0)
        , \\
        x_1(t_0) = 0, \\
    \end{cases}
\end{equation}
which is a linear time-varying system with solution \(x_1(t)\).

We now paraphrase Khalil's Theorem 10.1 \cite{khalil_nonlinear} on the
accuracy of perturbation analysis.
\begin{thm}
    Suppose \(f : [t_0,t_1] \times D \times [-I_0,I_0] \rightarrow \R^2\) is
    \(C^1\), and that the nominal problem (\ref{eqn:khalil-perturbation-nominal}) has a
    unique solution \(x_0(t) \in D\) on \([t_0,t_1]\).
    Then there exists \(I^\star > 0\) such that for all \(|I| < I^\star\), the
    solution \(x(t,I)\) to (\ref{eqn:khalil-setup}) satisfies
    \[
        \norm{x(t,I) - \left(x_0(t) + I x_1(t)\right)} \leq k |I^2|
    \]
    for some \(k > 0\).
\end{thm}

The above Theorem tells us that we can approximate the solution of the nonlinear
system by the Taylor approximation \(x_0(t) + I x_1(t)\).
In other words, when \(I\) is small enough solutions of the nonlinear system and
the Taylor approximation are the same up to order \(I^2\) along compact time
intervals.
Since we know that the acrobot behaves like a pendulum at \(I = 0\), we can
use this theory to more easily study the behaviour of orbits and prove energy
injection properties of our VNHC.
 
\subsection*{Perturbation Analysis for Oscillations}
The nominal pendulum (\ie when \(I = 0\)) oscillates whenever the
nominal energy (\ref{eqn:acrobot-nominal-E}) is less than \(E(\pi,0)\).
Hence, the domain of oscillations on \(\Gamma\) is given by
\[
    D := \left\{ (q_u,p_u) \in \Gamma | E(q_u,p_u) < E(\pi,0)\right\}
    .
\]
We wish to find a transformation \(T(q_u,p_u)\) on \(D\) into
\((\alpha,\mu)\)-coordinates where \(\alpha \in \mathbb{S}^1\) is a pseudo-angle
and \(\mu > 0\) is a pseudo-radius.
Since the energy of oscillation can be determined by the orbit's intersection
with the \(q_u\)-axis and \(q_u \in ]-\pi,\pi[\) on \(D\), 
we set \(\mu \in ]0,\pi[\) as in Figure \ref{fig:mu-intersection}.
The transformation we want is therefore a diffeomorphism of the form
\begin{align*}
    T : D\backslash \{(0,0)\} &\rightarrow \mathbb{S}^1 \times \, ]0,\pi], \\
    (q_u, p_u) &\mapsto (\alpha,\mu)
    .
\end{align*}

\begin{figure}
    \centering
    % TODO: Add a figure with one oscillation mark at a distance mu from
    % the origin to the intersection on the q-axis. We also draw the boundary of
    % D and shade the interior
    %\includestandalone[]{images/mu_intersection}
    \caption{The domain \(D\) where a pendulum oscillates.
    The pseudo-radius coordinate \(\mu\) corresponds to the
    intersection of an orbit of oscillation with the \(q_u\)-axis.}
    \label{fig:mu-intersection}
\end{figure}

The energy level set corresponding to the intersection \((q_u,p_u) = (\mu,0)\)
is 
\[
    \left\{(q_u,p_u) \in \Gamma \mid E(q_u,p_u) = 3mgl(1- \cos(\mu))\right\}
    ,
\]
which gives the relationship
\begin{equation}\label{eqn:oscillation-pu2}
    p_u^2 = 30m^2gl^3(\cos(q_u) - \cos(\mu))
    .
\end{equation}
On this level set, \(q_u\) ranges between \([-\mu,\mu]\) and can be uniquely
parameterized by \(q_u = \mu \cos(\alpha)\), where \(\alpha\) is our
pseudo-angle.
Substituting this into (\ref{eqn:oscillation-pu2}), we get
\[
    p_u^2 = 30m^2gl^3(\cos(\mu \cos(\alpha)) - \cos(\mu))
    .
\]
We want to find \(p_u\) as a function of \((\alpha,\mu)\); noting that we can
determine the sign of \(p_u\) from the sign of \(\sin(\alpha)\), we get the
(clockwise) parameterization
\begin{equation}\label{eqn:oscillation-pu}
    p_u = -\sign{\sin(\alpha)} \sqrt{30m^2gl^3 \left(\cos(\mu c_\alpha) - c_\mu\right)}
    ,
\end{equation}
which is, in fact, smooth.

We have thus found a transformation \(T\inv(\alpha,\mu) = (q_u,p_u)\).
We need the inverse of this map to get our diffeomorphism \(T(q_u,p_u)\).
Notice from (\ref{eqn:oscillation-pu2}) that
\[
    \cos(\mu) = -\frac{p_u^2}{30m^2gl^3} + \cos(q_u) =: C_\mu(q_u,p_u)
    .
\]
Since \(\mu \in ]0,\pi]\), we can uniquely express \(\mu\) by
\begin{equation}\label{eqn:mu-qupu}
    \mu = \arccos\left(C_\mu(q_u,p_u)\right)
    .
\end{equation}
Next we need to find \(\alpha\). 
Recall that 
\[
    \cos(\alpha) = \frac{q_u}{\mu}
    ,
\] 
which means 
\[
    \sin(\alpha) = \pm \sqrt{1 - \frac{q_u^2}{\mu^2}}
    .
\]
Notice from (\ref{eqn:oscillation-pu}) that 
\(\sign{\sin(\alpha)} = -\sign{p_u}\).
We use these facts to deduce that
\begin{equation}\label{eqn:alpha-qupu}
    \alpha = \left.
        \arctan_2\left( -\sign{p_u}\sqrt{1 - \frac{q_u^2}{\mu^2}}, \frac{q_u}{\mu}\right)
        \right|_{\mu = \arccos(C_\mu(q_u,p_u))}
    ,
\end{equation}
and thus have our transformation \(T(q_u,p_u)\) into
\((\alpha,\mu)\)-coordinates on \(D\).

The acrobot's constrained dynamics in
\((\alpha,\mu)\)-coordinates can be computed by evaluating
\[
    \left.\begin{cases}
        \dot{\mu} = 
        \begin{bmatrix} \pdiff{\mu}{q_u} & \pdiff{\mu}{p_u} \end{bmatrix}
        \begin{bmatrix} \dot{q}_u \\ \dot{p}_u \end{bmatrix}
        , \\
        \dot{\alpha} = \begin{bmatrix} \pdiff{\alpha}{q_u} & \pdiff{\alpha}{p_u} \end{bmatrix}
        \begin{bmatrix} \dot{q}_u \\ \dot{p}_u \end{bmatrix}
        .
    \end{cases} \right|_{(q_u,p_u) = T\inv(\mu,\alpha)}
\]
These dynamics are incredibly complicated and difficult to express, so we will
write them simply as
\[
    \begin{cases}
        \dot{\mu} = f_\mu(\alpha,\mu,I)
        ,\\
        \dot{\alpha} = f_\alpha(\alpha,\mu,I)
        .
    \end{cases}
\]

To find the nominal system we set \(I = 0\); 
MATLAB's symbolic toolbox operations tell us that
\begin{equation}\label{eqn:acrobot-nominal-ma-dynamics}
    \begin{cases}
        \dot{\mu} = 0
        , \\
        \dot{alpha} = \sqrt{\frac{6g}{5l}} 
        \sqrt{\frac{\cos(\mu\cos(\alpha)) - \cos(\mu)}{\mu^2 \sin(\alpha)^2}}
        .
    \end{cases}
\end{equation}

\subsection*{Perturbation Analysis for Rotations}

\section{Experimental Results}
% TODO: This might be a separate chapter

%/========== /Acrobot ==========/%
% vim: set tw=80 ts=4 sw=4 sts=0 et ffs=unix :


% Concluding remarks and future work
%! TEX root = main.tex

%/========== Conclusion ==========/%

\chapter{Conclusion}\label{ch:conclusion}
% TODO: Summarize the work

\section{Limitations of this Work}


\section{Future Research}


%/========== /Conclusion ==========/%
% vim: set ts=3 sw=3 sts=0 et ffs=unix :


% Appendices
\appendix
%! TEX root = main.tex

%/========== Appendix: Bounding I =========/%
\chapter{Bounding the Acrobot Control Parameter}\label{apx:I-bounds}
% TODO: Prove the bounds on I using theta_dot 

% vim: set tw=80 ts=4 sw=4 sts=0 et ffs=unix :


%---------- Bibliography ----------%
%\newpage
% This adds a line for the Bibliography in the Table of Contents.
\addcontentsline{toc}{chapter}{Bibliography}
\bibliography{bib}
\end{document}
%/========== /Main Document ==========/%

% vim: set tw=80 ts=4 sw=4 sts=0 et ffs=unix :
