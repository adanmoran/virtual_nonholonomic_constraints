%! TEX root = main.tex

%/========== Virtual Holonomic Constraints ==========/%
\section{Virtual Holonomic Constraints}\label{sec:vhcs}

\subsection{Lagrangian Framework Review}%
\label{sub:lagrangian_vhcs}

\begin{defn}\label{defn:vhc_order_k}
   A \textbf{virtual holonomic constraint (VHC) of order k} is a smooth relation 
   \( h : \mathcal{Q} \rightarrow \mathbb{R}^k\) where
   \( \text{rank}(dh_q) = k \, \forall q \in h^{-1}(0)\) and where there exists
   a smooth feedback controller \(\tau(q,\dot{q})\) such that the constraint
   manifold
   \[
      \Gamma = \left\{ (q,\dot{q}) \in T_q\mathcal{Q} \vert h(q) = 0, \, 
      dh_q \dot{q} = 0\right\}
   \]
   is positively invariant.
\end{defn}

\begin{defn}\label{defn:vhc_regular}
   A VHC is \textbf{regular} if system \textbf{TODO:
   reference to E-L mechanical system} with output \(e = h(q)\) has vector
   relative degree \(\{2,2,\ldots,2\}\).
   Equivalently, a VHC is regular if 
   \[
      dh_q D^{-1}(q)B(q)
   \]
   is nonsingular everywhere on \(\Gamma\).
\end{defn}

%/========== /Virtual Holonomic Constraints ==========/%
% vim: set tw=80 ts=3 sw=3 sts=0 et ffs=unix :
