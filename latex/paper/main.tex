%! TEX program = lualatex

\documentclass[journal,twoside,web]{ieeecolor}
\usepackage{generic}
%/========== Preamble ==========/%
\usepackage{cite}
\usepackage{amsmath,amssymb,amsfonts}
\usepackage{algorithmic}
\usepackage{graphicx}
\usepackage{textcomp}

%---------- Bibliography Style ----------%
\bibliographystyle{IEEEtran}

%---------- Header ----------% 
\newcommand*{\Title}{VNHC and Acrobot Title}
\markboth{\journalname, VOL. XX, NO. XX, XXXX 2021}
{Moran-MacDonald \MakeLowercase{\textit{et al.}}: \Title}

%---------- Special Commands ----------%
\newcommand*{\tpose}{^\mathsf{T}} 
\newcommand*{\inv}{^\mathsf{-1}}
\newcommand*{\Rt}[1]{[\R]_{#1}}
\newcommand*{\R}{\mathbb{R}}
\newcommand*{\n}{\mathbf{n}}
\newcommand*{\N}{\mathbb{N}}
\newcommand*{\Sone}{\mathbb{S}^1}
\newcommand*{\SxR}{\Sone \times \R}
\newcommand*{\Minv}{M^\mathsf{-1}}
\newcommand*{\Id}[1]{I_{#1}}
\newcommand*{\Zmat}[1]{\bm{0}_{#1}}
\newcommand*{\diff}[2]{\frac{d #1}{d #2}}
\newcommand*{\ddiff}[3]{\frac{d^2 #1}{d #2 d #3}}
\newcommand*{\pdiff}[2]{\frac{\partial #1}{\partial #2}}
%/========== /Preamble ==========/%

%/========== Main Document ==========/%
\begin{document}
\title{\Title}
\author{Adan Moran-MacDonald, \IEEEmembership{Member, IEEE}, Manfredi Maggiore
\IEEEmembership{Member, IEEE}*, and Xingbo Wang
\thanks{Manuscript submitted for review on \today.}
\thanks{A. Moran-MacDonald is with the Department of Electrical and Computer
    Engineering, University of Toronto, Toronto, ON, Canada (e-mail:
adan.moran@mail.utoronto.ca).}
\thanks{M. Maggiore is with the Department of Electrical and Computer
Engineering, University of Toronto, ON, Canada (e-mail:
maggiore@control.utoronto.ca).}
\thanks{X. Wang is with ??? (e-mail: ???).}
} %/author

\maketitle

%/========== Abstract ==========/%
\begin{abstract}
TODO: Abstract here.
\end{abstract}

\begin{IEEEkeywords}
TODO: keywords in alphabetical order, separated by commas.
\end{IEEEkeywords}

%/========== Introduction ==========/%
\section{Introduction}\label{sec:introduction}
\IEEEPARstart{T}ODO: intro.

\subsection{Notation}
We use the following notation in this paper. 
For \(T > 0\), the set of real numbers modulo is denote \(\Rt{T}\), with
\(\Rt{\infty} := \R\).
The gradient of the matrix-valued function 
\(A : \R^m \rightarrow \R^{n\times n}\) is the matrix of stacked partial
derivatives, 
\(\nabla_xA := (\pdiff{A}{x_1},\ldots,\pdiff{A}{x_m})\tpose \in \R^{nm \times n}\).
Given two matrices \(A \in \R^{n \times m}\) and \(B \in \R^{r \times s}\), the
Kronecker product \cite{kronprod} \(A \otimes B \in \R^{nr \times ms}\) is the
matrix
\[
    A \otimes B = \begin{bmatrix}
        A_{1,1}B & \cdots & A_{1,m} B \\
        \vdots & \ddots & \vdots \\
        A_{n,1} B & \cdots & A_{n,m} B
    \end{bmatrix} 
    .
\]

%/========== Problem Formulation ==========/%
\section{Problem Formulation}\label{sec:problem-formulation}
TODO: motivate injection/dissipation for energy regulation in an acrobot. Use
this to lead into VNHCs. e.g. represent torso as chain of a swing, pivot as
seat, and legs as human legs. Replicate fig3.3 from thesis but for acrobot: take
(qu,pu)-plane and show how a person moves their legs wrt theta, qu, or pu. Leave
the gymnast model until the end, or ignore it entirely.

Alternatively, ff human movement journal says legs move wrt qu, we can use that
as a foundation.

%/========== VNHC ==========/%
\section{Theory of VNHCs}\label{sec:vnhc}
TODO: Just like in the thesis, cover the theory of VNHCs. Justify our results compared
to those of Horn. Make it terse but legible. Cover simply actuated systems,
vnhcs, regularity, the theorems, etc.


\subsection{Simply Actuated Hamiltonian Systems}
Take an underactuated mechanical system modelled with generalized coordinates 
\(q = (q_1, \ldots, q_n)\) on the configuration manifold
\(\mathcal{Q} = \Rt{T_1} \times \cdot \Rt{T_n}\), where
\(T_i = 2\pi\) if \(q_i\) is an angle and \(T_i = \infty\) if \(q_i\) is a
length. The corresponding velocities are 
\(\dot{q} = (\dot{q}_1,\ldots,\dot{q}_n) \in \R^n\).

Suppose this system has Lagrangian
\(\mathcal{L}(q,\dot{q}) = 1/2 \dot{q}^T D(q) \dot{q} - P(q)\),
where \(P : \mathcal{Q} \rightarrow \mathbb{R}\) is the potential energy, and
the inertia matrix \(D : \mathcal{Q} \rightarrow \mathbb{R}^{n \times n}\)
is positive definite for all \(q \in \mathcal{Q}\).
By defining the \textit{conjugate of momentum} 
\(p_i := \pdiff{\mathcal{L}}{\dot{q}_i}\) so that the vector 
\(p = (p_1,\ldots,p_n) \in \R^n\) is given by \(p = D(q)\dot{q}\),
we find the \textit{Hamiltonian} of the system in \((q,p)\) coordinates is
\(\mathcal{H}(q,p) = 1/2 p\tpose D(q) p + P(q)\) \cite{landau_mechanics}.
The dynamics of the system in Hamiltonian form are
\begin{equation}\label{eqn:hamiltonian-eom-general}
    \begin{cases}
        \dot{q} = \nabla_p\mathcal{H} 
        , \\
        \dot{p} = -\nabla_q\mathcal{H} + \tau
        ,
    \end{cases}
\end{equation}
where \(\tau \in \R^k\) is a vector of generalized input forces and the input
matrix \(B \in \R^{n \times k}\) is assumed to be full rank \(k < n\).

It is easy to show that \eqref{eqn:hamiltonian-eom-general} reduces to
\begin{equation}\label{eqn:hamiltonian-full-dynamics}
     \begin{cases}
        \dot{q} = D\inv(q)p \\
        \dot{p} = -\frac{1}{2} (\Id{n} \otimes p\tpose) \nabla_q D\inv(q) p
        - \nabla_q P(q) + B(q) \tau
        . \\
    \end{cases}
\end{equation}

%/========== Acrobot ==========/%
\section{The Acrobot VNHC}\label{sec:acrobot}
specialize the theory of VNHCs to the acrobot. consider constraints which depend
only on pu, and show arctan vnhc and reduced dynamics. conclude with our
theorem. Make this section short and sweet.

%/========== Simulation ==========/%
\section{Simulation Results}\label{sec:simulations}

%/========== Experiments ==========/%
\section{Experimental Results}\label{sec:experiments}

%/========== Proof ==========/%
\section{Proof of Theorem \textbf{TODO: ref theorem}}\label{sec:proof}

%/========== Conclusion ==========/%
\section{Conclusion}\label{sec:conclusion}


%---------- Bibliography ----------%
\bibliography{bib}
\end{document}
%/========== /Main Document ==========/%
% vim: set tw=80 ts=4 sw=4 sts=0 et ffs=unix :
