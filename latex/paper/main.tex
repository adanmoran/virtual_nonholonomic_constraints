%! TEX program = lualatex

\documentclass[journal,twoside,web]{ieeecolor}

\usepackage{generic}
%/========== Preamble ==========/%
% Packages required by ieeecolor
\usepackage{amsmath}
\usepackage{amsfonts}
\usepackage{amssymb}
\usepackage{algorithmic}
\usepackage{graphicx}
\usepackage{textcomp}
\usepackage{cite}

%---------- More packages ----------%
% NOTE: cannot use amsthm for some reason
\usepackage{mathtools}
\usepackage{bm}

%---------- Bibliography Style ----------%
\bibliographystyle{IEEEtran}

%---------- Header ----------% 
\newcommand*{\Title}{VNHC and Acrobot Title}
\markboth{\journalname, VOL. XX, NO. XX, XXXX 2021}
{Moran-MacDonald \MakeLowercase{\textit{et al.}}: \Title}

%---------- Theorems ----------%
\newtheorem{thm}{Theorem}% Uncomment: reset numbering at each chapter
%\newtheorem{prop}[thm]{Proposition} % Propositions depend on Theorem number
\newtheorem{lemma}[thm]{Lemma} % Same with lemmas
\newtheorem{defn}[thm]{Definition} % Definitions depend on theorem number
\newtheorem{assm}{Assumption} % Assumptions do not reset 

%---------- Special Commands ----------%
\DeclarePairedDelimiter{\norm}{\lVert}{\rVert}
\DeclareMathOperator{\Rank}{rank}
\DeclareMathOperator{\Sign}{sgn}
\DeclareMathOperator{\Diag}{diag}
\newcommand*{\rank}[1]{\Rank\left(#1\right)}
\newcommand*{\sign}[1]{\Sign\left(#1\right)}
\newcommand*{\diag}[1]{\Diag\left(#1\right)}

\newcommand*{\tpose}{^\mathsf{T}} 
\newcommand*{\inv}{^\mathsf{-1}}
\newcommand*{\Rt}[1]{[\R]_{#1}}
\newcommand*{\R}{\mathbb{R}}
\newcommand*{\n}{\mathbf{n}}
\newcommand*{\N}{\mathbb{N}}
\newcommand*{\Sone}{\mathbb{S}^1}
\newcommand*{\SxR}{\Sone \times \R}
\newcommand*{\Minv}{M^\mathsf{-1}}
\newcommand*{\Id}[1]{I_{#1}}
\newcommand*{\Zmat}[1]{\bm{0}_{#1}}
\newcommand*{\diff}[2]{\frac{d #1}{d #2}}
\newcommand*{\ddiff}[3]{\frac{d^2 #1}{d #2 d #3}}
\newcommand*{\pdiff}[2]{\frac{\partial #1}{\partial #2}}

\newcommand*{\simpleB}{\begin{bmatrix}\Zmat{(n-k)\times k}\\ \Id{k}\end{bmatrix}}
\newcommand*{\pdmat}{(\Id{n} \otimes p\tpose)\nabla_q\Minv(q)p}
\newcommand*{\pudmat}{(\Id{n-k} \otimes p\tpose)\nabla_{q_u}\Minv(q)p}
%/========== /Preamble ==========/%

%/========== Main Document ==========/%
\begin{document}
\title{\Title}
\author{Adan Moran-MacDonald, \IEEEmembership{Member, IEEE}, Manfredi Maggiore
\IEEEmembership{Member, IEEE}*, and Xingbo Wang
\thanks{Manuscript submitted for review on \today.}
\thanks{A. Moran-MacDonald is with the Department of Electrical and Computer
    Engineering, University of Toronto, Toronto, ON, Canada (e-mail:
adan.moran@mail.utoronto.ca).}
\thanks{M. Maggiore is with the Department of Electrical and Computer
Engineering, University of Toronto, ON, Canada (e-mail:
maggiore@control.utoronto.ca).}
\thanks{X. Wang is with ??? (e-mail: ???).}
} %/author

\maketitle

%/========== Abstract ==========/%
\begin{abstract}
TODO: Abstract here.
\end{abstract}

\begin{IEEEkeywords}
TODO: keywords in alphabetical order, separated by commas.
\end{IEEEkeywords}

%/========== Introduction ==========/%
\section{Introduction}\label{sec:introduction}
\IEEEPARstart{T}ODO: intro.

\subsection{Notation}
We use the following notation in this article.
For \(A \in \R^{n\times m}\) and \(B \in \R^{p \times m}\),
define \([A;B] \in \R^{(n+p)\times m}\) as the matrix obtained by stacking \(A\)
on top of \(B\). 
For \(T > 0\), the set of real numbers modulo \(T\) is denoted \(\Rt{T}\), with
\(\Rt{\infty} := \R\).
The gradient of the matrix-valued function 
\(A : \R^m \rightarrow \R^{n\times n}\) is the block matrix of stacked partial
derivatives, 
\(\nabla_xA := [\pdiff{A}{x_1};\ldots;\pdiff{A}{x_m}] \in \R^{nm \times n}\).
Given two matrices \(A \in \R^{n \times m}\) and \(B \in \R^{r \times s}\), the
Kronecker product \cite{kronprod} \(A \otimes B \in \R^{nr \times ms}\) is the
matrix
\begin{equation}\label{eqn:kronprod}
    A \otimes B = \begin{bmatrix}
        A_{1,1}B & \cdots & A_{1,m} B \\
        \vdots & \ddots & \vdots \\
        A_{n,1} B & \cdots & A_{n,m} B
    \end{bmatrix} 
    .
\end{equation}
The Poisson bracket \cite{landau_mechanics} between the functions
\(f(q,p)\) and \(g(q,p)\) is
\begin{equation}\label{eqn:poisson-bracket}
    [f,g] := \sum \limits_{i=1}^n \pdiff{f}{p_i}\pdiff{g}{q_i} - 
        \pdiff{f}{q_i}\pdiff{g}{p_i}
    .
\end{equation}
Finally, the Kronecker delta \(\delta_i^j = 1\) if \(i = j\) and \(0\)
otherwise.

%/========== Problem Formulation ==========/%
\section{Problem Formulation}\label{sec:problem-formulation}
TODO: motivate injection/dissipation for energy regulation in an acrobot. Use
this to lead into VNHCs. e.g. represent torso as chain of a swing, pivot as
seat, and legs as human legs. Replicate fig3.3 from thesis but for acrobot: take
(qu,pu)-plane and show how a person moves their legs wrt theta, qu, or pu. Leave
the gymnast model until the end, or ignore it entirely.

Alternatively, ff human movement journal says legs move wrt qu, we can use that
as a foundation.

%/========== VNHC ==========/%
\section{Theory of VNHCs}\label{sec:vnhc}
TODO: Just like in the thesis, cover the theory of VNHCs. Justify our results compared
to those of Horn. Make it terse but legible. Cover simply actuated systems,
vnhcs, regularity, the theorems, etc.


\subsection{Simply Actuated Hamiltonian Systems}
Take a mechanical system modelled with generalized coordinates 
\(q = (q_1, \ldots, q_n)\) on the configuration manifold
\(\mathcal{Q} = \Rt{T_1} \times \cdot \Rt{T_n}\), where
\(T_i = 2\pi\) if \(q_i\) is an angle and \(T_i = \infty\) if \(q_i\) is a
displacement. The corresponding generalized velocities are 
\(\dot{q} = (\dot{q}_1,\ldots,\dot{q}_n) \in \R^n\).

Suppose this system has Lagrangian
\(\mathcal{L}(q,\dot{q}) = 1/2 \dot{q}^T D(q) \dot{q} - P(q)\),
where the potential energy 
\(P : \mathcal{Q} \rightarrow \mathbb{R}\) 
is smooth, and the inertia matrix 
\(D : \mathcal{Q} \rightarrow \mathbb{R}^{n \times n}\)
is positive definite for all \(q \in \mathcal{Q}\).
The \textit{conjugate of momentum} to \(q\) is the vector
\(p_i := \partial\mathcal{L}/\partial\dot{q}_i\) so that the vector 
\(p = D(q)\dot{q} \in \R^n\).
As per \cite{landau_mechanics}, 
the \textit{Hamiltonian} of the system in \((q,p)\) coordinates
is
\begin{equation}\label{eqn:hamiltonian}
    \mathcal{H}(q,p) = 1/2 p\tpose D(q) p + P(q)
    ,
\end{equation}
with dynamics
\begin{equation}\label{eqn:hamiltonian-eom-general}
    \begin{cases}
        \dot{q} = \nabla_p\mathcal{H} 
        , \\
        \dot{p} = -\nabla_q\mathcal{H} + B(q) \tau
        ,
    \end{cases}
\end{equation}
where \(\tau \in \R^k\) is a vector of generalized input forces and the input
matrix \(B : \mathcal{Q} \rightarrow \R^{n \times k}\) is full rank for all 
\(q \in \mathcal{Q}\).
If \(k < n\), we say the system is \textit{underactuated} with degree of
underactuation \((n-k)\).

It is easy to show using the matrix Kronecker product that
\eqref{eqn:hamiltonian-eom-general} expands to
\begin{equation}\label{eqn:hamiltonian-full-dynamics}
     \begin{cases}
        \dot{q} = D\inv(q)p \\
        \dot{p} = -\frac{1}{2} (\Id{n} \otimes p\tpose) \nabla_q D\inv(q) p
        - \nabla_q P(q) + B(q) \tau
        . \\
    \end{cases}
\end{equation}

Because \(\tau\) is transformed by \(B(q)\), it is not obvious how any
particular input force \(\tau_i\) affects the system.
To address this problem, we make the following assumptions.

\begin{assm}\label{assm:B-const}
    The input matrix \(B(q) \equiv B \in \R^{n\times k}\) is constant,
    full rank, and \(k < n\).
\end{assm}

\begin{assm}\label{assm:B-perp}
    There exists a right semi-orthogonal matrix 
    \(B^\perp \in \R^{(n-k)\times n}\)
    which is a left-annihilator for \(B\). 
\end{assm}

Note that Assumption \ref{assm:B-perp} requires the rows of \(B^\perp\) be unit vectors
that are mutually orthogonal. 
In the case that \(k = (n-1)\), the existence of any left annihilator 
\(A^0 \in \R^{1\times n}\) implies the left annihilator 
\(B^\perp := A^0/\norm{A^0}\) will be a unit vector satisfying this assumption.

The above assumptions allow us to define a
canonical coordinate transformation of ~\eqref{eqn:hamiltonian} 
which decouples the input forces.
To define this transformation we will make use of the following lemma.

\begin{lemma}\label{lemma:B-orthogonal}
    Suppose Assumption \ref{assm:B-const} holds. Then
    there exists a nonsingular matrix \(\hat{T} \in \R^{k \times k}\) 
    so that the regular feedback transformation 
    \[
        \tau = \hat{T} \hat{\tau}
    \] 
    has a new input matrix \(\hat{B}\) for \(\hat{\tau}\) which is left
    semi-orthogonal.  
    That is, \(\hat{B}\tpose \hat{B} = \Id{k}\).
\end{lemma}
\begin{proof}
    Since \(B\) is constant and full rank, it has a singular value decomposition 
    \(B = U\tpose \Sigma V\) where 
    \(\Sigma = [\diag{\sigma_1,\ldots,\sigma_k}; \Zmat{(n-k)\times k}]\),
    \(\sigma_i > 0\), and \(U \in R^{n \times n}\),
    \(V \in \R^{k \times k}\) are unitary matrices \cite{calculating_svd}.
    Defining \(T = \diag{1/\sigma_1^2,\ldots,1/\sigma_k^2}\) and assigning the
    regular feedback transformation \(\tau = V T \hat{\tau}\) yields a new input
    matrix \(\hat{B} = B V T\) for \(\hat{\tau}\) such that
    \(\hat{B}\tpose \hat{B} = T\tpose \Sigma\tpose \Sigma T = \Id{k}\).
\end{proof}

In light of Lemma \ref{lemma:B-orthogonal}, there is no loss of generality in
assuming that the input matrix is left semi-orthogonal.
Let now \(\mathbf{B} = [B^\perp; B\tpose]\).
Since \(B^\perp\) is a left annihilator of \(B\) and both \(B^\perp\) and
\(B\tpose\) are right semi-orthogonal, it is easy to show that \(\mathbf{B}\) is
orthogonal.

\begin{thm}\label{thm:simply-actuated}
    Take the Hamiltonian system ~\eqref{eqn:hamiltonian}.
    Under Assumptions \ref{assm:B-const} and \ref{assm:B-perp},
    the coordinate transformation
    \(\left(\tilde{q} = \mathbf{B}q, \tilde{p} = \mathbf{B}p\right)\)
    is a canonical transformation.
    The resulting dynamics are given by 
    \begin{gather}\label{eqn:simple-hamiltonian}
        \mathcal{H}(\tilde{q},\tilde{p}) = 
        \frac{1}{2} \tilde{p}\tpose \Minv(\tilde{q}) \tilde{p} + V(\tilde{q})
        , \\
       \begin{cases}
           \dot{\tilde{q}} = \Minv(\tilde{q})\tilde{p}
           , \\
           \dot{\tilde{p}} = -\frac{1}{2} (\Id{n} \otimes \tilde{p}\tpose)
           \nabla_{\tilde{q}} \Minv(\tilde{q}) \tilde{p} \\
           \phantom{---} - \nabla_{\tilde{q}} V(\tilde{q}) + \simpleB \tau
            ,
        \end{cases} \nonumber
    \end{gather}
    where 
    \(\Minv(\tilde{q}) := 
    \mathbf{B}D^{-1}(\mathbf{B}\tpose \tilde{q})\mathbf{B}\tpose\)
    and
    \(V(\tilde{q}) := P(\mathbf{B}\tpose \tilde{q})\).
\end{thm}
\begin{proof}
    Since \(\mathbf{B}\) is constant, this transformation satisfies
    \(\partial\tilde{q}_i/\partial p_m = \partial\tilde{p}_i/\partial q_m = 0\) for all 
    \(i,m \in \{1,\ldots,n\}\).
    This implies the Poisson brackets \([\tilde{q}_i, \tilde{q}_j]\)
    and \([\tilde{p}_i,\tilde{p}_j]\) are both zero.
    Then, since \(\mathbf{B}\) is orthogonal, 
    \([\tilde{p}_i, \tilde{q}_j] = (\mathbf{B}_i)\tpose (\mathbf{B}\tpose)_j
        = \delta_i^j\).
    By (45.10) in \cite{landau_mechanics}, this transformation is canonical and
    the new Hamiltonian is
    \(\mathcal{H}(\mathbf{B}\tpose \tilde{q}, \mathbf{B}\tpose \tilde{p})\).
    Finally, since \(\dot{\tilde{p}} = \mathbf{B} \dot{p}\), the input
    matrix for the system in \((\tilde{q},\tilde{p})\) coordinates is
    \(\mathbf{B}B = [\Zmat{(n-k)\times k}; \Id{k}]\), which proves the theorem.
\end{proof}

These new coordinates are called 
\textit{simply actuated coordinates}.
The first \((n-k)\) coordinates, labelled \(q_u\), are
the \textit{unactuated coordinates}; 
the remaining \(k\) coordinates, labelled \(q_a\), are the
\textit{actuated coordinates}.
The corresponding \((p_u, p_a)\) are the \textit{unactuated} and 
\textit{actuated momenta}, respectively.

\subsection{Virtual Nonholonomic Constraints}

In this section, we will take the system of inquiry to be a Hamiltonian
mechanical system in simply actuated coordinates.
We wish to find a controller which constraints the system according to a
function of both the configuration \(q \in \mathcal{Q}\) and the conjugate of
momentum \(p \in \R^n\).

\begin{defn}\label{defn:vnhc}
    A \textit{virtual nonholonomic constraint} (VNHC) \textit{of order \(k\)} is a
    relation \(h(q,p) = 0\) where \(h : \mathcal{Q}\times\R^n \rightarrow \R^k\) is
    \(C^2\), \(\rank{\left[ dh_q,\, dh_p \right]} = k\) for all 
    \((q,p) \in h\inv(0)\), and there exists a feedback controller \(\tau(q,p)\)
    rendering the \textit{constraint manifold} \(\Gamma\) invariant,
    where
    \[
        \Gamma = \left\{(q,p) \mid h(q,p) = 0, dh_q \dot{q} + dh_p \dot{p} = 0\right\}
        .
    \]
\end{defn}

The nonholonomic constraint \(h(q,p)\) is called \textit{virtual} because it is
enforced by the controller \(\tau(q,p)\), rather than being a constraint which
arises from the physics of the system.
The constraint manifold \(\Gamma\) is a \(2(n-k)\)-dimensional
closed embedded submanifold of \(\mathcal{Q} \times \R^n\).
Hence, the VNHC reduces the dimensionality of the dynamics from \(2n\) to
\(2(n-k)\). 
In particular, if \(k = (n-1)\), the constraint manifold is always
2-dimensional. 

We often want to stabilize the constraint within a neighbourhood of \(\Gamma\).
To see when this is possible, let us define the error output \(e = h(q,p)\).
If any component of \(e_i\) has relative degree 1, we may not be able
to stabilize \(\Gamma\) -- we can always guarantee \(e_i \to 0\), but not
necessarily \(\dot{e}_i \to 0\).
It is for this reason that we define the following special type of VNHC.

\begin{defn}
    A VNHC \(h(q,p) = 0\) of order \(k\) is \textit{regular} if the output 
    \(e = h(q,p)\) is of relative degree \(\{2,2.\ldots,2\}\) everywhere on the
    constraint manifold \(\Gamma\).
\end{defn}


The authors of
\cite{nhvc_dynamic_walking,hybrid_zero_dynamics_bipedal_nhvcs,nhvc_incline_walking}
observed that relations which use only the unactuated conjugate of momentum
often have vector relative degree \(\{2,\ldots,2\}\).
Indeed, we shall now provide a characterization of regularity which shows that
regular constraints cannot use the actuated momentum at all.

To ease notation in the rest of this section, we use the following shorthand:
\begin{align}
    \mathcal{M}(q,p) &:= (\Id{n-k} \otimes p\tpose)\nabla_{q_u}\Minv(q) 
        ,\\
    \mathcal{D}(q,p_u) &:= dh_q(q,p_u) \Minv(q) 
    .
\end{align}

\begin{thm}\label{thm:vnhc-regularity}
    A relation \(h(q,p) = 0\) for system ~\eqref{eqn:simple-hamiltonian}
    is a regular VNHC of order \(k\) if and only if \(dh_{p_a} = 0\) 
    and
    \[
        \rank{\left(\mathcal{D}(q,p_u) - dh_{p_u}\mathcal{M}(q,p)\right)\simpleB} = k
         ,
    \]
    everywhere on the constraint manifold \(\Gamma\).
\end{thm}
\begin{proof}
    Let \(e = h(q,p) \in \R^k\).
    If \(dh_{p_a} \neq \Zmat{k\times k}\) for some \((q,p) \in \Gamma\), 
    then \(\tau\) appears in \(\dot{e}\) and the VNHC is not of relative degree
    \(\{2,\ldots,2\}\). Suppose now that \(dh_{p_a} = \Zmat{k\times k}\).
    Then 
    \[
        \dot{e} = \mathcal{D}(q,p_u)p - 
        dh_{p_u}\left(\frac{1}{2}\mathcal{M}(q,p)p + \nabla_{q_u}V(q)\right)
        .
    \]
    Taking one further derivative provides
    \begin{align*}
        \ddot{e} 
            &= (\star) - 
            dh_{p_u}\left(\frac{1}{2}\diff{}{t}\left(\mathcal{M}(q,p)p\right)\right) 
                \\
            &+ \mathcal{D}(q,p_u)\simpleB \tau
    \end{align*}
    where \((\star)\) is a continuous function of \(q\) and \(p\).
    One can show that
    \begin{align*}
        dh_{p_u}\left(\frac{1}{2}\diff{}{t}\left(\mathcal{M}(q,p)p\right)\right)
        &= (\star) +\\
        &dh_{p_u} \mathcal{M}(q,p)\simpleB \tau
        .
    \end{align*}
    Hence,
    \[
       \ddot{e} = (\star) +
       \left(\mathcal{D}(q,p_u) - dh_{p_u}\mathcal{M}(q,p)\right) \simpleB \tau
        ,
    \]
    which we write as \( \ddot{e} = E(q,p) + H(q,p)\tau\) for appropriate \(E\)
    and \(H\).
    From the definition of regularity, the VNHC \(h\) is regular 
    when \(e\) is of relative degree \(\{2,\ldots,2\}\), which is true 
    if and only if the matrix premultiplying \(\tau\) is nonsingular, and hence
    that \(H\) is invertible. This proves the theorem.
\end{proof}

Under additional mild conditions (see \cite{vhcs_for_el_systems}), a regular VNHC of
order \(k\) can be stabilized by the output-linearizing phase-feedback
controller
\begin{equation}
    \tau(q,p) = -H\inv(q,p)\left(E(q,p) + k_p e + k_d \dot{e}\right)
    ,
\end{equation}
where \(k_p, k_d > 0\) are control parameters which can be tuned on the
resulting error dynamices \(\ddot{e} = -k_p e - k_d \dot{e}\).

In Section \ref{sec:acrobot} we will enforce a regular constraint on the
acrobot of the form \(h(q,p) = q_a - f(q_u,p_u)\), where the actuators track a
function of the unactuated variables.
Intuitively then, the constrained dynamics should be parameterized by \((q_u, p_u)\).
Unfortunately, \(\dot{q}_u\) depends on \(p_a\), and for general systems one
cannot solve explicitly for \(p_a\) in terms of \((q_u,p_u)\) because
the \(\dot{p}\) dynamics contains the coupling term 
\((\Id{n} \otimes p\tpose)\nabla_{q_u}M(q)p\). 

We now introduce an assumption which allows us to solve for \(p_a\) as a
function of \((q_u,p_u)\), which in turn allows us to find the constrained
dynamics.

\begin{assm}\label{assm:inertially-actuated}
The inertia matrix does not depend on the unactuated coordinates, so that 
\(\nabla_{q_u}M(q) = \Zmat{n(n-k) \times n}\).
\end{assm}

\begin{thm}\label{thm:zero-dynamics}
    Let \(\mathcal{H}\) be a mechanical system in simply actuated
    coordinates satisfying Assumption \ref{assm:inertially-actuated}. 
    Let \(h(q,p_u) = 0\) be a regular VNHC of order \(k\) with constraint
    manifold \(\Gamma\). Suppose that on \(\Gamma\) one can solve for
    \(q_a\) as a function \(q_a = f(q_u,p_u)\).
    Then the constrained dynamics are given by
    \begin{equation}\label{eqn:qpu-dynamics}
        \left.\begin{aligned}
                \dot{q}_u &= \begin{bmatrix}
                    \Id{(n-k)} & \Zmat{(n-k) \times k}
                \end{bmatrix}\Minv(q)p \\
            \dot{p}_u &= -\nabla_{q_u}V(q) \\
            \end{aligned}{}\right|_{\begin{array}{c}
                q_a = f(q_u,p_u) \\ 
                p_a = g(q_u,p_u) \\
            \end{array}}
            ,
    \end{equation}
    where
    \begin{equation}\label{eqn:g-qpu}
    \begin{aligned}
        &g(q_u,p_u) := \\
        &\left(\mathcal{D}(q,p_u) \simpleB \right)\inv 
               \\
        &\left.\left(dh_{p_u} \nabla_{q_u}V(q) - \mathcal{D}(q,p_u)
        \begin{bmatrix}
            I_{n-k} \\
            \Zmat{k \times (n-k)} \\
    \end{bmatrix} p_u\right)\right|_{qa = f(q_u,p_u)}
        .
    \end{aligned}
    \end{equation}
\end{thm}
\begin{proof}
    Setting \(e = h(q,p_u)\) and using Assumption
    \ref{assm:inertially-actuated}, we find that
    \(\dot{e} = \mathcal{D}(q,p_u)p - dh_{p_u}\nabla_{q_u}V(q)\).
    Notice that
    \(\mathcal{D}(q,p_u)p = \mathcal{D}(q,p_u)[\Zmat{(n-k)\times k}; \Id{k}]p_a
    + \mathcal{D}(q,p_u)[\Id{n-k};\Zmat{k \times (n-k)}] p_u\).
    Since \(h(q,p_u)\) is regular, \(\mathcal{D}(q,p_u)\) is invertible.
    Taking \(e = \dot{e} = 0\), solving for \(p_a\), and setting 
    \(q_a = f(q_u,p_u)\) completes the proof.
\end{proof}

\textbf{Comparison with existing literature}: Horn et.al. provide constrained
dynamics for VNHCs in \cite{nhvc_incline_walking}.
In fact, our constrained dynamics \eqref{eqn:qpu-dynamics} coincide with their
system (17) when one chooses the special case \(\theta_1 = q_u\) and 
\(\theta_2 = p_u\).
Other than this distinction, our assumptions are identical to theirs.
In particular, their assumption \textbf{H2} is what we call
regularity.
The only real distinction is that our constrained dynamics
are explicit functions of the Hamiltonian phase coordinates
\((q_u,p_u)\).
This representation will be beneficial when we apply the theory of VNHCs to the
acrobot.

%/========== Acrobot ==========/%
\section{The Acrobot VNHC}\label{sec:acrobot}
specialize the theory of VNHCs to the acrobot. consider constraints which depend
only on pu, and show arctan vnhc and reduced dynamics. conclude with our
theorem. Make this section short and sweet.

%/========== Simulation ==========/%
\section{Simulation Results}\label{sec:simulations}

%/========== Experiments ==========/%
\section{Experimental Results}\label{sec:experiments}

%/========== Proof ==========/%
\section{Proof of Theorem \textbf{TODO: ref theorem}}\label{sec:proof}

%/========== Conclusion ==========/%
\section{Conclusion}\label{sec:conclusion}


%---------- Bibliography ----------%
\bibliography{bib}
\end{document}
%/========== /Main Document ==========/%
% vim: set tw=80 ts=4 sw=4 sts=0 et ffs=unix :
