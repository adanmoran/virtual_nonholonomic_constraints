%! TEX program = lualatex

\documentclass[journal,twoside,web]{ieeecolor}

\usepackage{generic}
%/========== Preamble ==========/%
% Packages required by ieeecolor
\usepackage{amsmath}
\usepackage{amsfonts}
\usepackage{amssymb}
\usepackage{algorithmic}
\usepackage{graphicx}
\usepackage{textcomp}
\usepackage{cite}

%---------- More packages ----------%
% NOTE: cannot use amsthm for some reason
\usepackage{mathtools}
\usepackage{bm}

%---------- Bibliography Style ----------%
\bibliographystyle{IEEEtran}

%---------- Header ----------% 
\newcommand*{\Title}{VNHC and Acrobot Title}
\markboth{\journalname, VOL. XX, NO. XX, XXXX 2021}
{Moran-MacDonald \MakeLowercase{\textit{et al.}}: \Title}

%---------- Theorems ----------%
\newtheorem{thm}{Theorem}% Uncomment: reset numbering at each chapter
%\newtheorem{prop}[thm]{Proposition} % Propositions depend on Theorem number
\newtheorem{lemma}[thm]{Lemma} % Same with lemmas
\newtheorem{defn}[thm]{Definition} % Definitions depend on theorem number
\newtheorem{assm}{Assumption} % Assumptions do not reset 

%---------- Special Commands ----------%
\DeclarePairedDelimiter{\norm}{\lVert}{\rVert}
\DeclareMathOperator{\Rank}{rank}
\DeclareMathOperator{\Sign}{sgn}
\newcommand*{\rank}[1]{\Rank\left(#1\right)}
\newcommand*{\sign}[1]{\Sign\left(#1\right)}

\newcommand*{\tpose}{^\mathsf{T}} 
\newcommand*{\inv}{^\mathsf{-1}}
\newcommand*{\Rt}[1]{[\R]_{#1}}
\newcommand*{\R}{\mathbb{R}}
\newcommand*{\n}{\mathbf{n}}
\newcommand*{\N}{\mathbb{N}}
\newcommand*{\Sone}{\mathbb{S}^1}
\newcommand*{\SxR}{\Sone \times \R}
\newcommand*{\Minv}{M^\mathsf{-1}}
\newcommand*{\Id}[1]{I_{#1}}
\newcommand*{\Zmat}[1]{\bm{0}_{#1}}
\newcommand*{\diff}[2]{\frac{d #1}{d #2}}
\newcommand*{\ddiff}[3]{\frac{d^2 #1}{d #2 d #3}}
\newcommand*{\pdiff}[2]{\frac{\partial #1}{\partial #2}}

\newcommand*{\simpleB}{\begin{bmatrix}\Zmat{(n-k)\times k}\\ \Id{k}\end{bmatrix}}
\newcommand*{\pdmat}{(\Id{n} \otimes p\tpose)\nabla_q\Minv(q)p}
\newcommand*{\pudmat}{(\Id{n-k} \otimes p\tpose)\nabla_{q_u}\Minv(q)p}
%/========== /Preamble ==========/%

%/========== Main Document ==========/%
\begin{document}
\title{\Title}
\author{Adan Moran-MacDonald, \IEEEmembership{Member, IEEE}, Manfredi Maggiore
\IEEEmembership{Member, IEEE}*, and Xingbo Wang
\thanks{Manuscript submitted for review on \today.}
\thanks{A. Moran-MacDonald is with the Department of Electrical and Computer
    Engineering, University of Toronto, Toronto, ON, Canada (e-mail:
adan.moran@mail.utoronto.ca).}
\thanks{M. Maggiore is with the Department of Electrical and Computer
Engineering, University of Toronto, ON, Canada (e-mail:
maggiore@control.utoronto.ca).}
\thanks{X. Wang is with ??? (e-mail: ???).}
} %/author

\maketitle

%/========== Abstract ==========/%
\begin{abstract}
TODO: Abstract here.
\end{abstract}

\begin{IEEEkeywords}
TODO: keywords in alphabetical order, separated by commas.
\end{IEEEkeywords}

%/========== Introduction ==========/%
\section{Introduction}\label{sec:introduction}
\IEEEPARstart{T}ODO: intro.

\subsection{Notation}
We use the following notation in this paper. 
For \(T > 0\), the set of real numbers modulo is denote \(\Rt{T}\), with
\(\Rt{\infty} := \R\).
The gradient of the matrix-valued function 
\(A : \R^m \rightarrow \R^{n\times n}\) is the matrix of stacked partial
derivatives, 
\(\nabla_xA := (\pdiff{A}{x_1},\ldots,\pdiff{A}{x_m})\tpose \in \R^{nm \times n}\).
Given two matrices \(A \in \R^{n \times m}\) and \(B \in \R^{r \times s}\), the
Kronecker product \cite{kronprod} \(A \otimes B \in \R^{nr \times ms}\) is the
matrix
\[
    A \otimes B = \begin{bmatrix}
        A_{1,1}B & \cdots & A_{1,m} B \\
        \vdots & \ddots & \vdots \\
        A_{n,1} B & \cdots & A_{n,m} B
    \end{bmatrix} 
    .
\]

%/========== Problem Formulation ==========/%
\section{Problem Formulation}\label{sec:problem-formulation}
TODO: motivate injection/dissipation for energy regulation in an acrobot. Use
this to lead into VNHCs. e.g. represent torso as chain of a swing, pivot as
seat, and legs as human legs. Replicate fig3.3 from thesis but for acrobot: take
(qu,pu)-plane and show how a person moves their legs wrt theta, qu, or pu. Leave
the gymnast model until the end, or ignore it entirely.

Alternatively, ff human movement journal says legs move wrt qu, we can use that
as a foundation.

%/========== VNHC ==========/%
\section{Theory of VNHCs}\label{sec:vnhc}
TODO: Just like in the thesis, cover the theory of VNHCs. Justify our results compared
to those of Horn. Make it terse but legible. Cover simply actuated systems,
vnhcs, regularity, the theorems, etc.


\subsection{Simply Actuated Hamiltonian Systems}
Take an underactuated mechanical system modelled with generalized coordinates 
\(q = (q_1, \ldots, q_n)\) on the configuration manifold
\(\mathcal{Q} = \Rt{T_1} \times \cdot \Rt{T_n}\), where
\(T_i = 2\pi\) if \(q_i\) is an angle and \(T_i = \infty\) if \(q_i\) is a
length. The corresponding velocities are 
\(\dot{q} = (\dot{q}_1,\ldots,\dot{q}_n) \in \R^n\).

Suppose this system has Lagrangian
\(\mathcal{L}(q,\dot{q}) = 1/2 \dot{q}^T D(q) \dot{q} - P(q)\),
where \(P : \mathcal{Q} \rightarrow \mathbb{R}\) is the potential energy, and
the inertia matrix \(D : \mathcal{Q} \rightarrow \mathbb{R}^{n \times n}\)
is positive definite for all \(q \in \mathcal{Q}\).
By defining the \textit{conjugate of momentum} 
\(p_i := \pdiff{\mathcal{L}}{\dot{q}_i}\) so that the vector 
\(p = (p_1,\ldots,p_n) \in \R^n\) is given by \(p = D(q)\dot{q}\),
we find the \textit{Hamiltonian} of the system in \((q,p)\) coordinates
is\cite{landau_mechanics}
\begin{equation}\label{eqn:hamiltonian}
    \mathcal{H}(q,p) = 1/2 p\tpose D(q) p + P(q).
\end{equation}
The dynamics of the system in Hamiltonian form are
\begin{equation}\label{eqn:hamiltonian-eom-general}
    \begin{cases}
        \dot{q} = \nabla_p\mathcal{H} 
        , \\
        \dot{p} = -\nabla_q\mathcal{H} + B(q) \tau
        ,
    \end{cases}
\end{equation}
where \(\tau \in \R^k\) is a vector of generalized input forces and the input
matrix \(B : \mathcal{Q} \rightarrow \R^{n \times k}\) is full rank for all 
\(q \in \mathcal{Q}\).

It is easy to show using the matrix Kronecker product that
\eqref{eqn:hamiltonian-eom-general} reduces to
\begin{equation}\label{eqn:hamiltonian-full-dynamics}
     \begin{cases}
        \dot{q} = D\inv(q)p \\
        \dot{p} = -\frac{1}{2} (\Id{n} \otimes p\tpose) \nabla_q D\inv(q) p
        - \nabla_q P(q) + B(q) \tau
        . \\
    \end{cases}
\end{equation}

Because \(\tau\) is transformed by \(B(q)\), it is not in general clear how any
particular input force \(\tau_i\) will affect the dynamics of the system. 
The following definition provides a class of systems where the effect of input
forces is made explicit.

\begin{defn}
    Let \(\mathcal{H}\) be an \(n\)-DOF Hamiltonian system 
    with \(k \leq n\) actuators. 
    A set of canonical coordinates \((q,p)\) for this system
    are said to be \textit{simply actuated coordinates} if the
    input matrix \(B(q) \in \R^{n \times k}\) is of the form
    \[
        B(q) = \simpleB    
        .
    \]

\end{defn}

The first \((n-k)\) coordinates, labelled \(q_u\), are called the
\textit{unactuated coordinates}. The remaining \(k\) coordinates, labelled
\(q_a\), are called the \textit{actuated coordinates}. When grouping them
together, we will always put them in the order \((q_u, q_a)\) to fit with 
the definition. 
The corresponding \((p_u, p_a)\) are called the \textit{unactuated} and 
\textit{actuated momenta}, respectively.

Under the following assumptions on the input matrix, we will show that there is
a canonical transformation of ~\eqref{eqn:hamiltonian} into simply actuated
coordinates.

\begin{assm}\label{assm:B-const}
    The input matrix \(B(q) \equiv B \in \R^{n\times k}\) is constant,
    full rank, and \(k < n\).
\end{assm}

\begin{assm}\label{assm:B-perp}
    There exists a matrix 
    \(B^\perp \in \R^{(n-k)\times n}\)
    which is right semi-orthogonal 
    \(\left(B^\perp(B^\perp)\tpose = \Id{(n-k)}\right)\)
    and which is a left-annihilator for \(B\). 
    That is, \(B^\perp B = \Zmat{(n-k) \times k}\).
\end{assm}

Assumption \ref{assm:B-perp} requires the rows of \(B^\perp\) to be unit vectors
that are mutually orthogonal. 
In the case that \(k = (n-1)\), the existence of any left annihilator 
\(A^0 \in \R^{1\times n}\) implies the left annihilator 
\(B^\perp := A^0/\norm{A^0}\) will be a unit vector satisfying Assumption \ref{assm:B-perp}.

\begin{lemma}\label{lemma:B-orthogonal}
    Suppose Assumption \ref{assm:B-const} holds. Then
    there exists a nonsingular matrix \(\hat{T} \in \R^{k \times k}\) 
    so that the regular feedback transformation 
    \[
        \tau = \hat{T} \hat{\tau}
    \] 
    has a new input matrix \(\hat{B}\) for \(\hat{\tau}\) which is left
    semi-orthogonal.  
    That is, \(\hat{B}\tpose \hat{B} = \Id{k}\).
\end{lemma}
\begin{proof}
    \textbf{TODO: prove this in a terse way}.
\end{proof}

In light of Lemma \ref{lemma:B-orthogonal}, there is no loss of generality
by making the following assumption.
\begin{assm}\label{assm:B-orthogonal}
    Assume that the input matrix \(B\) is
    left semi-orthogonal, i.e.~\(B\tpose B = I_k\). 
\end{assm}

Let now \(\mathbf{B} \in \R^{n\times n}\) be the following matrix:
\[
    \mathbf{B} = 
    \begin{bmatrix}
        B^\perp \\
        B\tpose \\
    \end{bmatrix}
    .
\]
Since \(B^\perp\) is a left annihilator of \(B\) and both \(B^\perp\) and
\(B\tpose\) are right semi-orthogonal, it is easy to show that \(\mathbf{B}\) is
orthogonal.
\begin{proof}
\[
    \mathbf{B}\mathbf{B}\tpose = 
    \begin{bmatrix}
        B^\perp (B^\perp)\tpose & B^\perp B \\
        (B^\perp B)\tpose & B\tpose B
    \end{bmatrix} = \Id{n}
    .
\]
Hence, \(\mathbf{B}\) is invertible with \(\mathbf{B}\inv = \mathbf{B}\tpose\).
\end{proof}

The following theorem shows that \(\mathbf{B}\) provides a canonical
transformation into simply actuated coordinates, so that only the actuated momenta
are affected by the input forces.

\begin{thm}\label{thm:simply-actuated}
    Under Assumptions \ref{assm:B-const},\ref{assm:B-perp}, and
    \ref{assm:B-orthogonal}, the Hamiltonian system ~\eqref{eqn:hamiltonian}
    has simply actuated canonical coordinates 
    \(\left(\tilde{q} = \mathbf{B}q, \tilde{p} = \mathbf{B}p\right)\). 
    The resulting dynamics are given by 
    \begin{gather}\label{eqn:simple-hamiltonian}
        \mathcal{H}(\tilde{q},\tilde{p}) = 
        \frac{1}{2} \tilde{p}\tpose \Minv(\tilde{q}) \tilde{p} + V(\tilde{q})
        , \\
       \begin{cases}
           \dot{\tilde{q}} = \Minv(\tilde{q})\tilde{p}
           , \\
           \dot{\tilde{p}} = -\frac{1}{2} (\Id{n} \otimes \tilde{p}\tpose)
           \nabla_{\tilde{q}} \Minv(\tilde{q}) \tilde{p}
           - \nabla_{\tilde{q}} V(\tilde{q}) + \simpleB \tau
            ,
        \end{cases} \nonumber
    \end{gather}
    where 
    \(\Minv(\tilde{q}) := \mathbf{B}D^{-1}(\mathbf{B}\tpose
    \tilde{q})\mathbf{B}\tpose\)
    and
    \(V(\tilde{q}) := P(\mathbf{B}\tpose \tilde{q})\).
\end{thm}
\begin{proof}
    \textbf{TODO: prove this in a terse way.}
\end{proof}
Throughout the rest of this paper, we will assume the system we are studying is
in simply actuated coordinates.

\subsection{Virtual Nonholonomic Constraints}

\begin{defn}
    A \textit{virtual nonholonomic constraint} (VNHC) \textit{of order \(k\)} is a
    relation \(h(q,p) = 0\) where \(h : \mathcal{Q}\times\R^n \rightarrow \R^k\) is
    \(C^2\), \(\rank{\left[ dh_q,\, dh_p \right]} = k\) for all 
    \((q,p) \in h\inv(0)\), and there exists a feedback controller \(\tau(q,p)\)
    rendering the \textit{constraint manifold} \(\Gamma\) invariant,
    where
    \[
        \Gamma = \left\{(q,p) \mid h(q,p) = 0, dh_q \dot{q} + dh_p \dot{p} = 0\right\}
        .
    \]
\end{defn}

From the definition of VNHCs, one finds that the constraint manifold \(\Gamma\)
is a \(2(n-k)\)-dimensional closed embedded submanifold of 
\(\mathcal{Q} \times \R^n\). 

%TODO: tie this together 

\begin{thm}\label{thm:vnhc-regularity}
    A relation \(h(q,p) = 0\) for system ~\eqref{eqn:simply-actuated-hamiltonian}
    is a regular VNHC of order \(k\) if and only if \(dh_{p_a} = 0\) 
    and
    \[
        \rank{\left(dh_q \Minv(q) - 
          dh_{p_u} (\Id{n-k} \otimes p\tpose)\nabla_{q_u}\Minv(q) 
         \right)\simpleB} = k
         ,
    \]
    everywhere on the constraint manifold \(\Gamma\).
\end{thm}
\begin{proof}
    \textbf{TODO: prove this in a terse way}.
\end{proof}

The constrained dynamics for a general VNHC were derived by 
Horn et. al \cite{nhvc_dynamic_walking}.
In our case, we will be enforcing regular constraints where the actuators of the
acrobot track a function of the unactuated variables. 
Intuitively, the dynamics should be parameterized by \((q_u, p_u)\).
Unfortunately, \(\dot{q}_u\) depends on \(p_a\), and for general systems one
cannot solve explicitly for \(p_a\) in terms of \((q_u,p_u)\). 
This is because the \(\dot{p}\) dynamics contains the coupling term 
\((\Id{n} \otimes p\tpose)\nabla_{q_u}M(q)p\). 

We now introduce an assumption so we can solve explicitly for the constrained
dynamics.

\begin{assm}\label{assm:inertially-actuated}
    The Hamiltonian system has an inertia matrix that does not depend on the
    unactuated coordinates:
    \[
        \nabla_{q_u}M(q) = \Zmat{n(n-k) \times n}
        .
    \]
\end{assm}

\begin{thm}\label{thm:zero-dynamics}
    Let \(\mathcal{H}\) be a mechanical system in simply actuated
    coordinates satisfying Assumption \ref{assm:inertially-actuated}. 
    Let \(h(q,p_u) = 0\) be a regular VNHC of order \(k\) with constraint
    manifold \(\Gamma\). Suppose that on \(\Gamma\) one can solve for
    \(q_a\) as a function \(q_a = f(q_u,p_u)\).
    Then the constrained dynamics are given by
    \begin{equation}\label{eqn:qpu-dynamics}
        \left.\begin{aligned}
                \dot{q}_u &= \begin{bmatrix}
                    \Id{(n-k)} & \Zmat{(n-k) \times k}
                \end{bmatrix}\Minv(q)p \\
            \dot{p}_u &= -\nabla_{q_u}V(q) \\
            \end{aligned}{}\right|_{\begin{array}{c}
                q_a = f(q_u,p_u) \\ 
                p_a = g(q_u,p_u) \\
            \end{array}}
            ,
    \end{equation}
    where
    \begin{equation}\label{eqn:g-qpu}
    \begin{aligned}
        &g(q_u,p_u) := \\
           &\left(dh_q \Minv(q) \simpleB \right)\inv 
               \\
           &\left.\left(dh_{p_u} \nabla_{q_u}V(q) - dh_q \Minv(q)
        \begin{bmatrix}
            I_{n-k} \\
            \Zmat{k \times (n-k)} \\
    \end{bmatrix} p_u\right)\right|_{qa = f(q_u,p_u)}
        .
    \end{aligned}
    \end{equation}
\end{thm}
\begin{proof}
    This is a special case of Theorem 1 in \cite{nhvc_dynamic_walking} using
    \(\theta_1 = q_u\) and \(\theta_2 = p_u\).
    The proof is ommitted.
\end{proof}


%/========== Acrobot ==========/%
\section{The Acrobot VNHC}\label{sec:acrobot}
specialize the theory of VNHCs to the acrobot. consider constraints which depend
only on pu, and show arctan vnhc and reduced dynamics. conclude with our
theorem. Make this section short and sweet.

%/========== Simulation ==========/%
\section{Simulation Results}\label{sec:simulations}

%/========== Experiments ==========/%
\section{Experimental Results}\label{sec:experiments}

%/========== Proof ==========/%
\section{Proof of Theorem \textbf{TODO: ref theorem}}\label{sec:proof}

%/========== Conclusion ==========/%
\section{Conclusion}\label{sec:conclusion}


%---------- Bibliography ----------%
\bibliography{bib}
\end{document}
%/========== /Main Document ==========/%
% vim: set tw=80 ts=4 sw=4 sts=0 et ffs=unix :
