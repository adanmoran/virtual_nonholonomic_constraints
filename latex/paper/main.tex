%! TEX program = lualatex

\documentclass[journal,twoside,web]{ieeecolor}
\usepackage{generic}
%/========== Preamble ==========/%
\usepackage{cite}
\usepackage{amsmath,amssymb,amsfonts}
\usepackage{algorithmic}
\usepackage{graphicx}
\usepackage{textcomp}
%---------- Bibliography Style ----------%
\def\BibTeX{{\rm B\kern-.05em{\sc i\kern-.025em b}\kern-.08em
    T\kern-.1667em\lower.7ex\hbox{E}\kern-.125emX}}
%---------- Header ----------% 
\newcommand*{\Title}{VNHC and Acrobot Title}
\markboth{\journalname, VOL. XX, NO. XX, XXXX 2021}
{Moran-MacDonald \MakeLowercase{\textit{et al.}}: \Title}
%/========== /Preamble ==========/%

%/========== Main Document ==========/%
\begin{document}
\title{\Title}
\author{Adan Moran-MacDonald, \IEEEmembership{Member, IEEE}, Manfredi Maggiore
\IEEEmembership{Fellow, IEEE}*, and Xingbo Wang
\thanks{Manuscript submitted for review on \today.}
\thanks{A. Moran-MacDonald is with the Department of Electrical and Computer
    Engineering, University of Toronto, Toronto, ON, Canada (e-mail:
adan.moran@mail.utoronto.ca).}
\thanks{M. Maggiore is with the Department of Electrical and Computer
Engineering, University of Toronto, ON, Canada (e-mail:
maggiore@control.utoronto.ca).}
\thanks{X. Wang is with ??? (e-mail: ???).}
} %/author

\maketitle

%/========== Abstract ==========/%
\begin{abstract}
TODO: Abstract here.
\end{abstract}

\begin{IEEEkeywords}
TODO: keywords in alphabetical order, separated by commas.
\end{IEEEkeywords}

%/========== Introduction ==========/%
\section{Introduction}\label{sec:introduction}
\IEEEPARstart{T}ODO: intro.

\subsection{Notation}

%/========== Problem Formulation ==========/%
\section{Problem Formulation}\label{sec:problem-formulation}
TODO: motivate injection/dissipation for energy regulation in an acrobot. Use
this to lead into VNHCs. e.g. represent torso as chain of a swing, pivot as
seat, and legs as human legs. Replicate fig3.3 from thesis but for acrobot: take
(qu,pu)-plane and show how a person moves their legs wrt theta, qu, or pu. Leave
the gymnast model until the end, or ignore it entirely.

Alternatively, ff human movement journal says legs move wrt qu, we can use that
as a foundation.

%/========== VNHC ==========/%
\section{Theory of VNHCs}\label{sec:vnhc}
Just like in the thesis, cover the theory of VNHCs. Justify our results compared
to those of Horn. Make it terse but legible. Cover simply actuated systems,
vnhcs, regularity, the theorems, etc.

%/========== Acrobot ==========/%
\section{The Acrobot VNHC}\label{sec:acrobot}
specialize the theory of VNHCs to the acrobot. consider constraints which depend
only on pu, and show arctan vnhc and reduced dynamics. conclude with our
theorem. Make this section short and sweet.

%/========== Simulation ==========/%
\section{Simulation Results}\label{sec:simulations}

%/========== Experiments ==========/%
\section{Experimental Results}\label{sec:experiments}

%/========== Proof ==========/%
\section{Proof of Theorem \textbf{TODO: ref theorem}}\label{sec:proof}

%/========== Conclusion ==========/%
\section{Conclusion}\label{sec:conclusion}

\end{document}
%/========== /Main Document ==========/%
% vim: set tw=80 ts=4 sw=4 sts=0 et ffs=unix :
