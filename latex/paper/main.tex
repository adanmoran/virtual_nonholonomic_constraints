%! TEX program = lualatex

\documentclass[journal,twoside,web]{ieeecolor}

\usepackage{generic}
%/========== Preamble ==========/%
% Packages required by ieeecolor
\usepackage{amsmath}
\usepackage{amsfonts}
\usepackage{amssymb}
\usepackage{algorithmic}
\usepackage{graphicx}
\usepackage{textcomp}
\usepackage{cite}

%---------- More packages ----------%
% NOTE: cannot use amsthm for some reason
\usepackage{mathtools}
\usepackage{bm}

%---------- Bibliography Style ----------%
\bibliographystyle{IEEEtran}

%---------- Header ----------% 
\newcommand*{\Title}{VNHC and Acrobot Title}
\markboth{\journalname, VOL. XX, NO. XX, XXXX 2021}
{Moran-MacDonald \MakeLowercase{\textit{et al.}}: \Title}

%---------- Theorems ----------%
\newtheorem{thm}{Theorem}% Uncomment: reset numbering at each chapter
\newtheorem{prop}[thm]{Proposition} % Propositions depend on Theorem number
\newtheorem{lemma}[thm]{Lemma} % Same with lemmas
\newtheorem{defn}[thm]{Definition} % Definitions depend on theorem number
\newtheorem{assm}{Assumption} % Assumptions do not reset 

%---------- Special Commands ----------%
\DeclarePairedDelimiter{\norm}{\lVert}{\rVert}
\DeclareMathOperator{\Rank}{rank}
\DeclareMathOperator{\Sign}{sgn}
\DeclareMathOperator{\Diag}{diag}
\newcommand*{\rank}[1]{\Rank\left(#1\right)}
\newcommand*{\sign}[1]{\Sign\left(#1\right)}
\newcommand*{\diag}[1]{\Diag\left(#1\right)}

\newcommand*{\tpose}{^\mathsf{T}} 
\newcommand*{\inv}{^\mathsf{-1}}
\newcommand*{\Rt}[1]{[\R]_{#1}}
\newcommand*{\R}{\mathbb{R}}
\newcommand*{\n}{\mathbf{n}}
\newcommand*{\N}{\mathbb{N}}
\newcommand*{\Sone}{\mathbb{S}^1}
\newcommand*{\SxR}{\Sone \times \R}
\newcommand*{\Minv}{M^\mathsf{-1}}
\newcommand*{\Id}[1]{I_{#1}}
\newcommand*{\Zmat}[1]{\bm{0}_{#1}}
\newcommand*{\diff}[2]{\frac{d #1}{d #2}}
\newcommand*{\ddiff}[3]{\frac{d^2 #1}{d #2 d #3}}
\newcommand*{\pdiff}[2]{\frac{\partial #1}{\partial #2}}

\newcommand*{\simpleB}{\begin{bmatrix}\Zmat{(n-k)\times k}\\ \Id{k}\end{bmatrix}}
\newcommand*{\pdmat}{(\Id{n} \otimes p\tpose)\nabla_q\Minv(q)p}
\newcommand*{\pudmat}{(\Id{n-k} \otimes p\tpose)\nabla_{q_u}\Minv(q)p}
%/========== /Preamble ==========/%

%/========== Main Document ==========/%
\begin{document}
\title{\Title}
\author{Adan Moran-MacDonald, \IEEEmembership{Member, IEEE}, Manfredi Maggiore
\IEEEmembership{Member, IEEE}*, and Xingbo Wang
\thanks{Manuscript submitted for review on \today.}
\thanks{A. Moran-MacDonald is with the Department of Electrical and Computer
    Engineering, University of Toronto, Toronto, ON, Canada (e-mail:
adan.moran@mail.utoronto.ca).}
\thanks{M. Maggiore is with the Department of Electrical and Computer
Engineering, University of Toronto, ON, Canada (e-mail:
maggiore@control.utoronto.ca).}
\thanks{X. Wang is with ??? (e-mail: ???).}
} %/author

\maketitle

%/========== Abstract ==========/%
\begin{abstract}
TODO: Abstract here.
\end{abstract}

\begin{IEEEkeywords}
TODO: keywords in alphabetical order, separated by commas.
\end{IEEEkeywords}

%/========== Introduction ==========/%
\section{Introduction}\label{sec:introduction}
\IEEEPARstart{T}ODO: intro.

\subsection{Notation}
We use the following notation and terminology in this article.
A matrix \(A \in \R^{n \times m}\) is \textit{right semi-orthogonal} if
\(A A\tpose = \Id{m}\) and is \textit{left semi-orthogonal} if 
\(A\tpose A = \Id{n}\).
For \(A \in \R^{n\times m}\) and \(B \in \R^{p \times m}\),
we define \([A;B] \in \R^{(n+p)\times m}\) as the matrix obtained by stacking \(A\)
on top of \(B\). 
For \(T > 0\), the set of real numbers modulo \(T\) is denoted \(\Rt{T}\), with
\(\Rt{\infty} := \R\).
The gradient of a matrix-valued function 
\(A : \R^m \rightarrow \R^{n\times n}\) is the block matrix of stacked partial
derivatives, 
\(\nabla_xA := [\pdiff{A}{x_1};\ldots;\pdiff{A}{x_m}] \in \R^{nm \times n}\).
Given two matrices \(A \in \R^{n \times m}\) and \(B \in \R^{r \times s}\), the
Kronecker product \cite{kronprod} \(A \otimes B \in \R^{nr \times ms}\) is the
matrix
\begin{equation}\label{eqn:kronprod}
    A \otimes B = \begin{bmatrix}
        A_{1,1}B & \cdots & A_{1,m} B \\
        \vdots & \ddots & \vdots \\
        A_{n,1} B & \cdots & A_{n,m} B
    \end{bmatrix} 
    .
\end{equation}
The Poisson bracket \cite{landau_mechanics} between the functions
\(f(q,p)\) and \(g(q,p)\) is
\begin{equation}\label{eqn:poisson-bracket}
    [f,g] := \sum \limits_{i=1}^n \pdiff{f}{p_i}\pdiff{g}{q_i} - 
        \pdiff{f}{q_i}\pdiff{g}{p_i}
    .
\end{equation}
Finally, the Kronecker delta \(\delta_i^j = 1\) if \(i = j\) and \(0\)
otherwise.

%/========== Problem Formulation ==========/%
\section{Problem Formulation}\label{sec:problem-formulation}
%TODO: motivate injection/dissipation for energy regulation in an acrobot. Use
%this to lead into VNHCs. e.g. represent torso as chain of a swing, pivot as
%seat, and legs as human legs. Replicate fig3.3 from thesis but for acrobot: take
%(qu,pu)-plane and show how a person moves their legs wrt theta, qu, or pu. Leave
%the gymnast model until the end, or ignore it entirely.
%
%Alternatively, ff human movement journal says legs move wrt qu, we can use that
%as a foundation.
%Ideas: Human movement says appropriate regulation of pendulum allows energy to
%be pumped into system. Pendulum length rapidly increased on downward swing, then
%rapidly decreased on upward swing (I think these mean backward and forward).
%They plot pendulum length rdot/r vs "body angle" theta to determine energy
%injection, which hints at some relationship between (r,rdot) and (theta,thetad).
%Indeed, the peak value of rdot/r did not occur at a fixed value of theta, nor
%once anyone reached a particular speed; it occured at the same thetadot/theta for all
%participants of the study. This perfectly motivates VNHCs for robots, because we
%can shape (r,rtdot) as functions of (theta,thetad).
%Start motivation from energy injection - how do humans inject energy? How does
%acrobot model gymnast? use human movement to motivate tracking functions of both
%position and velocity rather than time.

In gymnastics terminology, a ``giant" is the motion a gymnast performs to
achieve full rotations around a horizontal bar \cite{usagym_giant}. 
A gymnast will begin by hanging at rest, then swing their legs
appropriately to gain energy over time.
The authors of \cite{pendulum_length_giant_gymnastics} modelled the gymnast as a
variable length pendulum, and studied how the pendulum's length changes as a
function of the orientation of a gymnast's joints.
Labeling the pendulum length by \(r\) and the gymnast's body orientation
by \(\theta\), they discovered experimentally that the value \(\dot{r}/r\) has
the highest impact on the magintude of energy injection; they also discovered
that the peak value of \(\dot{r}/r\) occured at the same value of
\(\dot{\theta}/\theta\) for all gymnasts.
In other words, gymnasts appear to move their legs as a function of their body
angle and velocity; this is the mechanism by which they gain enough energy to
rotate around the bar.

Control theorists typically want a system's actuators to track a fixed
trajectory over time. 
If we were attempting design a gymnastics robot which
could replicate human behaviour, the results of
\cite{pendulum_length_giant_gymnastics} indicate that we should not use this
standard approach.
Instead, we should have our robot's legs follow a function of its
body angle and angular velocity.
That is exactly what we set out to do in this article.

The simplest way to model the gymnast is as a variable-length pendulum, but a
more realistic model is that of a two-link acrobot (Figure \ref{fig:acrobot}).
The acrobot's first link represents the gymnast's torso, the bottom link represents
their legs, and it is actuated at the center joint (the hips).

\begin{figure}
    \centering
    \includegraphics[width=\linewidth]{acrobot_gymnast.png}
    \caption{The two-link acrobot as a model for a gymnast.
    Image modified from \cite{xingbo_thesis}.}
    \label{fig:acrobot}
\end{figure}

In this article, our goal is to design a leg controller for the acrobot so that
the resulting motion mimics that of a human performing a giant.
Previous attempts at solving this problem have involved
trajectory tracking, set-point tracking, or energy
pumping regulation (see
\cite{energy_pumping_robotic_swinging,swingup_giant_acrobot,dynamical_servo_acrobot_vc,control_giant_two_link_gymnastic_robot}
).
While these approaches all result in the acrobot performing rotations on the
bar, none of the are particularly human-like.
Wang designed a controller which
forces the hips to track a function of the torso angle using virtual holonomic
constraints \cite{xingbo_thesis};
however, truly human-like hip motion would also take into account the torso velocity.
This style of control action is captured by the method of virtual nonholonomic
constraints.

Virtual nonholonomic constraints (VNHCs) have been used for human-robot interaction
\cite{vnhc_human_robot_cooperation,psd_based_vnhc_redundant_manipulator,haptic_vnhc},
error-reduction on time-delayed systems \cite{vnhc_time_delay_teleop},
and they have shown marked improvements to the field of bipedal locomotion 
\cite{nhvc_dynamic_walking,
hybrid_zero_dynamics_bipedal_nhvcs,output_nhvc_bipedal_control}.
Indeed, they produce more robust bipedal motion than
controllers which do not depend on velocity \cite{nhvc_incline_walking}.
In particular, VNHCs may be capable of injecting and
dissipating energy from a system in a robust manner, all while producing
realistic biological motion. 

Here then is the problem we will solve: design a virtual nonholonomic constraint
so that the acrobot performs giant-like motion.
We will rigorously prove the acrobot ``gains energy" according to the following
definition.

\begin{defn}\label{defn:energy-gain}
    Let \(\mathcal{Q}\) be an
    \(n\)-dimensional generalized cylinder.
    Let \(f : \mathcal{Q} \rightarrow \mathcal{Q}\times\R^n\) be a smoth vector
    field and let \(D \subset M\) be open.
    The system described by \(\dot{x} = f(x)\) 
    \textit{gains energy on \(D\)} if, 
    for all compact sets \(K \subset D\) and for almost every initial
    condition \(x(0) \in K\), there exists \(T > 0\) such
    that \(x(t) \notin K \, (\forall t > T)\).
    The system \textit{loses energy on \(D\)} if it gains energy in
    negative-time.
\end{defn}

Before embarking on this design problem, let us summarize the relevant theory of
VNHCs.

%/========== VNHC ==========/%
\section{Theory of VNHCs}\label{sec:vnhc}

\subsection{Simply Actuated Hamiltonian Systems}
Take a mechanical system modelled with generalized coordinates 
\(q = (q_1, \ldots, q_n)\) on a configuration manifold
\(\mathcal{Q} = \Rt{T_1} \times \cdot \Rt{T_n}\), where
\(T_i = 2\pi\) if \(q_i\) is an angle and \(T_i = \infty\) if \(q_i\) is a
displacement. The corresponding generalized velocities are 
\(\dot{q} = (\dot{q}_1,\ldots,\dot{q}_n) \in \R^n\).

Suppose this system has Lagrangian
\(\mathcal{L}(q,\dot{q}) = 1/2~\dot{q}^T D(q) \dot{q} - P(q)\),
where the potential energy 
\(P : \mathcal{Q} \rightarrow \mathbb{R}\) 
is smooth, and the inertia matrix 
\(D : \mathcal{Q} \rightarrow \mathbb{R}^{n \times n}\)
is smooth and positive definite for all \(q \in \mathcal{Q}\).
The \textit{conjugate of momentum} to \(q\) is the vector
\(p := \partial\mathcal{L}/\partial\dot{q} = D(q) \dot{q} \in \R^n\).
As per \cite{landau_mechanics}, 
the \textit{Hamiltonian} of the system in \((q,p)\) coordinates
is
\begin{equation}\label{eqn:hamiltonian}
    \mathcal{H}(q,p) = \frac{1}{2} p\tpose D\inv(q) p + P(q)
    ,
\end{equation}
with dynamics
\begin{equation}\label{eqn:hamiltonian-eom-general}
    \begin{cases}
        \dot{q} = \nabla_p\mathcal{H} 
        , \\
        \dot{p} = -\nabla_q\mathcal{H} + B(q) \tau
        ,
    \end{cases}
\end{equation}
where \(\tau \in \R^k\) is a vector of generalized input forces and the input
matrix \(B : \mathcal{Q} \rightarrow \R^{n \times k}\) is full rank for all 
\(q \in \mathcal{Q}\).
If \(k < n\), we say the system is \textit{underactuated} with degree of
underactuation \((n-k)\).

It is easy to show using the matrix Kronecker product that
\eqref{eqn:hamiltonian-eom-general} expands to
\begin{equation}\label{eqn:hamiltonian-full-dynamics}
     \begin{cases}
        \dot{q} = D\inv(q)p 
        , \\
        \dot{p} = -\frac{1}{2} (\Id{n} \otimes p\tpose) \nabla_q D\inv(q) p
        - \nabla_q P(q) + B(q) \tau
        . \\
    \end{cases}
\end{equation}

Because \(\tau\) is transformed by \(B(q)\), it is not obvious how any
particular input force \(\tau_i\) affects the system.
As a first step in addressing this problem, we make the following assumptions.

\begin{assm}\label{assm:B-const}
    The input matrix \(B(q) \equiv B \in \R^{n\times k}\) is constant,
    full rank, and \(k < n\).
\end{assm}

\begin{assm}\label{assm:B-perp}
    There exists a right semi-orthogonal matrix 
    \(B^\perp \in \R^{(n-k)\times n}\)
    which is a left-annihilator for \(B\). 
\end{assm}

Note that Assumption \ref{assm:B-perp} requires the rows of \(B^\perp\) be unit vectors
that are mutually orthogonal. 
When \(k = (n-1)\), Assumption \ref{assm:B-perp} can be removed because it is
automatically implied by Assumption \ref{assm:B-const}.

The above assumptions allow us to define a
canonical coordinate transformation of ~\eqref{eqn:hamiltonian} 
which decouples the input forces.
To define this transformation we will make use of the following lemma.

\begin{lemma}\label{lemma:B-orthogonal}
    Suppose Assumption \ref{assm:B-const} holds. Then
    there exists a nonsingular matrix \(\hat{T} \in \R^{k \times k}\) 
    so that the regular feedback transformation 
    \[
        \tau = \hat{T} \hat{\tau}
    \] 
    has a new input matrix \(\hat{B}\) for \(\hat{\tau}\) which is left
    semi-orthogonal.  
\end{lemma}
\begin{proof}
    Since \(B\) is constant and full rank, it has a singular value decomposition 
    \(B = U\tpose \Sigma V\) where 
    \(\Sigma = [\diag{\sigma_1,\ldots,\sigma_k}; \Zmat{(n-k)\times k}]\),
    \(\sigma_i > 0\), and \(U \in R^{n \times n}\),
    \(V \in \R^{k \times k}\) are unitary matrices \cite{calculating_svd}.
    Defining \(T = \diag{1/\sigma_1^2,\ldots,1/\sigma_k^2}\) and assigning the
    regular feedback transformation \(\tau = V T \hat{\tau}\) yields a new input
    matrix \(\hat{B} = B V T\) for \(\hat{\tau}\) such that
    \(\hat{B}\tpose \hat{B} = T\tpose \Sigma\tpose \Sigma T = \Id{k}\).
\end{proof}

In light of Lemma \ref{lemma:B-orthogonal}, there is no loss of generality in
assuming that the input matrix is left semi-orthogonal.
Now, let \(\mathbf{B} := [B^\perp; B\tpose]\).
Since \(B^\perp\) is a left annihilator of \(B\) and both \(B^\perp\) and
\(B\tpose\) are right semi-orthogonal, one can easily show that \(\mathbf{B}\) is
an orthogonal matrix.

\begin{thm}\label{thm:simply-actuated}
    Take the Hamiltonian system ~\eqref{eqn:hamiltonian} and suppose
    Assumptions \ref{assm:B-const} and \ref{assm:B-perp} hold.
    The coordinate transformation
    \(\left(\tilde{q} = \mathbf{B}q, \tilde{p} = \mathbf{B}p\right)\)
    is a canonical transformation and the resulting dynamics are given by 
    \begin{gather}\label{eqn:simple-hamiltonian}
        \mathcal{H}(\tilde{q},\tilde{p}) = 
        \frac{1}{2} \tilde{p}\tpose \Minv(\tilde{q}) \tilde{p} + V(\tilde{q})
        , \\
       \begin{cases}
           \dot{\tilde{q}} = \Minv(\tilde{q})\tilde{p}
           , \\
           \dot{\tilde{p}} = -\frac{1}{2} (\Id{n} \otimes \tilde{p}\tpose)
           \nabla_{\tilde{q}} \Minv(\tilde{q}) \tilde{p} \\
           \phantom{---} - \nabla_{\tilde{q}} V(\tilde{q}) + \simpleB \tau
            ,
        \end{cases} \nonumber
    \end{gather}
    where 
    \(\Minv(\tilde{q}) := 
    \mathbf{B}D^{-1}(\mathbf{B}\tpose \tilde{q})\mathbf{B}\tpose\)
    and
    \(V(\tilde{q}) := P(\mathbf{B}\tpose \tilde{q})\).
\end{thm}
\begin{proof}
    Since \(\mathbf{B}\) is constant, this transformation satisfies
    \(\partial\tilde{q}_i/\partial p_j = \partial\tilde{p}_i/\partial q_j = 0\) for all 
    \(i,j \in \{1,\ldots,n\}\).
    This implies the Poisson brackets \([\tilde{q}_i, \tilde{q}_j]\)
    and \([\tilde{p}_i,\tilde{p}_j]\) are both zero.
    Then, since \(\mathbf{B}\) is orthogonal, 
    \([\tilde{p}_i, \tilde{q}_j] = (\mathbf{B}_i)\tpose (\mathbf{B}\tpose)_j
        = \delta_i^j\).
    By (45.10) in \cite{landau_mechanics}, this transformation is canonical and
    the new Hamiltonian is
    \(\mathcal{H}(\mathbf{B}\tpose \tilde{q}, \mathbf{B}\tpose \tilde{p})\).
    Finally, since \(\dot{\tilde{p}} = \mathbf{B} \dot{p}\), the input
    matrix for the system in \((\tilde{q},\tilde{p})\) coordinates is
    \(\mathbf{B}B = [\Zmat{(n-k)\times k}; \Id{k}]\), which proves the theorem.
\end{proof}

We call the \((\tilde{q},\tilde{p})\) coordinates
\textit{simply actuated coordinates}.
The first \((n-k)\) configuration variables in \(\tilde{q}\), labelled \(q_u\),
are the \textit{unactuated coordinates}; 
the remaining \(k\) configuration variables, labelled \(q_a\), are the
\textit{actuated coordinates}.
The corresponding \((p_u, p_a)\) in \(\tilde{p}\) are the \textit{unactuated}
and \textit{actuated momenta}, respectively.

\subsection{Virtual Nonholonomic Constraints}

The notion of a nonholonomic constraint which can be stabilized via state feedback
was first described by Griffin and Grizzle in \cite{nhvc_dynamic_walking}.
Horn et. al later extended their results in
\cite{hybrid_zero_dynamics_bipedal_nhvcs} and
derived the dynamics of constrained systems.
In this section we reformulate these ideas for the Hamiltonian framework,
because the theory is cleaner when using unactuated and actuated momenta.
For this reason, we take the system of inquiry to be a Hamiltonian
mechanical system in simply actuated coordinates, as in
\eqref{eqn:simple-hamiltonian}.
For simplicity of notation, we relabel \((\tilde{q},\tilde{p})\) as \((q,p)\),
where we now have \((q,p) = (q_u,q_a,p_u,p_a)\).

\begin{defn}\label{defn:vnhc}
    A \textit{virtual nonholonomic constraint} (VNHC) \textit{of order \(k\)} is a
    relation \(h(q,p) = 0\) where \(h : \mathcal{Q}\times\R^n \rightarrow \R^k\) is
    \(C^2\), \(\rank{\left[ dh_q,\, dh_p \right]} = k\) for all 
    \((q,p) \in h\inv(0)\), and there exists a feedback controller \(\tau(q,p)\)
    rendering the \textit{constraint manifold} \(\Gamma\) invariant,
    where
    \begin{equation}
        \Gamma = \left\{(q,p) \mid h(q,p) = 0, dh_q \dot{q} + dh_p \dot{p} = 0\right\}
        .
    \end{equation}
\end{defn}

The constraint manifold is a \(2(n-k)\)-dimensional
closed embedded submanifold of \(\mathcal{Q} \times \R^n\).
A VNHC thereby allows us to study a reduced-order model of the system: it reduces
the original \(2n\) differential equations to \(2(n-k)\) equations.
In particular, if \(k = (n-1)\), the constraint manifold is \textit{always}
2-dimensional and its dynamics can be plotted on a plane. 

We often want to stabilize a constraint within some neighbourhood of \(\Gamma\).
To see when this is possible, let us define the error output \(e = h(q,p)\).
If any component of \(e_i\) has relative degree 1, we may not be able
to stabilize \(\Gamma\) -- we can always guarantee \(e_i \to 0\), but not
necessarily \(\dot{e}_i \to 0\).
It is for this reason that we define the following special type of VNHC.

\begin{defn}
    A VNHC \(h(q,p) = 0\) of order \(k\) is \textit{regular} if the output 
    \(e = h(q,p)\) is of relative degree \(\{2,2.\ldots,2\}\) everywhere on the
    constraint manifold \(\Gamma\).
\end{defn}

The authors of
\cite{nhvc_dynamic_walking,hybrid_zero_dynamics_bipedal_nhvcs}
observed that relations which use only the unactuated conjugate of momentum
often have vector relative degree \(\{2,\ldots,2\}\).
Indeed, we shall now provide a characterization of regularity which shows that
regular constraints cannot use the actuated momentum at all.

To ease notation in the rest of this section, we use the following shorthand:
\begin{align}
    \mathcal{A}(q,p_u) &:= dh_q(q,p_u) \Minv(q) 
        ,\\
    \mathcal{M}(q,p) &:= (\Id{n-k} \otimes p\tpose)\nabla_{q_u}\Minv(q) 
    .
\end{align}

\begin{thm}\label{thm:vnhc-regularity}
    A relation \(h(q,p) = 0\) for system ~\eqref{eqn:simple-hamiltonian}
    is a regular VNHC of order \(k\) if and only if 
    \(dh_{p_a} = \Zmat{k \times k}\) 
    and
    \[
        \rank{\left(\mathcal{A}(q,p_u) - dh_{p_u}\mathcal{M}(q,p)\right)\simpleB} = k
         ,
    \]
    everywhere on the constraint manifold \(\Gamma\).
\end{thm}
\begin{proof}
    \textbf{TODO: should we make this a ``proof sketch"?}
    Let \(e = h(q,p) \in \R^k\).
    If \(dh_{p_a} \neq \Zmat{k\times k}\) for some \((q,p) \in \Gamma\), 
    then \(\tau\) appears in \(\dot{e}\) and the VNHC is not of relative degree
    \(\{2,\ldots,2\}\). Suppose now that \(dh_{p_a} = \Zmat{k\times k}\).
    Then 
    \(\dot{e} = \mathcal{A}(q,p_u)p - 
     dh_{p_u}\left(1/2~\mathcal{M}(q,p)p + \nabla_{q_u}V(q)\right)\).
    Taking one further derivative provides
    \( \ddot{e} = (\star) - 
        dh_{p_u}\left(1/2~d/dt~\left(\mathcal{M}(q,p)p\right)\right) 
        + \mathcal{A}(q,p_u)[\Zmat{(n-k)\times k};\Id{k}] \tau\),
    where \((\star)\) is a continuous function of \(q\) and \(p\).
    One can further show that
    \(dh_{p_u}\left(1/2~d/dt~\left(\mathcal{M}(q,p)p\right)\right)
        = (\star) + dh_{p_u}\mathcal{M}(q,p)[\Zmat{(n-k)\times k};
        \Id{k}]\tau\).
    Hence,
    \[
       \ddot{e} = (\star) +
       \left(\mathcal{A}(q,p_u) - dh_{p_u}\mathcal{M}(q,p)\right) \simpleB \tau
        ,
    \]
    which we write as \( \ddot{e} = E(q,p) + H(q,p)\tau\) for appropriate \(E\)
    and \(H\).
    From the definition of regularity, the VNHC \(h\) is regular 
    when \(e\) is of relative degree \(\{2,\ldots,2\}\), which is true 
    if and only if the matrix premultiplying \(\tau\) is nonsingular, and hence
    that \(H\) is invertible. This proves the theorem.
\end{proof}

Under additional mild conditions (see \cite{vhcs_for_el_systems}), a regular VNHC of
order \(k\) can be stabilized by the output-linearizing phase-feedback
controller
\begin{equation}
    \tau(q,p) = -H\inv(q,p)\left(E(q,p) + k_p e + k_d \dot{e}\right)
    ,
\end{equation}
where \(k_p, k_d > 0\) are control parameters which can be tuned on the
resulting error dynamices \(\ddot{e} = -k_p e - k_d \dot{e}\).

In Section \ref{sec:acrobot} we will enforce a regular constraint on the
acrobot of the form \(h(q,p) = q_a - f(q_u,p_u)\), where the actuators track a
function of the unactuated variables.
Intuitively then, the constrained dynamics should be parameterized by \((q_u, p_u)\).
Unfortunately, \(\dot{q}_u\) depends on \(p_a\), and for general systems one
cannot solve explicitly for \(p_a\) in terms of \((q_u,p_u)\) because
the \(\dot{p}\) dynamics contains the coupling term 
\((\Id{n} \otimes p\tpose)\nabla_{q}M(q)p\). 

We now introduce an assumption which allows us to solve for \(p_a\) as a
function of \((q_u,p_u)\), which in turn allows us to find the constrained
dynamics.

\begin{assm}\label{assm:inertially-actuated}
The inertia matrix does not depend on the unactuated coordinates, so that 
\(\nabla_{q_u}M(q) = \Zmat{n(n-k) \times n}\).
\end{assm}

\begin{thm}\label{thm:zero-dynamics}
    Let \(\mathcal{H}\) be a mechanical system in simply actuated
    coordinates satisfying Assumption \ref{assm:inertially-actuated}. 
    Let \(h(q,p_u) = 0\) be a regular VNHC of order \(k\) with constraint
    manifold \(\Gamma\). Suppose that on \(\Gamma\) one can solve for
    \(q_a\) as a function \(q_a = f(q_u,p_u)\).
    Then the constrained dynamics are given by
    \begin{equation}\label{eqn:qpu-dynamics}
        \left.\begin{aligned}
                \dot{q}_u &= 
                \left[\Id{(n-k)} ~ \Zmat{(n-k) \times k}\right]
                \Minv(q)p \\
            \dot{p}_u &= -\nabla_{q_u}V(q) \\
            \end{aligned}{}\right|_{\begin{array}{c}
                q_a = f(q_u,p_u) \\ 
                p_a = g(q_u,p_u) \\
            \end{array}}
            ,
    \end{equation}
    where
    \begin{align}\label{eqn:g-qpu}
        &g(q_u,p_u) := 
        \left(\mathcal{A}(q,p_u)[\Zmat{(n-k)\times k};\Id{k}] \right)\inv 
        (dh_{p_u} \nabla_{q_u}V(q) \nonumber
        \\
        &- \mathcal{A}(q,p_u)[\Id{(n-k)};\Zmat{k \times(n-k)}]p_u
        \left.)\right|_{q_a = f(q_u,p_u)}
        .
    \end{align}
\end{thm}
\begin{proof}
    Setting \(e = h(q,p_u)\) and using Assumption
    \ref{assm:inertially-actuated}, we find that
    \(\dot{e} = \mathcal{A}(q,p_u)p - dh_{p_u}\nabla_{q_u}V(q)\).
    Notice that
    \(\mathcal{A}(q,p_u)p = \mathcal{A}(q,p_u)[\Zmat{(n-k)\times k}; \Id{k}]p_a
    + \mathcal{A}(q,p_u)[\Id{n-k};\Zmat{k \times (n-k)}] p_u\).
    Since \(h(q,p_u)\) is regular, \(\mathcal{A}(q,p_u)\) is invertible.
    Taking \(e = \dot{e} = 0\), solving for \(p_a\), and setting 
    \(q_a = f(q_u,p_u)\) completes the proof.
\end{proof}

\textbf{Comparison with existing literature}: Horn et.al. provide the constrained
dynamics for VNHCs in \cite{nhvc_incline_walking}.
Their assumption \textbf{H2} is what we call regularity, and our requirement
that one can solve for \(q_a = f(q_u,p_u)\) on \(\Gamma\) implies their
assumption \textbf{H3} holds true.
The only real distinction between this section and their work
is that our constrained dynamics are explicit functions of the Hamiltonian phase
coordinates \((q_u,p_u)\).
In fact, our constrained dynamics \eqref{eqn:qpu-dynamics} coincide with their
system (17) when one chooses the special case \(\theta_1 = q_u\) and 
\(\theta_2 = p_u\).
This explicit representation will be beneficial when we apply the theory of
VNHCs to the acrobot.

%/========== Acrobot ==========/%
\section{The Acrobot VNHC}\label{sec:acrobot}
specialize the theory of VNHCs to the acrobot. consider constraints which depend
only on pu, and show arctan vnhc and reduced dynamics. 
give definition of energy injection/dissipation. conclude with our
theorem. Make this section short and sweet.

TODO: after writing S2, re-introduce what we want to do with acrobot (energy
injection for giant stabilization)

For our model of the acrobot, we assume the acrobot iscomprised of two massless
rods of equal length \(l\), with equal point masses \(m\) at the tips.
We call this a \textit{simple} acrobot, which is displayed in Figure
\ref{fig:simple-acrobot-model}.
We will also ignore any frictional forces at both the hip and shoulder joints. 
Finally, it is important to note that a real gymnast cannot swing their legs in
full circles, though they are usually flexible enough to raise them parallel to
the floor; 
for this reason, we assume that \(q_a \in [-Q_a, Q_a]\) where 
\(Q_a \in [\frac{\pi}{2}, \pi[\). 

\begin{figure}
    \centering
    %\includegraphics[width=0.5\textwidth]{images/simple_acrobot_model}
    \caption{A simple acrobot has massless rods of equal length \(l\) and 
    equal masses \(m\) at the tips.}
    \label{fig:simple-acrobot-model}
\end{figure}

This system has inertia matrix 
\(M\), potential function \(V\) (with respect to
the horizontal bar), and input matrix \(B\) given as follows:
\begin{align}\label{eqn:acrobot-inertia}
    M(q) &= \begin{bmatrix}
        ml^2\left(3+2\cos(q_a)\right) & 
        ml^2\left(1+\cos(q_a)\right) \\
        ml^2\left(1+\cos(q_a)\right) &
        ml^2
    \end{bmatrix} 
    , \\
    \label{eqn:acrobot-potential}
    V(q) &= -mgl\left(2\cos(q_u)+\cos(q_u+q_a)\right)
    , \\
    \label{eqn:acrobot-B}
    B &= [0;1]
    .
\end{align}
The conjugate of momenta is \(p = (p_u,p_a) = M(q)\dot{q}\).
The dynamics of the acrobot in \((q,p)\) coordinates are given in
~\eqref{eqn:acrobot-hamiltonian}, where
for shorthand, we write \(c_u := \cos(q_u)\), \(c_a := \cos(q_a)\), and 
\(c_{ua} := \cos(q_u + q_a)\); likewise, \(s_u := \sin(q_u)\), 
\(s_a := \sin(q_a)\), and \(s_{ua} := \sin(q_u + q_a)\).
\begin{align}\label{eqn:acrobot-hamiltonian}
    \mathcal{H}(q,p) &= \frac{1}{2}p\tpose \Minv(q) p -
    mgl\left(2 c_u + c_{ua}\right)
    , \\
     &\begin{cases}
        \dot{q} = \Minv(q) p 
        ,\\
        \dot{p}_u = -mgl\left(2s_u + s_{ua}\right) 
        ,\\
        \dot{p}_a =-\frac{1}{2}p\tpose \nabla_{q_a}\Minv(q) p
        - mgl s_{ua} + \tau,
    \end{cases} \nonumber
\end{align}
The control input is a force \(\tau \in \R\) affecting only the dynamics of
\(p_a\), representing a torque acting on the hip joint.
This means \((q,p)\) are simply actuated coordinates inside the phase space
\(\mathcal{Q} \times \mathcal{P}\) where
\(\mathcal{Q} = \mathcal{Q}_u \times \mathcal{Q}_a 
:= \Sone \times \Sone\), and
\(\mathcal{P} = \mathcal{P}_u \times \mathcal{P}_a
:= \R \times \R\).
Hence, we can apply the theory of VNHCs from Section \ref{sec:vnhc} to design an
energy-injecting controller.
Since we need our VNHC to be regular, the following proposition will prove
useful.
\begin{prop}\label{prop:acrobot-fpu-regular}
    A relation \(h(q,p) = q_a - f(p_u) = 0\) 
    with \(f \in C^2\left(\mathcal{P}_u; \mathcal{Q}_a\right)\) is a regular
    VNHC of order 1 for the simple acrobot.
\end{prop}
\begin{proof}
    The regularity matrix for this VNHC evaluates to
    \(\frac{(1+c_a)\partial_{q_u}f(q_u,p_u) + (3+2c_a)}{ml^2(2-c_a^2)}\).
    Since \(\partial_{q_u} f = 0\), this matrix is strictly positive for all values
    of \(q_a\), and hence is full rank everywhere on the constraint manifold.
\end{proof}

%/========== Simulation ==========/%
\section{Simulation Results}\label{sec:simulations}

%/========== Experiments ==========/%
\section{Experimental Results}\label{sec:experiments}

%/========== Proof ==========/%
\section{Proof of Theorem \textbf{TODO: ref theorem}}\label{sec:proof}

%/========== Conclusion ==========/%
\section{Conclusion}\label{sec:conclusion}


%---------- Bibliography ----------%
\bibliography{bib}
\end{document}
%/========== /Main Document ==========/%
% vim: set tw=80 ts=4 sw=4 sts=0 et ffs=unix :
