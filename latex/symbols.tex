%! TEX root = main.tex

%/========== Symbols ==========/%

% Set the heading so it appears in the Table of Contents
\addcontentsline{toc}{chapter}{List of Symbols}
\chapter*{List of Symbols}

% Define a table for the symbols using the booktabs convention
\noindent
\begin{table}[h!]
\resizebox{\textwidth}{!}{%
\centering
\begin{tabular}{ll}
   \toprule
   Symbol & Definition \\
   \midrule
   \(\n\) & The index set \(\{1,\ldots,n\}\) of natural numbers up to \(n\). \\
   \(\R^n\) & Real numbers in \(n\) dimensions. \\
   \(\Rt{T}\) & Real numbers modulo \(T > 0\), with \(\Rt{\infty} = \R\). \\
   \(\mathbb{S}^1\) & The unit circle, equivalent to \(\Rt{2\pi}\). \\
   \(\mathcal{Q}\) & The configuration manifold of a system. \\
   \(\R^{n \times m}\) & The space of real-valued matrices with \(n\) rows and \(m\) columns. \\
   \(\Id{n}\) & The \(n \times n\) identity matrix. \\
   \(\Zmat{n \times m}\) & The \(n \times m\) matrix of all zeros. \\
   \(M_i\) & If \(M\) is a vector, the \(i\)th element of \(M\). \\
         & If \(M\) is a matrix, the \(i\)th column of \(M\). \\
   \(M_{i,j}\) & The value of row \(i\), column \(j\) for the matrix \(M\). \\
   \(\dot{x}\) & Derivative of \(x\) with respect to time \(t\). \\
   \(\nabla_v F\) & If \(F\) is \(\R\)-valued, the gradient of \(F\) with respect to \(v\). \\
            & If \(F : \R^m \rightarrow \R^{n\times n}\), the block matrix gradient \((\pdiff{F}{v_1},\ldots,\pdiff{F}{v_m}) \in \R^{nm \times n}\). \\
   \(dF_v\) & Total differential (Jacobian) of \(F\), equivalent to \((\nabla_v F)\tpose\). \\
   \(\Hess F\) & If \(F : \R^n \rightarrow \R\), the \(n \times n\) Hessian matrix of double derivatives of \(F\). \\
               & If \(F : \R^n \rightarrow \R^k\), the block matrix \((\Hess F_1, \ldots, \Hess F_k) \in \R^{n \times nk}\). \\
   \(\partial_v \partial_w F\) & Derivative matrix of \(F : \R^n \times \R^m \rightarrow \R\), with \((i,j)\) element \(\ppdiff{F}{v_i}{w_j}\). \\
   \(\delta_{i,j}\) & The Kronecker delta: 1 if \(i = j\) and 0 otherwise. \\
   \(\otimes\) & The matrix kronecker product (see Appendix \ref{appendix:matrix-kron-prod}). \\
   \bottomrule
\end{tabular}

}% resizebox
\end{table}

%---------- Nomenclature ----------%
%\chapter*{Nomenclature}
%
%% Define a table for the symbols using the booktabs convention
%\noindent
%\begin{table}[h!]
%\resizebox{\textwidth}{!}{%
%\centering
%\begin{tabular}{ll}
%   \toprule
%   Acronym & Meaning \\
%   \midrule
%   DOF & Degrees of freedom
%   EOC & Equation of constraint \\
%   EOM & Equation of motion \\
%   ODE & Ordinary differential equation \\
%   VNHC & Virtual nonholonomic constraint \\
%   WLOG & Without loss of generality \\
%   \bottomrule
%\end{tabular}
%
%}% resizebox
%\end{table}
%/========== /Symbols ==========/%
% vim: set ts=3 sw=3 sts=0 et ffs=unix :
