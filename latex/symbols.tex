%! TEX root = main.tex

%/========== Symbols ==========/%

% Set the heading so it appears in the Table of Contents
\addcontentsline{toc}{chapter}{List of Symbols}
\chapter*{List of Symbols}

% Define a table for the symbols using the booktabs convention
\noindent
\begin{table}[h!]
\resizebox{\textwidth}{!}{%
\centering
\begin{tabular}{ll}
   \toprule
   Symbol & Definition \\
   \midrule
   \(\R^n\) & Real numbers in \(n\) dimensions. \\
   \(\R^{n \times m}\) & Real-valued matrix with \(n\) rows and \(m\) columns. \\
   \(\Rt{T}\) & Real numbers modulo \(T > 0\), with \(\Rt{\infty} = \R\). \\
   \(\mathbb{S}^1\) & The unit circle, equivalent to \(\Rt{2\pi}\). \\
   \([M]_{i,j}\) & The value of row \(i\), column \(j\) for the matrix \(M\). \\
   \(\dot{x}\) & Derivative of \(x(t,y)\) with respect to time \(t\). \\
   \(x'\) & Derivative of \(x(t,y)\) with respect to a non-time input \(y\). \\
   \(\mathcal{Q}\) & The configuration manifold of a mechanical system. \\
   \(q_u\) & Unactuated coordinates. \\
   \(q_a\) & Actuated coordinates. \\
   \(p_u\) & Conjugate of momentum to \(q_u\). \\
   \(p_a\) & Conjugate of momentum to \(q_a\). \\
   \bottomrule
\end{tabular}

}% resizebox
\end{table}

%/========== /Symbols ==========/%
% vim: set ts=3 sw=3 sts=0 et ffs=unix :
