%! TEX root = main.tex

%/========== Symbols ==========/%

% Set the heading so it appears in the Table of Contents
\addcontentsline{toc}{chapter}{List of Symbols}
\chapter*{List of Symbols}

% Define a table for the symbols using the booktabs convention
\noindent
\begin{table}[h!]
\resizebox{\textwidth}{!}{%
\centering
\begin{tabular}{ll}
   \toprule
   Symbol & Definition \\
   \midrule
   \(\R^n\) & Real numbers in \(n\) dimensions. \\
   \(\Rt{T}\) & Real numbers modulo \(T > 0\), with \(\Rt{\infty} = \R\). \\
   \(\mathbb{S}^1\) & The unit circle, equivalent to \(\Rt{2\pi}\). \\
   \(\R^{n \times m}\) & The space of real-valued matrices with \(n\) rows and \(m\) columns. \\
   \(\Id{n}\) & The \(n \times n\) identity matrix. \\
   \(\Zmat{n \times m}\) & The \(n \times m\) matrix of all zeros. \\
   \(M_{i,j}\) & The value of row \(i\), column \(j\) for the matrix \(M\). \\
   \(A_i\) & If \(A\) is a vector, the \(i\)th element of \(M\). \\
         & If \(A\) is a matrix, the \(i\)th row of \(M\). \\
   \([v_i]\) & If \(v_i \in \R\), the column vector with value \(v_i\) at position \(i\). \\
       & If \(v_i\) is a row vector, the matrix with row \(i\) given by \(v_i\). \\
   \(\dot{x}\) & Derivative of \(x\) with respect to time \(t\). \\
   \(\nabla_v F\) & Gradient of \(F)\) with respect to the vector \(v\). \\
   \bottomrule
\end{tabular}

}% resizebox
\end{table}

%/========== /Symbols ==========/%
% vim: set ts=3 sw=3 sts=0 et ffs=unix :
