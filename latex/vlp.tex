%! TEX root = main.tex

%/========== Variable Length Pendulum ==========/%
\chapter{Application of VNHCS: The Variable Length Pendulum}\label{sec:vlp}
\section{Motivation}
The variable length pendulum (VLP) is modelled as a point with mass \(m\)
connected to a pivot by a massless rod of length \(l\).

% TODO: Why is this a good problem to study? What is the problem? What would a human do on a standing swing?

Before computing the dynamics of the VLP, we must make it clear that this
VLP model uses \(l\) directly as the control input to the system, rather than the
generalized forces \(\tau\) described in Section \ref{sec:lagrangian-mechanics}.
Since the control input is not considered a configuration variable of the
system, the configuration of the VLP is
the angle \(q \in \mathbb{S}^1\) between the rod and the vertical axis. We set
\(q = 0\) when the mass is at the bottom of its swing.

With this in mind, we can now compute the Hamiltonian dynamics of the system.
The cartesian position of the mass is given by \(x = (l\sin(q),l\cos(q))\),
which has velocity 
\(\dot{x} = (\dot{l}\sin(q) + l\cos{q}\dot{q}, \dot{l}\cos(q) - l\sin{q}\dot{q})\)
Computing the kinetic energy \(T\) yields
\[
   T(q,\dot{q}) = \frac{1}{2}m\norm{\dot{x}}^2 = \frac{1}{2}m\left(\dot{l}^2 +
   l^2\dot{q}^2\right)
\]
The potential energy \(P\) with respect to the pivot (under a gravitational
acceleration \(g\)) is
\[
   P(q) = -mgl\cos(q)
\]
which means, the Lagrangian is
\[
   \mathcal{L}(q,\dot{q}) = \frac{1}{2}m\left(\dot{l}^2 + l^2\dot{q}^2\right)
      + mgl\cos(q)
\]
Computing the conjugate of momentum to \(q\), we get 
\[
   p = \pdiff{\mathcal{L}}{\dot{q}} = ml^2\dot{q}
\]
Since \(q\) is not actuated by any generalized force \(\tau\), the system
is simply actuated. 
Solving for the Hamiltonian \(\mathcal{H} = \dot{q}p - \mathcal{L}\)
and applying Theorem \ref{thm:simply-actuated}, we find the dynamics of the VLP
are given in Hamiltonian form by (\ref{eqn:vlp-hamiltonian}). 
\begin{align}\label{eqn:vlp-hamiltonian}
   \mathcal{H}(q,p) &= \frac{p^2}{2ml^2} - \frac{1}{2}m\dot{l}^2 - mgl\cos(q) \\
     &\begin{cases}
        \dot{q} = \frac{p}{ml^2} \\
        \dot{p} = -mgl\sin(q) \\
   \end{cases} \nonumber
\end{align}


\section{The VLP Constraint}
% TODO: Go through the development of the constraint and prove it gains energy
% TODO: Explain how we came up with the constraint first. Talk about the child
% on a swing, the paper showing the time-optimal control input, and describe why
% we want to remove time from the equation. Discuss the lmin and lmax choices.
% Why do we need to use a VNHC instead of a VHC: we are modelling the
% time-optimal controller in a time-independent way, which requires knowledge of
% which quadrant in (q,p)-space we're in. This means using the angle theta,
% which is inherently a VNHC.

\begin{thm}
For the variable-length pendulum, define \(\theta := \arctan_2(p,q)\). 
A VNHC of the form \(l = l(\theta)\) injects energy if there exists \(l_{avg} \in \R_{>0}\) such that
\[
   \left(l(\theta) - l_{avg}\right)sin(2\theta) \leq 0 \text{ }\forall \theta \in \mathbb{S}^1
\]
with the property that the inequality is strict for almost every \(\theta\).
\end{thm}
\begin{proof}
    Choose, as a candidate anti-Lyapunov function, the energy for the average-length pendulum 
    \[
       E_{avg}(q,p) := \frac{1}{2}\frac{p^2}{m l_{avg}^2} 
                    + m g l_{avg} (1-\cos(q))
    \]

    which is non-negative and has derivative 
    \[
      \dot{E}_{avg} = \frac{-g\sin(q)p \left(l(\theta)^3 - l_{avg}^3\right)}
                 {l_{avg}^2l(\theta)^2}
    \]

    We will show that \(E_{avg}\) is increasing.

    % VBOX UNDERFULL WARNING HERE, IGNORE IT
    Observe that \(\sign{\sin(q)p} = \sign{\sin(2\theta)}\) and, 
    by Lemma \textbf{TODO: REF LEMMA},
    \( \sign{l(\theta)^3 - l_{avg}^3} 
     = \sign{ l(\theta) - l_{avg}}\). 

     Then the derivative of \(E_{avg}\) is almost always positive, since
     \begin{align*}
        \label{eq:}
        \sign{\dot{E}_{avg}} &= \sign{-\sin(q)p \left(l(\theta)^3 - l_{avg}^3\right)} \\
                    &= -\sign{\sin(2\theta) \left(l(\theta) - l_{avg}\right)} \\
                    &\geq 0 \text{ (by assumption)}
     \end{align*}

      Hence, \(E_{avg}\) is an anti-Lyapunov function with positive derivative,  
      so the variable-length pendulum is gaining energy.
\end{proof}

\section{Simulation Results}

%/========== /Variable Length Pendulum ==========/%
% vim: set ts=3 sw=3 sts=0 et tw=80 ffs=unix :
