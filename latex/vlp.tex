%! TEX root = main.tex

%/========== Variable Length Pendulum ==========/%
\chapter{Application of VNHCS: The Variable Length Pendulum}\label{sec:vlp}
\section{Motivation}
% TODO: Why is this a good problem to study? What is the problem? What would a human do on a standing swing?
\section{The VLP Constraint}
% TODO: Go through the development of the constraint and prove it gains energy

\begin{thm}
For the variable-length pendulum, define \(\theta := \arctan_2(p,q)\). 
A VNHC of the form \(l = l(\theta)\) injects energy if there exists \(l_{avg} \in \R_{>0}\) such that
\[
   \left(l(\theta) - l_{avg}\right)sin(2\theta) \leq 0 \text{ }\forall \theta \in \mathbb{S}^1
\]
with the property that the inequality is strict for almost every \(\theta\).
\end{thm}
\begin{proof}
    Choose, as a candidate anti-Lyapunov function, the energy for the average-length pendulum 
    \[
       E_{avg}(q,p) := \frac{1}{2}\frac{p^2}{m l_{avg}^2} 
                    + m g l_{avg} (1-\cos(q))
    \]

    which is non-negative and has derivative 
    \[
      \dot{E}_{avg} = \frac{-g\sin(q)p \left(l(\theta)^3 - l_{avg}^3\right)}
                 {l_{avg}^2l(\theta)^2}
    \]

    We will show that \(E_{avg}\) is increasing.

    % VBOX UNDERFULL WARNING HERE, IGNORE IT
    Observe that \(\sign{\sin(q)p} = \sign{\sin(2\theta)}\) and, 
    by Lemma \textbf{TODO: REF LEMMA},
    \( \sign{l(\theta)^3 - l_{avg}^3} 
     = \sign{ l(\theta) - l_{avg}}\). 

     Then the derivative of \(E_{avg}\) is almost always positive, since
     \begin{align*}
        \label{eq:}
        \sign{\dot{E}_{avg}} &= \sign{-\sin(q)p \left(l(\theta)^3 - l_{avg}^3\right)} \\
                    &= -\sign{\sin(2\theta) \left(l(\theta) - l_{avg}\right)} \\
                    &\geq 0 \text{ (by assumption)}
     \end{align*}

      Hence, \(E_{avg}\) is an anti-Lyapunov function with positive derivative,  
      so the variable-length pendulum is gaining energy.
\end{proof}

\section{Simulation Results}

%/========== /Variable Length Pendulum ==========/%
% vim: set ts=3 sw=3 sts=0 et tw=80 ffs=unix :
