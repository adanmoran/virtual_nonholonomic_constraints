%! TEX root = main.tex

%/========== Variable Length Pendulum ==========/%
\chapter{Application of VNHCS: The Variable Length Pendulum}\label{sec:vlp}
\section{Motivation}
The variable length pendulum (VLP) is a classical underactuated dynamical system
which is often used to model the motion of a person on a swing
\cite{pumping_swing_standing_squatting,how_to_pump_a_swing}.
The VLP also represents the swinging of a crane, the (simplified) motion of a
gymnast on a bar \cite{pendulum_length_giant_gymnastics}, and the
tuned-mass-damper systems which stabilize skyscrapers
\cite{vlp_tuned_mass_damper}, among others.

The motion of the VLP has been well studied (see for instance
\cite{dynamics_periodic_vlp}), and many control mechanisms exist
to stabilize trajectories of the system. While many of these control mechanism
are time-dependent, \cite{vlp_energy_shaping}
offers a time-independent technique to inject energy into the system and
stabilize desired energy level sets. The authors designed a controller through a
technique called \textit{energy shaping}, and proved that their control
mechanism would allow the VLP to achieve any desired energy level set.
However, the energy injection mechanism is
ad-hoc in the sense that it is not derived from any human-like behaviour. In
this chapter, we will use VNHCs to provably stabilize energy level sets as well;
the difference is that the control mechanism we derive will stay true to human
motion and can be implemented by any person on a swing.

\section{Dynamics of the Variable Length Pendulum}
% Derive the dynamics of the VLP in Hamiltonian form
We will model the VLP as a point mass \(m\)
connected to a fixed pivot by a massless rod of varying length \(l\) with angle 
\(q \in \mathbb{S}^1\) from the vertical, as is seen in Figure
\ref{fig:vlp_model}. 
We will also ignore any damping and frictional forces in this model.
In a realistic VLP, the rod length \(l\) varies between some minimum
length \(\underline{l} \geq 0\) and some maximum length 
\(\overline{l} > \underline{l}\). The configuration of the VLP is the vector
\(\mathbf{q} := (q,l) \in \mathbb{S}^1 \times [\underline{l},\overline{l}]\).

\begin{figure}
   \centering
   \includestandalone[width=0.4\textwidth]{images/vlp_model}
   \caption{The representation of the variable length pendulum as a mass at the
      tip of a massless rod.}\label{fig:vlp_model}
\end{figure}

Using this configuration, we will compute the Hamiltonian dynamics of the system.
The cartesian position of the mass at the tip of the pendulum
is given by \(x = (l\sin(q),l\cos(q))\), while its velocity is
\(\dot{x} = (\dot{l}\sin(q) + l\cos{q}\dot{q}, \dot{l}\cos(q) - l\sin{q}\dot{q})\).
Computing the kinetic energy \(T\) yields
\[
   T(\mathbf{q},\dot{\mathbf{q}}) = 
   \frac{1}{2}m\norm{\dot{x}}^2 = \frac{1}{2}m\left(\dot{l}^2 + l^2\dot{q}^2\right)
\]
The potential energy \(P\) with respect to the pivot (under a gravitational
acceleration \(g\)) is
\[
   P(\mathbf{q}) = -mgl\cos(q)
\]
Collecting the kinetic energy into a quadratic form, we get the Lagrangian
\[
   \mathcal{L}(\mathbf{q},\dot{\mathbf{q}}) 
   = \frac{1}{2} \dot{\mathbf{q}}\tpose D(\mathbf{q})\dot{\mathbf{q}} - P(\mathbf{q})
   = \frac{1}{2}
   \begin{bmatrix} \dot{q} & \dot{l} \end{bmatrix}
   \begin{bmatrix}
      ml^2 & 0 \\
      0 & m \\
   \end{bmatrix}
   \begin{bmatrix} 
      \dot{q} \\ \dot{l}
   \end{bmatrix}
   + mgl\cos(q)
\]
Computing the conjugate of momenta to \(\mathbf{q}\), we get 
\[
   \mathbf{p} := \begin{bmatrix} p \\ p_l \end{bmatrix} 
   = \begin{bmatrix} ml^2\dot{q} \\ m\dot{l} \end{bmatrix} 
\]
Performing the Legendre transform on \(\mathcal{L}\) and setting
\(M(\mathbf{q}) := D(\mathbf{q})\), \(V(\mathbf{q}) := P(\mathbf{q})\),
we find the Hamiltonian is equal to the total mechanical energy of the system:
\[
   \mathcal{H} = E = \frac{1}{2}\dot{\mathbf{p}}\tpose \Minv(\mathbf{q})
   \dot{\mathbf{p}}+V(\mathbf{q})
\]
Taking the appropriate derivatives, the dynamics of the VLP in Hamiltonian form
are described in (\ref{eqn:vlp-hamiltonian-with-pl}). 
\begin{align}\label{eqn:vlp-hamiltonian-with-pl}
   \mathcal{H} &= \frac{1}{2} \begin{bmatrix} p & p_l \end{bmatrix}
      \begin{bmatrix}
         \frac{1}{ml^2}  & 0 \\
         0 & \frac{1}{m}
      \end{bmatrix} \begin{bmatrix} p \\ p_l \end{bmatrix} - mgl\cos(q) \\
     &\begin{cases}
        \dot{q} = \frac{p}{ml^2} \\
        \dot{l} = \frac{p_l}{m} \\
        \dot{p} = -mgl\sin(q) \\
        \dot{p}_l = \frac{p^2}{ml^3} + mg\cos(q) + \tau \\
   \end{cases} \nonumber
\end{align}
The control input is a force \(\tau \in \mathbb{R}\) affecting the dynamics of
\(p_l\), acting colinearly with direction of the rod.
We assume the force does not affect the dynamics of \(p\) in any way -
that is, there is no lateral force such as when a person is pushed on a swing.
In this way, the dynamics of \((q,p)\) and \((l,p_l)\) are decoupled. 

This decoupling is extremely useful in simplifying the dynamics. Since one can
write set \(\dot{p}_l\) to any desired function, we can suppose (with some abuse
of notation) that \(l\) is tracking some function \(l(t)\). 
The closed loop dynamics of the system will be described exclusively by
\((\dot{q},\dot{p})\) with \(l = l(t)\); 
the fact that \(p_l\) does not appear in these closed-loop dynamics allows us to
ignore the subdynamics \((\dot{l},\dot{p}_l)\) entirely.

What this means is we can treat \(l\) as the control input directly, rather than
modelling it as a configuration variable. Re-deriving the Hamiltonian and the
dynamics with this in mind, we get the system 
(\ref{eqn:vlp-hamiltonian}) with phase \((q,p) \in \mathbb{S}^1 \times \R\). 
Note that the control input \(l(t)\) and its derivative \(\dot{l}(t)\) are both
known variables.
\begin{align}\label{eqn:vlp-hamiltonian}
   \mathcal{H}(q,p) &= \frac{p^2}{ml^2} - \frac{1}{2}\dot{l}^2 - mgl\cos(q) \\
     &\begin{cases}
        \dot{q} = \frac{p}{ml^2} \\
        \dot{p} = -mgl\sin(q) \\
      \end{cases}\nonumber
\end{align}
In particular, the Hamiltonian of this simplified model is no longer equal to
the total mechanical energy of the system, which is given by
(\ref{eqn:vlp-energy}).
\begin{equation}\label{eqn:vlp-energy}
   E(q,p) = \frac{p^2}{ml^2} + \frac{1}{2}\dot{l}^2 - mgl\cos(q)
\end{equation}

\section{The VLP Constraint}
% TODO: Go through the development of the constraint and prove it gains energy
% TODO: Explain how we came up with the constraint first. Talk about the child
% on a swing, the paper showing the time-optimal control input, and describe why
% we want to remove time from the equation. Discuss the lmin and lmax choices.
% Why do we need to use a VNHC instead of a VHC: we are modelling the
% time-optimal controller in a time-independent way, which requires knowledge of
% which quadrant in (q,p)-space we're in. This means using the angle theta,
% which is inherently a VNHC.

\begin{thm}
For the variable-length pendulum, define \(\theta := \arctan_2(p,q)\). 
A VNHC of the form \(l = l(\theta)\) injects energy if there exists \(l_{avg} \in \R_{>0}\) such that
\[
   \left(l(\theta) - l_{avg}\right)sin(2\theta) \leq 0 \text{ }\forall \theta \in \mathbb{S}^1
\]
with the property that the inequality is strict for almost every \(\theta\).
\end{thm}
\begin{proof}
    Choose, as a candidate anti-Lyapunov function, the energy for the average-length pendulum 
    \[
       E_{avg}(q,p) := \frac{1}{2}\frac{p^2}{m l_{avg}^2} 
                    + m g l_{avg} (1-\cos(q))
    \]

    which is non-negative and has derivative 
    \[
      \dot{E}_{avg} = \frac{-g\sin(q)p \left(l(\theta)^3 - l_{avg}^3\right)}
                 {l_{avg}^2l(\theta)^2}
    \]

    We will show that \(E_{avg}\) is increasing.

    % VBOX UNDERFULL WARNING HERE, IGNORE IT
    Observe that \(\sign{\sin(q)p} = \sign{\sin(2\theta)}\) and, 
    by Lemma \textbf{TODO: REF LEMMA},
    \( \sign{l(\theta)^3 - l_{avg}^3} 
     = \sign{ l(\theta) - l_{avg}}\). 

     Then the derivative of \(E_{avg}\) is almost always positive, since
     \begin{align*}
        \label{eq:}
        \sign{\dot{E}_{avg}} &= \sign{-\sin(q)p \left(l(\theta)^3 - l_{avg}^3\right)} \\
                    &= -\sign{\sin(2\theta) \left(l(\theta) - l_{avg}\right)} \\
                    &\geq 0 \text{ (by assumption)}
     \end{align*}

      Hence, \(E_{avg}\) is an anti-Lyapunov function with positive derivative,  
      so the variable-length pendulum is gaining energy.
\end{proof}

\section{Simulation Results}

%/========== /Variable Length Pendulum ==========/%
% vim: set ts=3 sw=3 sts=0 et tw=80 ffs=unix :
