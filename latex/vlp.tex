%! TEX root = main.tex

%/========== Variable Length Pendulum ==========/%
\chapter{Application of VNHCS: The Variable Length Pendulum}\label{ch:vlp}
\section{Motivation}
The variable length pendulum (VLP) is a classical underactuated dynamical system
which is often used to model the motion of a person on a swing
\cite{pumping_swing_standing_squatting,how_to_pump_a_swing}.
The VLP also represents the motion of the load at the end of a crane, 
the (simplified) motion of a gymnast on a bar
\cite{pendulum_length_giant_gymnastics}, and the tuned-mass-damper systems which
stabilize skyscrapers \cite{vlp_tuned_mass_damper}.

The motion of the VLP has been well studied (see for instance
\cite{dynamics_periodic_vlp}), and many control mechanisms exist
to stabilize trajectories of the system. While many of these controllers 
are time-dependent, \citet{vlp_energy_shaping}
offer a time-independent technique to inject energy into the VLP. 
They design a controller through a technique called \textit{energy shaping}
and prove that it stabilizes any desired energy level set.
However, their control input depends on a pre-specified target energy and requires
knowledge of the current total energy of the VLP.
This makes their energy injection mechanism ``ad-hoc" in the sense that it is
tailored very specifically to the VLP, and is not generalizable to a larger 
methodology.

It may be better to base the control design on natural biological behaviour.
In this chapter we will make use of the general VNHC framework developed in Chapter
\ref{ch:vnhcs} to add and remove energy from the VLP in a time-independent
manner.
We'll show that, unlike energy shaping, VNHCs can be used to stabilize energy
levels while maintaining the structured motion of a human on a swing.

\section{Dynamics of the Variable Length Pendulum}
% Derive the dynamics of the VLP in Hamiltonian form
We will model the VLP as a point mass \(m\)
connected to a fixed pivot by a massless rod of varying length \(l\) with angle 
\(q \in \Sone\) from the vertical, as in Figure
\ref{fig:vlp-model}. 
We will ignore any damping and frictional forces in this model.
In a realistic VLP, the rod length \(l\) varies between some minimum
length \(\underline{l} \geq 0\) and some maximum length 
\(\overline{l} > \underline{l}\). The configuration of the VLP is the vector
\(\mathbf{q} := (q,l) \in \Sone \times [\underline{l},\overline{l}]\).

\begin{figure}
   \centering
   \includestandalone[width=0.5\textwidth]{images/vlp_model}
   \caption{The variable length pendulum is a mass attached to the
      tip of a massless rod which can change length.}\label{fig:vlp-model}
\end{figure}

Using this configuration, we will compute the Hamiltonian dynamics of the system.
The Cartesian position of the mass at the tip of the pendulum
is given by \(x = (l\sin(q),-l\cos(q))\), while its velocity is
\(\dot{x} = (\dot{l}\sin(q) + l\cos(q)\dot{q}, -\dot{l}\cos(q) + l\sin(q)\dot{q})\).
Computing the kinetic energy \(T\) yields
\[
   T(\mathbf{q},\dot{\mathbf{q}}) = 
   \frac{1}{2}m\norm{\dot{x}}^2 = \frac{1}{2}m\left(\dot{l}^2 + l^2\dot{q}^2\right)
   .
\]
The potential energy \(P\) with respect to the pivot (under a gravitational
acceleration \(g\)) is
\[
   P(\mathbf{q}) = -mgl\cos(q)
   .
\]
Collecting the kinetic energy into a quadratic form, we get the Lagrangian
\[
   \mathcal{L}(\mathbf{q},\dot{\mathbf{q}}) 
   = \frac{1}{2} \dot{\mathbf{q}}\tpose D(\mathbf{q})\dot{\mathbf{q}} - P(\mathbf{q})
   = \frac{1}{2}
   \begin{bmatrix} \dot{q} & \dot{l} \end{bmatrix}
   \begin{bmatrix}
      ml^2 & 0 \\
      0 & m \\
   \end{bmatrix}
   \begin{bmatrix} 
      \dot{q} \\ \dot{l}
   \end{bmatrix}
   + mgl\cos(q)
   .
\]
Computing the conjugate of momenta to \(\mathbf{q}\), we get 
\[
   \mathbf{p} := \begin{bmatrix} p \\ p_l \end{bmatrix} 
   = \begin{bmatrix} ml^2\dot{q} \\ m\dot{l} \end{bmatrix} 
   .
\]
Performing the Legendre transform on \(\mathcal{L}\) and setting
\(M(\mathbf{q}) := D(\mathbf{q})\), \(V(\mathbf{q}) := P(\mathbf{q})\),
we get the Hamiltonian ~\eqref{eqn:simply-actuated-hamiltonian} whose
dynamics ~\eqref{eqn:simply-actuated-dynamics} resolve to  
\begin{align}\label{eqn:vlp-hamiltonian-with-pl}
   \mathcal{H} &= \frac{1}{2} \begin{bmatrix} p & p_l \end{bmatrix}
      \begin{bmatrix}
         \frac{1}{ml^2}  & 0 \\
         0 & \frac{1}{m}
      \end{bmatrix} \begin{bmatrix} p \\ p_l \end{bmatrix} - mgl\cos(q)
      , \\
     &\begin{cases}
        \dot{q} = \frac{p}{ml^2} \\
        \dot{l} = \frac{p_l}{m} \\
        \dot{p} = -mgl\sin(q) \\
        \dot{p}_l = \frac{p^2}{ml^3} + mg\cos(q) + \tau
        . \\
   \end{cases} \nonumber
\end{align}
The control input is a force \(\tau \in \R\) affecting the dynamics of
\(p_l\), acting collinearly with the rod.
We assume the force does not affect the dynamics of \(p\) in any way -
that is, the control input cannot enact any lateral force on the pendulum.
This makes the VLP into an underactuated mechanical system with degree of
underactuation one. 
It is also a useful assumption because it means \((\mathbf{q},\mathbf{p})\) 
are simply actuated coordinates, which allows us to apply the theory of VNHCs we
developed in Chapter \ref{ch:vnhcs}.

Let us define the VNHC \(l = L(q,p)\), by which we mean we are
actually defining the VNHC \(h(\mathbf{q},\mathbf{p}) = l - L(q,p) = 0\) of
order 1.
The VLP satisfies
\(\nabla_q \Minv(\mathbf{q}) = \Zmat{2 \times 2}\).
By Theorem \ref{thm:vnhc-regularity},
\(h(\mathbf{q},\mathbf{p})\) is a regular VNHC whenever 
\(L(q,p)\) is \(C^2\) because
\[
   dh_{\mathbf{q}} \Minv(\mathbf{q})B = 
   \begin{bmatrix}
      -\pdiff{L}{q} & 1
   \end{bmatrix}
   \begin{bmatrix}
      \frac{1}{ml^2}  & 0 \\
      0 & \frac{1}{m}
   \end{bmatrix} 
   \begin{bmatrix}
      0 \\ 
      1
   \end{bmatrix}
   = \frac{1}{m}
\]
is always full rank.
The constraint manifold \(\Gamma\) is diffeomorphic to 
\(\SxR\), which is parameterized by the unactuated
phase \((q,p)\).
By Theorem \ref{thm:zero-dynamics}, the constrained dynamics
are described entirely by \((\dot{q},\dot{p})\) with \(l\)
replaced by \(L(q,p)\):
\begin{equation}\label{eqn:vlp-hamiltonian}
   \begin{cases}
      \dot{q} = \frac{p}{m L^2} \\
      \dot{p} = -mgL\sin(q)
      . \\ 
   \end{cases}
\end{equation}
Note that we suppress the function notation of \(L(q,p)\) for clarity.

The total mechanical energy of the system restricted to the constraint manifold
is given by ~\eqref{eqn:vlp-energy}, where \(L(q,p)\) and \(\dot{L}(q,p)\) are
known.
\begin{equation}\label{eqn:vlp-energy}
   E(q,p) = \frac{1}{2} \frac{p^2}{mL^2} + \frac{1}{2}\dot{L}^2 - mgL\cos(q)
   .
\end{equation}
In the rest of this chapter, we will derive a \(C^2\) function \(L(q,p)\) based
on natural human motion.
This function will produce constrained dynamics that inject energy into the VLP.

% TODO: Describe the oscillation and rotation motion of a pendulum in the
% qp-plane, and give a figure showing the orbiting motion for each of these.
% Then describe what the orbit should do under energy injection (spiral away /
% begin rotation outward) and energy dissipation (spiral in / rotate inward).

\section{The VLP Constraint}
% Go through the development of the constraint and prove it gains energy
To motivate why a VNHC could inject energy into the VLP in a human-like
manner, we will examine a person standing on a swing.
As can be seen in Figure \ref{fig:child-vlp}, a person's center of mass moves closer
to the swing's pivot when they stand, and moves away from the pivot when they
squat.
This is equivalent to the VLP model from Figure \ref{fig:vlp-model}, where
standing and squatting correspond to shortening and lengthening the pendulum
respectively.
\begin{figure}
   \centering
   \begin{subfigure}[t]{0.45\textwidth}
      \includegraphics[]{images/child_vlp_standing.png}
      \caption{A person standing on a swing has their center of mass 
      close to the pivot.}
   \end{subfigure}
   \hfill
   \begin{subfigure}[t]{0.45\textwidth}
      \includegraphics[]{images/child_vlp_squatting.png}
      \caption{When a person squats on a swing, their center of mass extends
      away from the pivot.}
   \end{subfigure}
   \caption{The VLP representation of a person on a standing swing.}
   \label{fig:child-vlp}
\end{figure}

The action of regulating pendulum length to inject energy into the VLP is known as
``pumping". \citet{pumping_swing_standing_squatting} asked whether
the pumping strategy performed by children is time-optimal, assuming the
children could squat or stand instantaneously. 
Indeed, they discovered that children increase the height of their swing as fast
as is physically possible.

A child's optimal pumping strategy is the following: 
they stand at the lowest point of the swing, and squat at the highest point.
Looking at the VLP representation, the pendulum shortens at the bottom of the
swing, and lengthens at the top. 
For an intuitive explanation, conservation of angular momentum indicates that
shortening the pendulum at the bottom forces the mass to gain speed to
compensate for the reduced length \cite{how_to_pump_a_swing}.
Energy is not conserved in this process, so the pendulum gains kinetic energy
and reaches a higher point at the peak of its swing.
Lengthening the pendulum when it reaches this peak means gravity
imparts a larger angular momentum to the mass by the time it reaches the bottom
of its swing, which in turn is converted to a higher velocity when the
pendulum is shortened.
By alternating these processes, the pendulum experiences an average net gain in
rotational energy.

Notice that the child's pumping strategy requires knowledge of
when the system is at the ``bottom" or ``top" of the swing. 
Since being at the bottom is equivalent to having angle \(q = 0\) and being
at the top is equivalent to having momentum \(p = 0\), a controller based on
this strategy will necessarily involve the full unactuated phase \((q,p)\). 
This is why we must use VNHCs instead of other methods (such as VHCs) 
to perform this maneuver.

The time-optimal controller from \cite{pumping_swing_standing_squatting} is, in
our notation,
\[
   L^\star(q,p) := -\sign{qp}
   ,
\]
which is a piecewise-continuous controller that varies between \(\pm 1\). 
We could set our constraint to be \(l = L^\star(q,p)\), but this is not a VNHC
because it is not \(C^2\).
Additionally, it would force us to assume that \(l \in \{-1,0,1\}\) and that one
can switch \(l\) instantaneously.
Since we need to enforce the constraint using the physical input \(\tau\) (which
would ideally emulate realistic human motion), we cannot use \(L^\star\) as our
VNHC.
We will instead find an alternate representation of \(L^\star\) which can be
converted into a VNHC, and which allows 
\(l \in [\underline{l},\overline{l}]\).

Figure \ref{fig:vlp-optimal-controller-qp-plane} displays \(L^\star(q,p)\) 
on the \((q,p)\)-plane.
Note that the length remains constant inside each quadrant and changes only when it
crosses one of the axes. Using this fact, we can redefine
the time-optimal pumping strategy as a function of 
\(\theta := \arctan_2(p,q)\).
Abusing notation, we denote this by \(L^\star(\theta)\), which is defined in
~\eqref{eqn:vlp-optimal-controller}.
Figure \ref{fig:vlp-optimal-controller} shows the graph of \(L^\star(\theta)\), 
where now the length varies between \([\underline{l},\overline{l}]\) rather than 
\(\{-1,0,1\}\). 

\begin{equation}\label{eqn:vlp-optimal-controller}
   L^\star(\theta):= \begin{cases}
      \overline{l} & \theta \in [-\frac{\pi}{2},0[ \cup [\frac{\pi}{2}, \pi[ \\
      \underline{l} & \theta \in [-pi, -\frac{\pi}{2}[ \cup [0,\frac{\pi}{2}[ 
      .
   \end{cases}
\end{equation}

\begin{figure}
   \centering
   \begin{subfigure}[t]{0.45\textwidth}
      \includestandalone[width=\linewidth]{images/vlp_optimal_controller_qp_plane}
      \caption{The time-optimal controller \(L^\star(q,p)\) mapped onto the
      \((q,p)\)-plane. Here, red corresponds to \(L^\star(q,p) = -1\) and blue
      to \(L^\star(q,p) = 1\).}
      \label{fig:vlp-optimal-controller-qp-plane}
   \end{subfigure}
   \hfill
   \begin{subfigure}[t]{0.45\textwidth}
      \includestandalone[width=\linewidth]{images/vlp_optimal_controller}
      \caption{The time-optimal controller converted to the alternate
      representation \(L^\star(\theta)\).}
      \label{fig:vlp-optimal-controller}
   \end{subfigure}
   \caption{The time-optimal controller for a standing swing as derived by
      \cite{pumping_swing_standing_squatting}. The colour red
      corresponds to standing, blue to squatting, and 
      \(\theta := \arctan_2(p,q)\) is the angle of the VLP phase in the
      \((q,p)\)-plane.}
\end{figure}

We now define a continuous function which
approximates \(L^\star(\theta)\). 
Let \(\Delta l := (\overline{l} - \underline{l})/2\) and 
\(l_{\text{avg}} := (\overline{l} + \underline{l})/2\).
Let \(T \in \, ]0,\frac{\pi}{2}]\) be a parameter of our choosing.
By intelligently attaching sinusoids of frequency \(\omega = \frac{\pi}{T}\) to
\(L^\star(\theta)\) (see Figure \ref{fig:vlp-T-controller}), we get a family of
\(C^1\) constraints \(L_T(\theta)\) parameterized
by \(T\): 
\begin{equation}\label{eqn:vlp-T-controller}
   L_T(\theta) := \begin{cases}
      \overline{l} & \theta \in \left[-\frac{\pi}{2} + \frac{T}{2}, -\frac{T}{2}\right] 
      \cup \left[\frac{\pi}{2} + \frac{T}{2}, \pi - \frac{T}{2}\right] \\
      \underline{l} & \theta \in \left[-\pi + \frac{T}{2}, -\frac{\pi}{2} - \frac{T}{2}\right] 
      \cup \left[\frac{T}{2}, \frac{\pi}{2} - \frac{T}{2}\right] \\
      -\Delta l \sin(\omega(\theta + \pi)) + l_{\text{avg}} & \theta \in
      \left[-\pi,-\pi + \frac{T}{2}\right] \\
      -\Delta l \sin(\omega \theta) + l_\text{avg} & \theta \in [-\frac{T}{2},
      \frac{T}{2}] \\
      \Delta l \sin(\omega(\theta - a)) + l_\text{avg} & 
      \theta \in \left[a - \frac{T}{2}, a + \frac{T}{2}\right] \text{ for } 
      a \in \left\{-\frac{\pi}{2}, \frac{\pi}{2}\right\} \\
      -\Delta l \sin(\omega(\theta-\pi)) & \theta \in \left[\pi - \frac{T}{2},\pi\right] 
      .
   \end{cases}
\end{equation}

\begin{figure}
   \centering
   \includestandalone[width=0.7\textwidth]{images/vlp_T_controller}
   \caption{The continuous VLP constraint \(l = L_T(\theta)\).}
   \label{fig:vlp-T-controller}
\end{figure}

This family of constraints approximates \(L^\star(\theta)\) because 
\[
   \lim\limits_{T \rightarrow 0} L_T(\theta) = L^\star(\theta)
   .
\]
Unfortunately, while \(L_T(\theta)\) is continuously-differentiable, 
it is not twice-differentiable for most values of \(T\).
If we wish to use it as a VNHC, we must ensure that either the generalized
forces \(\tau\) acting on \(p_l\) can be discontinuous (which is certainly
achievable by humans), or we must find a value of \(T\) where this constraint is
at least \(C^2\).
Thankfully, setting \(T = \frac{\pi}{2}\) yields the smooth function 
\(L_{\frac{\pi}{2}}(\theta)\), which can be simplified from
~\eqref{eqn:vlp-T-controller} into
\begin{equation}\label{eqn:vlp-smoothed-controller}
   L_\frac{\pi}{2}(\theta) = -\Delta l \sin(2\theta) + l_{\text{avg}}
   .
\end{equation}
This smooth constraint is plotted for demonstration in Figure
\ref{fig:vlp-smoothed-controller}.

\begin{figure}
   \centering
   \includestandalone[width=0.6\textwidth]{images/vlp_smoothed_controller}
   \caption{The smoothed VLP constraint \(l = L_\frac{\pi}{2}(\theta)\).}
   \label{fig:vlp-smoothed-controller}
\end{figure}

Because \(L_\frac{\pi}{2}(\theta)\) is smooth and it approximates
\(L^\star(\theta)\), we set our VNHC to be
\[
   h(\mathbf{q},\mathbf{p}) = l - L_\frac{\pi}{2}\left(\theta(q,p)\right)
   .
\]

We can now prove our VNHC injects energy into the VLP. 
As part of the proof, we will require the following lemma.

\begin{lemma}\label{lemma:sign-of-cube}
   For any \(x,y \in \R\),
   \[
      \sign{x^3 - y^3} = \sign{x-y}
      .
   \]
\end{lemma}
\begin{proof}
   Observe that \(x^3 - y^3 =  (x-y)(x^2 + xy + y^2)\).
   The inequality \(x^2 + xy + y^2 \geq 0\) holds because
   \[
      x^2 + xy + y^2 = \left(x  + \frac{y}{2}\right)^2 + \frac{3y^2}{4} \geq 0
      ,
   \]
   which proves the lemma.
\end{proof}

We will show that the constrained dynamics trace out a curve on the
\((q,p)\)-plane which is diverging from the origin.  
This implies that the momentum \(p\) is increasing in magnitude whenever the
curve hits the \(p\)-axis, which in turn means the VLP is gaining energy on
average.

\begin{thm}\label{thm:vlp-energy-stabilization}
   Define \(\theta := \arctan_2(p,q)\).
   Let \(L : \Sone \rightarrow [\underline{l},\overline{l}]\) be a
   \(C^2\) function of \(\theta\).
   A regular VNHC of the form 
   \(h(\mathbf{q},\mathbf{p}) = l - L(\theta)\) for the VLP injects energy 
   on the constraint manifold \(\Gamma \simeq \SxR\) if there
   exists \(l_\text{avg} \in [\underline{l},\overline{l}]\) such that 
   \begin{equation}\label{eqn:vlp-energy-gain-condition}
      \left(l_\text{avg} - L(\theta)\right)sin(2\theta) \geq 0 
      \text{ }\forall \theta \in \Sone
      ,
   \end{equation}
   with the property that the inequality is strict except at 
   \(\theta \in \{0, \frac{\pi}{2}, \frac{\pi}, \frac{3\pi}{2}\}\).  
   If instead 
   \[
    \left(l_\text{avg} - L(\theta)\right)sin(2\theta) \leq 0 
         \text{ }\forall \theta \in \Sone
         ,
   \]
   the VNHC dissipates energy on \(\Gamma\).
\end{thm}
\begin{proof}
We will first show that, under the given assumptions, the origin of the
constrained dynamics~\eqref{eqn:vlp-hamiltonian} is a repeller, and for that
we consider the negative time system obtained
from~\eqref{eqn:vlp-hamiltonian} by reversing the sign of the vector field:
\begin{equation}\label{eqn:negative-time-vlp}
   \begin{cases}
      \dot{q} = -\frac{p}{m L^2} \\
      \dot{p} = mgL\sin(q)
      . \\ 
   \end{cases}
\end{equation}
The state space of~\eqref{eqn:negative-time-vlp} is 
\(\SxR\), and the system has two equilibria,
\((q,p) = (0,0)\) and \((q,p) = (\pi,0)\). 
We need to show that the equilibrium \((q,p) = (0,0)\) is
asymptotically stable for~\eqref{eqn:negative-time-vlp}. 

Consider the Lyapunov function candidate
\[
    E_\text{avg}(q,p) := \frac{1}{2}\frac{p^2}{m l_\text{avg}^2} 
        + m g l_\text{avg} (1-\cos(q))
   ,
\]
which corresponds to the energy of VLP when the pendulum length is fixed at
\(l_\text{avg}\). 
The function \(E_\text{avg}\) is positive definite at \((0,0)\), and has compact
sublevel sets on \(\SxR\).  
The derivative of \(E_\text{avg}\) along~\eqref{eqn:negative-time-vlp} is
\begin{equation}\label{eqn:eavg-dot-neg}
   \dot{E}_\text{avg} = \frac{g\sin(q)p \left(L(\theta)^3 -
     l_\text{avg}^3\right)} {l_\text{avg}^2L(\theta)^2}
   .
\end{equation}
It is easy to show that
\[
   \sign{\sin(q)p} = \sign{\sin(2\theta)}
   .
\]
Furthermore, by Lemma \ref{lemma:sign-of-cube} we have
\[ 
    \sign{L(\theta)^3 - l_\text{avg}^3} = \sign{L(\theta) - l_\text{avg}}
    ,
\]
and therefore,
\begin{align}\label{eqn:vlp-proof-neg-inv}
    \sign{\dot{E}_\text{avg}} &= 
       \sign{\sin(q)p \left(L(\theta)^3 - l_\text{avg}^3\right)}
    \nonumber \\
    &= \sign{\sin(2\theta) \left(L(\theta) - l_\text{avg}\right)}
    \nonumber \\
    &= -\sign{(l_\text{avg} - L(\theta))\sin(2\theta)} 
    \nonumber \\
    &\leq 0 \text{ (by assumption)}
   .
\end{align}
We have thus shown that \(\dot{E}_\text{avg} \leq 0\), and therefore
the equilibrium \((q,p)=(0,0)\) is stable for~\eqref{eqn:negative-time-vlp}.
We now apply the Krasovskii-LaSalle invariance principle, and consider the
largest subset of 
\( Z = \left\{(q,p) \in \SxR \mid \dot{E}_\text{avg}(q,p) = 0
\right\} \). 
We see from~\eqref{eqn:eavg-dot-neg}
that \(\dot{E}_\text{avg}(q,p)\) is zero when \(p = 0\), \(\sin(q) = 0\), or
\(L(\theta) = l_\text{avg}\). 
This latter condition by assumption is met when 
\(\theta \in \{0,\pi/2,\pi, 3\pi/2\}\), or equivalently when either 
\(q = 0\) or \(p = 0\). 
All in all, we have that
\[
   Z= \left\{(q,p) \in \SxR \mid q =0\right\} 
   \cup \left\{(q,p) \in \SxR \mid p =0\right\} 
   .
\]
It is easily seen that the largest invariant subset of \(Z\) is the union of the
two equilibria
\[
   \Omega =  \{(0,0)\}\cup \{(\pi,0)\}
   .
\]  
Since all sublevel sets of \(E_\text{avg}\) are compact, the
Krasovskii-LaSalle invariance principle implies that the set
\(\Omega\) is globally attractive \cite{krasovskii_lasalle}. 
Since the two equilibria are isolated, and since \((q,p)=(0,0)\) is stable, we
deduce that the equilibrium \((q,p)=(0,0)\) is asymptotically stable for the
negative time system~\eqref{eqn:negative-time-vlp}, and thus it is a repeller
for the constrained dynamics in~\eqref{eqn:vlp-hamiltonian}.


We now prove that almost all solutions of the reduced
dynamics~\eqref{eqn:vlp-hamiltonian} escape compact sets in finite
time, \ie, that for each compact subset \(K\) of 
\(\SxR\), and for almost every condition 
\( (q_0,p_0) \in K \), there exists \( T > 0\) such that
\( (q(t),p(t)) \notin K \) for all \( t > T \), where \( (q(t),p(t)) \)
denotes the solution of~\eqref{eqn:vlp-hamiltonian} with 
\( (q(0),p(0)) = (q_0,p_0)\). 
In what follows, we denote by \( x = (q,p) \) the state of the reduced
dynamics~\eqref{eqn:vlp-hamiltonian}. 
Also, we denote by \(\Pi^-\) the stable manifold of the equilibrium \((\pi,0)\)
for system~\eqref{eqn:vlp-hamiltonian}, defined as
\[
   \Pi^- := \left\{ x(0) \in \SxR \mid
   \lim_{t \to \infty} x(t) = (\pi, 0) \right\}.
\]
The linearization of system~\eqref{eqn:vlp-hamiltonian} at the
equilibrium \((\pi,0)\) has a system matrix given by
\begin{equation}\label{eqn:linearized-vlp}
   \begin{bmatrix}
      0 & -\frac{1}{mL(0)^2} \\
      -m g L(0) & 0 \\
   \end{bmatrix}
   ,
\end{equation}
whose spectrum is \(\{\pm \sqrt{{g}/{L(0)}}\}\).  
Since one eigenvalue of ~\eqref{eqn:linearized-vlp} is positive and one is
negative, the stable manifold theorem implies that the stable manifold 
\( \Pi^- \) is a one-dimensional immersed submanifold of
\(\SxR\), and thus is a set of measure zero
\cite{stable_manifold}.

Let \( K \subset \SxR\) be an arbitrary compact set and let 
\( x(0) \in K \backslash \left(\Pi^- \cup \left\{(0,0)\right\}\right)\).
Since \( x(0) \notin \Pi^- \), we have that
\begin{equation}\label{eqn:vlp-x-does-not-converge}
   x(t) \nrightarrow_{t \to \infty} (\pi,0).
\end{equation}
Suppose, by way of contradiction, that for each \(T > 0\) there exists 
\(t' > T\) such that \(x(t') \in K\).  
Letting
\[
   k := \max\limits_{x \in K} E_\text{avg}(x)
   ,
\] 
and \(E_k =\{ x \mid E_\text{avg}(x) \leq k\} \), we have that 
\( K \subset E_k \), and thus \(x(t') \in E_k\).
We have shown earlier that the function \( E_\text{avg} \) is nonincreasing
along solutions of~\eqref{eqn:negative-time-vlp}, which implies that
\( E_\text{avg}(x(t)) \) is nondecreasing for solutions of the reduced
dynamics~\eqref{eqn:vlp-hamiltonian}, and therefore the half orbit \(x([0,t'])\)
is contained fully within \(E_k\).  
In particular, \(x(T) \in E_k\). Since this is true for each \( T >0 \), we
deduce that \(x(t) \in E_k\) for all \(t \in [0,\infty[\).  
Since \(E_k\) is compact, \(x(t)\) has a positive limit set in \(E_k\) which, by
the Poincar\'{e}-Bendixson theorem, is either an equilibrium or a closed orbit
\cite{poincare_bendixson}.  
We will show this positive limit set must be an equilibrium and thereby reach a
contradiction.

In order to rule out closed orbits, suppose that the reduced
dynamics~\eqref{eqn:vlp-hamiltonian} have a periodic solution \( z(t)\). 
Since \( E_\text{avg}(z(t)) \) is nondecreasing and \(z(t)\)
is periodic, it must be that \( E_\text{avg}(z(t)) \) is constant, or
\(\dot E_\text{avg}(z(t)) \equiv 0\). 
We have shown earlier
that\footnote{We have shown it for the negative time
system~\eqref{eqn:negative-time-vlp}, but changing the sign of the
vector field does not change the set where 
\(\dot E_\text{avg} =0\) nor the invariant sets.}
\[
   \{x \mid \dot E_\text{avg}(x) = 0\} = \left\{(q,p) \mid q = 0\right\}
   \cup \left\{(q,p) \mid p =0\right\},
\]
and the only invariant subset of this set is the union of two disjoint
equilibria. 
Since the orbit \(z(\R)\) is an invariant set which is
connected, \(z(\R)\) must be an equilibrium and cannot be a nontrivial
closed orbit.

Returning to the Poincar\`e-Bendixson theorem, the positive limit set
of \( x(t)\) must be an equilibrium and,
by~\eqref{eqn:vlp-x-does-not-converge}, this equilibrium must be 
\( (0,0)\). 
Since we have chosen \(x(0) \neq (0,0)\), and since \( (q,p) = (0,0) \) is a
repeller, \( x(t) \) cannot converge to \( (0,0)\), which gives a contradiction.

We conclude that the VNHC injects energy into the VLP on 
\(\SxR\).
By flipping the inequality of ~\eqref{eqn:vlp-energy-gain-condition}
we find the VLP is gaining energy in negative-time, so the VNHC is
dissipating energy.
\end{proof}

\begin{cor}
   Recall that
   \[
      L_\frac{\pi}{2}(\theta) := -\Delta l \sin(2\theta) + l_\text{avg}
      ,
   \]
   and define 
   \[
      L^{-}_\frac{\pi}{2}(\theta) := \Delta l \sin(2\theta) + l_\text{avg}
      .
   \]
   The VNHC \(l = L_\frac{\pi}{2}(\theta)\) injects energy into the VLP,
   while \(l = L^{-}_\frac{\pi}{2}(\theta)\) dissipates energy.
\end{cor}
\begin{proof}
   Both VNHCs satisfy Theorem \ref{thm:vlp-energy-stabilization} because
   \[
      \left(l_\text{avg} - L_\frac{\pi}{2}(\theta)\right)\sin(2\theta) = 
      \Delta l \sin^2(2\theta) \geq 0
      ,
   \] 
   and
   \[
      \left(l_\text{avg} - L^{-}_\frac{\pi}{2}(\theta)\right)\sin(2\theta) = 
      - \Delta l \sin^2(2\theta) \leq 0
      ,
   \] 
   where the inequalities are strict everywhere except the coordinate axes.
\end{proof}

% TODO: Add a remark:
% Our controller not only injects energy, but makes the lower equilibrium a
% repeller which means the "bad" set of x(0) does not contain a nbhd of (0,0),
% so anything near (0,0) will start to gain eenrgy aka ANY slight deviation from
% 0 will gain energy.


The class of VNHCs which satisfies Theorem \ref{thm:vlp-energy-stabilization} is
illustrated graphically in Figure \ref{fig:vlp-energy-in-out}. 
To stabilize specific energy level sets, one simple approach is to switch
between injection and dissipation VNHCs when the momentum \(p\) reaches
a pre-determined value at the bottom of the swing.
For the VNHCs we designed in this chapter, this means toggling
between \(L_\frac{\pi}{2}(\theta)\) and \(L^{-}_\frac{\pi}{2}(\theta)\),
with some hysteresis to avoid infinite switching.  

\begin{figure}
   \centering
   \includestandalone[width=0.5\textwidth]{images/vlp_energy_in_out}
   \caption{Any VNHC of the form \(l = L(\theta)\) where \(L(\theta)\)
      is entirely contained within
      the green (yellow) regions will inject (dissipate) energy.}
      \label{fig:vlp-energy-in-out}
\end{figure}

Theorem \ref{thm:vlp-energy-stabilization} provides an alternate
explanation for why the optimal pumping strategy \(L^\star(\theta)\) works
so well at injecting energy: it maximizes the derivative of \(E_\text{avg}\)
under the restriction \(l \in [\underline{l},\overline{l}]\), so that the orbit
in the \((q,p)\)-plane diverges from the origin as fast as possible. 

Let us define \((L^\star)^{-}(\theta)\) by swapping the order of 
\(\underline{l}\) and \(\overline{l}\) in \(L^\star(\theta)\): 
\[
   (L^\star)^-(\theta) := \begin{cases}
      \underline{l} & \theta \in [-\frac{\pi}{2},0[ \cup [\frac{\pi}{2}, \pi[ \\
      \overline{l} & \theta \in [-pi, -\frac{\pi}{2}[ \cup [0,\frac{\pi}{2}[ 
      .
   \end{cases}
\]
Since this function \textit{minimizes} the derivative of \(E_\text{avg}\) under
the restriction \(l \in [\underline{l},\overline{l}]\), one might predict that 
\((L^\star)^{-}(\theta)\) is the optimal energy dissipation strategy for the VLP.
This is, in fact, true. \citet{pumping_swing_standing_squatting} showed 
that squatting at the lowest point of a swing and standing at the highest
(instead of standing and squatting, respectively) produces the
time-optimal trajectory for \textit{stopping} a standing swing. 

All together, the theory developed in this chapter shows that VNHCs can
replicate the time-optimal pumping/dissipation strategies performed by humans on
swings.
Furthermore, we see that VNHCs are a powerful tool for creating simple energy
stabilization techniques based on natural human motion.

\section{Simulation Results}
% TODO: Show the simulations of the VLP VNHCs (multiple of them). How does the
% energy injection time of our sin(2theta) VNHC compare to the energy injection
% time of the optimal one? How does the VNHC version compare in robustness with
% the time-based one from the pumping paper?

%/========== /Variable Length Pendulum ==========/%
% vim: set ts=3 sw=3 sts=0 et tw=80 ffs=unix :
