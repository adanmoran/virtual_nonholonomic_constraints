%! TEX program = lualatex

\documentclass[a4paper,12pt]{article}

%/========== Preamble ==========/%
%----------- Set Main Font ----------%
\usepackage{times} % Times New Roman

%----------- Choose Margins ----------%
\usepackage[a4paper,margin=1.87cm]{geometry}

%---------- Bibliography Style ----------%
\usepackage[style=ieee,backend=biber]{biblatex}
\addbibresource{bib.bib}

%----------- Math Packages -----------%
\usepackage{mathtools}

%---------- Remove indenting from paragraphs ----------%
% Also makes each new paragraph start with an vertical space.
\usepackage[parfill]{parskip}

%/========== Main Document ==========/%
\begin{document}

%---------- Title -----------%
\title{Ontario Graduate Scholarship: Plan of Study}
\author{Adan Moran-MacDonald}
\date{}
\maketitle

%---------- Main Content ----------%
\begin{large}
\textbf{Background}
\end{large}


Mechanical systems can have their motion constrained in two ways. First, the
constraint can be a function of their position, such as how the joints of a
robotic arm cannot go through each other. This type of constraint is called
``holonomic”. The second type of constraint involves both position and velocity,
like the fact that a car cannot normally slide sideways. This type of constraint
is called ``nonholonomic”.

It has been shown that one can create a holonomic constraint which did not exist
before through a control algorithm (see for example \cite{vhcs_for_el_systems}).
These constraints are called ``virtual" holonomic constraints (VHC) since they
are enforced by the robot’s actuators, as opposed to being enforced by laws of
physics. VHCs have been used to control walking robots \cite{stable-walking},
autonomous bicycles \cite{bicycle}, gymnastics robots \cite{xingbo-thesis}, and
snake robots \cite{snake-robot} among others. Mohammadi \textit{et.al.} showed
that a single VHC can result in several motion patterns, from which one can be
chosen if the robot is initialized with a configuration that is ``close enough”
to the desired motion pattern \cite{manfredi-orbit-stabilization}. However,
their method does not work for all starting configurations of the robot.

\begin{large} \textbf{Proposal} \end{large}

We propose to investigate a method that can stabilize a desired motion pattern
for a VHC from any starting configuration of the robot. This involves using
virtual nonholonomic constraints (VNHC) to maneuver the robot into a position
where the desired motion pattern can be stabilized. 

VNHCs have been studied for application to human-robot cooperation in
\cite{vnhc-human-robot-coop} and walking robots in \cite{vnhc-biped-robot}. Horn
\textit{et.al.} have also examined properties of VNHCs for general mechanical
systems \cite{hybrid_zero_dynamics_bipedal_nhvcs} and then applied their results
to walking robots. Our approach is novel since we will study VNHCs on general
mechanical systems with the aim of stabilizing any desired motion pattern on a
VHC. We will prove our results theoretically, then demonstrate them
experimentally on the gymnastics robot built by Wang \cite{xingbo-thesis}.

I am currently at the beginning stages of this thesis. We are studying
mathematical conditions whereby a general mechanical system constrained by a
VNHC will gain or lose energy over time. This will allow us to create a control
algorithm to add or remove energy from a robot. In doing so we will position the
robot configuration to stabilize any desired motion pattern which comes from a
VHC. After completing this general mathematical formulation, we will apply the
results to the gymnastics robot created by Wang \cite{xingbo-thesis}. We will
create virtual constraints that will make the robot perform a backflip routine
around a bar to confirm that our rigorous mathematical results are applicable to
real life autonomous systems. 

The advancement in mathematics from this thesis will also enable improvements to
control of other robotic systems. It will provide a more computationally
feasible method to stabilizing movement patterns than what was created in
\cite{manfredi-orbit-stabilization}. 

%---------- Bibliography ----------%
\printbibliography
\end{document}

% vim: set tw=80 ts=4 sw=4 sts=0 et ffs=unix :
