%! TEX program = lualatex

\documentclass[a4paper,12pt]{article}

%/========== Preamble ==========/%
%----------- Set Main Font ----------%
\usepackage{times} % Times New Roman

%----------- Choose Margins ----------%
\usepackage[a4paper,margin=1.87cm]{geometry}

%---------- Bibliography Style ----------%
\usepackage[style=ieee,backend=biber]{biblatex}
\addbibresource{bib.bib}

%----------- Math Packages -----------%
\usepackage{mathtools}

%---------- Remove indenting from paragraphs ----------%
% Also makes each new paragraph start with an vertical space.
\usepackage[parfill]{parskip}

%/========== Main Document ==========/%
\begin{document}

%---------- Title -----------%
\title{Ontario Graduate Scholarship: Plan of Study}
\author{Adan Moran-MacDonald}
\date{}
\maketitle

%---------- Main Content ----------%
\begin{large}
\textbf{Background}
\end{large}

When a gymnast is performing a swing-up routine on a bar, they do not kick their
feet at specified time intervals. Rather, a more natural human motion is to
swing their legs based on their current speed and position. In doing so, they
are able to generate enough momentum to perform backflips on the bar. When
controlling mechanical systems, standard practice is to have a robot's actuators
track a position dependent on time. Much like the gymnast, this tends to
produce unnatural motion. Furthermore, it is liable to timin errors if the
motion is interrupted in any way. The more recent techniques of virtual constraints 
can rectify these issues and produce realistic, biologically inspired motion.

Mechanical systems can have their motion constrained in two ways. Holonomic 
constraints restrict position, while nonholonomic constraints restrict
velocities (such as how a car cannot normally slide sideways). Previous research
has shown that one can enforce holonomic constraints through a robot's
actuators \cite{vhcs_for_el_systems}. These are called ``virtual holonomic
constraints" (VHCs). VHCs have been used to control walking robots \cite{stable-walking},
autonomous bicycles \cite{bicycle}, gymnastics robots \cite{xingbo-thesis}, and
snake robots \cite{snake-robot} among others. Likewise, in my 
master's thesis \textbf{TODO: HOW TO CITE?} I developed a general theory for
enforcing virtual nonholonomic constraints (VNHCs). I designed constraints which
injected energy into the gymnastics robot created by 
Wang \cite{xingbo-thesis}, enabling it to perform backflips on a bar.

\begin{large} \textbf{Proposal} \end{large}

Mohammadi \textit{et.al.} showed that a single VHC can result in several motion
patterns, from which one pattern can be chosen if the robot is initialized with
a configuration that is ``close enough” to the desired motion pattern
\cite{manfredi-orbit-stabilization}. However, their method does not work for all
starting configurations of the robot. We propose to investigate a method to
stabilize and transition between desired motion patterns from any starting
configuration of the robot. Rather than using VHCs, we will use VNHCs to
generate these motions.

VNHCs have been studied for application to human-robot cooperation in
\cite{vnhc-human-robot-coop} and walking robots in \cite{vnhc-biped-robot}. Horn
\textit{et.al.} have also examined properties of VNHCs for general mechanical
systems \cite{hybrid_zero_dynamics_bipedal_nhvcs} and then applied their results
to walking robots. Our approach is novel since we will study VNHCs on general
mechanical systems with the aim of transitioning between desired motion patterns.

We will first study mathematical conditions under which on can transition
between two VNHCs. These conditions will allow us to create a control algorithm
to switch between motion patterns. After completing this mathematical
formulation, we will apply the results to the gymanstics robot built by Wang
\cite{xingbo-thesis}. We will use the theory to create control algorithms which
enable this robot to perform an entire gymnastics routine by transitioning
between different VNHCs.

The advancement in mathematics from this research will enable improvements to
control of other robotic systems. The ability to transition between complex
constraints will allow for more natural expressive motion in many autonomous
systems and may become a standard technique for controlling biologically
inspired robots.

%---------- Bibliography ----------%
\printbibliography
\end{document}

% vim: set tw=80 ts=4 sw=4 sts=0 et ffs=unix :
