%! TEX program = lualatex

\documentclass[a4paper,12pt]{article}

%/========== Preamble ==========/%
%----------- Set Main Font ----------%
\usepackage{times} % Times New Roman

%----------- Choose Margins ----------%
\usepackage[a4paper,margin=1.87cm]{geometry}

%---------- Bibliography Style ----------%
\usepackage[style=ieee,backend=biber]{biblatex}
\addbibresource{bib.bib}

%----------- Math Packages -----------%
\usepackage{mathtools}

%---------- Remove indenting from paragraphs ----------%
% Also makes each new paragraph start with an vertical space.
\usepackage[parfill]{parskip}

%/========== Main Document ==========/%
\begin{document}

%---------- Title -----------%
\title{Ontario Graduate Scholarship: Plan of Study}
\author{Adan Moran-MacDonald}
\date{}
\maketitle

%---------- Main Content ----------%
\begin{large}
\textbf{Background}
\end{large}

\textbf{TODO: CITATIONS FOR THIS PARAGRAPH}
When a gymnast is performing a swing-up routine on a bar, they do not kick their
feet at specified time intervals. Rather, a more natural human motion is to
smoothly swing their legs based on their current speed and position
\cite{pendulum_length_giant_gymnastics}. In doing so, they are able to generate
enough momentum to perform backflips on the bar. When controlling mechanical
systems, standard practice is to have a robot's actuators track a function of
time. Much like the gymnast, it is often more natural to track a
function of the robot's current speed and position. The more recent techniques
of virtual constraints offer this ability and produce realistic, biologically
plausible motion.

Mechanical systems can be constrained in two ways. Holonomic constraints
restrict position (for instance, car tires remain a fixed distance apart), while
nonholonomic constraints restrict velocities (such as how those tires cannot
slide sideways). Previous research has shown that one can enforce holonomic
constraints through a robot's actuators, using the method of "virtual holonomic
constraints" (VHCs) \cite{vhcs_for_el_systems}. VHCs have been used to control
walking robots \cite{stable-walking}, autonomous bicycles \cite{bicycle},
gymnastics robots \cite{xingbo-thesis}, and snake robots \cite{snake-robot}
among others. These VHCs are a special case of virtual nonholonomic constraints
(VNHCs); in my master's thesis \cite{my-thesis} I developed a general theory for
enforcing VNHCs and designed one which injects energy into the gymnastics robot
created by Wang \cite{xingbo-thesis}, thereby enabling it to perform backflips
on a bar.

\begin{large} \textbf{Proposal} \end{large}

Mohammadi \textit{et.al.} \cite{manfredi-orbit-stabilization} showed that a
single VHC can result in several motion patterns (known as ``orbits"), from
which one specific motion can be chosen if the robot is initialized with a
configuration that is ``close enough" to it. However, their method does not
work for all starting configurations of the robot. We propose to extend this
work and investigate a method to transition between VNHCs; doing so should allow
one to start anywhere and move the configuration close enough to stabilize the
chosen orbit.

VNHCs have been studied with applications to human-robot cooperation
\cite{vnhc-human-robot-coop} and walking robots \cite{vnhc-biped-robot}. Horn
\textit{et.al.} \cite{hybrid_zero_dynamics_bipedal_nhvcs} have also examined
properties of VNHCs for general mechanical systems and applied their results to
stabilizing a chosen gait on walking robots. Our approach is novel since we will
assume one has already stabilized a VNHC and develop techniques to transition to
a different VNHC while maintaining pre-specified safety constraints.

First we will study mathematical conditions under which on can transition
between two VNHCs. These conditions will allow us to create a control algorithm
to switch between motion patterns. After completing this mathematical
formulation, we will test the theory on a gymnastics robot by designing a
gymnastics routine which is performed by transitioning between different VNHCS. 

The advancement in mathematics from this research will enable improvements to
control of other robotic systems. The ability to transition between complex
constraints will allow for more natural expressive motion in many autonomous
systems and may become a standard technique for controlling biologically
inspired robots.

%---------- Bibliography ----------%
\printbibliography
\end{document}

% vim: set tw=80 ts=4 sw=4 sts=0 et ffs=unix :
