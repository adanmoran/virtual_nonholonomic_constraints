%! TEX program = lualatex

\documentclass[a4paper,12pt]{article}

%/========== Preamble ==========/%
%----------- Set Main Font ----------%
\usepackage{times} % Times New Roman

%----------- Choose Margins ----------%
\usepackage[a4paper,margin=1.87cm]{geometry}

%---------- Bibliography Style ----------%
\usepackage[style=ieee,backend=biber]{biblatex}
\addbibresource{bib.bib}

%----------- Math Packages -----------%
\usepackage{mathtools}

%---------- Remove indenting from paragraphs ----------%
% Also makes each new paragraph start with an vertical space.
\usepackage[parfill]{parskip}

%---------- Add name to Header ----------%
\usepackage{atbegshi,picture}
\AtBeginShipout{\AtBeginShipoutUpperLeft{%
        \put(\dimexpr\paperwidth-5.5cm\relax,-1.2cm){\makebox{\setlength{\fboxrule}{0pt}\framebox{Adan
        Moran-MacDonald}}}%
}}

%/========== Main Document ==========/%
\begin{document}

%---------- Title -----------%
\title{Energy Regulation for Biologically-inspired Robotics}
\author{}
\date{}
\maketitle

\vspace{-1cm} % Move the following text up
%\textbf{What is your area of research? What sort of problems do you want to
%    address? WHy is the area important, and why are the applications important?
%    What is special about your approach, and how is it different from
%    existing/competing approaches? What is a sample application of the outcomes
%of your thesis?}
%\textbf{Detailed but concise. Provide background to position your research
%    within current knowledge of the field. State significance of research to the
%    field. Outline approach to be taken (citing literature). Clearly state
%differences btwn masters and this work.}

%---------- Main Content ----------%
\begin{large}
\textbf{Background}
\end{large}

When a gymnast wants to do backflips around a horizontal bar, they start by
performing a technique called a ``giant" \cite{usagym_giant}. 
The gymnast swings their legs based on their current speed and position, thereby
generating enough momentum to rotate around the bar.

Now imagine the gymnast is a robot, and their creator is teaching them to
perform giants like a human.
If the roboticist had studied classical control theory, they would plot a human
gymnast's leg angle as a trajectory over time and tell the robot to synchronize
its legs with this trajectory.
While this is the standard approach in robotics, it is not appropriate for
biologically-inspired robots: it is susceptible to timing delays, it is not
robust to external disturbances, and it is \textit{not what humans do}.
After all, human gymnasts do not have an internal stopwatch telling them when to
move their legs.
%If our robot were pushed mid-swing, its legs would continue along the
%time-based trajectory but the leg motion would desynchronize with
%the swinging of the torso, making the robot slow down.
Instead, existing research suggests that human gymnasts actually move their legs
as a function of their body angle and velocity
\cite{pendulum_length_giant_gymnastics}. 
Rather than using time-based motion, a clever roboticist would attempt to
emulate this natural human behaviour, which is well described by the method
of virtual constraints.

There are two types of constraints on a mechanical system:
\textit{holonomic} constraints restrict position (e.g. a snake robot can slither
on the ground but cannot fly),
while \textit{nonholonomic} constraints restrict both position and
velocity (e.g. an autonomous car with normal wheels cannot slide sideways).
It is often possible to use a robot's actuators to enforce a desired constraint
which is engineered to achieve a safety or motion planning goal.
Since this constraint is enforced by the robot's actuators, and not by the
physics of the system, it is known as a ``virtual" constraint
\cite{vhcs_for_el_systems}.
Virtual holonomic constraints (VHCs) have been used to control walking robots
\cite{stable_walking}, autonomous bicycles \cite{bicycle},
helicopters \cite{helicopter}, and snake robots \cite{snake_robot},
among other applications.
Unfortunately, VHCs cannot adequately recreate many animal behaviours.
For example, they cannot perfectly recreate the giant motion of gymnastics 
because they do not incorporate information about the gymnast's velocity
\cite{xingbo_thesis}.
This is where virtual nonholonomic constraints (VNHCs) are most useful.

The modern concept of VNHCs was described by
Griffin and Grizzle \cite{nhvc_dynamic_walking} in 2015, though there are
references to preliminary versions going back as early as the year 2000
\cite{vnhc_human_robot_coop}.
VNHCs have been of most notable use in bipedal walking robots,
where they show marked improvements in the robustness of walking gaits
when compared to previous control techniques
\cite{nhvc_gait_optimization,output_nhvc_bipedal_control}.
They have also been used for error-reduction in time-delayed teleoperation
\cite{vnhc_time_delay_teleop} and in the field of human-robot interaction
\cite{psd_based_vnhc_redundant_manipulator,haptic_vnhc}.
Horn \textit{et.~al.}~\cite{hybrid_zero_dynamics_bipedal_nhvcs} derived the 
equations of motion for robots constrained by VNHCs, and they used VNHCs to
improve the gait of walking robots on a variable-slope terrain
\cite{nhvc_incline_walking}.

\begin{large} \textbf{Proposal} \end{large}

Virtual constraints have proven useful for generating biologically-realistic
behaviour in robotic systems.
In my master's thesis I showed that VNHCs also allow for 
\textit{energy regulation}, i.e., they can inject energy into (or dissipate
energy from) a certain class of mechanical systems \cite{my-thesis}.
One such mechanical system is a gymnastics robot called the
\textit{acrobot}, for which I designed a VNHC that generates giant-like motion. 
This VNHC enables the acrobot to gain energy and perform backflips on a
horizontal bar.

Using VNHCs for \textit{energy injection} is beneficial because it allows robots to
safely and reliably increase their momentum.
Likewise, designing VNHCs for \textit{energy dissipation} allows robots to
safely and realistically slow down.
In this PhD, I propose to further study the energy regulation properties of VNHCs.
First, we will find mathematical conditions under which VNHCs are guaranteed to
inject or dissipate energy in fixed-base robots.
Then we will design a means of transitioning between two different VNHCs,
enabling us to regulate a robot's energy in a provably safe manner.
We will experimentally verify the results of this theory on the acrobot by
designing a gymnastics routine which is performed solely by transitioning
between different virtual constraints.
Finally, we will extend the theory to mobile animal-like robots,
and we will build a robotic monkey which uses VNHCs to swing on monkey bars just
like a real monkey would swing through trees.

The advancements in mathematics from this research will bring improvements
to the control of all autonomous systems. 
The ability to safely transition between complex constraints by adding or
removing energy will allow for more expressive motion, and may become a standard
technique for controlling biologically inspired robots.

\newpage
%---------- Bibliography ----------%
\printbibliography
\end{document}

% vim: set tw=80 ts=4 sw=4 sts=0 et ffs=unix :
