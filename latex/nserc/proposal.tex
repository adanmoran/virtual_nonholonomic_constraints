%! TEX program = lualatex

\documentclass[a4paper,12pt]{article}

%/========== Preamble ==========/%
%----------- Set Main Font ----------%
\usepackage{times} % Times New Roman

%----------- Choose Margins ----------%
\usepackage[a4paper,margin=1.87cm]{geometry}

%---------- Bibliography Style ----------%
\usepackage[style=ieee,backend=biber]{biblatex}
\addbibresource{bib.bib}

%----------- Math Packages -----------%
\usepackage{mathtools}

%---------- Remove indenting from paragraphs ----------%
% Also makes each new paragraph start with an vertical space.
\usepackage[parfill]{parskip}

%---------- Add name to Header ----------%
\usepackage{atbegshi,picture}
\AtBeginShipout{\AtBeginShipoutUpperLeft{%
        \put(\dimexpr\paperwidth-5.5cm\relax,-1.2cm){\makebox{\setlength{\fboxrule}{0pt}\framebox{Adan
        Moran-MacDonald}}}%
}}

%/========== Main Document ==========/%
\begin{document}

%---------- Title -----------%
\title{Energy Regulation for Biologically-inspired Robotics}
\author{}
\date{}
\maketitle

\vspace{-1cm} % Move the following text up
\textbf{What is your area of research? What sort of problems do you want to
    address? WHy is the area important, and why are the applications important?
    What is special about your approach, and how is it different from
    existing/competing approaches? What is a sample application of the outcomes
of your thesis?}

%---------- Main Content ----------%
\begin{large}
\textbf{Background}
\end{large}

When a gymnast wants to do backflips around a horizontal bar, they start by
performing a technique called a ``giant" \cite{usagym_giant}. 
The gymnast swings their legs based on their current speed and position, thereby
generating enough momentum to rotate around the bar.

Now imagine the gymnast is a robot, and their creator is teaching them to act
like a human.
If the roboticist had studied classical control theory, they would plot a
human gymnast's hip angle as a trajectory over time and tell the robot to synchronize
its legs with this trajectory.
While this approach often works for simple systems, it is not great for
biologically-inspired robots: it can cause timing delays, it is not robust to
many external disturbances, and it is \textit{not what humans do}.
After all, human gymnasts do not have an internal stopwatch telling them when to
extend or retract their legs.
Instead, existing research suggests that human gymnasts actually move their legs
as a function of their body angle and velocity
\cite{pendulum_length_giant_gymnastics}.
Rather than using time-based motion, a clever roboticist would attempt to
emulate this more natural human behaviour.
% by driveing robot's legs according to some function of its torso speed and position.
This type of behaviour is well described by the method of virtual constraints.

There are two types of constraints on a mechanical system:
\textit{holonomic} constraints restrict position (e.g. a snake robot can slither
but not fly),
while \textit{nonholonomic} constraints restrict position and
velocity (e.g. cars cannot slide sideways).
One can design a constraint they want their robot to satisfy and enforce it
using the robot's actuators.
Since this constraint is not enforced by physics, it is known as a ``virtual"
constraint \cite{vhcs_for_el_systems}.
Virtual holonomic constraints (VHCs) have been used to control walking robots
\cite{stable-walking}, autonomous bicycles \cite{bicycle},
gymnastics robots \cite{xingbo-thesis}, and snake robots \cite{snake-robot}.
Virtual nonholonomic constraints (VNHCs) have been used for human-robot
cooperation \cite{vnhc-human-robot-coop} and for improving robustness of walking
robots \cite{vnhc-biped-robot,hybrid_zero_dynamics_bipedal_nhvcs}, among other
applications.

\begin{large} \textbf{Proposal} \end{large}

Virtual constraints have proven useful for generating biologically-inspired
behaviour in robotic systems.
In my master's thesis, I demonstrated that VNHCs can inject/dissipate energy
into/from robots:
I designed a VNHC which enabled a gymnastics robot to gain energy
and perform backflips on a horizontal bar \cite{my-thesis}.
Using VNHCs for \textit{energy injection} is beneficial because it allows robots to
safely increase their momentum.
Likewise, designing VNHCs for \textit{energy dissipation} allows robots to
safely slow down in a realistic and reliable manner. 

We propose to study the energy injection and dissipation properties of VNHCs.
First, we will find mathematical conditions under which VNHCs are guaranteed to
inject/dissipate energy in arbitrary mechanical systems.
Then we will design a means of transitioning between two different VNHCs,
enabling us to regulate energy in a provably safe manner.
Finally, we will test the theory on two robots: a gymnastics robot and a robotic
monkey.
For the gymnastics robot (which is already built), we will design a gymnastics
routine which is performed solely by transitioning between different virtual
constraints.
After this, we will build the robotic monkey and design virtual constraints
which allow it to swing on monkey bars, just like real monkeys swing through
trees.

The advancements in mathematics from this research will bring improvements to
the control of autonomous systems. 
The ability to safely transition between complex constraints by adding or
removing energy will allow for more expressive motion, and may become a standard
technique for controlling biologically inspired robots.

%---------- Bibliography ----------%
\printbibliography
\end{document}

% vim: set tw=80 ts=4 sw=4 sts=0 et ffs=unix :
