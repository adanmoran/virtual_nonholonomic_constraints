%! TEX program = lualatex

\documentclass[a4paper,12pt]{article}

%/========== Preamble ==========/%
%----------- Set Main Font ----------%
\usepackage{times} % Times New Roman

%----------- Choose Margins ----------%
\usepackage[a4paper,margin=1.87cm]{geometry}

%---------- Bibliography Style ----------%
\usepackage[style=ieee,backend=biber]{biblatex}
\addbibresource{bib.bib}

%----------- Math Packages -----------%
\usepackage{mathtools}

%---------- Remove indenting from paragraphs ----------%
% Also makes each new paragraph start with an vertical space.
\usepackage[parfill]{parskip}

%---------- Add name to Header ----------%
\usepackage{atbegshi,picture}
\AtBeginShipout{\AtBeginShipoutUpperLeft{%
        \put(\dimexpr\paperwidth-5.5cm\relax,-1.2cm){\makebox{\setlength{\fboxrule}{0pt}\framebox{Adan
        Moran-MacDonald}}}%
}}

%/========== Main Document ==========/%
\begin{document}

%---------- Title -----------%
\subsection*{Part I: Contributions to Research and Development}
\subsubsection*{d. Non-peer-reviewed Contributions}
\textbf{Moran-MacDonald, A.} \& Alajaji, F. (2018). The R\'{e}nyi Deterministic
Information Bottleneck and Geometric Clustering. NSERC summer research project,
Queen's University.

\subsection*{Part II: Most Significant Contributions to Research \& Development}
%\textit{Describe your role in research, clarify your contribution, provide
%    details on significance of tech report, indicate collaborations, discuss
%relevance to engineering practice (if appropriate)}
``The R\'{e}nyi Deterministic Information Bottleneck and Geometric
Clustering" is a technical report about a new information-theoretic tool for
clustering datapoints.
I wrote this report in its entirety, and I contributed all the theoretical and
simulation results (with abundant advice from my supervisor, Fady Alajaji).
This report was used by two undergraduate groups at Queen's University for their
honours theses in 2019, and it motivated further research in [1].

[1] J.J. Weng, F. Alajaji and T. Linder, \textit{An information bottleneck
problem with R\'{e}nyi's Entropy}, Proceedings of the IEEE International
Symposium on Information Theory, Melbourne, Australia, July 2021.

\subsection*{Part III: Applicant's Statement}
\subsubsection*{Research Experience}
%\textit{Describe scientific and engineering abilities you gained in past
%    research experience, including projects, honours thesis, coop reports. If
%    you have work experience, discuss relevance of that to your proposed field
%of study and what benefits you gained.}
The following list outlines the scientific and engineering abilities I gained in
my research and professional roles.
\begin{itemize}
    \item \textbf{MASc Research, University of Toronto, 2018 -- 2021}.
        In my master's degree, I learned several important research skills:
        my supervisor taught me to correctly formulate research questions;
        I learned the tenacity and dedication required to solve a
        complex problem; I developed my creativity by attacking problems from
        multiple directions; I learned to read papers, perform a literature
        review, think critically about the literature, and apply existing
        results to my research; I improved my time management skills, practiced
        explaining my work (both orally and in writing), and improved my
        presentation skills for the thesis defense; and I learned to write
        research papers and a thesis, which are important tools for communicating my
        work to the engineering community.
        I also learned several engineering skills, including advanced control
        techniques, simulation of physical systems in MATLAB, and implementation
        of control systems on a physical robot.

    \item \textbf{Radar Developer, University of Toronto, 2018}.
        In the group project for a graduate course, I worked with three other
        students to write software which reads data from a radar device.
        This project improved my project management skills, as I kept our group
        meetings organized and separated the project tasks in such a way
        that everyone was challenged appropriately.
        I learned to read hardware documentation and to wire a device from
        schematics.
        I wrote networking code in Python to interface with the
        device, and validated the hardware's capabilities for myself by
        creating appropriate test environments. 
        All these skills will be beneficial for my PhD research, where I will
        collaborate with labmates to build a robotic monkey.

    \item \textbf{Undergraduate Thesis, Queen's University, 2017 -- 2018}.
        For my undergraduate thesis, I worked with my peers to design an image
        filtering algorithm by applying theoretical results from information theory.
        During this project I learned to perform a basic literature review,
        and to manage deadlines with a supervisor.
        I also learned to write reports in LaTeX,
        and my group worked together on code optimization: we learned to
        convert equations into matrix form to take advantage of hardware
        optimization; we parallelized our code so our algorithm would run faster
        through multithreading; we ported MATLAB code to C++ to achieve even
        faster results; and we created a GUI so we could test our code quickly
        and easily.

    \item \textbf{Software Developer, Neptec Technologies Corp., 2016 -- 2017}.
        My experience at Neptec was integral to my career because it convinced
        me to pursue research in robotics.
        Working at Neptec exposed me to many areas of electrical and computer
        engineering.
        I worked on several embedded software projects, my favourite of which
        was a control systems project.
        After completing this work, I decided to do a master's (and a PhD) so I
        could learn as much about control as possible.
        At Neptec I also learned to work with large codebases,
        and I became extremely proficient at software development, a skill which
        has served me well in all aspects of my research.
        
\end{itemize}

\subsubsection*{Relevant Activities}
%\textit{Describe professional, academic, extracurricular activities that
%    demonstrate communication, interpersonal and leadership skills, e.g.
%    teaching, mentoring, managing projects, volunteer work, committee work,
%participation in organizations and clubs, or work experience}.

Over the course of my career, I have enjoyed several extracurricular
roles which developed my communication, interpersonal, and leadership skills.
Here is a brief description of some of these roles.
\begin{itemize}
\item \textbf{Lionel Massey Fund Co-chair, Massey College, September 2021 -- Present}.
    I am one of six co-chairs for the Lionel Massey Fund, which is the events
    committee for Massey College (an interdisciplinary college at the University
    of Toronto).
    I have already organized 13 events for members of the college, and will
    continue to organize events throughout the year.
    This role has taught me the logistics and leadership skills required to
    manage large events for up to 60 people.
    I have also learned to make events accessible to everyone, and to
    communicate with attendees so I can meet their needs.

\item \textbf{Jiu-Jitsu Instructor, 2019 -- Present}. 
    I am an instructor at the University of Toronto Jiu-Jitsu club. I teach
    martial arts while fostering a safe, friendly atmosphere that
    encourages learning through action.
    This role has refined my communication and interpersonal skills:
    it has taught me to create learning plans for my students;
    break down difficult concepts into manageable chunks; 
    and to reward students for trying their best, even if they make a mistake.

\item \textbf{Teaching Assistant, 2017 -- Present}.
    I have been a teaching assistant for 14 engineering courses in the past 5 years.
    Each course I teach improves my communication skills, and each tutorial gives
    me the opportunity to make the course material more fun and engaging.
    I focus most on attaining high levels of student participation by being
    kind and non-judgemental, which makes students feel
    comfortable enough to ask more questions.

\item \textbf{Science Payload Designer, Queen’s University Satellite Design
    Team (QSAT), 2017 -- 2018}. 
    I took on a mentorship role at QSAT, where I taught C++, Git, and information
    theory to the team's software developers. I helped them organize
    the satellite codebase, managed all code reviews, and provided advice for my
    peers rather than writing software myself.

\item \textbf{Head Programming Mentor, FRC Team 2809, 2013 -- 2015}. 
    In this role, I mentored high school students in Object Oriented programming
    for robotics software. I taught students to program a robot in under 6
    weeks, which earned us 3rd place at the Ontario provincial competition in
    2014.
    In this role I learned to lead my students by giving them agency over the
    project, and to communicate with people from all over the world in
    high-stress situations.

\end{itemize}

I take pride in my teaching, communication, and leadership abilities, yet I
remain aware that these can always be improved.
Obtaining an NSERC scholarship will provide me with the financial liberty to
participate in more volunteer roles, where I can improve my skills while giving back to my
community.

\end{document}

% vim: set tw=80 ts=4 sw=4 sts=0 et ffs=unix :
