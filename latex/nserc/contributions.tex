%! TEX program = lualatex

\documentclass[a4paper,12pt]{article}

%/========== Preamble ==========/%
%----------- Set Main Font ----------%
\usepackage{times} % Times New Roman

%----------- Choose Margins ----------%
\usepackage[a4paper,margin=1.87cm]{geometry}

%---------- Bibliography Style ----------%
\usepackage[style=ieee,backend=biber]{biblatex}
\addbibresource{bib.bib}

%----------- Math Packages -----------%
\usepackage{mathtools}

%---------- Remove indenting from paragraphs ----------%
% Also makes each new paragraph start with an vertical space.
\usepackage[parfill]{parskip}

%---------- Add name to Header ----------%
\usepackage{atbegshi,picture}
\AtBeginShipout{\AtBeginShipoutUpperLeft{%
        \put(\dimexpr\paperwidth-5.5cm\relax,-1.2cm){\makebox{\setlength{\fboxrule}{0pt}\framebox{Adan
        Moran-MacDonald}}}%
}}

%/========== Main Document ==========/%
\begin{document}

%---------- Title -----------%
\subsection*{Part I: Contributions to Research and Development}
\subsubsection*{d. Non-peer-reviewed Contributions}
\textbf{Moran-MacDonald, A.} \& Alajaji, F. (2018). The R\'{e}nyi Deterministic
Information Bottleneck and Geometric Clustering. NSERC summer research project,
Queen's University.

\subsection*{Part II: Most Significant Contributions to Research \& Development}
\textit{Describe your role in research, clarify your contribution, provide
    details on significance of tech report, indicate collaborations, discuss
relevance to engineering practice (if appropriate)}
``The R\'{e}nyi Deterministic Information Bottleneck and Geometric
Clustering" is a technical report in which we develop a new
information-theoretic tool for clustering datapoints.
I wrote the report in its entirety, and I produced the theoretical results with
the help and advice of my supervisor, Fady Alajaji.
In the report, I summarized the existing theory on the
\textit{information bottleneck} (IB) and its variant, the \textit{deterministic
information bottleneck} (DIB), which are tools from information theory that
have been touted as a rigorous foundation for Deep Neural Networks. 
The DIB in particular is useful for clustering data.
Its equations are defined using an information-theoretic concept called
``Shannon's Mutual Information" (SMI).
In this report, Alajaji suggested we extend the equations by using a more
general concept called ``R\'{e}nyi's Mutual Information" to get the aptly-named
\textit{R\'{e}nyi's deterministic information bottleneck} (RDIB).
I compared the efficacy of the DIB and RDIB for clustering data, and we found
that the RDIB is more robust when the probability distribution of the data model 
does not match the true distirbution which generated the data.
This report was used by two undergraduate thesis groups at Queen's University in
2019, and it motivated further published research [1].

[1] J.J. Weng, F. Alajaji and T. Linder, \textit{An information bottleneck
problem with R\'{e}nyi's Entropy}, Proceedings of the IEEE International
Symposium on Information Theory, Melbourne, Australia, July 2021.

\subsection*{Part III: Applicant's Statement}
\subsubsection*{Research Experience}
\textit{Describe scientific and engineering abilities you gained in past
    research experience, including projects, honours thesis, coop reports. If
    you have work experience, discuss relevance of that to your proposed field
of study and what benefits you gained.}
Masters Research skills
Undergrad thesis research skills
Neptec skills

\subsubsection*{Relevant Activities}
\textit{Describe professional, academic, extracurricular activities that
    demonstrate communication, interpersonal and leadership skills, e.g.
    teaching, mentoring, managing projects, volunteer work, committee work,
participation in organizations and clubs, or work experience}.

Over the course of my career, I have enjoyed several extracurricular roles which
developed my communication, interpersonal, and leadership skills.
What follows is a brief description of some of these roles, along with a summary
of what I learned.
\begin{itemize}
\item \textbf{Lionel Massey Fund Co-chair, Massey College, Sep. 2021 -- Present}.
    I am one of six co-chairs for the Lionel Massey Fund (LMF) -- the social
    committee for Massey College. Thus far I have dramatically improved my
    communication and leadership skills by organizing 13 orientation events for
    all 150 members of the college. I have learned to communicate problems
    early, and to ask people what they want or expect so I know how to best
    serve their needs.

\item \textbf{Teaching Assistant positions, 2017 -- Present}.
    I have been a teaching assistant for 14 engineering courses in the past 5 years.
    Some of these were in-person, some were online, but in both cases I always
    learned a new ways to make the material more engaging.
    I focus most on attaining student participation, and I actively learn each
    student's name to make them feel comfortable asking questions. When teaching
    pre-recorded material, I use many examples and make the lectures as easily
    digestible as possible to account for the difficulties of online
    interactions.

\item \textbf{Jiu-Jitsu Instruction}. 
    I am an instructor at the University of Toronto Jiu-Jitsu club. I teach
    people to be confident in what their bodies can accomplish while fostering a
    safe, friendly atmosphere that encourages learning through action.
    This role has refined my communication and interpersonal skills:
    it has taught me to create learning plans for my students;
    break down difficult concepts into manageable chunks; 
    and to reward mistakes, because the best way to learn is to try, fail, and
    then try again anyway.

\item \textbf{Radar Developer}
    TODO: managed the project, led the software design, and made sure everyone
    was learning and was participating so they would have a feeling of agency
    over the project.

    TODO: These need to be edited for more than just teaching, but also to show
    what I learned and how I showed leadership and communication.
\item \textbf{Queen’s University Satellite Design Team}. 
    I taught C++ to the team’s software manager and other developers. I worked
    with the manager to create good design patterns for the satellite’s
    software. I also taught three first-year students how to use information
    theory to improve one of the algorithms used on the satellite.

\item \textbf{Resource Manager at Queen’s University}. 
    I created new workbooks which intuitively explained difficult concepts for 9
    first-year engineering courses. The workshops were attended by over 200
    first-year students.
    TODO: might be unnecessary

\item \textbf{Head Programming mentor for FIRST Team 2809}. 
    I mentored high school students in Object Oriented programming and robotics
    software. Together we programmed a robot in under 6 weeks and placed 3rd at
    the Ontario provincial competition in 2014.
    TODO: Learned to let go of pride, give power to students, and communicate
    with other teams in a high-stress scenario.



I take pride in my teaching abilities. I always look for teaching opportunities
and strive to become a better educator by reducing knowledge to its
fundamentals. It is one of my goals in life to teach and learn as much as
possible; the variety of roles I have taken on are a demonstration of this. By
teaching APS106 for the third summer in a row, I aim to give students confidence
in their programming abilities while improving my own teaching skills.

\end{document}

% vim: set tw=80 ts=4 sw=4 sts=0 et ffs=unix :
