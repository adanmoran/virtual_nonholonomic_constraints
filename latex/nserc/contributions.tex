%! TEX program = lualatex

\documentclass[a4paper,12pt]{article}

%/========== Preamble ==========/%
%----------- Set Main Font ----------%
\usepackage{times} % Times New Roman

%----------- Choose Margins ----------%
\usepackage[a4paper,margin=1.87cm]{geometry}

%---------- Bibliography Style ----------%
\usepackage[style=ieee,backend=biber]{biblatex}
\addbibresource{bib.bib}

%----------- Math Packages -----------%
\usepackage{mathtools}

%---------- Remove indenting from paragraphs ----------%
% Also makes each new paragraph start with an vertical space.
\usepackage[parfill]{parskip}

%---------- Add name to Header ----------%
\usepackage{atbegshi,picture}
\AtBeginShipout{\AtBeginShipoutUpperLeft{%
        \put(\dimexpr\paperwidth-5.5cm\relax,-1.2cm){\makebox{\setlength{\fboxrule}{0pt}\framebox{Adan
        Moran-MacDonald}}}%
}}

%/========== Main Document ==========/%
\begin{document}

%---------- Title -----------%
\subsection*{Part I: Contributions to Research and Development}
\subsubsection*{d. Non-peer-reviewed Contributions}
\textbf{Moran-MacDonald, A.} \& Alajaji, F. (2018). The R\'{e}nyi Deterministic
Information Bottleneck and Geometric Clustering. NSERC summer research project,
Queen's University.

\subsection*{Part II: Most Significant Contributions to Research \& Development}
\textit{Describe your role in research, clarify your contribution, provide
    details on significance of tech report, indicate collaborations, discuss
relevance to engineering practice (if appropriate)}
``The R\'{e}nyi Deterministic Information Bottleneck and Geometric
Clustering" is a technical report about a new information-theoretic tool for
clustering datapoints.
My contribution was substantial: I produced the theoretical and simulation
results (with lots of advice from my supervisor, Fady Alajaji), and then I wrote
the report in its entirety.
%In the report, I summarized the existing theory on the
%\textit{information bottleneck} (IB) and its variant, the \textit{deterministic
%information bottleneck} (DIB), which are tools from information theory that
%have been touted as a rigorous foundation for Deep Neural Networks. 
%The DIB in particular is useful for clustering data.
%Its equations are defined using an information-theoretic concept called
%``Shannon's Mutual Information" (SMI).
%In this report, Alajaji suggested we extend the equations by using a more
%general concept called ``R\'{e}nyi's Mutual Information" to get the aptly-named
%\textit{R\'{e}nyi's deterministic information bottleneck} (RDIB).
%I compared the efficacy of the DIB and RDIB for clustering data, and we found
%that the RDIB is more robust when the probability distribution of the data model 
%does not match the true distirbution which generated the data.
This report was used by two undergraduate groups at Queen's University for their
honours theses in 2019. 
It also motivated further research, which was published in [1].

[1] J.J. Weng, F. Alajaji and T. Linder, \textit{An information bottleneck
problem with R\'{e}nyi's Entropy}, Proceedings of the IEEE International
Symposium on Information Theory, Melbourne, Australia, July 2021.

\subsection*{Part III: Applicant's Statement}
\subsubsection*{Research Experience}
\textit{Describe scientific and engineering abilities you gained in past
    research experience, including projects, honours thesis, coop reports. If
    you have work experience, discuss relevance of that to your proposed field
of study and what benefits you gained.}
The following list outlines the scientific and engineering abilities I gained in
my research and professional roles.
\begin{itemize}
    \item \textbf{MASc Research, University of Toronto, 2018 -- 2021}
        TODO.
        Research skills: time management, critical thinking, coming up with
        well-formulated research question, tenacity and creativity
        (always ``attacking" the problem from multiple directions), dedication
        to solving a complex problem, explaining
        my work both orally and in writing, presentation skills, 
        literature reviews and how to read papers
        Engineering skills: nonlinear control techniques, implementation of
        control systems on hardware

    \item \textbf{Radar Developer, 2018}
        TODO:
        Managing a software project, implementing software for a hardware
        system, validating documentation (i.e. never trust it blindly).

    \item \textbf{Undergraduate Thesis, Queen's University, 2017 -- 2018}
        TODO:
        First time doing a research project. Collaborating with fellow
        researchers. Coming up with an application for existing research. Basic
        literature review skills, report writing. Code optimization (turning
        equations into matrix form for speed, parallelization, C++ vs MATLAB),
        creating a GUI for users.

    \item \textbf{Software Developer, Neptec Technologies Crop, 2016 -- 2017}
        My experience at Neptec was incredibly beneficial.
        I worked on several embedded software projects, including a control
        systems project. The exposure to various areas of electrical and
        computer engineering is what convinced me to pursue research in the
        first place.
        I learned to work on large software projects, where multiple people work
        together on the same topic.
        I also became extremely proficient at software development, a skill
        which has served me well in my research experiments on physical robots.
        
\end{itemize}

\subsubsection*{Relevant Activities}
\textit{Describe professional, academic, extracurricular activities that
    demonstrate communication, interpersonal and leadership skills, e.g.
    teaching, mentoring, managing projects, volunteer work, committee work,
participation in organizations and clubs, or work experience}.

Over the course of my career, I have enjoyed several extracurricular roles which
developed my communication, interpersonal, and leadership skills.
What follows is a brief description of some of these roles, along with a summary
of what I learned.
\begin{itemize}
\item \textbf{Lionel Massey Fund Co-chair, Massey College, September 2021 -- Present}.
    I am one of six co-chairs for the Lionel Massey Fund (LMF) -- the social
    committee for Massey College, a graduate college at the University of
    Toronto. Thus far I have dramatically improved my
    communication and leadership skills by managing 13 orientation events for
    150 members of the college. I have learned to explicitly communicate my needs
    with others, and to ask questions to understand their needs so we all succeed.

\item \textbf{Teaching Assistant, 2017 -- Present}.
    I have been a teaching assistant for 14 engineering courses in the past 5 years.
    Each course has taught me better teaching skills, along with new ways of
    making the material more engaging.
    I focus most on attaining high student participation, and I find
    commonalities between myself and students so they feel comfortable asking
    questions. 

\item \textbf{Jiu-Jitsu Instructor, 2019 -- Present}. 
    I am an instructor at the University of Toronto Jiu-Jitsu club. I teach
    people martial arts while fostering a safe, friendly atmosphere that
    encourages learning through action.
    This role has refined my communication and interpersonal skills:
    it has taught me to create learning plans for my students;
    break down difficult concepts into manageable chunks; 
    and reward mistakes, because the best way to learn is to try, fail, and
    to always try again.

\item \textbf{Science Payload Designer, Queen’s University Satellite Design
    Team, 2017 -- 2018}. 
    In this role, I taught C++ and information theory to the team’s software
    developers. I also learned project management skills, because I organized
    the satellite codebase but took on a supportive role for my peers (rather
    than writing the software myself).

\item \textbf{Head Programming Mentor, FIRST Team 2809, 2013 -- 2015}. 
    I mentored high school students in Object Oriented programming for robotics
    software. Together we programmed a robot in under 6 weeks and placed 3rd at
    the Ontario provincial competition in 2014.
    In this role I learned to give my students agency over the project, and to
    communicate with people from all over the world in high-stress situations.

\end{itemize}

I take pride in my teaching, communication, and leadership abilities.
Obtaining an NSERC scholarship will provide me with the financial liberty to
participate in more volunteer roles, where I can better myself while giving back to my
community.

\end{document}

% vim: set tw=80 ts=4 sw=4 sts=0 et ffs=unix :
