%! TEX program = lualatex

\documentclass[a4paper,12pt]{article}

%/========== Preamble ==========/%
%----------- Set Main Font ----------%
\usepackage{times} % Times New Roman

%----------- Choose Margins ----------%
\usepackage[a4paper,margin=1.87cm]{geometry}

%---------- Bibliography Style ----------%
\usepackage[style=ieee,backend=biber]{biblatex}
\addbibresource{bib.bib}

%----------- Math Packages -----------%
\usepackage{mathtools}

%---------- Remove indenting from paragraphs ----------%
% Also makes each new paragraph start with an vertical space.
\usepackage[parfill]{parskip}

%---------- Add name to Header ----------%
\usepackage{atbegshi,picture}
\AtBeginShipout{\AtBeginShipoutUpperLeft{%
        \put(\dimexpr\paperwidth-5.5cm\relax,-1.2cm){\makebox{\setlength{\fboxrule}{0pt}\framebox{Adan
        Moran-MacDonald}}}%
}}

%/========== Main Document ==========/%
\begin{document}

%---------- Title -----------%
\subsection*{Part I: Contributions to Research and Development}
\subsubsection*{d. Non-peer-reviewed Contributions}
\textbf{Moran-MacDonald, A.} \& Alajaji, F. (2018). The R\'{e}nyi Deterministic
Information Bottleneck and Geometric Clustering. NSERC summer research project,
Queen's University.

\subsection*{Part II: Most Significant Contributions to Research \& Development}
\textit{Describe your role in research, clarify your contribution, provide
    details on significance of tech report, indicate collaborations, discuss
relevance to engineering practice (if appropriate)}
``The R\'{e}nyi Deterministic Information Bottleneck and Geometric
Clustering" is a technical report in which we develop a new
information-theoretic tool for clustering datapoints.
I wrote the report in its entirety, and I produced the theoretical results with
the help and advice of my supervisor, Fady Alajaji.
In the report, I summarized the existing theory on the
\textit{information bottleneck} (IB) and its variant, the \textit{deterministic
information bottleneck} (DIB), which are tools from information theory that
have been touted as a rigorous foundation for Deep Neural Networks. 
The DIB in particular is useful for clustering data.
Its equations are defined using an information-theoretic concept called
``Shannon's Mutual Information" (SMI).
In this report, Alajaji suggested we extend the equations by using a more
general concept called ``R\'{e}nyi's Mutual Information" to get the aptly-named
\textit{R\'{e}nyi's deterministic information bottleneck} (RDIB).
I compared the efficacy of the DIB and RDIB for clustering data, and we found
that the RDIB is more robust when the probability distribution of the data model 
does not match the true distirbution which generated the data.
This report was used by two undergraduate thesis groups at Queen's University in
2019, and it motivated further published research [1].

[1] J.J. Weng, F. Alajaji and T. Linder, \textit{An information bottleneck
problem with R\'{e}nyi's Entropy}, Proceedings of the IEEE International
Symposium on Information Theory, Melbourne, Australia, July 2021.

\subsection*{Part III: Applicant's Statement}
\subsubsection*{Research Experience}
\textit{Describe scientific and engineering abilities you gained in past
    research experience, including projects, honours thesis, coop reports. If
    you have work experience, discuss relevance of that to your proposed field
of study and what benefits you gained.}

\subsubsection*{Relevant Activities}
\textit{Describe professional, academic, extracurricular activities that
    demonstrate communication, interpersonal and leadership skills, e.g.
    teaching, mentoring, managing projects, volunteer work, committee work,
participation in organizations and clubs, or work experience}.

\end{document}

% vim: set tw=80 ts=4 sw=4 sts=0 et ffs=unix :
