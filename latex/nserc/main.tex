%! TEX program = lualatex

\documentclass[a4paper,12pt]{article}

%/========== Preamble ==========/%
%----------- Set Main Font ----------%
\usepackage{times} % Times New Roman

%----------- Choose Margins ----------%
\usepackage[a4paper,margin=1.87cm]{geometry}

%---------- Bibliography Style ----------%
\usepackage[style=ieee,backend=biber]{biblatex}
\addbibresource{bib.bib}

%----------- Math Packages -----------%
\usepackage{mathtools}

%---------- Remove indenting from paragraphs ----------%
% Also makes each new paragraph start with an vertical space.
\usepackage[parfill]{parskip}

%---------- Add name to Header ----------%
\usepackage{atbegshi,picture}
\AtBeginShipout{\AtBeginShipoutUpperLeft{%
        \put(\dimexpr\paperwidth-5.5cm\relax,-1.2cm){\makebox{\setlength{\fboxrule}{0pt}\framebox{Adan
        Moran-MacDonald}}}%
}}

%/========== Main Document ==========/%
\begin{document}

%---------- Title -----------%
\title{Energy Regulation for Biologically-inspired Robotics}
\author{}
\date{}
\maketitle

%---------- Main Content ----------%
\vspace{-1cm}
\begin{large}
\textbf{Background}
\end{large}

When a gymnast on a horizontal bar wants to swing up to do backflips, they
typically perform a technique called a ``giant".
This technique involves the gymnast swinging their legs based on their current
speed and position, thereby generating enough momentum to rotate around the bar
\cite{pendulum_length_giant_gymnastics}.  

Standard practice in robotics begins with finding a trajectory over time
replicating a desired behaviour, then using the robot's actuators to track this
time-based trajectory. 
While this often works for simple systems, it can cause timing delays, it is not
robust to external disturbances, and it is not what humans do.
A more natural approach is to constrain the robot's actuators according to
some function of speed and position.
This type of behaviour is well described by the method of virtual constraints.

There are two types of constraints on a mechanical system:
\textit{holonomic} constraints restrict position (e.g. a snake robot can slither
but not fly),
while \textit{nonholonomic} constraints restrict position and
velocity (e.g. cars cannot slide sideways).
One can design a specific constraint for a robot and enforce it using the
actuators.
Since this constraint is not enforced by physics, it is known as a ``virtual"
constraint
\cite{vhcs_for_el_systems}.
Virtual holonomic constraints (VHCs) have been used to control walking robots
\cite{stable-walking}, autonomous bicycles \cite{bicycle},
gymnastics robots \cite{xingbo-thesis}, and snake robots \cite{snake-robot}.
Virtual nonholonomic constraints (VNHCs) have been used for human-robot
cooperation \cite{vnhc-human-robot-coop} and for improving robustness of walking
robots \cite{vnhc-biped-robot,hybrid_zero_dynamics_bipedal_nhvcs}, among other
applications.

\begin{large} \textbf{Proposal} \end{large}

Virtual constraints have proven useful for generating biologically-inspired
behaviour in robotic systems.
In my master's thesis, I demonstrated that VNHCs can also inject energy into
robots:
I designed a VNHC which enabled a gymnastics robot to gain energy
and perform backflips on a horizontal bar \cite{my-thesis}.
Using VNHCs for \textit{energy injection} is beneficial because it allows robots to
safely increase their momentum.
Likewise, designing VNHCs for \textit{energy dissipation} allows robots to
safely slow down in a realistic and reliable manner. 

We propose to study the energy injection and dissipation properties of VNHCs.
First, we will find mathematical conditions under which VNHCs are guaranteed to
inject/dissipate energy in abstract mechanical systems.
Then we will design a means of safely transitioning between two different VNHCs.
Finally, we will test the theory on a gymnastics robot by designing a gymnastics
routine which is performed solely by transitioning between different virtual
constraints.

The advancements in mathematics from this research will bring improvements to
the control of autonomous systems. 
The ability to safely transition between complex constraints by adding or
removing energy will allow for more expressive motion, and may become a standard
technique for controlling biologically inspired robots.

%---------- Bibliography ----------%
\printbibliography
\end{document}

% vim: set tw=80 ts=4 sw=4 sts=0 et ffs=unix :
