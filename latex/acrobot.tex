%! TEX root = main.tex

%/========== Acrobot ==========/%

\chapter{Application of VNHCs: The Acrobot}\label{ch:acrobot}
\section{Motivation}
The acrobot is a two-link pendulum, actuated at the center joint (as in Figure
\ref{fig:acrobot-model}). 
Since its first description in 1990
\cite{nonlinear_controllers_nonintegrable_acrobot}, the acrobot has become a
benchmark problem in control theory; 
it is an underactuated mechanical system which produces complex nonlinear motion
from an easy-to-describe model.
The acrobot models a gymnast on a bar, since it represents a torso (top link)
and legs (bottom link) with motion generated by the swinging of the legs at the
hips. 
It is also one of the simplest models for a biped walking robot
\cite{toward_framework_biped_locomotion}.

\begin{figure}
    \centering
    \includestandalone[width=0.7\textwidth]{images/acrobot_model}
    \caption{The general acrobot model, represented by two weighted rods
    differing in both length and mass.}%
    \label{fig:acrobot-model}
\end{figure}

Controlling the acrobot is a nontrivial task because it is not feedback
linearizable \cite{nonlinear_controllers_nonintegrable_acrobot}. 
Several researchers have studied the swing-up problem of driving the acrobot to
its equilibrium point above the bar using partial feedback linearization
\cite{swingup_problem_acrobot}, energy-based control
\cite{swingup_acrobot_pendulum, swingup_acrobot_energy}, and through studying
human motion \cite{swingup_giant_acrobot, motion_control_gymnastic_skill}.

In gymnastics terminology, a ``giant" is the motion a gymnast performs to
achieve full rotations around the bar \cite{usagym_giant}. 
We are interested in using VNHCs to generate giant motion, with the aim of
stabilizing desired energy levels.
The control of giant motion for the acrobot has been studied in
\cite{energy_pumping_robotic_swinging, swingup_giant_acrobot}, 
and some authors have used virtual holonomic constraints to achieve this
behaviour
\cite{dynamical_servo_acrobot_vc, control_giant_two_link_gymnastic_robot,
xingbo_thesis}. 
However, these controllers are neither intuitive nor easy to
design:
\cite{control_giant_two_link_gymnastic_robot} defines a constraint by inverting
a trajectory in time onto the state space; 
\cite{dynamical_servo_acrobot_vc} requires a cascade controller to stabilize
both a constraint and a desired limit cycle in the state space; 
and \cite{xingbo_thesis} enforces the giant by adding an extra state to estimate
velocity, which increases the dimensionality of the problem in a crude
approach to using VNHCs.

In this chapter we will design a physically-intuitive VNHC which generates giant
motion and prove the acrobot gains energy. 
In the process of completing this proof, we will arrive at a promising method
which might one day be useful for generating energy-injecting VNHCs on arbitrary
mechanical systems.

\section{Dynamics of the Acrobot}
Suppose we are given an acrobot as in Figure \ref{fig:acrobot-model} modelling a
gymnast hanging on a horizontal bar, where the ``torso" has moment of
inertia \(J_u\) and the ``leg" has moment of inertia \(J_a\) (each with respect
to their own center of mass).
Let \(q_u \in \Sone\) be the shoulder angle and \(q_a \in \Sone\) 
be the hip angle, where only \(q_a\) is actuated. 
Collecting them together provides the configuration
\(q = (q_u,q_a) \in \Sone \times \Sone\). 
The acrobot has inertia matrix \(D\), potential function \(P\) (with respect to
the horizontal bar), and input matrix \(B\)  given as
follows \cite{xingbo_thesis}:
\begin{align}\label{eqn:general-acrobot-inertia}
    D(q) &= \begin{bmatrix}
      m_al_u^2 + 2m_a\cos(q_a)l_u l_{c_a} + m_al_{c_a}^2 + m_ul_{c_u}^2 + J_u + J_a &
      m_al_{c_a}^2 + m_al_ul_{c_a}\cos(q_a) + J_a \\
      m_al_{c_a}^2 + m_al_ul_{c_a}\cos(q_a) + J_a &
      m_al_{c_a}^2 + J_a
    \end{bmatrix} 
    , \\
    \label{eqn:general-acrobot-potential}
    P(q) &= g\left(m_al_{c_a}(1 - \cos(q_u+q_a)) + 
        (m_al_u + m_ul_{c_u})(1-\cos(q_u))\right) 
    , \\
    B(q) &= \begin{bmatrix} 0 \\ 1 \end{bmatrix}
    .
\end{align}

While this is the most general representation of an acrobot, the dynamics
are unwieldy.
To make rigorous analysis of these dynamics more tractable, we begin by assuming
the acrobot is comprised of two massless rods of equal length \(l\), with equal
point masses \(m\) at the tips.
We call this a \textit{simple} acrobot, which is displayed in Figure
\ref{fig:simple-acrobot-model}.
We will also ignore any frictional forces at both the hip and shoulder joints. 
Finally, it is important to note that a real gymnast cannot swing their legs in
full circles, though they are usually flexible enough to raise them parallel to
the floor; 
for this reason, we assume that \(q_a \in [-Q_a, Q_a]\) where 
\(Q_a \in [\frac{\pi}{2}, \pi[\). 

\begin{figure}
    \centering
    \includestandalone[width=0.5\textwidth]{images/simple_acrobot_model}
    \caption{A simple acrobot has massless rods of equal length \(l\) and 
    equal masses \(m\) at the tips.}
    \label{fig:simple-acrobot-model}
\end{figure}

Since we are now working with a simple acrobot, 
we have \(l_{c_u} = l_{c_a} = l_u = l_a = l\) and
\(m_u = m_a = m\). 
On top of this, the moments of inertia \(J_u\) and \(J_a\) of the rods vanish.
Reducing
~\eqref{eqn:general-acrobot-inertia}-\ref{eqn:general-acrobot-potential}
yields the simplified inertia matrix \(D_s\) and potential function \(P_s\),
where
\begin{align}
    D_s(q) &= \begin{bmatrix}
        ml^2\left(3+2\cos(q_a)\right) & 
        ml^2\left(1+\cos(q_a)\right) \\
        ml^2\left(1+\cos(q_a)\right) &
        ml^2
    \end{bmatrix} 
    , \\
    P_s(q) &= -mgl\left(2\cos(q_u)+\cos(q_u+q_a)\right)
    .
\end{align}
\begin{notation}
    For shorthand, we write \(c_u := \cos(q_u)\), \(c_a := \cos(q_a)\), and 
    \(c_{ua} := \cos(q_u + q_a)\). Likewise, \(s_u := \sin(q_u)\), 
    \(s_a := \sin(q_a)\), and \(s_{ua} := \sin(q_u + q_a)\).
\end{notation}

Defining \(M(q) := D_s(q)\) and \(V(q) := P_s(q)\), we find the conjugate of momenta 
is \(p = (p_u, p_a) = M(q)\dot{q}\).
The dynamics in \((q,p)\) coordinates are given by
\begin{align}\label{eqn:acrobot-hamiltonian}
    \mathcal{H}(q,p) &= \frac{1}{2}p\tpose \Minv(q) p -
    mgl\left(2 c_u + c_{ua}\right)
    , \\
     &\begin{cases}
        \dot{q} = \Minv(q) p \\
        \dot{p}_u = -mgl\left(2s_u + s_{ua}\right) \\
        \dot{p}_a =-\frac{1}{2}p\tpose \nabla_{q_a}\Minv(q) p
        - mgl s_{ua} + \tau,
    \end{cases} \nonumber
\end{align}
where the inverse inertia matrix is
\begin{equation}\label{eqn:Minv}
    \Minv(q) = \frac{1}{ml^2\left(2-c_a^2\right)}
    \begin{bmatrix}
        1 &
        -\left(1+c_a\right) \\
        -\left(1+c_a\right) &
        3+2c_a
    \end{bmatrix}
    .
\end{equation}
The control input is a force \(\tau \in \R\) affecting only the dynamics of
\(p_a\), representing a torque acting on the hip joint.
This means \((q,p)\) are simply actuated coordinates inside the phase space
\(\mathcal{Q} \times \mathcal{P}\) where
\(\mathcal{Q} = \mathcal{Q}_u \times \mathcal{Q}_a 
:= \Sone \times \Sone\), and
\(\mathcal{P} = \mathcal{P}_u \times \mathcal{P}_a
:= \R \times \R\).
This allows us to apply the theory of VNHCs from Chapter
\ref{ch:vnhcs}.

Let us define the VNHC \(h(q,p) = q_a - f(q_u,p_u)\) of order 1, where
\(f \in C^2\left(\mathcal{Q}_u \times \mathcal{P}_u; \mathcal{Q}_a\right)\).
Since \(\nabla_{q_u}\Minv(q) = \Zmat{2\times 2}\),
Theorem \ref{thm:vnhc-regularity} tells us that this VNHC will be regular
when the regularity matrix
\[
    dh_q \Minv(q) \begin{bmatrix}0 \\ 1 \end{bmatrix}
    ,
\]
is of full rank \(1\) on the constraint manifold \(\Gamma\).
Given that
\(dh_q = \begin{bmatrix} -\partial_{q_u} f & 1 \end{bmatrix}\),
the regularity matrix evaluates to the scalar equation
\begin{equation}\label{eqn:regularity-matrix-acrobot}
    \frac{(1+c_a)\partial_{q_u}f(q_u,p_u) + (3+2c_a)}{ml^2(2-c_a^2)}
    .
\end{equation}
This is full rank if and only if the numerator does not
change sign.
The following proposition provides a sufficient condition for regularity.

\begin{prop}\label{prop:acrobot-fpu-regular}
    A relation \(h(q,p) = q_a - f(p_u) = 0\) 
    with \(f \in C^2\left(\mathcal{P}_u; \mathcal{Q}_a\right)\) is a regular
    VNHC of order 1 for the simple acrobot.
\end{prop}
\begin{proof}
    Since \(\partial_{q_u} f = 0\), the regularity equation
    ~\eqref{eqn:regularity-matrix-acrobot} is strictly positive for all values
    of \(q_a\), and hence is full rank everywhere on the constraint manifold.
    By Theorem \ref{thm:vnhc-regularity}, \(h\) is a regular VNHC of order 1.
\end{proof}
Proposition \ref{prop:acrobot-fpu-regular} will be useful later, as we will not
need to check regularity if we design a function of the unactuated momentum. 

The acrobot is noticeably more complex than the VLP, as
the dynamics of \((q_u,p_u)\) and \((q_a,p_a)\) are coupled through \(\Minv(q)\).
Because of this, the constrained dynamics of an arbitrary VNHC may not be easy
to write out.
In the rest of this chapter, our goal is to design the function \(f(q_u,p_u)\)
based on the natural human motion of a gymnast, with one caveat: 
we must be able to prove the constrained dynamics will inject energy into the
acrobot.

\section{Previous Constraint Approaches}
Let us examine some of the existing approaches to generating giant motion for
the acrobot, since these may be viable candidates on which to base a VNHC.

One initial approach to controlling the acrobot is to model it as a
variable-length pendulum by collapsing the two rods and masses into one
equivalent center of mass (ECM), as in Figure \ref{fig:acrobot-ecm}.
This seems a reasonable model reduction, since the length from the pivot to the
ECM changes depending on the angle \(q_a\) of the leg.
Indeed, \citet{swingup_giant_acrobot} use this approach to design a trajectory
for the ECM, then determine which leg angles \(q_a(t)\) are required to generate
that trajectory.
Following in their footsteps, we might consider using the results
from Chapter \ref{ch:vlp} to find the leg angles that allow the ECM to gain
energy. 
Then we could apply Theorem \ref{thm:vlp-energy-stabilization} to prove
the acrobot is gaining energy.
\begin{figure}
    \centering
    \includestandalone[width=0.5\textwidth]{images/acrobot_ecm}
    \caption{A simple acrobot modelled as a VLP with equivalent center of mass \(2m\). 
        The length of the VLP changes according to \(q_a\).}
    \label{fig:acrobot-ecm}
\end{figure}

Unfortunately, the VLP is not a true representation of the acrobot.
The effective length of the ECM is 
\[
    l_e(q_a) := l\sqrt{\frac{5}{4} + c_a}
    ,
\]
and its effective angle is
\[
    q_e := \arctan_2\left(s_u + \frac{1}{2}s_{ua}, -c_u - \frac{1}{2}c_{ua}\right)
    .
\]
There are two important notes to consider based on these equations. 
First, Figure \ref{fig:acrobot-vlp-symmetry} shows that for each pose of the
VLP representation, there are two configurations
of the acrobot which give the same effective length and angle.
This means the acrobot and the VLP are not equivalent representations;
designing a VNHC that injects energy using the ECM may not produce human-like
leg motion on the acrobot.

Second, if we were to compute the conjugate of momenta
\(p_{l_e}\) to \(l_e\) and \(p_e\) to \(q_e\), we would see the torque input
\(\tau\) appearing in both of their dynamic equations.
In the VLP model from Chapter \ref{ch:vlp}, the control input only
affects the dynamics of the length variable.
If we want to design a VNHC for this system, we cannot use any of the results
from Chapter \ref{ch:vlp} because the VLP models do not match.

\begin{figure}
    \centering
    \includestandalone[width=0.3\textwidth]{images/acrobot_vlp_symmetry}
    \caption{The equivalent center of mass of the acrobot generally has two configurations
        which correspond to the same effective length and angle. These
        configurations are symmetric about the line connecting the pivot to the
        ECM.}
    \label{fig:acrobot-vlp-symmetry}
\end{figure}

Since we cannot apply the results of Chapter \ref{ch:vlp} to simplify the proof
of energy injection, and the resulting ECM motion may not even produce
realistic leg motion, this model reduction is ineffective for our purposes. 

Let us turn next to the thesis of \citet{xingbo_thesis}, who designs a VHC to enforce a
so-called ``tap" motion with the purpose of injecting energy into the acrobot. 
First, he defines a compensator variable \(s\) which tracks \(\dot{q}_u\), so
that he can use the theory of VHCs with the extended configuration 
\((q_u,q_a,s)\).
He then finds \(h_1, h_2 \in \R_{>0}\) to define the
normalized radius \(\rho\) and normalized angle \(\xi\) in the
\((q_u, s)\)-plane.
These normalized variables are given by
\begin{align*}
    \rho &:= \sqrt{h_1 q_u^2 + h_2 s^2}
    , \\
    \xi &:= \arctan_2(h_2 s, h_1 q_u)
    . 
\end{align*}
He then sets the VHC to be \(h(q) = q_a - f_\text{rad}(\rho)f_\text{ang}(\xi)\)
with the control parameters \(\bar{q}_u\) and \(\rho_0\), where
\begin{align}
    \label{eqn:xingbo-frad}
    f_\text{rad}(\rho) &:= \tanh^2(\rho/\rho_0)
    , \\
    \label{eqn:xingbo-fang}
    f_\text{ang}(\xi) &:= 
    \begin{cases}
        0 & -\pi < \xi \leq 0 \\
        \bar{q}_u \exp\left(1 - \frac{1}{1-(\frac{4\xi}{\pi} - 1)^2}\right) 
          & 0 < \xi \leq \frac{\pi}{2} \\
        0 & \frac{\pi}{2} < \xi \leq \pi
        .
    \end{cases}
\end{align}

While this constraint shows promising experimental results and it accurately
emulates true human motion, \citeauthor{xingbo_thesis}
does not provide analytical proof that the acrobot will gain energy.
His lack of analysis is tied to the fact that the constrained
dynamics are incredibly complicated.
In fact, just showing the constraint is regular is a challenging task.
While we could very easily convert his VHC into a VNHC by replacing \(s\) with
\(p_u\), we would run into the same problem. 
Since we want our constraint to \textit{provably} inject energy, we must forgo
this type of constraint in favour of something less complex.

\section{The Acrobot Constraint}
One may be tempted to design a constraint of the form \(q_a = \bar{q}_a\sin(\theta)\) 
with \(\theta := \arctan_2(p_u,q_u)\), since a similar approach 
was so effective for the VLP in Chapter \ref{ch:vlp}.
Unfortunately, this constraint is not regular, and it is difficult to find any
VNHC of the form \(q_a = f(\theta)\) where regularity can be proven easily.
Instead, we will develop a constraint \(h(q,p) = q_a - f(p_u)\) because these
constraints are always regular (as per Proposition \ref{prop:acrobot-fpu-regular}). 
 
To design this constraint, let us begin (perhaps unexpectedly) by examining a person on
a seated swing.
The person extends their legs when the swing moves forwards, and retracts their
legs when the swing moves backwards.
As the swing gains speed, the person leans their body back while
extending their legs.
This allows them to bring their legs higher, shortening the distance
from their center of mass to the pivot and adding more energy to the swing.
When the swing moves backward, they sit up and fully retract their legs
underneath them \cite{how_to_pump_a_swing}.

Now imagine the person's torso is affixed to the swing's rope so they are
always upright. 
Imagine further that the swing has no seat at all, allowing the person to extend
their legs beneath them. 
This position is identical to that of a gymnast on a bar, which is why we can
use leg motion from the seated swing to design a controller for the acrobot.

The acrobot's legs are rigid rods which cannot retract, so we emulate the person
on a swing by pivoting the legs toward the direction of motion. 
To account for how a person leans back at higher speeds, the legs should pivot to an
angle proportional to the swing's speed.
Since the direction of motion is entirely determined by \(p_u\), 
one such VNHC which emulates this process is \(q_a = \bar{q}_a\arctan( I p_u)\),
displayed in Figure \ref{fig:qa-arctan}.
Here, \(\bar{q}_a \in ]0,\frac{2 Q_a}{\pi}]\) and \(I \in \R\) is a fixed
control parameter.

This constraint does not perfectly recreate giant motion, during which
the gymnast's legs are almost completely extended \cite{usagym_giant}
-- it instead pivots the legs partially during rotations.
However, the behaviour looks similar enough that the constraint should provide a
reasonable foundation for injecting energy into the acrobot.
It is for this reason that we choose our acrobot's constraint to be
\begin{equation}\label{eqn:acrobot-constraint}
    h(q,p) = q_a - \bar{q}_a \arctan(I p_u)
    .
\end{equation}

\begin{figure}
    \centering
    \includestandalone[width=0.5\textwidth]{images/qa_arctan}
    \caption{The acrobot constraint \(q_a = \bar{q}_a \arctan(I p_u)\).}
    \label{fig:qa-arctan}
\end{figure}

Let us now compute the constrained dynamics under
~\eqref{eqn:acrobot-constraint}.
Note that \(dh_q = \begin{bmatrix}0 & 1\end{bmatrix}\), while
\[
    dh_{p_u} = \frac{-\bar{q}_a I}{1 + I^2 p_u^2}
    .
\]
Inserting these into ~\eqref{eqn:g-qupu}, we get the solution for \(p_a\) on the
constraint manifold:
\[
    p_a(q_u,p_u) = \frac{
        (1+c_a)(1+I^2 p_u^2)p_u - m^2gl^3\bar{q}_a I (2-c_a^2)(2s_u + s_{ua})
    }{ml^2(3+2c_a)(1+I^2 p_u^2)}
    .
\]
The dynamics for \(p_u\) do not contain \(p_a\), so they remain unchanged.
The constrained dynamics for \(q_u\) are given by 
\begin{equation*}
    \dot{q}_u = e_1\tpose \Minv(q) \begin{bmatrix}
                    p_u \\ p_a(q_u,p_u)
                \end{bmatrix} %\\
    ,
\end{equation*}
which can be simplified into 
\begin{equation*}
    \dot{q_u} = \frac{(1+I^2 p_u^2)p_u + m^2gl^3\bar{q}_a I(2s_u + s_{ua})(1+c_a) }{ml^2(1+I^2 p_u^2)(3+2c_a)}
    .
\end{equation*}
Hence, the constrained dynamics for the acrobot under
~\eqref{eqn:acrobot-constraint} are
\begin{equation}\label{eqn:acrobot-constrained-dynamics}
\left.\begin{cases}
    \dot{q}_u &= \frac{(1+I^2 p_u^2)p_u + m^2gl^3\bar{q}_a I(2s_u + s_{ua})(1+c_a) }
            {ml^2(1+I^2 p_u^2)(3+2c_a)}
        \\
    \dot{p}_u &= - m g l (2s_u + s_{ua})
    \end{cases} \right|_{q_a = \bar{q}_a\arctan(Ip_u)}
    .
\end{equation}

These dynamics do not always gain energy -- we need certain conditions on our
control parameter \(I\) to guarantee this is true.

\begin{thm}\label{thm:acrobot-energy-stabilization}
    Consider the simple acrobot ~\eqref{eqn:acrobot-hamiltonian} constrained by
    the VNHC ~\eqref{eqn:acrobot-constraint}, whose
    constraint manifold is \(\Gamma \simeq \SxR\).
    Let 
    \[
        E(q_u,p_u) := \frac{p_u^2}{10ml^2} + 3mgl(1 - \cos(q_u))
        ,
    \]
    be the energy of the simple pendulum obtained by setting \(I = 0\).
\begin{enumerate}
    \item There exists \(I > 0\) small enough that 
    ~\eqref{eqn:acrobot-constraint} injects energy into the acrobot on
    \[
        \mathcal{O} := \left\{(q_u,p_u) \in \Gamma 
        \mid E(q_u,p_u) < E(\pi,0) \right\}
        .
    \]
    If instead \(I < 0\), the VNHC dissipates energy on \(\mathcal{O}\).
\item Define \(b : \SxR_{> 0} \rightarrow \R\) by
    \[
        b(\beta,\rho_0) := 
        \frac{5m^2 g l^3 \bar{q}_a \left(
            m^2gl^3\left(18s_\beta^2 + 30c_\beta(1 - c_\beta)\right)
            - c_\beta\rho_0^2
        \right)}{
        |\rho_0|\sqrt{\rho_0^2 - 30m^2gl^3(1 - c_\beta)}
        }
        ,
    \]
    and set \(S(\rho_0) := \int \limits_{0}^{2\pi} b(\sigma,\rho_0)d\sigma\).
    Fix \(\bar{\rho} > \sqrt{60m^2gl^3}\).
    Suppose there exists \(\epsilon > 0\) so that \(S(\rho_0) > \epsilon\) for
    all \(\rho_0 \in \, \left]\sqrt{60m^2gl^3}, \bar{\rho}\right]\).
    Then there exists \(I > 0\) small enough that
    ~\eqref{eqn:acrobot-constraint} injects energy into the acrobot on
    \[
        \Omega := \left\{(q_u,p_u) \in \Gamma 
        \mid E(q_u,p_u) < E(0,\bar{\rho})\right\}
        .
    \]
    If instead \(I < 0\), the VNHC dissipates energy on \(\Omega\).
\end{enumerate}
\end{thm}

A small enough fixed value \(I > 0\) will enable the acrobot to 
escape any compact subset of \(\mathcal{O}\) in finite time. 
What's more, the acrobot's phase will flow towards the homoclinic orbit
corresponding to \(E(\pi,0)\), as in Figure
\ref{fig:acrobot-oscillation-domain}.
Hence, the first result of Theorem \ref{thm:acrobot-energy-stabilization} states
that all acrobots constrained by ~\eqref{eqn:acrobot-constraint} 
will gain enough energy to begin rotating around the bar.
In the worst case, the acrobot will at least perform a swing-up routine to reach
the \((q_u,p_u) = (\pi,0)\) equilibrium.

If we want the acrobot to achieve giants with energy
\(E(0,\bar{\rho})\), the acrobot needs to satisfy the assumption on the
integral of \(b(\beta,\rho_0)\).
In this case, orbits will escape compact subsets of \(\Omega\) in finite time
until the acrobot eventually reaches a momentum of at least \(\bar{\rho}\).
One example of this behaviour is displayed in Figure \ref{fig:acrobot-omega}.
Note that the value of \(I\) will depend on \(\bar{\rho}\).

\begin{figure}
    \centering
    \begin{subfigure}[t]{0.45\textwidth}
        \includestandalone[width=\linewidth]{images/acrobot_oscillation_domain}
        \caption{The set \(\mathcal{O}\). An orbit starting in this set (blue)
            will reach the level set \(E(\pi,0)\) of the nominal pendulum in
            finite time.}
        \label{fig:acrobot-oscillation-domain}
    \end{subfigure}
    \hfill
    \begin{subfigure}[t]{0.45\textwidth}
        \includestandalone[width=\linewidth]{images/acrobot_omega}
        \caption{The set \(\Omega\). Orbits starting in this set will
            reach the level set \(E(0,\bar{\rho})\) of the nominal pendulum in
            finite time.}
            \label{fig:acrobot-omega}
    \end{subfigure}
    \caption{The sets defined by Theorem \ref{thm:acrobot-energy-stabilization}
        on which the acrobot gains energy.}
\end{figure}

Unfortunately, the proof used in Chapter \ref{ch:vlp} does not
readily transfer to the acrobot.
Proving Theorem \ref{thm:acrobot-energy-stabilization} requires an intelligent
change of coordinates and the use perturbation
theory \cite{khalil_nonlinear}.
We provide the full proof of this theorem in Chapter \ref{sec:acrobot-proof}.
 
\section{Proof of Theorem \ref{thm:acrobot-energy-stabilization}}\label{sec:acrobot-proof}
In an effort to make the proof of Theorem \ref{thm:acrobot-energy-stabilization}
as clear as possible, we will break it down into the following segments:
\begin{enumerate}
    \item Background on perturbation theory.
    \item Perturbation analysis for oscillations.
    \item Perturbation analysis for rotations.
\end{enumerate}

When \(I = 0\), the constrained acrobot behaves like a single
pendulum with masses at a distance \(l\) and \(2l\) from the pivot 
(Figure \ref{fig:acrobot-I0}) whose energy
\begin{equation}\label{eqn:acrobot-nominal-E}
    E(q_u,p_u) = \frac{p_u^2}{10ml^2} + 3mgl(1 - \cos(q_u))
    ,
\end{equation}
is conserved.
Level sets of \(E\) are ellipses on the \((q_u,p_u)\)-plane
when \(E(q_u,p_u) < E(\pi,0)\), which we call ``oscillations";
and they are open curves when \(E(q_u,p_u) > E(\pi,0)\), which we call
``rotations". 
Examples of these can be seen in Figure \ref{fig:pendulum-level-sets}.

\begin{figure}
    \centering
    \begin{subfigure}[t]{0.45\textwidth}
        \includestandalone[]{images/acrobot_I_zero}
        \caption{A simple pendulum with two masses.}
        \label{fig:acrobot-I0}
    \end{subfigure}
    \hfill
    \begin{subfigure}[t]{0.5\textwidth}
        \includestandalone[]{images/pendulum_level_sets}
        \caption{Level sets of \(E\).
            The blue ellipse is an oscillation; the red curves
            are rotations; and the black ellipse is the homoclinic
            orbit with energy \(E(\pi,0)\).}
        \label{fig:pendulum-level-sets}
    \end{subfigure}
    \caption{Our constrained acrobot is a simple pendulum when \(I = 0\).}
\end{figure}

Using a method developed by \citet{dynamic_vhcs_stabilize_closed_orbits},
we can find a change of coordinates \((q_u,p_u) \to (\alpha, \mu)\) with the
following properties:
\begin{itemize}
    \item The energy of oscillation in \((\alpha,\mu)\) coordinates is 
        uniquely defined by \(\mu\), which remains constant along oscillations
        of the simple pendulum.
    \item \(\alpha\) is a pseudo-angle living in \(\Sone\) which is
        always increasing along oscillations.
\end{itemize}
The coordinates \((\alpha,\mu)\) can be thought of as deformed polar coordinates
adapted to level sets of \(E\) below the value \(E(\pi,0)\), 
where \(\mu\) specifies the level set and \(\alpha\) identifies a point on that
level set.
Once we have these coordinates in hand, we can use perturbation theory to prove
that \(\mu\) increases on \(\mathcal{O}\) when \(I > 0\) is small enough.

Likewise, we will find a second set of coordinates \((\beta, \rho)\) for
rotations, where \(\beta \in \Sone\) is always increasing along solutions,
\(\rho\) is constant along rotations of the simple pendulum, and \(\rho\)
increases for the acrobot when \(I > 0\) is small enough.
Using both \((\alpha,\mu)\) and \((\beta,\rho)\), we will show the acrobot is
gaining energy on \(\Omega\).

\subsection{Background on Perturbation Theory}
Nonlinear systems like the acrobot are difficult (or even impossible) to solve
analytically.
Perturbation theory allows one to understand the behaviour of nonlinear systems
by studying a simpler nominal system. 
Solutions of the nonlinear system can often be approximated by taking a Taylor
expansion around a solution of the nominal system.

\citet{khalil_nonlinear} considers a system of the form
\begin{equation}\label{eqn:khalil-setup}
    \begin{cases}
        \dot{x} = f(t,x,I), \\
        x(t_0) = \eta(I),
    \end{cases}
\end{equation}
where 
\(f : [t_0,t_1] \times D \times [-I_0,I_0] \rightarrow \R^n\) is  
smooth\footnote{Khalil actually considers \(f(t,x,\epsilon)\) which is ``sufficiently
    smooth". To more easily connect his theory to the acrobot constraint, we
    assume smoothness of \(f\) and replace \(\epsilon\) with \(I\).}
on a domain \(D \subset \R^n\).
In the context of the acrobot, \(x\) is any pair of coordinates which make
our analysis convenient;
\(I\) is the control parameter of the system;
the function \(f\) is the constrained dynamics in \(x\)-coordinates;
and \(\eta(I) \equiv \eta_0\) is a constant initial condition.

Setting \(I = 0\) we get the nominal system
\begin{equation}\label{eqn:khalil-perturbation-nominal}
    \begin{cases}
        \dot{x}_0 = f(t,x,0) ,\\
        x_0(t_0) = \eta_0 ,
    \end{cases}
\end{equation}
which we need to solve for the explicit solution \(x_0(t,\eta_0)\) on
\([t_0,t_1]\).
We assume this solution is contained in \(D\).

Now consider the solution \(x(t,\eta_0,I)\) to ~\eqref{eqn:khalil-setup}.
Performing a first-order Taylor series expansion at \(I = 0\), we get
\(x(t,\eta_0,I) = x_0(t,\eta_0) + I x_1(t,\eta_0) + R(t,\eta_0,I)\),
where the remainder term \(R(t,\eta_0,I)\) is smooth and \(O(I^2)\), \ie, 
\[
    \lim \limits_{I \to 0} \frac{R(t,\eta_0,I)}{I} = 0
    .
\]
One can recover \(x_1(t,\eta_0)\) as the solution to the scalar
time-varying ODE
\begin{equation}\label{eqn:khalil-perturbation-firstorder}
    \begin{cases}
        \dot{x}_1 = \pdiff{f}{x}(t,x_0(t,\eta_0),0)x_1 + \pdiff{f}{I}(t,x_0(t,\eta_0),0)
        , \\
        x_1(t_0,\eta_0) = 0
        .
    \end{cases}
\end{equation}

We now paraphrase Khalil's Theorem 10.1 \cite{khalil_nonlinear} on the
accuracy of perturbation analysis.
\begin{thm}\label{thm:khalil-perturbation}
    Fix \(t_0, t_1 \in \mathbb{R}\) with \(t_0 < t_1\).
    Suppose \(f : [t_0,t_1] \times D \times [-I_0,I_0] \rightarrow \R^2\) is
    \(C^1\), and that the nominal system
    ~\eqref{eqn:khalil-perturbation-nominal} has a unique solution
    \(x_0(t,\eta_0) \in D\) on \([t_0,t_1]\).
    Then there exists \(I^\star > 0\) such that, for all \(I\) with 
    \(|I| < I^\star\), the solution \(x(t,\eta_0,I)\) to
    ~\eqref{eqn:khalil-setup} satisfies
    \[
        \norm{x(t,\eta_0,I) - \left(x_0(t,\eta_0) + I x_1(t,\eta_0)\right)} \leq k |I^2|
    \]
    for some \(k > 0\).
\end{thm}

Theorem \ref{thm:khalil-perturbation} tells us that we can approximate the
solution of the nonlinear system by the Taylor approximation 
\(x(t,\eta_0,I) \approx x_0(t,\eta_0) + I x_1(t,\eta_0)\).
When \(I\) is small enough, solutions of the nonlinear system and
the Taylor approximation are the same up to order \(I^2\) along compact time
intervals.
This is equivalent to saying that the remainder term \(R(t,\eta_0,I)\) is
bounded, because 
\[
    \norm{x(t,\eta_0,I) - \left(x_0(t,\eta_0) + I x_1(t,\eta_0)\right)} 
    = \norm{R(t,\eta_0,I)} \leq k |I^2|
    .
\]
Since we know that the acrobot behaves like a pendulum at \(I = 0\), we can
use this theory to prove energy injection properties on the acrobot by studying
the simple pendulum.
 
\subsection{Perturbation Analysis for Oscillations}
As we have seen, setting \(I = 0\) turns the acrobot into a nominal pendulum. 
This pendulum oscillates whenever the nominal energy
~\eqref{eqn:acrobot-nominal-E} is less than \(E(\pi,0)\). 
Hence, the domain of oscillations for the nominal pendulum is given by
\[
    \mathcal{O} := \left\{ (q_u,p_u) \in \Gamma \mid E(q_u,p_u) < E(\pi,0)\right\}
    ,
\]
where \(\Gamma \simeq \SxR\) is the constraint manifold of the
acrobot.
Recall that orbits of a pendulum are level sets of \(E\). 
From Figure \ref{fig:pendulum-level-sets}, we know that oscillation level sets
look like ellipses contained in \(\mathcal{O}\).
Since perturbation theory requires us to analyze the nominal pendulum, we will
make our analysis convenient by changing coordinates into a
pseudo-radius \(\mu > 0\) which remains constant on level sets of \(E\), 
along with a pseudo-angle \(\alpha \in \Sone\) satisfying \(\dot{\alpha} > 0\) on
\(\mathcal{O}\).

\subsubsection*{Pseudo-Polar Coordinates}

The energy of oscillation can be determined by the orbit's intersection
\(\mu\) with the \(q_u\)-axis (as in Figure \ref{fig:mu-intersection}).
Since \(q_u \in ]-\pi,\pi[\) on \(\mathcal{O}\), we set \(\mu \in ]0,\pi[\).
The transformation we want is therefore a diffeomorphism of the form
\begin{align*}
    T : \mathcal{O} \backslash \{(0,0)\} &\rightarrow \Sone \times \, ]0,\pi[, \\
    (q_u, p_u) &\mapsto (\alpha,\mu)
    .
\end{align*}

\begin{figure}
    \centering
    \includestandalone[width=0.5\textwidth]{images/mu_intersection}
    \caption{The domain \(\mathcal{O}\) (blue) where a pendulum oscillates.
    The pseudo-radius \(\mu\) corresponds to the
    intersection of an oscillation (red) with the \(q_u\)-axis.
    The psuedo-angle \(\alpha\) is taken to be clockwise positive because 
    oscillations of a pendulum move clockwise on \(\mathcal{O}\).}
    \label{fig:mu-intersection}
\end{figure}

The energy level set corresponding to the intersection \((q_u,p_u) = (\mu,0)\)
is 
\[
    \left\{(q_u,p_u) \in \Gamma \mid E(q_u,p_u) = 3mgl(1- \cos(\mu))\right\}
    ,
\]
which gives the relationship
\begin{equation}\label{eqn:oscillation-pu2}
    p_u^2 = 30m^2gl^3\left(\cos(q_u) - \cos(\mu)\right)
    .
\end{equation}
On this level set, \(q_u\) ranges between \([-\mu,\mu]\) and can be uniquely
parameterized by \(q_u = \mu \cos(\alpha)\), where \(\alpha\) is our
desired pseudo-angle.
Substituting this into ~\eqref{eqn:oscillation-pu2}, we get
\[
    p_u^2 = 30m^2gl^3\left(\cos(\mu \cos(\alpha)) - \cos(\mu)\right)
    .
\]
We want to find \(p_u\) as a function of \((\alpha,\mu)\); noting that we can
determine the sign of \(p_u\) from the sign of \(\sin(\alpha)\), we get the
(clockwise) parameterization
\begin{equation}\label{eqn:oscillation-pu}
    p_u = -\sign{\sin(\alpha)} \sqrt{30m^2gl^3 \left(\cos(\mu c_\alpha) - c_\mu\right)}
    ,
\end{equation}
which is smooth for all \(\mu \in ]0,\pi[\).

We have thus found a transformation \(T\inv(\alpha,\mu) = (q_u,p_u)\).
We need the inverse of this map to get our diffeomorphism \(T(q_u,p_u)\).
Notice from ~\eqref{eqn:oscillation-pu2} that
\[
    \cos(\mu) = -\frac{p_u^2}{30m^2gl^3} + \cos(q_u) =: C_\mu(q_u,p_u)
    .
\]
Since \(\mu \in ]0,\pi[\), we can uniquely express \(\mu\) by
\begin{equation}\label{eqn:mu-qupu}
    \mu = \arccos\left(C_\mu(q_u,p_u)\right)
    .
\end{equation}
Next we need to find \(\alpha\). 
Recall that 
\begin{equation}\label{eqn:acrobot-cosalpha-qu}
    \cos(\alpha) = \frac{q_u}{\mu}
    ,
\end{equation}
which means 
\begin{equation}\label{eqn:acrobot-sinalpha-qu}
    \sin(\alpha) = \pm \sqrt{1 - \frac{q_u^2}{\mu^2}}
    .
\end{equation}
Using ~\eqref{eqn:oscillation-pu}, we determine that
\(\sign{\sin(\alpha)} = -\sign{p_u}\).
Putting together ~\eqref{eqn:acrobot-cosalpha-qu} and
~\eqref{eqn:acrobot-sinalpha-qu}, we deduce that
\begin{equation}\label{eqn:alpha-qupu}
    \alpha = \left.
        \arctan_2\left( -\sign{p_u}\sqrt{1 - \frac{q_u^2}{\mu^2}}, \frac{q_u}{\mu}\right)
        \right|_{\mu = \arccos(C_\mu(q_u,p_u))}
    .
\end{equation}
Thus, ~\eqref{eqn:mu-qupu} and ~\eqref{eqn:alpha-qupu} define 
our transformation into \((\alpha,\mu)\)-coordinates on \(\mathcal{O}\).

The acrobot's constrained dynamics in
\((\alpha,\mu)\)-coordinates can be computed by evaluating
\begin{equation*}
\left.\begin{cases}
    \dot{\mu} &= \pdiff{\mu(q_u,p_u)}{q_u}\dot{q_u} +
         \pdiff{\mu(q_u,p_u)}{p_u}\dot{p}_u
    \\
    \dot{\alpha} &= \pdiff{\alpha(q_u,p_u)}{q_u}\dot{q}_u + 
        \pdiff{\alpha(q_u,p_u)}{p_u}\dot{p}_u
    \end{cases}\right|_{(q_u,p_u) = T\inv(\mu,\alpha)}
    .
\end{equation*}
with \((\dot{q}_u,\dot{p}_u)\) given by 
 ~\eqref{eqn:acrobot-constrained-dynamics}.

These dynamics are much too large to write out, so we simply denote them by
\begin{align}\label{eqn:acrobot-mu-dot}
    \dot{\mu} &= f_\mu(\alpha,\mu,I)
    ,\\
    \label{eqn:acrobot-alpha-dot}
    \dot{\alpha} &= f_\alpha(\alpha,\mu,I)
    .
\end{align}
The dynamics of the nominal pendulum arise from evaluating
\(f_\mu\) and \(f_\alpha\) at \(I = 0\).
MATLAB's symbolic toolbox evaluates the nominal dynamics as
\begin{align}\label{eqn:acrobot-mu-dot-nom}
    \dot{\mu} &= 0
    , \\
    \label{eqn:acrobot-alpha-dot-nom}
    \dot{\alpha} &= \sqrt{\frac{6g}{5l}} 
        \sqrt{\frac{\cos(\mu\cos(\alpha)) - \cos(\mu)}
            {\mu^2 \sin(\alpha)^2}}
    .
\end{align}
One can confirm that \(\dot{\alpha} > 0\) for every \(\mu \in ]0,\pi[\).
By continuity of ~\eqref{eqn:acrobot-alpha-dot}, there exists \(I_1 > 0\) small
enough that \(\dot{\alpha}\) remains positive on \(\mathcal{O}\) for 
\(I \in [-I_1,I_1]\).

\subsubsection*{Time Scaling}

We have observed that \(\dot{\alpha} > 0\) on \(\mathcal{O}\) for small enough \(I\). 
Hence, we can use \(\alpha\) as our time variable by reparameterizing \(t\) as a
function \(t(\alpha)\).
This allows us to study the evolution of \(\mu\) as a function of \(\alpha\)
rather than a function of time.
Setting \(\hat{\mu}(\alpha) := \mu(t(\alpha))\) yields the dynamics
\[
    \diff{\hat{\mu}}{\alpha} = 
    \diff{\mu}{t} \diff{t}{\alpha}
    .
\] 
This reduces the system \((\dot{\alpha},\dot{\mu})\) into the scalar
time-varying ODE 
\begin{equation}\label{eqn:muhat-dot}
    \begin{cases}
        \diff{\hat{\mu}}{\alpha} 
        = \frac{f_\mu(\tau(\alpha),\mu,I)}{f_\alpha(\tau(\alpha),\mu,I)}
        =: g(\alpha,\mu,I)
        , \\
        \hat{\mu}(0) = \mu_0
        .
    \end{cases}
\end{equation}

\subsubsection*{Perturbation Analysis of the Time Scaled System}

In the spirit of perturbation analysis, we will expand our time-scaled system
\(\hat{\mu}(\alpha,\mu_0,I)\).
We begin by setting \(I = 0\) to find the nominal system
\[
\begin{cases}
    \dot{\hat{\mu}}_0 = g(\alpha,\hat{\mu}_0,0)
    , \\
    \hat{\mu}_0(0) = \mu_0
   .
\end{cases}
\]
Equations ~\eqref{eqn:acrobot-mu-dot-nom}-\eqref{eqn:acrobot-alpha-dot-nom})
reveal that \(g(\alpha,\mu,0) = 0\),
so the solution to this nominal system is 
\(\hat{\mu}_0(\alpha,\mu_0) \equiv \mu_0\) for all \(\alpha\).

We now take a first-order Taylor approximation of \(\hat{\mu}(\alpha,\mu_0,I)\)
around the nominal system \(\hat{\mu}_0(\alpha,\mu_0)\), which 
expands into 
\begin{equation}\label{eqn:acrobot-muhat-approx}
    \hat{\mu}(\alpha,\mu_0,I) = \hat{\mu}_0(\alpha,\mu_0) + I
    \hat{\mu}_1(\alpha,\mu_0)
    + R(\alpha,\mu_0,I)
    .
\end{equation}

By ~\eqref{eqn:khalil-perturbation-firstorder}, we know the function
\(\hat{\mu}_1(\alpha,\mu_0)\) is the solution to the linear time-varying scalar
system
\begin{equation}\label{eqn:acrobot-mu1-dot}
    \begin{cases}
        \dot{\hat{\mu}}_1 = 
        \pdiff{g}{\hat{\mu}}(\alpha,\mu_0,0)\hat{\mu}_1 + \pdiff{g}{I}(\alpha, \mu_0, 0)
        , \\
        \hat{\mu}_1(0) = 0
        .
    \end{cases}
\end{equation}
These dynamics are difficult to compute by hand, so we resort to MATLAB symbolic
computation to reveal that
\begin{align*}
    \pdiff{g}{\hat{\mu}}(\alpha,\mu_0,0) &= 0
    , \\
    \pdiff{g}{I}(\alpha, \mu_0, 0) &= K a(\alpha,\mu_0)
    ,
\end{align*}
where we define
\begin{align*}
    K &:= \frac{\bar{q}_a \sqrt{30m^2g l^3}}{15}
    , \\
    a(\alpha,\mu_0) &:= \frac{
        \mu_0 |\sin(\alpha)| \left(
        5 c_{\mu_0} \cos(\mu_0 c_\alpha) - 8 \cos(\mu_0c_\alpha)^2 + 3
    \right)
    }{
    \sin(\mu_0)\sqrt{\cos(\mu_0c_\alpha) - c_{\mu_0}}
    }
    .
\end{align*}
Hence, the solution to ~\eqref{eqn:acrobot-mu1-dot} is given by
\[
    \hat{\mu}_1(\alpha,\mu_0) =
    K \int \limits_0^\alpha a(\sigma,\mu_0)d\sigma
    .
\]
Therefore, the first-order Taylor approximation of \(\hat{\mu}(\alpha,\mu_0,I)\) is
\begin{equation}\label{eqn:acrobot-muhat-expanded}
    \hat{\mu}(\alpha,\mu_0,I)
    = \mu_0 + I K \int \limits_0^\alpha a(\sigma,\mu_0)d\sigma +
    R(\alpha,\mu_0,I)
    .
\end{equation}

\subsubsection{Poincar\'{e} Analysis}

Since \(\dot{\alpha} > 0\), there is a well-defined Poincar\'{e} map 
describing how the pseudo-radius \(\mu\) changes each time the orbit of \((q_u(t),p_u(t))\)
intersects the \(q_u\)-axis, \ie, when \(\alpha = 0\) or \(\alpha = \pi\).
Using our time-scaled system, we can define this Poincar\'{e} map as the change
in \(\hat{\mu}\) when \(\alpha\) increases by \(\pi\):
\begin{align*}
    P_\mathcal{O} : \,]0,\pi[ &\rightarrow \R_+,
    \\
    \mu_0 &\mapsto \hat{\mu}(\pi,\mu_0,I)
    .
\end{align*}
From our perturbation analysis, the Poincar\'{e} map -- also known as a
Poincar\'{e} section --  expands into
\[
    P_\mathcal{O}(\mu_0) = \mu_0 + I K \int\limits_0^\pi a(\sigma,\mu_0)d\sigma
    + R(\pi,\mu_0,I)
    .
\]
Let us define
\[
    Q(\mu_0) := \int \limits_0^\pi a(\sigma,\mu_0)d\sigma
    ,
\]
so that \(\hat{\mu}_1(\pi,\mu_0) = K Q(\mu_0)\).
Note that \(K\) is a positive constant which contains the acrobot's physical
parameters, while \(a(\alpha,\mu)\) is adimensional. 
This means \(Q(\mu_0)\) is identical for every acrobot.
If \(Q(\mu_0)\) is strictly positive, then \(\hat{\mu}_1(\pi,\mu_0)\) will be
positive no matter the values of \(m\), \(g\), \(l\), or \(\bar{q}_a\).

We numerically compute \(Q(\mu_0)\) for \(\mu_0 \in [10^{-10}, \pi - 10^{-3}]\)
in Figure \ref{fig:acrobot-Q}. 
We see that it is strictly positive and monotonically increasing, with
an asymptote at \(\mu_0 = \pi\). 
Simulations with smaller \(\mu_0\) result in an infinite integral error due
to a division by zero, so we believe that \(Q(\mu_0)\) is in fact positive for
all \(\mu_0\).

\begin{figure}
    \centering
    \includegraphics[width=0.8\textwidth]{images/Qmu.png}
    \caption{The plot of \(Q(\mu_0)\).}
    \label{fig:acrobot-Q}
\end{figure}

This fact, coupled with knowledge that \(R(\pi,\mu_0,I)\) is \(O(I^2)\),
implies there exists \(I > 0\) small enough that
\[
    P_\mathcal{O}(\mu_0) = \mu_0 + IKQ(\mu_0) + R(\pi,\mu_0,I) > \mu_0
    .
\]
Unfortunately, perturbation analysis does not guarantee 
that \(I\) is the same for all \(\mu_0 \in \, ]0,\pi[\), since the value of
\(R(\pi,\mu_0,I)\) could become more negative for other initial conditions.
We will deal with this next.

\subsubsection*{Energy Gain on \(\mathcal{O}\)}

The Poincar\'{e} section \(P_\mathcal{O}\) allows us understand the evolution of
\(\hat{\mu}(\alpha,\mu_0,I)\) by studying the evolution of the discrete time
system
\begin{equation}\label{eqn:muhat-discrete}
    \begin{cases}
        \mu(n+1) := P_\mathcal{O}(\mu(n)) 
        = \mu(n) + I K Q(\mu(n)) + R(\pi,\mu(n),I)
        , \\
        \mu(0) = \mu_0
        .
    \end{cases}
\end{equation}
Tolerating the abuse of notation, \(\mu(n)\) represents the distance along the
\(q_u\)-axis when an orbit of the constrained dynamics intersects the
\(q_u\)-axis for the \(n\)th time, assuming the orbit was initialized at
\((q_u,p_u) = (\mu_0,0)\).

Proving the constrained dynamics gain energy on \(\mathcal{O}\) is equivalent to
showing \(\mu(n)\) eventually reaches \(\pi\).
That is, we want to find \(I^\star \in\, ]0, I_1]\) where, for all 
\(\delta > 0\) and \(\mu_0 \in ]0,\pi-\delta]\), there exists \(N > 0\) so that
for all \(n \geq N\), \(\mu(n) \notin [0,\pi-\delta]\).
A sufficient condition for this characterization is to show 
\(P_\mathcal{O}(\mu_0) > \mu_0\) for all \(\mu_0 \in\, ]0,\pi[\),

Linearizing ~\eqref{eqn:muhat-discrete} at \(\mu = 0\) yields
\[
    P^\prime_\mathcal{O}(0) = 1 + I K Q^\prime(0) +
    \pdiff{R}{\mu}(\pi,0,I)
    .
\]
If there is some value of \(I \in ]0,I_1]\) for which this term is greater than
\(1\), then \(0\) is a repeller of the discrete time system.
To show such a value of \(I\) exists, we first compute 
\[
    Q^\prime(0) = \int \limits_0^\pi \pdiff{a}{\mu}(\sigma,0) d\sigma
    .
\]
MATLAB reveals that 
\[
    \lim \limits_{\mu_0 \to 0^+}
    \pdiff{a}{\mu}(\alpha,\mu_0) 
    = -\frac{\sqrt{2}}{2} \left(11\sin(\alpha)^2 - 6\right)
    ,
\]
which means 
\[
    Q^\prime(0)  = \frac{\pi}{2\sqrt{2}} > 0
    .
\]
Since the remainder term \(R(\alpha,\mu_0,I)\) is \(O(I^2)\), 
its partial derivative \(\pdiff{R}{\mu}(\pi,\mu_0,I)\) is also \(O(I^2)\).
Hence, it can be written in the form
\(I^2 \tilde{R}(\mu_0,I)\) where \(\tilde{R}(\mu_0,I)\) is smooth and
zero at \(I = 0\).
Since \(I\) is assumed to be small, \(I^2\) shrinks rapidly;
thus, there exists \(I_2 \in ]0,I_1]\) where 
\(I_2 K Q^\prime(0) + (I_2)^2 \tilde{R}(0,I) > 0\).
Hence,
\[
    P^\prime_\mathcal{O}(0) \geq 1 + IKQ^\prime(0) + I^2 \tilde{R}(0,I) 
    > 1
    ,
\] 
for all \(I \in ]0, I_2]\), .

We have shown \(0\) is a repeller of the discrete time system, which means
there exists some (unknown) \(\epsilon > 0\) where the interval
\(]0,\epsilon[\) is negatively invariant for ~\eqref{eqn:muhat-discrete}.
What's more, all solutions starting in this interval will flow towards the value
\(\mu = \epsilon\), so we must have \(P_\mathcal{O}(\mu_0) > \mu_0\) for all 
\(\mu_0 \in \, ]0,\epsilon[\).

Next, recall that \(R(\pi,\mu_0,I)\) is smooth in all its parameters as well as
being \(O(I^2)\).
Therefore, it is bounded below on the compact set \(\mu_0 \in [\epsilon,\pi]\)
by some value \(R(\pi,\underbar{\mu},I)\). 
Theorem \ref{thm:khalil-perturbation} asserts that there exists some
\(I_3 > 0\) and \(r > 0\) so that, for all \(I \in [-I_3, I_3]\),
\[
    R(\pi,\mu_0,I) \geq R(\pi, \underbar{\mu}, I) > -I^2 r
    .
\]
Furthermore, \(Q(\mu_0)\) is a strictly increasing function, so
for\(\mu_0 \in [\epsilon,\pi[\) we have
\[
    P_\mathcal{O}(\mu_0) > \mu_0 + I K Q(\epsilon) - I^2 r
    .
\]
Choosing \(I^\star \in\, ]0,\min\left\{I_2,I_3\right\}]\) so that
\(I^\star K Q(\epsilon) - (I^\star)^2 r \geq 0\)
means that
\( P_\mathcal{O}(\mu_0) > \mu_0\)
for all \(\mu_0 \in [\epsilon,\pi[\).

We have thus shown that \(P_\mathcal{O}(\mu_0) > \mu_0\) for all
\(\mu_0 \in ]0,\pi[\).
Looking at ~\eqref{eqn:muhat-discrete}, this means
\(\mu(n+1) > \mu(n)\) and all solutions of the discrete time system will flow
towards \(\mu = \pi\).
Hence, orbits of the constrained dynamics will eventually escape
\(\mathcal{O}\).
We conclude that, for all \(m\), \(g\), \(l\), \(\bar{q}_a\), 
there exists \(I > 0\) small enough that the constraint 
~\eqref{eqn:acrobot-constraint} is injecting energy on \(\mathcal{O}\).

By the same arguments presented in this section, the Poincar\'{e} section
satisfies \( P_\mathcal{O}(\mu_0) < \mu_0 \) when \(I < 0\). 
In this case, the constraint dissipates energy on \(\mathcal{O}\).

% Empty proof environment to put QED square on the right of the page
\begin{proof}[\unskip\nopunct]
\end{proof}

\subsection{Perturbation Analysis for Rotations}
Figure \ref{fig:pendulum-level-sets} reminds us that rotations of the 
nominal pendulum obtained by setting \(I = 0\) look like open curves on some
rotation region \(\mathcal{R} \subset \Gamma\).
One can compute this rotation domain to be the set
\[
    \mathcal{R} := \left\{ (q_u,p_u) \in \Gamma \mid E(q_u,p_u) > E(\pi,0)\right\}
    .
\]
Performing a similar process to what we did for oscillations,
we wish to find a new set of coordinates \((\beta,\rho)\)
where the pseudo-radius \(\rho \in \mathbb{R}\) remains constant on level sets
of \(E\) and the pseudo-angle \(\beta \in \Sone\) is always increasing on
\(\mathcal{R}\).

\subsubsection*{Pseudo-Polar Coordinates}

In the oscillation region \(\mathcal{O}\) we showed that the nominal pendulum's
energy (and hence, its orbit) is uniquely determined by the intersection point
\(\mu\) with the \(q_u\)-axis;
likewise, on \(\mathcal{R}\) the energy is determined by the intersection
\(\rho\) on the \(p_u\)-axis, as in Figure \ref{fig:rho-intersection}.

\begin{figure}
    \centering
    \includestandalone[]{images/rho_intersection}
    \caption{The domain \(\mathcal{R}\) (blue) where a pendulum rotates. The
        pseudo-radius \(\rho\) corresponds to the intersection of an orbit of
        rotation with the \(p_u\)-axis. The pseudo-angle \(\beta\) selects a
        point on the rotation.}
    \label{fig:rho-intersection}
\end{figure}

Note that the boundary of \(\mathcal{R}\) intersects the \(p_u\) axis when 
\(p_u^2 = 60m^2 g l^3\). 
The level set of this boundary is \(E(\pi,0)\), the homoclinic orbit for the
upright equilibrium of the pendulum.
Hence, we must have \(\rho > \sqrt{60 m^2 g l^3}\) if a
rotation has momentum \(p_u > 0\) 
and \(\rho < -\sqrt{60 m^2 g l^3}\) if it has momentum \(p_u < 0\).
The energy level set associated with \(\rho\) is 
\[
    \left\{(q_u,p_u) \in \Gamma \mid E(q_u,p_u) = \frac{\rho^2}{10ml^2}\right\}
    ,
\]
which gives the relationship
\begin{equation}\label{eqn:rotation-pu2}
    \frac{p_u^2}{10m l^2} + 30mgl(1 - c_u) = \frac{\rho^2}{10 ml^2}
    .
\end{equation}
On this level set, \(q_u\) takes all values on \(\Sone\), so our angle of
rotation is uniquely parameterized by \(\beta = q_u\).
Since \(\rho\) does not change sign along the rotation, we have the smooth
relationship
\begin{align}\label{eqn:rotation-Tinv}
    q_u &= \beta
    , \\
    p_u &= \sign{\rho}\sqrt{\rho^2 - 30 m^2 g l^3\left(1 - c_\beta \right)}
    .
\end{align}

Inverting this relationship gives the smooth transformation
\begin{align}\label{eqn:rotation-T}
    \beta &= q_u
    , \\
    \rho &= \sign{p_u}\sqrt{p_u^2 + 30 m^2 g l^3 \left(1 - c_u \right)}
    .
\end{align}

Computing the acrobot's constrained dynamics in \((\beta,\rho)\)-coordinates and
setting \(I = 0\) yields the dynamics of the nominal pendulum.
MATLAB evaluates those dynamics as
\begin{align}\label{eqn:acrobot-rot-mu-dot-nom}
    \dot{\rho} &= 0
    , \\
    \label{eqn:acrobot-rot-alpha-dot-nom}
    \dot{\beta} &=  \sign{\rho} 
    \frac{\sqrt{\rho^2 - 30m^2gl^3(1 - c_\beta)}}{5ml^2}
    .
\end{align}
As expected, \(\dot{\beta}\) does not change sign on \(\mathcal{R}\) because the
orbits always flow clockwise.
If \(\rho > 0\), the rotation curve goes from \(\beta = -\pi\) to 
\(\beta = \pi\); 
if \(\rho < 0\), it goes from \(\beta = \pi\) to \(\beta = -\pi\).
By continuity of ~\eqref{eqn:rotation-T}, there exists \(I_1 > 0\) small enough
that \(\dot{\beta}\) is non-zero and does not change sign on \(\mathcal{R}\) for
all \(I \in [-I_1,I_1]\).

\subsubsection*{Time Scaling}

Using \(\beta\) as our new time variable (via a time reparameterization
\(t = t(\beta)\)) produces the time-scaled system
\(\hat{\rho}(\beta,\rho_0,I) = \rho(t(\beta),\rho_0,I)\) with dynamics
\[
    \begin{cases}
        \diff{\hat{\rho}}{\beta} = \frac{\dot{\rho}}{\dot{\beta}}
        , \\
        \hat{\rho}(0) = \rho_0
        .
    \end{cases}
\]

\subsubsection*{Perturbation Analysis of the Time Scaled System}

In the spirit of perturbation analysis, we expand the time-scaled system 
\(\hat{\rho}(\beta,\rho_0,I)\). 
From ~\eqref{eqn:khalil-perturbation-nominal} we know the nominal system at
\(I = 0\) is
\[
    \begin{cases} 
        \dot{\hat{\rho}}_0 = 0
        , \\
        \hat{\mu}_0(0) = \rho_0
        ,
    \end{cases}
\]
with solution \(\hat{\rho}_0(\beta,\rho_0) \equiv \rho_0\) for all \(\beta\).
Likewise, ~\eqref{eqn:khalil-perturbation-firstorder} 
tells us that \(\hat{\rho}_1(\beta,\rho_0)\) is the solution to the linear
time-varying scalar system
\begin{equation}\label{eqn:acrobot-rho1-dot}
  \begin{cases}
    \dot{\hat{\rho}}_1 =
    \frac{5m^2 g l^3 \bar{q}_a \left(
        m^2gl^3\left(18s_\beta^2 + 30c_\beta(1 - c_\beta)\right)
        - c_\beta\rho_0^2
    \right)}{
    |\rho_0|\sqrt{\rho_0^2 - 30m^2gl^3(1 - c_\beta)}
    }
    =: b(\beta,\rho_0)
     , \\
     \hat{\rho}_1(0) = 0
     ,
 \end{cases}
\end{equation}
whose solution is
\[
    \hat{\rho}_1(\beta,\rho_0) = \int \limits_0^\beta b(\sigma,\rho_0)d\sigma
    .
\]
Therefore, the first-order Taylor approximation of
\(\hat{\rho}(\beta,\rho_0,I)\) is
\begin{equation}\label{eqn:acrobot-rhohat-approx}
    \hat{\rho}(\beta,\rho_0,I) = \hat{\rho}_0(\beta,\rho_0) +
    I\hat{\rho}_1(\beta,\rho_0) + R(\beta,\rho_0,I)
    ,
\end{equation}
where \(R(\beta,\rho_0,I)\) is smooth and \(O(I^2)\).

\subsubsection*{Poincar\'{e} Analysis}

Suppose we initialize the acrobot at \((\beta,\rho) = (0,\rho_0)\).
One full rotation amounts to \(\beta\) traversing \(2\pi\) rad in a clockwise
direction; that is, when \(\beta\) goes from \(0\) to \(\sign{\rho_0}2\pi\)).
Since \(\dot{\beta}\) does not change sign on \(\mathcal{R}\),
there is a well-defined Poincar\'{e} map describing how the pseudo-radius
\(\rho\) changes each time the orbit \((q(t),p(t))\) hits the \(p_u\) axis, \ie,
every time \(\beta\) changes by \(2\pi\).
We define this Poincar\'{e} map as
\(P_\mathcal{R}(\rho_0) := \hat{\rho}(\sign{\rho_0}2\pi,\rho_0,I)\),
which expands into
\[
    P_\mathcal{R}(\rho_0) = \rho_0 + I \hat{\rho}_1(\sign{\rho_0}2\pi, \rho_0)
    + R(\sign{\rho_0}2\pi,\rho_0,I)
    .
\]

The function \(b(\beta,\rho_0)\) is even and
\(2\pi\)-periodic in \(\beta\), which implies that
\begin{align*}
    \hat{\rho}_1(\sign{\rho_0}2\pi, \rho_0) &=
    \int \limits_0^{\sign{\rho_0}2\pi} b(\sigma,\rho_0)d\sigma
    , \\
     &=\sign{\rho_0}\int\limits_0^{2\pi}b(\sigma,\rho_0)d\sigma
     , \\
     &= \sign{\rho_0} \hat{\rho}_1(2\pi,\rho_0)
     .
\end{align*}
Defining \(S(\rho_0) := \hat{\rho}_1(2\pi,\rho_0)\),
we get
\[
    P_\mathcal{R}(\rho_0) = \rho_0 + I\sign{\rho_0}S(\rho_0) 
    + R(\sign{\rho_0}2\pi,\rho_0,I)
    .
\]

Recall the assumption that there exist \(\bar{\rho} > 0\) and
\(\epsilon > 0\) where \(S(\rho_0) \geq \epsilon\) for all 
\(\rho_0 \in \,]\sqrt{60m^2gl^3},\bar{\rho}]\).
Let us define
\[
    D := \left[-\bar{\rho},\sqrt{60m^2gl^3}\right[ \, 
    \cup \, 
    \left]\sqrt{60m^2gl^3},\bar{\rho}\right]
    .
\]
Since \(b(\beta,\rho_0)\) is even in \(\rho_0\),
\(S(\rho_0) = S(\abs{\rho_0})\);
hence, \(S(\rho_0) \geq \epsilon\) for every \(\rho_0 \in D\).

Finally, recall that \(R(\sign{\rho_0}2\pi,\rho_0,I)\) is \(O(I^2)\), so 
for each fixed \(\rho_0 \in D\)
there exists \(I > 0\) small enough that 
\( IS(\rho_0) + R(\sign{\rho_0}2\pi,\rho_0,I) > 0\).
For this value of \(I\), we have that
\[
    \begin{cases}
        P_\mathcal{R}(\rho_0) > \rho_0 & \rho_0 > 0
        ,\\
        P_\mathcal{R}(\rho_0) < \rho_0 & \rho_0 < 0
        .\\
    \end{cases}
\]
Unfortunately, this \(I\) may not work for all \(\rho_0 \in D\). We deal
with this issue next.

\subsubsection*{Energy Gain on a Subset of \(\mathcal{R}\)}

The Poincar\'{e} section \(P_\mathcal{R}\) allows us understand the evolution of
\(\hat{\rho}(\beta,\rho_0,I)\) by studying the evolution of the discrete time
system
\begin{equation}\label{eqn:rhohat-discrete}
    \begin{cases}
        \rho(n+1) := P_\mathcal{R}(\rho(n)) 
        = \rho(n) + I\sign{\rho_0}S(\rho(n)) + 
            R(\sign{\rho_0}2\pi,\rho(n),I)
        , \\
        \rho(0) = \rho_0
        .
    \end{cases}
\end{equation}
Tolerating the abuse of notation, \(\rho(n)\) represents the distance along the
\(p_u\)-axis when an orbit of the constrained dynamics intersects the
\(p_u\)-axis for the \(n\)th time, assuming the orbit was initialized at
\((q_u,p_u) = (0,\rho_0)\).

We wish to show \(\abs{\rho(n)}\) eventually reaches \(\bar{\rho}\).
That is, we wish to find some \(I^\star \in \,]0,I_1]\) making
\(\abs{P_\mathcal{R}(\rho_0)} > \abs{\rho_0}\) for all 
\(\rho_0 \in D\)

We begin with examining the case where \(\rho_0 > 0\).
Recall that \(R(\beta,\rho_0,I)\) is \(O(I^2)\) and smooth in all
its parameters.
Therefore, it is bounded below on the compact set
\(\rho_0 \in [\sqrt{60m^2gl^3},\bar{\rho}]\) by some value
\(R(2\pi,\underbar{\rho},I)\).
Theorem \ref{thm:khalil-perturbation} asserts that there exists some 
\(I_2 > 0\) and \(r > 0\) so that, for all \(I \in [-I_2,I_2]\),
\[
    R(2\pi,\rho_0,I) \geq R(2\pi,\underbar{\rho},I) > -I^2r
    .
\]
Since \(S(\rho_0) \geq \epsilon\), we find that
\[
    P_\mathcal{R}(\rho_0) > \rho_0 + I\epsilon -I^2r
    ,
\]
for all \(\rho_0 \in ]\sqrt{60m^2gl^3},\bar{\rho}]\).
Choosing \(I_3 \in ]0,I_2]\) so that 
\(I_3\epsilon - (I_3)^2 r \geq 0\) means that
\(P_\mathcal{R}(\rho_0) > \rho_0\)
for all \(\rho_0 \in ]\sqrt{60m^2gl^3},\bar{\rho}]\)

We now examine the case when \(\rho_0 < 0\), where we wish to show that 
\(P_\mathcal{R}(\rho_0) < \rho_0\).
The remainder \(R(\beta,\rho_0,I)\) is bounded above on the compact set 
\(\rho_0 \in [-\bar{\rho},-\sqrt{60m^2gl^3}]\), and 
Theorem \ref{thm:khalil-perturbation} asserts that there exists some 
\(I_4 > 0\) and \(r^\prime > 0\) so that, for all \(I \in [-I_4,I_4]\), 
\[
    R(-2\pi,\rho_0,I) < I^2 r^\prime
    .
\]
Hence,
\begin{align*}
    P_\mathcal{R}(\rho_0)
    &= \rho_0 - IS(\rho_0) + R(-2\pi,\rho_0,I)
    , \\
    &< \rho_0 - IS(\epsilon) + I^2r^\prime
    .
\end{align*}
Choosing \(I_5 \in ]0,I_4]\) so that 
\((I_5)^2r^\prime - I_5 S(\epsilon) \leq 0\) makes
\(P_\mathcal{R}(\rho_0) < \rho_0\) for all 
\(\rho_0 \in [-\bar{\rho},-\sqrt{60m^2gl^3}[\).

Finally, choosing \(I^\star \in\, ]0,\min\left\{I_3,I_5\right\}]\) yields that
\(\abs{P_\mathcal{R}(\rho_0)} > \abs{\rho_0}\) for all \(\rho_0 \in D\), as
desired.
Looking at the discrete-time system ~\eqref{eqn:rhohat-discrete}, 
a solution of the system will flow away from the origin until it reaches a
magnitude \(\abs{\rho(n)} = \bar{\rho}\).
Hence, the constrained dynamics will eventually escape the set
\(\omega \subset \mathcal{R}\) defined by
\[
    \omega := \left\{(q_u,p_u) \in \Gamma
    \mid E(\pi,0) < E(q_u,p_u) < E(0,\bar{\rho}) \right\}
    ,
\]
which is displayed in Figure \ref{fig:acrobot-little-omega}.

We conclude that there exists \(I > 0\) small enough that the constraint
~\eqref{eqn:acrobot-constraint} is injecting energy on \(\omega\).

\begin{figure}
    \centering
    \includestandalone[width=0.5\textwidth]{images/acrobot_little_omega}
    \caption{The set \(\omega \subset \mathcal{R}\) on which the acrobot will gain energy.
    }
    \label{fig:acrobot-little-omega}
\end{figure}

By the same arguments presented in this section, the Poincar\'{e} section
satisfies \(\abs{P_\mathcal{R}(\rho_0)} < \abs{\rho_0}\) when \(I < 0\).
In this case, the constraint dissipates energy on \(\omega\).

\subsection{Energy Gain on \(\Omega\)}
Define \(E_\pi\) to be the energy level set associated with \(E(\pi,0)\).
We will prove the acrobot gains energy on the set 
\begin{align*}
    \Omega &= \left\{(q_u,p_u) \in \Gamma
    \mid E(q_u,p_u) < E(0,\bar{\rho}) \right\}
    ,\\
    &= \mathcal{O} \cup E_\pi \cup \omega
    .
\end{align*}
There exists some control input \(I_1 > 0\) for which the acrobot is gaining
energy on \(\mathcal{O}\). 
Hence, there are no closed orbits in \(\mathcal{O}\) and almost all orbits 
hit \(E_\pi\) in finite time.
Likewise, there exists some \(I_2 > 0\) for which the acrobot is gaining energy
on \(\omega\), so there are no closed orbits in \(\omega\) and almost every
orbit is flowing away from \(E_\pi\).
Choosing \(I^\star \in ]0,\min\left\{I_1,I_2\right\}]\) means the acrobot will gain energy on
both \(\mathcal{O}\) as well as \(\omega\).

We can prove energy gain on \(\Omega\) by showing two properties of the system.
First, that the set \(\Pi^+\) of initial conditions converging to  \((\pi,0)\)
is of measure zero in \(\Gamma\);
second, that \(E_\pi\) is not a positively-invariant set.
If both these conditions are true, then orbits flowing from \(\mathcal{O}\) into
\(E_\pi\) will generally swing past \(E_\pi\) into \(\omega\).

To show the first condition, we will simply prove that \((\pi,0)\) is an
unstable equilibrium of the constrained dynamics.
Taking the Jacobian of the constrained dynamics
~\eqref{eqn:acrobot-constrained-dynamics} at \((\pi,0)\) yields
\[
    J = \begin{bmatrix}
        -\frac{6mgl\bar{q}_aI}{5} & \frac{1 - 2m^2gl^3\bar{q}_a^2 I^2}{5ml^2} \\
        3mgl & mgl\bar{q}_aI
    \end{bmatrix}
    ,
\]
which has the characteristic polynomial
\[
    \det\left(\lambda \Id{2} - J\right)
    = \lambda^2 + \frac{mgl\bar{q}_a I}{5} \lambda - 3g
    .
\]
By Descartes' rule of signs, this polynomial has one root with positive real
part. 
The equilibrium \((\pi,0)\) is therefore unstable, so the set \(\Pi^+\) of
initial conditions converging to \((\pi,0)\) is one-dimensional, and hence is
of measure zero in \(\Gamma\).

For the second condition, let us seek a contradiction by 
supposing that \(E_\pi\) is a positively-invariant set. 
Let \(x(t)\) be an orbit of the constrained dynamics initialized at 
\(x(0) \in \mathal{O}\).
Since the constrained dynamics gain energy on \(\mathcal{O}\), there exists a
time \(T > 0\) such that \(x(T) \in E_\pi\).
By the assumed positive invariance of \(E_\pi\), this means 
\(x(t) \in E_\pi\) for all \(t \in [T,\infty[\).
Since \(E_\pi\) is a closed curve containing the equilibrium \((\pi,0)\) and no
other equilibria, \(x(t)\) must converge to \((\pi,0)\).
The initial condition \(x(0) \in \mathcal{O}\) was arbitrary, so this implies
that \(\mathcal{O} \subset \Pi^+\).
But \(\Pi^+\) is a set of measure zero, while \(\mathcal{O}\) has positive
measure. 
This is a contradiction: \(E_\pi\) cannot be positively-invariant.

Non-invariance of \(E_\pi\) and instability of \((\pi,0)\) implies that 
almost all orbits initialized in \(\mathcal{O}\) will reach \(E_\pi\) and
eventually swing into \(\omega\), rather than stay in \(E_\pi\) and converge to
\((\pi,0)\). 
We know that almost all orbits starting in \(\omega\) will reach 
\(E(0,\bar{\rho})\) in finite time;
we conclude that almost all orbits starting in \(\Omega\) will depart the energy
escaping \(\Omega\) all together.
Hence, the acrobot is gaining energy on \(\Omega\).

If \(I < 0\), orbits in \(\mathcal{O}\) will converge to the origin and orbits in
\(\omega\) will flow towards \(E_\pi\).
However, the upright equilibrium \((\pi,0)\) will still be unstable, and
\(E_\pi\) will remain a non-invariant set. 
This means that almost all orbits in \(\omega\) will flow past \(E_\pi\) into
\(\mathcal{O}\), so the constrained system will be gaining energy in
negative-time. Thus, the constraint is dissipating energy on \(\Omega\).

\section{Experimental Results}
% TODO: This might be a separate chapter

%/========== /Acrobot ==========/%
% vim: set tw=80 ts=4 sw=4 sts=0 et ffs=unix :
