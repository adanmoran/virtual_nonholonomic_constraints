%! TEX root = main.tex

%/========== Acrobot ==========/%

\chapter{Application of VNHCs: The Acrobot}\label{ch:acrobot}
\section{Motivation}
The acrobot is a two-link pendulum, actuated at the center joint (as in Figure
\ref{fig:acrobot-model}). 
Since its first description in 1990
\cite{nonlinear_controllers_nonintegrable_acrobot}, the acrobot has become a
benchmark problem in control theory; 
it is an underactuated mechanical system which produces complex nonlinear motion
from an easy-to-describe model.
The acrobot models a gymnast on a bar, since it represents a torso (top link)
and legs (bottom link) with motion generated by the swinging of the legs at the
hips. 
It is also one of the simplest models for a biped walking robot
\cite{toward_framework_biped_locomotion}.

\begin{figure}
    \centering
    \includestandalone[width=0.7\textwidth]{images/acrobot_model}
    \caption{The general acrobot model, represented by two weighted rods
    differing in both length and mass.}%
    \label{fig:acrobot-model}
\end{figure}

Controlling the acrobot is a nontrivial task, since it is not feedback
linearizable \cite{nonlinear_controllers_nonintegrable_acrobot}. 
Several researchers have studied the swing-up problem of driving the acrobot to
its equilibrium point above the bar using partial feedback linearization
\cite{swingup_problem_acrobot}, energy-based control
\cite{swingup_acrobot_pendulum, swingup_acrobot_energy}, and through studying
human motion \cite{swingup_giant_acrobot, motion_control_gymnastic_skill}.

In gymnastics terminology, a ``giant" is the motion a gymnast performs to
achieve full rotations around the bar \cite{usagym_giant}. 
We are interested in using VNHCs to generate giant motion, with the aim of
stabilizing desired energy levels.
The control of giant motion for the acrobot has been studied in
\cite{energy_pumping_robotic_swinging, swingup_giant_acrobot}, 
and some authors have used virtual holonomic constraints to achieve this
behaviour
\cite{dynamical_servo_acrobot_vc, control_giant_two_link_gymnastic_robot,
xingbo_thesis}. 
However, these controllers are neither intuitive nor easy to
design:
\cite{control_giant_two_link_gymnastic_robot} defines a constraint by inverting
a trajectory in time onto the state space; 
\cite{dynamical_servo_acrobot_vc} requires a cascade controller to stabilize
both a constraint and a desired limit cycle in the state space; 
and \cite{xingbo_thesis} enforces the giant by adding an extra state to estimate
velocity, which increases the dimensionality of the problem in a crude
approach to using VNHCs.

In this chapter we will design a physically-intuitive VNHC which generates giant
motion and prove the acrobot gains energy. 
In the process of completing this proof, we will arrive at a promising method
which might one day be useful for generating energy-injecting VNHCs on arbitrary
mechanical systems.

\section{Dynamics of the Acrobot}
Suppose we are given an acrobot as in Figure \ref{fig:acrobot-model} modelling a
gymnast hanging on a horizontal bar, where the ``torso" has moment of
inertia \(J_u\) and the ``leg" has moment of inertia \(J_a\) (each with respect
to their own center of mass).
Let \(q_u \in \mathbb{S}^1\) be the shoulder angle and \(q_a \in \mathbb{S}^1\) 
be the hip angle, where only \(q_a\) is actuated. 
Collecting them together provides the configuration
\(q = (q_u,q_a) \in \mathbb{S}^1 \times \mathbb{S}^1\). 
The acrobot has inertia matrix \(D\), potential function \(P\) (with respect to
the horizontal bar), and input matrix \(B\)  given as
follows \cite{xingbo_thesis}:
\begin{align}\label{eqn:general-acrobot-inertia}
    D(q) &= \begin{bmatrix}
      m_al_u^2 + 2m_a\cos(q_a)l_u l_{c_a} + m_al_{c_a}^2 + m_ul_{c_u}^2 + J_u + J_a &
      m_al_{c_a}^2 + m_al_ul_{c_a}\cos(q_a) + J_a \\
      m_al_{c_a}^2 + m_al_ul_{c_a}\cos(q_a) + J_a &
      m_al_{c_a}^2 + J_a
    \end{bmatrix} 
    , \\
    \label{eqn:general-acrobot-potential}
    P(q) &= g\left(m_al_{c_a}(1 - \cos(q_u+q_a)) + 
        (m_al_u + m_ul_{c_u})(1-\cos(q_u))\right) 
    , \\
    B(q) &= \begin{bmatrix} 0 \\ 1 \end{bmatrix}
    .
\end{align}

While this is the most general representation of an acrobot, the dynamics
become unwieldy.
To make rigorous analysis of these dynamics more tractable, we begin by assuming
the acrobot is comprised of two massless rods of equal length \(l\), with equal
point masses \(m\) at the tips.
We call this a \textit{simple} acrobot, which is displayed in Figure
\ref{fig:simple-acrobot-model}.
We will also ignore any frictional forces at both the hip and shoulder joints. 
Finally, it is important to note that a real gymnast cannot swing their legs in
full circles; 
for this reason, we assume that \(q_a \in [-Q_a, Q_a]\) where 
\(Q_a \in ]-\pi, \pi[\). 

\begin{figure}
    \centering
    \includestandalone[width=0.7\textwidth]{images/simple_acrobot_model}
    \caption{A simple acrobot has massless rods of equal length \(l\) and 
    equal masses \(m\) at the tips.}
    \label{fig:simple-acrobot-model}
\end{figure}

Since we are now working with a simple acrobot, 
we have \(l_{c_u} = l_{c_a} = l_u = l_a = l\) and
\(m_u = m_a = m\). 
On top of this, the moments of inertia \(J_u\) and \(J_a\) of the rods vanish.
Reducing
(\ref{eqn:general-acrobot-inertia}-\ref{eqn:general-acrobot-potential})
yields the simplified inertia matrix \(D_s\) and potential function \(P_s\),
where
\begin{align}
    D_s(q) &= \begin{bmatrix}
        ml^2\left(3+2\cos(q_a)\right) & 
        ml^2\left(1+\cos(q_a)\right) \\
        ml^2\left(1+\cos(q_a)\right) &
        ml^2
    \end{bmatrix} 
    , \\
    P_s(q) &= -mgl\left(2\cos(q_u)+\cos(q_u+q_a)\right)
    .
\end{align}
\begin{notation}
    For shorthand, we write \(c_u := \cos(q_u)\), \(c_a := \cos(q_a)\), and 
    \(c_{ua} := \cos(q_u + q_a)\). Likewise, \(s_u := \sin(q_u)\), 
    \(s_a := \sin(q_a)\), and \(s_{ua} := \sin(q_u + q_a)\).
\end{notation}

Defining \(M(q) := D_s(q)\) and \(V(q) := P_s(q)\), we find the conjugate of momenta 
is \(p = (p_u, p_a) = M(q)\dot{q}\).
The dynamics in \((q,p)\) coordinates are given by
\begin{align}\label{eqn:acrobot-hamiltonian}
    \mathcal{H}(q,p) &= \frac{1}{2}p\tpose \Minv(q) p -
    mgl\left(2 c_u + c_{ua}\right)
    , \\
     &\begin{cases}
        \dot{q} = \Minv(q) p \\
        \dot{p}_u = -mgl\left(2s_u + s_{ua}\right) \\
        \dot{p}_a =-\frac{1}{2}p\tpose \nabla_{q_a}\Minv(q) p
        - mgl s_{ua} + \tau,
    \end{cases} \nonumber
\end{align}
where the inverse inertia matrix is
\begin{equation}\label{eqn:Minv}
    \Minv(q) = \frac{1}{ml^2\left(2-c_a^2\right)}
    \begin{bmatrix}
        1 &
        -\left(1+c_a\right) \\
        -\left(1+c_a\right) &
        3+2c_a
    \end{bmatrix}
    .
\end{equation}
The control input is a force \(\tau \in \R\) affecting only the dynamics of
\(p_a\), representing a torque acting on the hip joint.
This means \((q,p)\) are simply actuated coordinates inside the phase space
\(\mathcal{Q} \times \mathcal{P}\) where
\[
    \mathcal{Q} = \mathcal{Q}_u \times \mathcal{Q}_a := 
    \mathbb{S}^1 \times \mathbb{S}^1
    ,
\]
and
\[
    \mathcal{P} = \mathcal{P}_u \times \mathcal{P}_a
    := \R \times \R
    .
\]
This allows us once again to apply the theory of VNHCs from Chapter
\ref{ch:vnhcs}.

Let us define the VNHC \(h(q,p) = q_a - f(q_u,p_u)\) of order 1, where
\(f \in C^2\left(\mathcal{Q}_u \times \mathcal{P}_u; \mathcal{Q}_a\right)\).
Since \(\nabla_{q_u}\Minv(q) = \Zmat{2\times 2}\),
Theorem \ref{thm:vnhc-regularity} tells us that this VNHC will be regular
if the regularity matrix
\[
    dh_q \Minv(q) \begin{bmatrix}0 \\ 1 \end{bmatrix}
    ,
\]
is of full rank on the constraint manifold \(\Gamma\).
Then, since 
\[
    dh_q = \begin{bmatrix}
        -\partial_{q_u} f & 1
    \end{bmatrix}
    ,
\]
the regularity matrix evaluates to the scalar equation
\begin{equation}\label{eqn:regularity-matrix-acrobot}
    \frac{(1+c_a)\partial_{q_u}f(q_u,p_u) + (3+2c_a)}{ml^2(2-c_a^2)}
    .
\end{equation}
This is full rank on \(\Gamma\) if and only if the numerator does not
change sign.

A sufficient condition for regularity is when \(f\) is a function solely of \(p_u\),
because then \(\partial_{q_u}f = 0\) and (\ref{eqn:regularity-matrix-acrobot})
is strictly positive for all values of \(q_a\). 
This will be useful later, as we will not need to check regularity if we design
a function of the unactuated momentum. 

The acrobot is noticeably more complex than the VLP, as
the dynamics of \((q_u,p_u)\) and \((q_a,p_a)\) are coupled through \(\Minv(q)\).
Because of this, the constrained dynamics of an arbitrary VNHC may not be easy
to write out.
In the rest of this chapter, our goal is to design the function \(f(q_u,p_u)\)
based on the natural human motion of a gymnast, with one caveat: 
we must be able to prove the constrained dynamics will inject energy into the
acrobot.

\section{Previous Constraint Approaches}
% TODO: Give Xingbo's intuition, abstract away to qa = sin(theta) constraint
% based on VLP but show it doesn't work well in our framework
% TODO: Describe Xingbo's results and observe that his results were similar to a VNHC
% except they used VHC tools. We will modify his approach to use VNHCs and prove
% rigorously that we can stabilize any energy level set on the acrobot.
Let us examine some of the existing approaches to generating giant motion for
the acrobot, since these may be viable candidates on which to base a VNHC.

One initial approach to controlling the acrobot is to model it as a
variable-length pendulum by collapsing the two rods and masses into one
equivalent center of mass (ECM), as in Figure \ref{fig:acrobot-ecm}.
This seems a reasonable model reduction, since the length from the pivot to the
ECM changes depending on the angle \(q_a\) of the leg.
Indeed, \citet{swingup_giant_acrobot} use this approach to design a trajectory
for the ECM, then determine which leg angles \(q_a(t)\) are required to generate
that trajectory.
Following in their footsteps, we might consider using the results
from Chapter \ref{ch:vlp} to find the leg angles that allow the ECM to gain
energy. 
Then we could apply Theorem \ref{thm:vlp-energy-stabilization} to prove
the acrobot is gaining energy too.
\begin{figure}
    \centering
    \includestandalone[width=0.5\textwidth]{images/acrobot_ecm}
    \caption{A simple acrobot modelled as a VLP with equivalent center of mass \(2m\). 
        The length of the VLP changes according to \(q_a\).}
    \label{fig:acrobot-ecm}
\end{figure}

Unfortunately, the VLP is not a true representation of the acrobot.
The effective length of the ECM is 
\[
    l_e(q_a) := l\sqrt{\frac{5}{4} + c_a}
    ,
\]
and its effective angle is
\[
    q_e := \arctan_2\left(s_u + \frac{1}{2}s_{ua}, -c_u - \frac{1}{2}c_{ua}\right)
    .
\]
There are two important notes to consider based on these equations. 
First, Figure \ref{fig:acrobot-vlp-symmetry} shows that for each pose of the
VLP representation, there are two configurations
of the acrobot which give the same effective length and angle.
This means the acrobot and the VLP are not equivalent representations;
designing a VNHC that injects energy using the ECM may not produce human-like
leg motion on the acrobot.

Second, if we were to compute the conjugate of momenta
\(p_{l_e}\) to \(l_e\) and \(p_e\) to \(q_e\), we would see the torque input
\(\tau\) appearing in both of their dynamic equations.
In the VLP model from Chapter \ref{ch:vlp}, the control input only
affects the dynamics of the length variable.
If we want to design a VNHC for this system, we cannot use any of the results
from Chapter \ref{ch:vlp} because the VLP models don't match.

\begin{figure}
    \centering
    \includestandalone[width=0.3\textwidth]{images/acrobot_vlp_symmetry}
    \caption{The equivalent center of mass of the acrobot generally has two configurations
        which correspond to the same effective length and angle. These
        configurations are symmetric about the line connecting the pivot to the
        ECM.}
    \label{fig:acrobot-vlp-symmetry}
\end{figure}

Since we cannot apply the results of Chapter \ref{ch:vlp} to simplify the proof
of energy injection, and the resulting ECM motion may not even produce
realistic leg motion, this model reduction is ineffective for our purposes. 

Let us turn next to the thesis of \citet{xingbo_thesis}, who designs a VHC to enforce a
so-called ``tap" motion with the purpose of injecting energy into the acrobot. 
First, he defines a compensator variable \(s\) which tracks \(\dot{q}_u\), so
that he can use the theory of VHCs with the extended configuration 
\((q_u,q_a,s)\).
He then finds \(h_1, h_2 \in \R_{>0}\) to define the
normalized radius \(\rho\) and normalized angle \(\xi\) in the
\((q_u, s)\)-plane.
These normalized variables are given by
\begin{align*}
    \rho &:= \sqrt{h_1 q_u^2 + h_2 s^2}
    , \\
    \xi &:= \arctan_2(h_2 s, h_1 q_u)
    . 
\end{align*}
He then sets the VHC to be \(h(q) = q_a - f_\text{rad}(\rho)f_\text{ang}(\xi)\)
with the control parameters \(\bar{q}_u\) and \(\rho_0\), where
\begin{align}
    \label{eqn:xingbo-frad}
    f_\text{rad}(\rho) &:= \tanh^2(\rho/\rho_0)
    , \\
    \label{eqn:xingbo-fang}
    f_\text{ang}(\xi) &:= 
    \begin{cases}
        0 & -\pi < \xi \leq 0 \\
        \bar{q}_u \exp\left(1 - \frac{1}{1-(\frac{4\xi}{\pi} - 1)^2}\right) 
          & 0 < \xi \leq \frac{\pi}{2} \\
        0 & \frac{\pi}{2} < \xi \leq \pi
        .
    \end{cases}
\end{align}

While this constraint shows promising experimental results and it accurately
emulates true human motion, \citeauthor{xingbo_thesis}
does not provide an analytical proof that the acrobot will gain energy.
His lack of analysis is tied to the fact that the constrained
dynamics are incredibly complicated.
In fact, just showing the constraint is regular is a challenging task.
While we could very easily convert his VHC into a VNHC by replacing \(s\) with
\(p_u\), we would run into the same problem. 
Since we want our constraint to \textit{provably} inject energy, we must forgo
this type of constraint in favour of something less complex.

\section{The Acrobot Constraint}
% TODO: Give the constraint, state the theorem, and show the outline of the
% proof (both as a sketch and in a figure)
One may be tempted to design a constraint \(q_a = f(\theta)\) 
with \(\theta := \arctan_2(p_u,q_u)\), since it was so effective for the VLP in
Chapter \ref{ch:vlp}.
It can be challenging to show this type of constraint is regular.
Recall that all constraints of the form \(h(q,p) = q_a - f(p_u)\)
are regular, since the regularity equation (\ref{eqn:regularity-matrix-acrobot})
becomes strictly positive.
A momentum-based constraint approximates human motion may still be likely to
inject energy into the acrobot.
 
To design the acrobot constraint, let us imagine what a human does on
a seated swing.
When they are moving forward they extend their legs, and when they are moving
backward they retract them. 
In reality, the motion is slightly more complicated than that.:
as the swing gains speed, the person leans their body back while
extending their legs.
This allows them to bring their legs higher, shortening the distance
from their center of mass to the pivot and adding more energy to the swing.
When the swing moves backward, they sit up and fully retract their legs
underneath them.

Now imagine we have affixed the torso of the acrobot to the swing's rope, so 
that the legs can pivot both forwards and backwards.  
In this case, the legs would pivot in the direction of motion
proportionally to the swing's speed.
One VNHC which emulates this process is \(q_a = \bar{q}_a\arctan_2( I p_u)\), as
in Figure \ref{fig:qa-arctan}.
Here, \(\bar{q}_a \in ]0,\frac{2 Q_a}{\pi}]\) so that 
\(q_a \in [-Q_a, Q_a]\) and \(I > 0 \in \R\) is a fixed control parameter.

This constraint is not a perfect emulation of a gymnast's movement -- during
a giant the gymnast's legs are almost completely extended, while the constraint
retracts the legs partially during rotations.
However, the motion looks similar enough that this constraint should provide a
decent foundation for injecting energy into the acrobot.
It is for this reason that we choose our acrobot's constraint to be
\begin{equation}\label{eqn:acrobot-constraint}
    h(q,p) = q_a - \bar{q}_a \arctan_2(I p_u)
    .
\end{equation}

\begin{figure}
    \centering
    \includestandalone[width=0.5\textwidth]{images/qa_arctan}
    \caption{The acrobot constraint \(q_a = \bar{q}_a \arctan_2(I p_u)\).}
    \label{fig:qa-arctan}
\end{figure}

Let us now compute the constrained dynamics for this
(\ref{eqn:acrobot-constraint}).
% TODO: should I do this for a generic h(q_a,p_u)? That's how I did it in my
% notes...
Note that \(dh_q = \begin{bmatrix}0 & 1\end{bmatrix}\tpose\), while
\[
    dh_{p_u} = \frac{-\bar{q}_a I}{1 + I^2 p_u^2}
    .
\]
Applying these to (\ref{eqn:g-qupu}), we get the solution for \(p_a\) on the
constraint manifold as
\[
    p_a(q_u,p_u) = \frac{
        (1+c_a)(1+I^2 p_u^2)p_u - m^2gl^3\bar{q}_a I (2-c_a^2)(2s_u + s_{ua})
    }{ml^2(3+2c_a)(1+I^2 p_u^2)}
    .
\]

\section{Proving the Acrobot Gains Energy}
% TODO: Prove this in sections
% TODO: You don't have to use the proof of theta_dot being positive. Stick to
% Manfredi's proof, and when you say "for I small enough" put in a footnote that
% says "in fact, we can be more precise and set bounds for I given as I <
% min{...}. See appendix for more detail."
% Then, we can have an appendix where we derive the bounds on I so that
% theta_dot is always positive. This will help the reader keep track and not
% bother them too much.

\section{Experimental Results}
% TODO: This might be a separate chapter

%/========== /Acrobot ==========/%
% vim: set tw=80 ts=4 sw=4 sts=0 et ffs=unix :
