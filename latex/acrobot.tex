%! TEX root = main.tex

%/========== Acrobot ==========/%

\chapter{Application of VNHCs: The Acrobot}\label{ch:acrobot}
\section{Motivation}
The acrobot is a two-link pendulum, actuated at the center joint (as in Figure
\ref{fig:acrobot-model}. Since its first description in 1990 
\cite{nonlinear_controllers_nonintegrable_acrobot}, the
acrobot has become a benchmark problem in control theory; it is an
underactuated mechanical system which produces
complex nonlinear motion from an easy-to-describe model.
The acrobot models a gymnast on a bar,
since it represents a torso (top link) and legs (bottom link) with motion
generated by the swinging of the legs at the hips. It is also one of the
simplest models for a biped walking robot
\cite{toward_framework_biped_locomotion}.

\begin{figure}
    \centering
    \includestandalone[width=0.7\textwidth]{images/acrobot_model}
    \caption{The acrobot model.}%
    \label{fig:acrobot-model}
\end{figure}

Controlling the acrobot is a nontrivial task, since it is not feedback
linearizable \cite{nonlinear_controllers_nonintegrable_acrobot}. Many
researchers have studied the swing-up problem of driving the acrobot to its
equilibrium point above the bar using partial feedback linearization
\cite{swingup_problem_acrobot}, energy-based control
\cite{swingup_acrobot_pendulum, swingup_acrobot_energy}, and through studying
human motion \cite{swingup_giant_acrobot, motion_control_gymnastic_skill}.

In gymnastics terminology, a ``giant" is when the gymnast performs full
rotations around the bar with their body almost fully extended
\cite{usagym_giant}. We are interested in studying the energy stabilization
problem for the acrobot by using VNHCs to generate giant motion. The control of
giant motion for the acrobot has been studied thoroughly 
\cite{energy_pumping_robotic_swinging, swingup_giant_acrobot,
control_giant_two_link_gymnastic_robot}, including several studies which use
virtual (holonomic) constraints to achieve this behaviour
\cite{dynamical_servo_acrobot_vc, control_giant_two_link_gymnastic_robot,
xingbo_thesis}. However, these controllers are neither intuitive nor easy to design:
\cite{control_giant_two_link_gymnastic_robot} defines a constraint by inverting
a trajectory in time onto the state space; \cite{dynamical_servo_acrobot_vc}
requires a cascade controller to stabilize both a constraint and a desired limit
cycle in the state space; and \cite{xingbo_thesis} enforces the giant by adding
an extra state to estimate the velocity, which increases the dimensionality of
the problem as a crude approach to using VNHCs.

In this chapter we will design a physically-intuitive VNHC which generates giant
motion. We will prove that this VNHC injects energy into the acrobot;
in the process of completing this proof,
we will describe a promising method which may be used to generate desirable 
VNHCs in the future.

\section{Dynamics of the Acrobot}
% TODO: Derive the acrobot dynamics in Hamiltonian form
For the acrobot model displayed in Figure \ref{fig:acrobot-model}, 
the COM of each link lies somewhere along the link itself. 
While this is the most general representation of an acrobot, the dynamics are
somewhat complicated and are difficult to use when proving the results in the 
rest of this chapter. For simplicity, we will
assume the acrobot is comprised of two massless rods of equal length with 
equal masses at the tips, as in Figure \ref{fig:simple-acrobot-model}; we call
this a \textbf{simple} acrobot.

\begin{figure}
    \centering
    \includestandalone[width=0.5\textwidth]{images/simple_acrobot_model}
    \caption{A simple acrobot with lengths \(l\) and masses of size \(m\).}
    \label{fig:simple-acrobot-model}
\end{figure}


\section{Previous Approaches}
% TODO: Give Xingbo's intuition, abstract away to qa = sin(theta) constraint
% based on VLP but show it doesn't work well in our framework
% TODO: Describe Xingbo's results and observe that his results were similar to a VNHC
% except they used VHC tools. We will modify his approach to use VNHCs and prove
% rigorously that we can stabilize any energy level set on the acrobot.

\section{The Acrobot Constraint}
% TODO: Give the constraint, state the theorem, and show the outline of the
% proof (both as a sketch and in a figure)
\section{Proving the Acrobot Gains Energy}
% TODO: Prove this in sections

\section{Experimental Results}
% TODO: This might be a separate chapter




























%\section{Energy Injection for the Acrobot}
%Suppose the acrobot is simple; 
%that is, assume that the acrobot's two links are of equal length \(l\)
%and of equal mass \(m\), with the mass concentrated at the tip of each rod.
%
%Let \(q = (q_u,q_a)\) where \(q_u\) is the shoulder angle and \(q_a\) is the hip angle.
%
%Then the acrobot has hamiltonian given by
%\[
%    \mathcal{H}(q,p) = \frac{1}{2} p^T M^{-1}(q) p + V(q)
%\]
%where
%\[
%    M^{-1}(q) = \frac{1}{m l^2 (2 - \cos^2(q_a))} 
%    \begin{bmatrix}
%        1              & -(1 + \cos(q_a)) \\
%        -(1 + \cos(q_a)) & (3 + 2\cos(q_a))
%    \end{bmatrix}
%\]
%and
%\[
%    V(q) = m g l (3 - 2 \cos(q_u) - \cos(q_u + q_a))
%\]
%
%Notice that the acrobot satisfies \(\partial M^{-1}(q)/ \partial q_u = 0\),
%which means it may be possible to enforce a regular solvable VNHC to inject energy
%into the acrobot.
%One such VNHC is
%\begin{align}\label{eqn:acrobot_vnhc}
%    \begin{split}
%    h(q,p) &= q_a - f(q_u,p_u) \\
%           &= q_a - \bar{q_a}\sin(\arctan(p_u,q_u)) \\
%           &= q_a - \frac{\bar{q_a}p_u}{\sqrt{q_u^2 + p_u^2}}
%    \end{split}
%\end{align}
%where \(\bar{q_a} \in ]0,\pi[\) is a parameter which dictates how much the 
%controlled link can swing back and forth.
%
%To compute the regularity and solvability conditions, first observe that since
%the acrobot has two coordinates and only \(p_a\) is actuatable, the control matrix
%\(B = [0, 1]^T\) is constant and full rank everywhere. This also implies that
%\(h\) is both regular and solvable if \(dh_q M^{-1}(q)B\) is full rank.
%
%Observe that
%\begin{align}
%    \begin{split}
%        dh_q M^{-1}(q)B &= \frac{1}{m l^2 (2 - \cos^2(q_a))}
%\left[ \frac{\bar{q_a}p_uq_u}{(q_u^2 + p_u^2)^{3/2}}, 1\right] 
%    \begin{bmatrix}
%        -(1 + \cos(q_a)) \\
%        (3 + 2\cos(q_a))
%    \end{bmatrix} \\
%                     &= \frac{ (3+2\cos(q_a))(q_u^2 + p_u^2)^{3/2} 
%                     - \bar{q_a}p_uq_u(1+cos(q_a))}
%                     {m l^2 (q_u^2+p_u^2)^{3/2} (2 - \cos^2(q_a))}
%    \end{split}
%\end{align}
%
%TODO: \textbf{Determine if this is full rank everywhere except at \((q_u,p_u) = (0,0)\)}
%
%Solving for \(p_a\) is straightforward. The computation is shown below.
%\begin{multline*}
%    p_a = \left(dh_q M^{-1}(q)B\right)^{-1}  \\
%        \left(\frac{m g l \bar{q_a}p_u q_u^2}{(q_u^2+p_u^2)^{3/2}} 
%            \left(2\sin(q_u) + sin(q_u + q_a) \right)
%        +
%            \frac{1}{m l^2 (2 - \cos^2(q_a))}
%            [ \frac{\bar{q_a}p_uq_u}{(q_u^2+p_u^2)^{3/2}}, 1] 
%            \begin{bmatrix}
%              1 \\
%              -(1 + \cos(q_a))
%          \end{bmatrix} \right)
%\end{multline*}
%\begin{align*}
%    &= -\frac
%    {
%        \frac{ m g l \bar{q_a}q_u^2(2\sin(q_u) + \sin(q_u+q_a))}{(q_u^2+p_u^2)^{3/2}}
%        + \frac{\bar{q_a} p_u^2 q_u}{m l^2 (q_u^2 + p_u^2)^{3/2} (2 -\cos^2(q_a))}
%        - \frac{(1+\cos(q_a)) p_u}{m l^2 (2 - \cos^2(q_a))}
%    }
%    {
%        \frac
%        {
%            (3 + 2\cos(q_a))(q_u^2+p_u^2)^{3/2} - \bar{q_a}(1 + \cos(q_a))q_up_u
%        }
%        {
%            m l^2 (q_u^2+p_u^2)^{3/2} (2 - \cos^2(q_a))
%        }
%    } \\
%    &= \frac
%    {
%        \frac
%        {
%            -m^2 g l^3 \bar{q_a}q_u^2(2\sin(q_u) + \sin(q_u+q_a))(2 - \cos^2(q_a))
%            - \bar{q_a}p_u^2 q_u 
%            + (1 + \cos(q_a))(q_u^2 + p_u^2)^{3/2} p_u
%        }
%        {
%            m l^2 (q_u^2+p_u^2)^{3/2} (2 - \cos^2(q_a))
%        }
%    }
%    {
%        \frac
%        {
%            (3 + 2\cos(q_a))(q_u^2+p_u^2)^{3/2} - \bar{q_a}(1 + \cos(q_a))q_up_u
%        }
%        {
%            m l^2 (q_u^2+p_u^2)^{3/2} (2 - \cos^2(q_a))
%        }
%    }
%\end{align*}
%
%\begin{equation*}
%    \Rightarrow p_a = \frac
%        {
%            (1 + \cos(q_a))(q_u^2 + p_u^2)^{3/2} p_u
%            -m^2 g l^3 \bar{q_a}q_u^2(2\sin(q_u) + \sin(q_u+q_a))(2 - \cos^2(q_a))
%            - \bar{q_a}q_up_u^2
%        }
%        {
%            (3 + 2\cos(q_a))(q_u^2+p_u^2)^{3/2} - \bar{q_a}(1 + \cos(q_a))q_up_u
%        }
%\end{equation*}
%
%With \(p_a\) in hand, it is now possible to solve for the zero dynamics under the constraint. To simplify the notation we define \(s_u := \sin(q_u)\),
%\(s_{ua} := \sin(q_u + q_a)\), 
%and \(c_a := \cos(q_a)\).
%\begin{align*}
%    \dot{q_u} &= \frac{1}{m l^2 (2 - c_a^2)} [ 1, -(1+c_a)]
%    \begin{bmatrix} p_u \\ p_a \end{bmatrix} \\
%    \dot{p_u} &= - m g l (2 s_u + s_{ua})
%\end{align*}
%Expanding out \(\dot{q_u}\) we get
%\begin{multline*}
%    \dot{q_u} = 
%    \frac{p_u}{m l^2 (2 - c_a^2)}\\
%    + \frac
%    {
%        m^2 g l^3 \bar{q_a}q_u^2(2s_u + s_{ua})(2 - c_a^2)(1 + c_a)
%        - (1+c_a)^2(q_u^2+p_u^2)^{3/2}p_u 
%        + \bar{q_a}(1+c_a)q_up_u^2
%    }
%    {
%        ml^2 (2-c_a^2) ((3 + 2c_a)(q_u^2+p_u^2)^{3/2} - \bar{q_a}(1 + c_a)q_up_u)
%    }
%\end{multline*}
%and, putting everything over one denominator,
%\begin{align*}
%    \dot{q_a} &= \frac
%    {
%        (3 + 2c_a)(q_u^2+p_u^2)^{3/2}p_u 
%        - (1+c_a)^2(q_u^2+p_u^2)^{3/2}p_u 
%        +m^2 g l^3 \bar{q_a}q_u^2(2s_u + s_{ua})(2 - c_a^2)(1 + c_a)
%    }
%    {
%        ml^2 (2-c_a^2) ((3 + 2c_a)(q_u^2+p_u^2)^{3/2} - \bar{q_a}(1 + c_a)q_up_u)
%    } \\
%    &= \frac
%    {
%        (2-c_a^2)\left((q_u^2+p_u^2)^{3/2}p_u 
%        +m^2 g l^3 \bar{q_a}q_u^2(2s_u + s_{ua})(1 + c_a)\right)
%    }
%    {
%        ml^2 (2-c_a^2) ((3 + 2c_a)(q_u^2+p_u^2)^{3/2} - \bar{q_a}(1 + c_a)q_up_u)
%    }
%\end{align*}
%Therefore, the zero-dynamics of the acrobot under the constraint 
%\(q_a = \sin(\arctan(p_u,q_u))\) are:
%\begin{align}\label{eqn:acrobot_eom_qp}
%\begin{split}
%    \dot{q_u} &= \frac
%    {
%        (q_u^2+p_u^2)^{3/2}p_u 
%        +m^2 g l^3 \bar{q_a}q_u^2(2s_u + s_{ua})(1 + c_a)
%    }
%    {
%        ml^2 \left((3 + 2c_a)(q_u^2+p_u^2)^{3/2} - \bar{q_a}(1 + c_a)q_up_u\right)
%    } \\
%    \dot{p_u} &= - m g l (2s_u + s_{ua})
%\end{split}
%\end{align}
%
%It may be convenient to represent these equations of motion  in polar coordinates.
%Letting \(r = \sqrt{q_u^2 + p_u^2}\) be the radius and 
%\(\theta = \arctan(p_u,q_u)\) be the angle in the \((q_u,p_u)\) plane and taking
%derivatives, it is easy to show that
%\begin{align}\label{eqn:dr_and_dt_general}
%\begin{split}
%    \dot{r} &= [\cos(\theta),\sin(\theta)]
%    \begin{bmatrix} \dot{q_u}(r,\theta) \\ \dot{p_u}(r,\theta) \end{bmatrix} \\
%    \dot{\theta} &= \frac{1}{r}[-\sin(\theta), \cos(\theta)]
%    \begin{bmatrix} \dot{q_u}(r,\theta) \\ \dot{p_u}(r,\theta) \end{bmatrix}
%\end{split}
%\end{align}
%
%By expanding out (\ref{eqn:dr_and_dt_general}) then substituting \(q_u = r\cos(\theta)\)
%and \(p_u = r\sin(\theta)\)
%it is straightforward computation to find the equations of motion in cylindrical \((r,\theta)\)
%coordinates. To simplify the notation yet again, define
%\(s(g)_\theta := \sin(g(\theta))\) and \(c(g)_\theta := \cos(g(\theta))\) 
%for any function \(g: \mathbb{R} \rightarrow \mathbb{R}\). The final equations
%of motion in \((r,\theta)\) coordinates are
%\begin{align}\label{eqn:acrobot_eom_rt}
%\begin{split}
%    \dot{r} &= \frac
%    {
%        r^2 c_\theta s_\theta 
%        + m^2 g l^3 (2 s(rc)_\theta + s(rc + \bar{q_a}s)_\theta)
%        \left(
%            (1+c(\bar{q_a}s)_\theta)\bar{q_a}c_\theta 
%            - (3 + 2c(\bar{q_a}s)_\theta)s_\theta r
%        \right)
%    }
%    {
%        m l^2 \left((3+2c(\bar{q_a}s)_\theta)r 
%        - \bar{q_a}(1+ c(\bar{q_a}s)_\theta)c_\theta s_\theta\right)
%    }\\
%    \dot{\theta} &= -\frac
%    {
%        r s_\theta^2
%        + m^2 g l^3 (2 s(rc)_\theta + s(rc + \bar{q_a}s)_\theta)
%        (3 + 2c(\bar{q_a}s)_\theta)c_\theta
%    }
%    {
%        m l^2 \left((3+2c(\bar{q_a}s)_\theta)r 
%        - \bar{q_a}(1+ c(\bar{q_a}s)_\theta)c_\theta s_\theta\right)
%    }
%\end{split}
%\end{align}

%/========== /Acrobot ==========/%
% vim: set tw=80 ts=4 sw=4 sts=0 et ffs=unix :
