%! TEX root = main.tex

%/========== Acrobot ==========/%

\chapter{Application of VNHCs: The Acrobot}\label{ch:acrobot}
\section{Motivation}
The acrobot is a two-link pendulum, actuated at the center joint (as in Figure
\ref{fig:acrobot-model}). 
Since its first description in 1990
\cite{nonlinear_controllers_nonintegrable_acrobot}, the acrobot has become a
benchmark problem in control theory; 
it is an underactuated mechanical system which produces complex nonlinear motion
from an easy-to-describe model.
The acrobot models a gymnast on a bar, since it represents a torso (top link)
and legs (bottom link) with motion generated by the swinging of the legs at the
hips. 
It is also one of the simplest models for a biped walking robot
\cite{toward_framework_biped_locomotion}.

\begin{figure}
    \centering
    \includestandalone[width=0.7\textwidth]{images/acrobot_model}
    \caption{The general acrobot model, represented by two weighted rods
    differing in both length and mass.}%
    \label{fig:acrobot-model}
\end{figure}

Controlling the acrobot is a nontrivial task because it is not feedback
linearizable \cite{nonlinear_controllers_nonintegrable_acrobot}. 
Several researchers have studied the swing-up problem of driving the acrobot to
its equilibrium point above the bar using partial feedback linearization
\cite{swingup_problem_acrobot}, energy-based control
\cite{swingup_acrobot_pendulum, swingup_acrobot_energy}, and through studying
human motion \cite{swingup_giant_acrobot, motion_control_gymnastic_skill}.

In gymnastics terminology, a ``giant" is the motion a gymnast performs to
achieve full rotations around the bar \cite{usagym_giant}. 
We are interested in using VNHCs to generate giant motion, with the aim of
stabilizing desired energy levels.
The control of giant motion for the acrobot has been studied in
\cite{energy_pumping_robotic_swinging, swingup_giant_acrobot}, 
and some authors have used virtual holonomic constraints to achieve this
behaviour
\cite{dynamical_servo_acrobot_vc, control_giant_two_link_gymnastic_robot,
xingbo_thesis}. 
However, these controllers are neither intuitive nor easy to
design:
\cite{control_giant_two_link_gymnastic_robot} defines a constraint by inverting
a trajectory in time onto the state space; 
\cite{dynamical_servo_acrobot_vc} requires a cascade controller to stabilize
both a constraint and a desired limit cycle in the state space; 
and \cite{xingbo_thesis} enforces the giant by adding an extra state to estimate
velocity, which increases the dimensionality of the problem in a crude
approach to using VNHCs.

In this chapter we will design a physically-intuitive VNHC which generates giant
motion and prove the acrobot gains energy. 
In the process of completing this proof, we will arrive at a promising method
which might one day be useful for generating energy-injecting VNHCs on arbitrary
mechanical systems.

\section{Dynamics of the Acrobot}
Suppose we are given an acrobot as in Figure \ref{fig:acrobot-model} modelling a
gymnast hanging on a horizontal bar, where the ``torso" has moment of
inertia \(J_u\) and the ``leg" has moment of inertia \(J_a\) (each with respect
to their own center of mass).
Let \(q_u \in \Sone\) be the shoulder angle and \(q_a \in \Sone\) 
be the hip angle, where only \(q_a\) is actuated. 
Collecting them together provides the configuration
\(q = (q_u,q_a) \in \Sone \times \Sone\). 
The acrobot has inertia matrix \(D\), potential function \(P\) (with respect to
the horizontal bar), and input matrix \(B\)  given as
follows \cite{xingbo_thesis}:
\begin{align}\label{eqn:general-acrobot-inertia}
    D(q) &= \begin{bmatrix}
      m_al_u^2 + 2m_a\cos(q_a)l_u l_{c_a} + m_al_{c_a}^2 + m_ul_{c_u}^2 + J_u + J_a &
      m_al_{c_a}^2 + m_al_ul_{c_a}\cos(q_a) + J_a \\
      m_al_{c_a}^2 + m_al_ul_{c_a}\cos(q_a) + J_a &
      m_al_{c_a}^2 + J_a
    \end{bmatrix} 
    , \\
    \label{eqn:general-acrobot-potential}
    P(q) &= g\left(m_al_{c_a}(1 - \cos(q_u+q_a)) + 
        (m_al_u + m_ul_{c_u})(1-\cos(q_u))\right) 
    , \\
    B(q) &= \begin{bmatrix} 0 \\ 1 \end{bmatrix}
    .
\end{align}

While this is the most general representation of an acrobot, the dynamics
are unwieldy.
To make rigorous analysis of these dynamics more tractable, we begin by assuming
the acrobot is comprised of two massless rods of equal length \(l\), with equal
point masses \(m\) at the tips.
We call this a \textit{simple} acrobot, which is displayed in Figure
\ref{fig:simple-acrobot-model}.
We will also ignore any frictional forces at both the hip and shoulder joints. 
Finally, it is important to note that a real gymnast cannot swing their legs in
full circles, though they are usually flexible enough to raise them parallel to
the floor; 
for this reason, we assume that \(q_a \in [-Q_a, Q_a]\) where 
\(Q_a \in [\frac{\pi}{2}, \pi[\). 

\begin{figure}
    \centering
    \includestandalone[width=0.5\textwidth]{images/simple_acrobot_model}
    \caption{A simple acrobot has massless rods of equal length \(l\) and 
    equal masses \(m\) at the tips.}
    \label{fig:simple-acrobot-model}
\end{figure}

Since we are now working with a simple acrobot, 
we have \(l_{c_u} = l_{c_a} = l_u = l_a = l\) and
\(m_u = m_a = m\). 
On top of this, the moments of inertia \(J_u\) and \(J_a\) of the rods vanish.
Reducing
~\eqref{eqn:general-acrobot-inertia}-\ref{eqn:general-acrobot-potential}
yields the simplified inertia matrix \(D_s\) and potential function \(P_s\),
where
\begin{align}
    D_s(q) &= \begin{bmatrix}
        ml^2\left(3+2\cos(q_a)\right) & 
        ml^2\left(1+\cos(q_a)\right) \\
        ml^2\left(1+\cos(q_a)\right) &
        ml^2
    \end{bmatrix} 
    , \\
    P_s(q) &= -mgl\left(2\cos(q_u)+\cos(q_u+q_a)\right)
    .
\end{align}
\begin{notation}
    For shorthand, we write \(c_u := \cos(q_u)\), \(c_a := \cos(q_a)\), and 
    \(c_{ua} := \cos(q_u + q_a)\). Likewise, \(s_u := \sin(q_u)\), 
    \(s_a := \sin(q_a)\), and \(s_{ua} := \sin(q_u + q_a)\).
\end{notation}

Defining \(M(q) := D_s(q)\) and \(V(q) := P_s(q)\), we find the conjugate of momenta 
is \(p = (p_u, p_a) = M(q)\dot{q}\).
The dynamics in \((q,p)\) coordinates are given by
\begin{align}\label{eqn:acrobot-hamiltonian}
    \mathcal{H}(q,p) &= \frac{1}{2}p\tpose \Minv(q) p -
    mgl\left(2 c_u + c_{ua}\right)
    , \\
     &\begin{cases}
        \dot{q} = \Minv(q) p \\
        \dot{p}_u = -mgl\left(2s_u + s_{ua}\right) \\
        \dot{p}_a =-\frac{1}{2}p\tpose \nabla_{q_a}\Minv(q) p
        - mgl s_{ua} + \tau,
    \end{cases} \nonumber
\end{align}
where the inverse inertia matrix is
\begin{equation}\label{eqn:Minv}
    \Minv(q) = \frac{1}{ml^2\left(2-c_a^2\right)}
    \begin{bmatrix}
        1 &
        -\left(1+c_a\right) \\
        -\left(1+c_a\right) &
        3+2c_a
    \end{bmatrix}
    .
\end{equation}
The control input is a force \(\tau \in \R\) affecting only the dynamics of
\(p_a\), representing a torque acting on the hip joint.
This means \((q,p)\) are simply actuated coordinates inside the phase space
\(\mathcal{Q} \times \mathcal{P}\) where
\(\mathcal{Q} = \mathcal{Q}_u \times \mathcal{Q}_a 
:= \Sone \times \Sone\), and
\(\mathcal{P} = \mathcal{P}_u \times \mathcal{P}_a
:= \R \times \R\).
This allows us to apply the theory of VNHCs from Chapter
\ref{ch:vnhcs}.

Let us define the VNHC \(h(q,p) = q_a - f(q_u,p_u)\) of order 1, where
\(f \in C^2\left(\mathcal{Q}_u \times \mathcal{P}_u; \mathcal{Q}_a\right)\).
Since \(\nabla_{q_u}\Minv(q) = \Zmat{2\times 2}\),
Theorem \ref{thm:vnhc-regularity} tells us that this VNHC will be regular
when the regularity matrix
\[
    dh_q \Minv(q) \begin{bmatrix}0 \\ 1 \end{bmatrix}
    ,
\]
is of full rank \(1\) on the constraint manifold \(\Gamma\).
Given that
\(dh_q = \begin{bmatrix} -\partial_{q_u} f & 1 \end{bmatrix}\),
the regularity matrix evaluates to the scalar equation
\begin{equation}\label{eqn:regularity-matrix-acrobot}
    \frac{(1+c_a)\partial_{q_u}f(q_u,p_u) + (3+2c_a)}{ml^2(2-c_a^2)}
    .
\end{equation}
This is full rank if and only if the numerator does not
change sign.
The following proposition provides a sufficient condition for regularity.

\begin{prop}\label{prop:acrobot-fpu-regular}
    A relation \(h(q,p) = q_a - f(p_u) = 0\) 
    with \(f \in C^2\left(\mathcal{P}_u; \mathcal{Q}_a\right)\) is a regular
    VNHC of order 1 for the simple acrobot.
\end{prop}
\begin{proof}
    Since \(\partial_{q_u} f = 0\), the regularity equation
    ~\eqref{eqn:regularity-matrix-acrobot} is strictly positive for all values
    of \(q_a\), and hence is full rank everywhere on the constraint manifold.
    By Theorem \ref{thm:vnhc-regularity}, \(h\) is a regular VNHC of order 1.
\end{proof}
Proposition \ref{prop:acrobot-fpu-regular} will be useful later, as we will not
need to check regularity if we design a function of the unactuated momentum. 

The acrobot is noticeably more complex than the VLP, as
the dynamics of \((q_u,p_u)\) and \((q_a,p_a)\) are coupled through \(\Minv(q)\).
Because of this, the constrained dynamics of an arbitrary VNHC may not be easy
to write out.
In the rest of this chapter, our goal is to design the function \(f(q_u,p_u)\)
based on the natural human motion of a gymnast, with one caveat: 
we must be able to prove the constrained dynamics will inject energy into the
acrobot.

\section{Previous Constraint Approaches}
Let us examine some of the existing approaches to generating giant motion for
the acrobot, since these may be viable candidates on which to base a VNHC.

One initial approach to controlling the acrobot is to model it as a
variable-length pendulum by collapsing the two rods and masses into one
equivalent center of mass (ECM), as in Figure \ref{fig:acrobot-ecm}.
This seems a reasonable model reduction, since the length from the pivot to the
ECM changes depending on the angle \(q_a\) of the leg.
Indeed, \citet{swingup_giant_acrobot} use this approach to design a trajectory
for the ECM, then determine which leg angles \(q_a(t)\) are required to generate
that trajectory.
Following in their footsteps, we might consider using the results
from Chapter \ref{ch:vlp} to find the leg angles that allow the ECM to gain
energy. 
Then we could apply Theorem \ref{thm:vlp-energy-stabilization} to prove
the acrobot is gaining energy.
\begin{figure}
    \centering
    \includestandalone[width=0.5\textwidth]{images/acrobot_ecm}
    \caption{A simple acrobot modelled as a VLP with equivalent center of mass \(2m\). 
        The length of the VLP changes according to \(q_a\).}
    \label{fig:acrobot-ecm}
\end{figure}

Unfortunately, the VLP is not a true representation of the acrobot.
The effective length of the ECM is 
\[
    l_e(q_a) := l\sqrt{\frac{5}{4} + c_a}
    ,
\]
and its effective angle is
\[
    q_e := \arctan_2\left(s_u + \frac{1}{2}s_{ua}, -c_u - \frac{1}{2}c_{ua}\right)
    .
\]
There are two important notes to consider based on these equations. 
First, Figure \ref{fig:acrobot-vlp-symmetry} shows that for each pose of the
VLP representation, there are two configurations
of the acrobot which give the same effective length and angle.
This means the acrobot and the VLP are not equivalent representations;
designing a VNHC that injects energy using the ECM may not produce human-like
leg motion on the acrobot.

Second, if we were to compute the conjugate of momenta
\(p_{l_e}\) to \(l_e\) and \(p_e\) to \(q_e\), we would see the torque input
\(\tau\) appearing in both of their dynamic equations.
In the VLP model from Chapter \ref{ch:vlp}, the control input only
affects the dynamics of the length variable.
If we want to design a VNHC for this system, we cannot use any of the results
from Chapter \ref{ch:vlp} because the VLP models do not match.

\begin{figure}
    \centering
    \includestandalone[width=0.3\textwidth]{images/acrobot_vlp_symmetry}
    \caption{The equivalent center of mass of the acrobot generally has two configurations
        which correspond to the same effective length and angle. These
        configurations are symmetric about the line connecting the pivot to the
        ECM.}
    \label{fig:acrobot-vlp-symmetry}
\end{figure}

Since we cannot apply the results of Chapter \ref{ch:vlp} to simplify the proof
of energy injection, and the resulting ECM motion may not even produce
realistic leg motion, this model reduction is ineffective for our purposes. 

Let us turn next to the thesis of \citet{xingbo_thesis}, who designs a VHC to enforce a
so-called ``tap" motion with the purpose of injecting energy into the acrobot. 
First, he defines a compensator variable \(s\) which tracks \(\dot{q}_u\), so
that he can use the theory of VHCs with the extended configuration 
\((q_u,q_a,s)\).
He then finds \(h_1, h_2 \in \R_{>0}\) to define the
normalized radius \(\rho\) and normalized angle \(\xi\) in the
\((q_u, s)\)-plane.
These normalized variables are given by
\begin{align*}
    \rho &:= \sqrt{h_1 q_u^2 + h_2 s^2}
    , \\
    \xi &:= \arctan_2(h_2 s, h_1 q_u)
    . 
\end{align*}
He then sets the VHC to be \(h(q) = q_a - f_\text{rad}(\rho)f_\text{ang}(\xi)\)
with the control parameters \(\bar{q}_u\) and \(\rho_0\), where
\begin{align}
    \label{eqn:xingbo-frad}
    f_\text{rad}(\rho) &:= \tanh^2(\rho/\rho_0)
    , \\
    \label{eqn:xingbo-fang}
    f_\text{ang}(\xi) &:= 
    \begin{cases}
        0 & -\pi < \xi \leq 0 \\
        \bar{q}_u \exp\left(1 - \frac{1}{1-(\frac{4\xi}{\pi} - 1)^2}\right) 
          & 0 < \xi \leq \frac{\pi}{2} \\
        0 & \frac{\pi}{2} < \xi \leq \pi
        .
    \end{cases}
\end{align}

While this constraint shows promising experimental results and it accurately
emulates true human motion, \citeauthor{xingbo_thesis}
does not provide analytical proof that the acrobot will gain energy.
His lack of analysis is tied to the fact that the constrained
dynamics are incredibly complicated.
In fact, just showing the constraint is regular is a challenging task.
While we could very easily convert his VHC into a VNHC by replacing \(s\) with
\(p_u\), we would run into the same problem. 
Since we want our constraint to \textit{provably} inject energy, we must forgo
this type of constraint in favour of something less complex.

\section{The Acrobot Constraint}
One may be tempted to design a constraint of the form 
\(q_a = \bar{q}_a\sin(\theta)\),
with \(\theta := \arctan_2(p_u,q_u)\), since a similar approach 
was so effective for the VLP in Chapter \ref{ch:vlp}.
Unfortunately, this constraint is not regular, and it is difficult to find any
VNHC of the form \(q_a = f(\theta)\) where regularity can be proven easily.
Instead, we will develop a constraint \(h(q,p) = q_a - f(p_u)\) because these
constraints are always regular (as per Proposition \ref{prop:acrobot-fpu-regular}). 
 
To design this constraint, let us begin (perhaps unexpectedly) by examining a person on
a seated swing.
The person extends their legs when the swing moves forwards, and retracts their
legs when the swing moves backwards.
As the swing gains speed, the person leans their body back while
extending their legs.
This allows them to bring their legs higher, shortening the distance
from their center of mass to the pivot and adding more energy to the swing.
When the swing moves backward, they sit up and fully retract their legs
underneath them \cite{how_to_pump_a_swing}.

Now imagine the person's torso is affixed to the swing's rope so they are
always upright. 
Imagine further that the swing has no seat at all, allowing the person to extend
their legs beneath them. 
This position is identical to that of a gymnast on a bar, which is why we can
use leg motion from the seated swing to design a controller for the acrobot.

The acrobot's legs are rigid rods which cannot retract, so we emulate the person
on a swing by pivoting the legs toward the direction of motion. 
To account for how a person leans back at higher speeds, the legs should pivot to an
angle proportional to the swing's speed.
Since the direction of motion is entirely determined by \(p_u\), 
one such VNHC which emulates this process is \(q_a = \bar{q}_a\arctan( I p_u)\),
displayed in Figure \ref{fig:qa-arctan}.
Here, \(\bar{q}_a \in ]0,\frac{2 Q_a}{\pi}]\) and \(I \in \R\) is a fixed
control parameter.

This constraint does not perfectly recreate giant motion, during which
the gymnast's legs are almost completely extended \cite{usagym_giant}
-- it instead pivots the legs partially during rotations.
However, the behaviour looks similar enough that the constraint should provide a
reasonable foundation for injecting energy into the acrobot.
It is for this reason that we choose our acrobot's constraint to be
\begin{equation}\label{eqn:acrobot-constraint}
    h(q,p) = q_a - \bar{q}_a \arctan(I p_u)
    .
\end{equation}

\begin{figure}
    \centering
    \includestandalone[width=0.5\textwidth]{images/qa_arctan}
    \caption{The acrobot constraint \(q_a = \bar{q}_a \arctan(I p_u)\).}
    \label{fig:qa-arctan}
\end{figure}

Let us now compute the constrained dynamics under
~\eqref{eqn:acrobot-constraint}.
Note that \(dh_q = \begin{bmatrix}0 & 1\end{bmatrix}\), while
\[
    dh_{p_u} = \frac{-\bar{q}_a I}{1 + I^2 p_u^2}
    .
\]
Inserting these into ~\eqref{eqn:g-qupu}, we get the solution for \(p_a\) on the
constraint manifold:
\[
    p_a(q_u,p_u) = \frac{
        (1+c_a)(1+I^2 p_u^2)p_u - m^2gl^3\bar{q}_a I (2-c_a^2)(2s_u + s_{ua})
    }{ml^2(3+2c_a)(1+I^2 p_u^2)}
    .
\]
The dynamics for \(p_u\) do not contain \(p_a\), so they remain unchanged.
The constrained dynamics for \(q_u\) are given by 
\begin{equation*}
    \dot{q}_u = e_1\tpose \Minv(q) \begin{bmatrix}
                    p_u \\ p_a(q_u,p_u)
                \end{bmatrix} %\\
    ,
\end{equation*}
which can be simplified into 
\begin{equation*}
    \dot{q_u} = \frac{(1+I^2 p_u^2)p_u + m^2gl^3\bar{q}_a I(2s_u + s_{ua})(1+c_a) }{ml^2(1+I^2 p_u^2)(3+2c_a)}
    .
\end{equation*}
Hence, the constrained dynamics for the acrobot under
~\eqref{eqn:acrobot-constraint} are
\begin{equation}\label{eqn:acrobot-constrained-dynamics}
\left.\begin{cases}
    \dot{q}_u &= \frac{(1+I^2 p_u^2)p_u + m^2gl^3\bar{q}_a I(2s_u + s_{ua})(1+c_a) }
            {ml^2(1+I^2 p_u^2)(3+2c_a)}
        \\
    \dot{p}_u &= - m g l (2s_u + s_{ua})
    \end{cases} \right|_{q_a = \bar{q}_a\arctan(Ip_u)}
    .
\end{equation}

These dynamics do not always gain energy; rather, energy gain is only
guaranteed under certain conditions on the size of \(I\).
First, let 
\[
    E(q_u,p_u) := \frac{p_u^2}{10ml^2} + 3mgl(1 - \cos(q_u))
    ,
\]
be the energy function obtained by setting \(I = 0\).
As we will see later, this is the mechanical energy of a simple pendulum with
two masses.
Next, define the set
\begin{equation}\label{eqn:oscillation-domain}
    \mathcal{O}_1 := \left\{(q_u,p_u) \in \SxR 
    \mid E(q_u,p_u) < E(\pi,0) \right\}
    ,
\end{equation}
as in Figure \ref{fig:acrobot-oscillation-domain}.
Finally, let \(\bar{\rho} > \sqrt{60m^2gl^3}\) and define the set
\begin{equation}\label{eqn:o-rhobar}
    \mathcal{O}_2(\bar{\rho}) := \left\{(q_u,p_u) \in \SxR
        \mid E(q_u,p_u) < E(0,\bar{\rho}) \right\}
    ,
\end{equation}
as in Figure \ref{fig:acrobot-o2}

\begin{figure}
    \centering
    \begin{subfigure}[t]{0.45\textwidth}
        \includestandalone[width=\linewidth]{images/acrobot_oscillation_domain}
        \caption{The set \(\mathcal{O}_1\). An orbit starting in this set (blue)
            will pass through the level set \(E(\pi,0)\) of the nominal pendulum.}
        \label{fig:acrobot-oscillation-domain}
    \end{subfigure}
    \hfill
    \begin{subfigure}[t]{0.45\textwidth}
        \includestandalone[width=\linewidth]{images/acrobot_omega}
        \caption{The set \(\mathcal{O}_2(\bar{\rho})\). Orbits starting in this
            set are guaranteed to reach the level set \(E(0,\bar{\rho})\) of the
            nominal pendulum.}
            \label{fig:acrobot-o2}
    \end{subfigure}
    \caption{The sets on which the acrobot gains energy, according to Theorem
        \ref{thm:acrobot-energy-stabilization}.}
\end{figure}

For small enough \(I\), our VNHC will inject energy on the open set
\(\mathcal{O}_1\) and (provided a suitable condition is met) on the open set
\(\mathcal{O}_2(\bar{\rho})\).

\begin{thm}\label{thm:acrobot-energy-stabilization}
    Consider the simple acrobot ~\eqref{eqn:acrobot-hamiltonian} constrained by
    the VNHC ~\eqref{eqn:acrobot-constraint}, whose
    constraint manifold is \(\Gamma \simeq \SxR\).
\begin{enumerate}
    \item There exists \(I^\star > 0\) such that, for all 
        \(I \in \,]0,I^\star]\), 
    ~\eqref{eqn:acrobot-constraint} injects energy into the acrobot on
    \(\mathcal{O}_1\).
    Moreover, almost every orbit will escape the closure of 
    \(\mathcal{O}_1\) in finite time.
    If instead \(I \in [-I^\star,0[\), the VNHC dissipates energy.
\item Define \(b : \SxR_{> 0} \rightarrow \R\) by
    \[
        b(\beta,\rho_0) := 
        \frac{5m^2 g l^3 \bar{q}_a \left(
            m^2gl^3\left(18s_\beta^2 + 30c_\beta(1 - c_\beta)\right)
            - c_\beta\rho_0^2
        \right)}{
        |\rho_0|\sqrt{\rho_0^2 - 30m^2gl^3(1 - c_\beta)}
        }
        ,
    \]
    and define \(S(\rho_0) := \int \limits_{0}^{2\pi} b(\sigma,\rho_0)d\sigma\).
    Fix \(\bar{\rho} > \sqrt{60m^2gl^3}\).
    Suppose there exists \(\epsilon > 0\) so that \(S(\rho_0) \geq \epsilon\) for
    all \(\rho_0 \in \, \left]\sqrt{60m^2gl^3}, \bar{\rho}\right]\).
    Then there exists \(I^\star > 0\) such that, for all 
    \(I \in \, ]0,I^\star]\)
    ~\eqref{eqn:acrobot-constraint} injects energy into the acrobot on
    \(\mathcal{O}_2(\bar{\rho})\).
    If instead \(I \in [-I^\star,0[\), the VNHC dissipates energy.
\end{enumerate}
\end{thm}

Notice that \(\mathcal{O}_1 \subset \mathcal{O}_2(\bar{\rho})\), yet
Theorem \ref{thm:acrobot-energy-stabilization} considers these sets separately.
This separation is advantageous because the first result holds for any
\(m\), \(g\), \(l\), and \(\bar{q}_a\). 
That is, regardless of one's choice of acrobot, 
a small enough fixed value \(I > 0\) will enable the acrobot to 
escape any compact subset of \(\mathcal{O}_1\) in finite time.
Therefore, the acrobot will display an oscillatory behaviour whereby the
amplitude of oscillation increases with time.
In so doing, the acrobot's orbit will exit in finite time the closure of
\(\mathcal{O}_1\) as in Figure \ref{fig:acrobot-oscillation-domain}.

In other words, the first result of Theorem
\ref{thm:acrobot-energy-stabilization} states that all acrobots constrained by
~\eqref{eqn:acrobot-constraint} will gain enough energy to begin rotating around
the bar.
In the worst case, the acrobot will at least perform a swing-up routine to reach
the unstable equilibrium at \((q_u,p_u) = (\pi,0)\).

The second result of Theorem
\ref{thm:acrobot-energy-stabilization} pertains to energy injection once the
acrobot has started rotating.
For the acrobot to achieve giants with energy
\(E(0,\bar{\rho})\), it must satisfy the assumption on the integral of
\(b(\beta,\rho_0)\).
The value of this integral depends on the acrobot's physical parameters.
If the assumption holds, orbits will escape compact subsets of
\(\mathcal{O}_2(\bar{\rho})\) in finite time until the acrobot eventually
reaches a momentum of at least \(\bar{\rho}\).
One example of this behaviour is displayed in Figure \ref{fig:acrobot-o2}.
Note that the size of \(I\) will depend on \(\bar{\rho}\).

The proof used in Chapter \ref{ch:vlp} does not
readily transfer to the acrobot.
Proving Theorem \ref{thm:acrobot-energy-stabilization} requires an intelligent
change of coordinates and the use perturbation
theory \cite{khalil_nonlinear}.
We provide the full proof of this theorem in Chapter \ref{sec:acrobot-proof}.
 
\section{Proof of Theorem \ref{thm:acrobot-energy-stabilization}}\label{sec:acrobot-proof}
To make the proof of Theorem \ref{thm:acrobot-energy-stabilization}
as clear as possible, we break it down into the following segments:
\begin{enumerate}
    \item Background on perturbation theory.
    \item Perturbation analysis for oscillations.
    \item Perturbation analysis for rotations.
\end{enumerate}

When \(I = 0\), the constrained acrobot behaves like a single
pendulum with masses at a distance \(l\) and \(2l\) from the pivot 
(Figure \ref{fig:acrobot-I0}) whose energy
\begin{equation}\label{eqn:acrobot-nominal-E}
    E(q_u,p_u) = \frac{p_u^2}{10ml^2} + 3mgl(1 - \cos(q_u))
    ,
\end{equation}
is conserved.
Level sets of \(E\) are ellipses on the \((q_u,p_u)\)-plane
when \(E(q_u,p_u) < E(\pi,0)\), which we call ``oscillations";
and they are open curves when \(E(q_u,p_u) > E(\pi,0)\), which we call
``rotations". 
Examples of these can be seen in Figure \ref{fig:pendulum-level-sets}.

\begin{figure}
    \centering
    \begin{subfigure}[t]{0.45\textwidth}
        \includestandalone[]{images/acrobot_I_zero}
        \caption{A simple pendulum with two masses.}
        \label{fig:acrobot-I0}
    \end{subfigure}
    \hfill
    \begin{subfigure}[t]{0.5\textwidth}
        \includestandalone[]{images/pendulum_level_sets}
        \caption{Level sets of \(E\).
            The blue ellipse is an oscillation; the red curves
            are rotations; and the black ellipse is the homoclinic
            orbit with energy \(E(\pi,0)\).}
        \label{fig:pendulum-level-sets}
    \end{subfigure}
    \caption{Our constrained acrobot is a simple pendulum when \(I = 0\).}
\end{figure}

Using a method developed by \citet{dynamic_vhcs_stabilize_closed_orbits},
we can find a change of coordinates \((q_u,p_u) \to (\alpha, \mu)\) with the
following properties:
\begin{itemize}
    \item The energy of oscillation in \((\alpha,\mu)\) coordinates is 
        uniquely defined by \(\mu\), which remains constant along oscillations
        of the simple pendulum.
    \item \(\alpha\) is a pseudo-angle living in \(\Sone\) which is
        always increasing along oscillations.
\end{itemize}
The coordinates \((\alpha,\mu)\) can be thought of as deformed polar coordinates
adapted to level sets of \(E\) below the value \(E(\pi,0)\), 
where \(\mu\) specifies the level set and \(\alpha\) identifies a point on that
level set.
Once we have these coordinates in hand, we can use perturbation theory to prove
that \(\mu\) increases on \(\mathcal{O}_1\) when \(I > 0\) is small enough.

Likewise, we will find a second set of coordinates \((\beta, \rho)\) for
rotations, where \(\beta \in \Sone\) is always increasing along solutions,
\(\rho\) is constant along rotations of the simple pendulum, and \(\rho\)
increases for the acrobot when \(I > 0\) is small enough.
Using both \((\alpha,\mu)\) and \((\beta,\rho)\), we will show the acrobot is
gaining energy on \(\mathcal{O}_2(\bar{\rho})\).

\subsection{Background on Perturbation Theory}
Nonlinear systems like the acrobot are difficult (or even impossible) to solve
analytically.
Perturbation theory allows one to understand the behaviour of nonlinear systems
by studying a simpler nominal system. 
Solutions of the nonlinear system can often be approximated by taking a Taylor
expansion around a solution of the nominal system.

\citet{khalil_nonlinear} considers a system of the form
\begin{equation}\label{eqn:khalil-setup}
    \begin{cases}
        \dot{x} = f(t,x,I), \\
        x(t_0) = \eta(I),
    \end{cases}
\end{equation}
where 
\(f : [t_0,t_1] \times D \times [-I_0,I_0] \rightarrow \R^n\) is  
smooth\footnote{Khalil actually considers \(f(t,x,\epsilon)\) which is ``sufficiently
    smooth". To more easily connect his theory to the acrobot constraint, we
    assume smoothness of \(f\) and replace \(\epsilon\) with \(I\).}
on a domain \(D \subset \R^n\).
In the context of the acrobot, \(x\) is any pair of coordinates which make
our analysis convenient;
\(I\) is the control parameter of the system;
the function \(f\) is the constrained dynamics in \(x\)-coordinates;
and \(\eta(I) \equiv \eta_0\) is a constant initial condition.

Setting \(I = 0\) we get the nominal system
\begin{equation}\label{eqn:khalil-perturbation-nominal}
    \begin{cases}
        \dot{x}_0 = f(t,x,0) ,\\
        x_0(t_0) = \eta_0 ,
    \end{cases}
\end{equation}
which we need to solve for the explicit solution \(x_0(t,\eta_0)\) on
\([t_0,t_1]\).
We assume this solution is contained in \(D\).

Now consider the solution \(x(t,\eta_0,I)\) to ~\eqref{eqn:khalil-setup}.
Performing a first-order Taylor series expansion at \(I = 0\), we get
\(x(t,\eta_0,I) = x_0(t,\eta_0) + I x_1(t,\eta_0) + R(t,\eta_0,I)\),
where the remainder term \(R(t,\eta_0,I)\) is smooth and \(O(I^2)\), \ie, 
\[
    \lim \limits_{I \to 0} \frac{R(t,\eta_0,I)}{I} = 0
    .
\]
One can recover \(x_1(t,\eta_0)\) as the solution to the scalar
time-varying ODE
\begin{equation}\label{eqn:khalil-perturbation-firstorder}
    \begin{cases}
        \dot{x}_1 = \pdiff{f}{x}(t,x_0(t,\eta_0),0)x_1 + \pdiff{f}{I}(t,x_0(t,\eta_0),0)
        , \\
        x_1(t_0,\eta_0) = 0
        .
    \end{cases}
\end{equation}

We now paraphrase Khalil's Theorem 10.1 \cite{khalil_nonlinear} on the
accuracy of perturbation analysis.
\begin{thm}\label{thm:khalil-perturbation}
    Fix \(t_0, t_1 \in \mathbb{R}\) with \(t_0 < t_1\).
    Suppose \(f : [t_0,t_1] \times D \times [-I_0,I_0] \rightarrow \R^2\) is
    \(C^1\), and that the nominal system
    ~\eqref{eqn:khalil-perturbation-nominal} has a unique solution
    \(x_0(t,\eta_0) \in D\) on \([t_0,t_1]\).
    Then there exists \(I^\star > 0\) such that, for all \(I\) with 
    \(|I| < I^\star\), the solution \(x(t,\eta_0,I)\) to
    ~\eqref{eqn:khalil-setup} satisfies
    \[
        \norm{x(t,\eta_0,I) - \left(x_0(t,\eta_0) + I x_1(t,\eta_0)\right)} \leq k |I^2|
    \]
    for some \(k > 0\).
\end{thm}

Theorem \ref{thm:khalil-perturbation} tells us that we can approximate the
solution of the nonlinear system by the Taylor approximation 
\(x(t,\eta_0,I) \approx x_0(t,\eta_0) + I x_1(t,\eta_0)\).
When \(I\) is small enough, solutions of the nonlinear system and
the Taylor approximation are the same up to order \(I^2\) along compact time
intervals.
This is equivalent to saying that the remainder term \(R(t,\eta_0,I)\) is
bounded, because 
\[
    \norm{x(t,\eta_0,I) - \left(x_0(t,\eta_0) + I x_1(t,\eta_0)\right)} 
    = \norm{R(t,\eta_0,I)} \leq k |I^2|
    .
\]
Since we know that the acrobot behaves like a pendulum at \(I = 0\), we can
use this theory to prove energy injection properties on the acrobot by studying
the simple pendulum.
 
\subsection{Perturbation Analysis for Oscillations}\label{sec:acrobot-proof-o1}
As we have seen, setting \(I = 0\) turns the acrobot into a nominal pendulum. 
This pendulum oscillates whenever the nominal energy
~\eqref{eqn:acrobot-nominal-E} is less than \(E(\pi,0)\). 
Hence, the domain of oscillations for the nominal pendulum is given by
\[
    \mathcal{O}_1 := \left\{ (q_u,p_u) \in \SxR \mid E(q_u,p_u) < E(\pi,0)\right\}
    .
\]
Recall that orbits of a pendulum are level sets of \(E\). 
From Figure \ref{fig:pendulum-level-sets}, we know that oscillation level sets
look like ellipses contained in \(\mathcal{O}_1\).
Since perturbation theory requires us to analyze the nominal pendulum, we will
make our analysis convenient by changing coordinates into a
pseudo-radius \(\mu > 0\) which remains constant on level sets of \(E\), 
along with a pseudo-angle \(\alpha \in \Sone\) satisfying \(\dot{\alpha} > 0\) on
\(\mathcal{O}_1\).

\subsubsection*{Pseudo-Polar Coordinates}

Once again, to define pseudo-polar coordinates, we consider the dynamics of the
pendulum obtained by setting \(I = 0\) in
~\eqref{eqn:acrobot-constrained-dynamics}, resulting in
\begin{equation}\label{eqn:acrobot-nominal-dynamics}
    \begin{cases}
        \dot{q}_u = \frac{p_u}{5ml^2}
        , \\
        \dot{p}_u = -3mgl\sin(q_u)
        . 
    \end{cases}
\end{equation}

The above is a simple mechanical system whose total energy is precisely the
function \(E(q_u,p_u)\) defined in ~\eqref{eqn:acrobot-nominal-E}.
The energy of oscillation can be determined by the intersection
\(\mu\) of an orbit of ~\eqref{eqn:acrobot-nominal-dynamics}
with the \(q_u\)-axis (as in Figure \ref{fig:mu-intersection}).
Since \(q_u \in ]-\pi,\pi[\) on \(\mathcal{O}_1\), we set \(\mu \in ]0,\pi[\).
The transformation we want is therefore a diffeomorphism of the form
\begin{align*}
    T : \mathcal{O}_1 \backslash \{(0,0)\} &\rightarrow \Sone \times \, ]0,\pi[, \\
    (q_u, p_u) &\mapsto (\alpha,\mu)
    .
\end{align*}

\begin{figure}
    \centering
    \includestandalone[width=0.5\textwidth]{images/mu_intersection}
    \caption{The domain \(\mathcal{O}_1\) (blue) where a pendulum oscillates.
    The pseudo-radius \(\mu\) corresponds to the
    intersection of an oscillation (red) with the \(q_u\)-axis.
    The psuedo-angle \(\alpha\) is taken to be clockwise positive because 
    oscillations of a pendulum move clockwise on \(\mathcal{O}_1\).}
    \label{fig:mu-intersection}
\end{figure}

The energy level set corresponding to the intersection \((q_u,p_u) = (\mu,0)\)
is 
\[
    \left\{(q_u,p_u) \in \SxR \mid E(q_u,p_u) = 3mgl(1- \cos(\mu))\right\}
    ,
\]
which gives the relationship
\begin{equation}\label{eqn:oscillation-pu2}
    p_u^2 = 30m^2gl^3\left(\cos(q_u) - \cos(\mu)\right)
    .
\end{equation}
On this level set, \(q_u\) ranges between \([-\mu,\mu]\) and can be uniquely
parameterized by \(q_u = \mu \cos(\alpha)\), where \(\alpha\) is our
desired pseudo-angle.
Substituting this into ~\eqref{eqn:oscillation-pu2}, we get
\[
    p_u^2 = 30m^2gl^3\left(\cos(\mu \cos(\alpha)) - \cos(\mu)\right)
    .
\]
We want to find \(p_u\) as a function of \((\alpha,\mu)\); noting that we can
determine the sign of \(p_u\) from the sign of \(\sin(\alpha)\), we get the
(clockwise) parameterization
\begin{equation}\label{eqn:oscillation-pu}
    p_u = -\sign{\sin(\alpha)} \sqrt{30m^2gl^3 \left(\cos(\mu c_\alpha) - c_\mu\right)}
    ,
\end{equation}
which is smooth for all \(\mu \in ]0,\pi[\).

We have thus found a transformation \(T\inv(\alpha,\mu) = (q_u,p_u)\).
We need the inverse of this map to get our diffeomorphism \(T(q_u,p_u)\).
Notice from ~\eqref{eqn:oscillation-pu2} that
\[
    \cos(\mu) = -\frac{p_u^2}{30m^2gl^3} + \cos(q_u) =: C_\mu(q_u,p_u)
    .
\]
Since \(\mu \in ]0,\pi[\), we can uniquely express \(\mu\) by
\begin{equation*}
    \mu = \arccos\left(C_\mu(q_u,p_u)\right)
    .
\end{equation*}
Next we need to find \(\alpha\). 
Recall that 
\begin{equation}\label{eqn:acrobot-cosalpha-qu}
    \cos(\alpha) = \frac{q_u}{\mu}
    ,
\end{equation}
which means 
\begin{equation}\label{eqn:acrobot-sinalpha-qu}
    \sin(\alpha) = \pm \sqrt{1 - \frac{q_u^2}{\mu^2}}
    .
\end{equation}
Using ~\eqref{eqn:oscillation-pu}, we determine that
\(\sign{\sin(\alpha)} = -\sign{p_u}\).
Putting together ~\eqref{eqn:acrobot-cosalpha-qu} and
~\eqref{eqn:acrobot-sinalpha-qu}, we deduce that
\begin{equation*}
    \alpha = \left.
        \arctan_2\left( -\sign{p_u}\sqrt{1 - \frac{q_u^2}{\mu^2}}, \frac{q_u}{\mu}\right)
        \right|_{\mu = \arccos(C_\mu(q_u,p_u))}
    .
\end{equation*}
Thus, our transformation into \((\alpha,\mu)\)-coordinates is
\begin{align}
        \label{eqn:alpha-qupu}
        \alpha &= \left.
        \arctan_2\left( -\sign{p_u}\sqrt{1 - \frac{q_u^2}{\mu^2}}, \frac{q_u}{\mu}\right)
        \right|_{\mu = \arccos(C_\mu(q_u,p_u))}
        ,\\
        \label{eqn:mu-qupu}
        \mu &= \arccos\left(C_\mu(q_u,p_u)\right)
        .
\end{align}

The acrobot's constrained dynamics in
\((\alpha,\mu)\)-coordinates can be computed by evaluating
\begin{equation*}
\left.\begin{cases}
    \dot{\alpha} &= \pdiff{\alpha(q_u,p_u)}{q_u}\dot{q}_u + 
        \pdiff{\alpha(q_u,p_u)}{p_u}\dot{p}_u
    \\
    \dot{\mu} &= \pdiff{\mu(q_u,p_u)}{q_u}\dot{q_u} +
         \pdiff{\mu(q_u,p_u)}{p_u}\dot{p}_u
    \end{cases}\right|_{(q_u,p_u) = T\inv(\alpha,\mu)}
    .
\end{equation*}
with \((\dot{q}_u,\dot{p}_u)\) given by 
 ~\eqref{eqn:acrobot-constrained-dynamics}.

These dynamics are much too large to write out, so we simply denote them by
\begin{align}
    \label{eqn:acrobot-alpha-dot}
    \dot{\alpha} &= f_\alpha(\alpha,\mu,I)
    ,\\
    \label{eqn:acrobot-mu-dot}
    \dot{\mu} &= f_\mu(\alpha,\mu,I)
    .
\end{align}
MATLAB's symbolic toolbox evaluates the nominal dynamics (when \(I = 0\)) as
\begin{align}
    \label{eqn:acrobot-alpha-dot-nom}
    \dot{\alpha} &= \sqrt{\frac{6g}{5l}} 
        \sqrt{\frac{\cos(\mu\cos(\alpha)) - \cos(\mu)}
            {\mu^2 \sin(\alpha)^2}}
    , \\
    \label{eqn:acrobot-mu-dot-nom}
    \dot{\mu} &= 0
    .
\end{align}
Note that ~\eqref{eqn:acrobot-alpha-dot-nom} has removable singularities at
\(\alpha \in \{0,\pi\}\).
Taking the limit as \(\alpha\) approaches these points yields
\[
    \lim \limits_{\alpha \to 0}\dot{\alpha} 
    = \lim \limits_{\alpha \to \pi} \dot{\alpha}
    = \sqrt{\frac{6g \sin(\mu)}{10l \mu}}
    ,
\]
which is smooth and well-defined for all \(\mu \in ]0,\pi[\).
Taking these removable singularities into account,
one can verify that \(\dot{\alpha} > 0\) for every \(\mu \in ]0,\pi[\).
By continuity of ~\eqref{eqn:acrobot-alpha-dot}, there exists \(I_1 > 0\) small
enough that \(\dot{\alpha}\) remains positive on \(\mathcal{O}_1\) for 
\(I \in [-I_1,I_1]\). 

These pseudo-polar coordinates can be extended smoothly to 
the boundary of \(\mathcal{O}_1\).
Notice that, for each \(\alpha \in \Sone\),
\begin{equation}\label{eqn:alpha-dot-boundary}
    \lim \limits_{\mu \to \pi} \dot{\alpha}
    = \sqrt{\frac{6g}{5l}} 
        \sqrt{\frac{\cos(\pi\cos(\alpha)) + 1}
            {\pi^2 \sin(\alpha)^2}}
    \geq 0
    ,
\end{equation}
with equality if and only if \(\alpha \in \{0,\pi\}\).
In other words, setting \(\mu = \pi\) dictates the behaviour of
the nominal pendulum on the homoclinic orbit with energy \(E(\pi,0)\).
Hence, for small enough \(I\) we have \(\dot{\alpha} \geq 0\) everywhere on the 
closure of \(\mathcal{O}_1\), with equality only at the upright
equilibrium \((q_u,p_u) = (\pi,0)\).
We will use this later to prove that all orbits starting in
\(\mathcal{O}_1\) will eventually begin rotating.

\subsubsection*{Time Scaling}

We have observed in ~\eqref{eqn:acrobot-mu-dot}-\eqref{eqn:acrobot-alpha-dot}
that \(\dot{\alpha} > 0\) on \(\mathcal{O}_1\) for small enough \(I\). 
Hence, we can use \(\alpha\) as our time variable by reparameterizing \(t\) as a
function \(t(\alpha)\).
This allows us to study the evolution of \(\mu\) as a function of \(\alpha\)
rather than a function of time.
Setting \(\hat{\mu}(\alpha) := \mu(t(\alpha))\) yields the dynamics
\[
    \diff{\hat{\mu}}{\alpha} = 
    \diff{\mu}{t} \diff{t}{\alpha}
    .
\] 
This reduces the system \((\dot{\alpha},\dot{\mu})\) into the scalar
time-varying ODE 
\begin{equation}\label{eqn:muhat-dot}
    \begin{cases}
        \diff{\hat{\mu}}{\alpha} 
        = \frac{f_\mu(\tau(\alpha),\mu,I)}{f_\alpha(\tau(\alpha),\mu,I)}
        =: g(\alpha,\mu,I)
        , \\
        \hat{\mu}(0) = \mu_0
        .
    \end{cases}
\end{equation}

\subsubsection*{Perturbation Analysis of the Time Scaled System}

In the spirit of perturbation analysis, we will expand our time-scaled system
\(\hat{\mu}(\alpha,\mu_0,I)\).
We begin by setting \(I = 0\) to find the nominal system
\[
\begin{cases}
    \pdiff{\hat{\mu}_0}{\alpha} = g(\alpha,\hat{\mu}_0,0)
    , \\
    \hat{\mu}_0(0) = \mu_0
   .
\end{cases}
\]
Equations ~\eqref{eqn:acrobot-mu-dot-nom}-\eqref{eqn:acrobot-alpha-dot-nom}
reveal that \(g(\alpha,\mu,0) = 0\),
so the solution to this nominal system is 
\(\hat{\mu}_0(\alpha,\mu_0) \equiv \mu_0\) for all \(\alpha\).

We now take a first-order Taylor approximation of \(\hat{\mu}(\alpha,\mu_0,I)\)
around the nominal \(\hat{\mu}_0(\alpha,\mu_0)\),
\begin{equation}\label{eqn:acrobot-muhat-approx}
    \hat{\mu}(\alpha,\mu_0,I) = \hat{\mu}_0(\alpha,\mu_0) + I
    \hat{\mu}_1(\alpha,\mu_0)
    + R(\alpha,\mu_0,I)
    .
\end{equation}

By ~\eqref{eqn:khalil-perturbation-firstorder}, we know the function
\(\hat{\mu}_1(\alpha,\mu_0)\) is the solution to the linear time-varying scalar
system
\begin{equation}\label{eqn:acrobot-mu1-dot}
    \begin{cases}
        \pdiff{\hat{\mu}_1}{\alpha} = 
        \pdiff{g}{\hat{\mu}}(\alpha,\mu_0,0)\hat{\mu}_1 + \pdiff{g}{I}(\alpha, \mu_0, 0)
        , \\
        \hat{\mu}_1(0) = 0
        .
    \end{cases}
\end{equation}
These dynamics are difficult to compute by hand, so we resort to MATLAB symbolic
computation to reveal that
\begin{align*}
    \pdiff{g}{\hat{\mu}}(\alpha,\mu_0,0) &= 0
    , \\
    \pdiff{g}{I}(\alpha, \mu_0, 0) &= K a(\alpha,\mu_0)
    ,
\end{align*}
where 
\begin{align*}
    K &:= \frac{\bar{q}_a \sqrt{30m^2g l^3}}{15}
    , \\
    a(\alpha,\mu_0) &:= \frac{
        \mu_0 |\sin(\alpha)| \left(
        5 c_{\mu_0} \cos(\mu_0 c_\alpha) - 8 \cos(\mu_0c_\alpha)^2 + 3
    \right)
    }{
    \sin(\mu_0)\sqrt{\cos(\mu_0c_\alpha) - c_{\mu_0}}
    }
    .
\end{align*}
Equation ~\eqref{eqn:acrobot-mu1-dot} can be solved by quadrature, giving
\[
    \hat{\mu}_1(\alpha,\mu_0) =
    K \int \limits_0^\alpha a(\sigma,\mu_0)d\sigma
    .
\]
Therefore, the first-order Taylor approximation of \(\hat{\mu}(\alpha,\mu_0,I)\) is
\begin{equation}\label{eqn:acrobot-muhat-expanded}
    \hat{\mu}(\alpha,\mu_0,I)
    = \mu_0 + I K \int \limits_0^\alpha a(\sigma,\mu_0)d\sigma +
    R(\alpha,\mu_0,I)
    .
\end{equation}

\subsubsection{Poincar\'{e} Analysis}

Since \(\dot{\alpha} > 0\), there is a well-defined Poincar\'{e} map 
describing how the pseudo-radius \(\mu\) changes each time the orbit of \((q_u(t),p_u(t))\)
intersects the \(q_u\)-axis, \ie, when \(\alpha = 0\) or \(\alpha = \pi\).
Using our time-scaled system ~\eqref{eqn:muhat-dot}, we can define this
Poincar\'{e} map as the change in \(\hat{\mu}\) when \(\alpha\) increases by
\(\pi\):
\begin{align*}
    P_\mathcal{O} : \,]0,\pi[ &\rightarrow \R_+,
    \\
    \mu_0 &\mapsto \hat{\mu}(\pi,\mu_0,I)
    .
\end{align*}
The Poincar\'{e} map expands into
\[
    P_\mathcal{O}(\mu_0) = \mu_0 + I K \int\limits_0^\pi a(\sigma,\mu_0)d\sigma
    + R(\pi,\mu_0,I)
    .
\]
Let us define
\begin{equation}\label{eqn:acrobot-Qmu}
    Q(\mu_0) := \int \limits_0^\pi a(\sigma,\mu_0)d\sigma
    .
\end{equation}
Note that \(K\) is a positive constant which contains the acrobot's physical
parameters \(m\), \(g\), \(l\), and \(\bar{q}_a\), 
while \(a(\alpha,\mu)\) is adimensional. 
This means \(Q(\mu_0)\) is identical for every acrobot.
If \(Q(\mu_0)\) is strictly positive, then \(\hat{\mu}_1(\pi,\mu_0)\) will be
positive for any acrobot.

We numerically compute \(Q(\mu_0)\) for \(\mu_0 \in [10^{-10}, \pi - 10^{-3}]\)
in Figure \ref{fig:acrobot-Q}. 
We see that it is strictly positive and monotonically increasing, with
an asymptote at \(\mu_0 = \pi\). 
Simulations with smaller \(\mu_0\) result in an infinite integral error due
to a division by zero, so we believe that \(Q(\mu_0)\) is in fact positive for
all \(\mu_0\).

\begin{figure}
    \centering
    \includegraphics[width=0.8\textwidth]{images/Qmu.png}
    \caption{The plot of \(Q(\mu_0)\).}
    \label{fig:acrobot-Q}
\end{figure}

\subsubsection*{Energy Gain on \(\mathcal{O}_1\)}

The Poincar\'{e} map \(P_\mathcal{O}\) allows us understand the evolution of
\(\hat{\mu}(\alpha,\mu_0,I)\) by studying the evolution of the discrete time
system
\begin{equation}\label{eqn:muhat-discrete}
        \mu_{n+1} := P_\mathcal{O}(\mu_n) 
        = \mu_n + I K Q(\mu_n) + R(\pi,\mu_n,I)
        ,
\end{equation}
with initial condition \(\mu_0\).
Here, \(\mu_n\) represents the distance along the
\(q_u\)-axis when an orbit of the constrained dynamics intersects the
\(q_u\)-axis for the \(n\)th time, assuming the orbit was initialized at
\((q_u,p_u) = (\mu_0,0)\).

Proving the constrained dynamics gain energy on \(\mathcal{O}_1\) is equivalent to
showing \(\mu_n\) eventually reaches \(\pi\).
That is, we want to find \(I^\star \in\, ]0, I_1]\) where, for all 
\(\delta > 0\) and \(\mu_0 \in ]0,\pi-\delta]\), there exists \(N > 0\) so that
for all \(n \geq N\), \(\mu_n \notin [0,\pi-\delta]\).

A sufficient conditions for this characterization is to prove that 
\(P_\mathcal{O}(\mu_0) \geq \mu_0 + \gamma\) for some \(\gamma > 0\).
We will perform our analysis in two sections.
First, we will show the origin is a repeller of ~\eqref{eqn:muhat-discrete}.
This implies that orbits near the origin flow away from it.
Then, we will use the sufficient condition above to guarantee all orbits reach
\(\mu = \pi\).

Linearizing ~\eqref{eqn:muhat-discrete} at \(\mu_0 = 0\) yields
\[
    P^\prime_\mathcal{O}(0) = 1 + I K Q^\prime(0) +
    R^\prime(\pi,0,I)
    ,
\]
where prime denotes differentiation with respect to \(\mu\).
If there is some value of \(I \in \,]0,I_1]\) for which this term is greater than
\(1\), then \(0\) is a repeller of the discrete time system.
To show such a value of \(I\) exists, we first compute 
\[
    Q^\prime(0) = \int \limits_0^\pi \pdiff{a}{\mu}(\sigma,0) d\sigma
    .
\]
Numerical computations reveal that 
\[
    \lim \limits_{\mu_0 \to 0^+}
    \pdiff{a}{\mu}(\alpha,\mu_0) 
    = -\frac{\sqrt{2}}{2} \left(11\sin(\alpha)^2 - 6\right)
    ,
\]
which means 
\[
    Q^\prime(0)  = \frac{\pi}{2\sqrt{2}} > 0
    .
\]
Since the remainder term \(R(\pi,0,I)\) is \(O(I^2)\), 
its partial derivative \(R^\prime(\pi,0,I)\) is also \(O(I^2)\).
Hence, it can be written in the form
\(I^2 \tilde{R}(I)\) where \(\tilde{R}(I)\) is smooth and
zero at \(I = 0\).
Thus, there exists \(I_2 \in \,]0,I_1]\) such that
\[
    I_2 K Q^\prime(0) + (I_2)^2 \tilde{R}(I) > 0
    .
\]
Hence,
\[
    P^\prime_\mathcal{O}(0) \geq 1 + IKQ^\prime(0) + I^2 \tilde{R}(I) 
    > 1
    ,
\] 
for all \(I \in \,]0, I_2]\).

We have shown \(0\) is a repeller of the discrete time system, which means
there exists some (unknown) \(\epsilon > 0\) where the interval
\(]0,\epsilon[\) is negatively invariant for ~\eqref{eqn:muhat-discrete}.
What's more, all solutions starting in this interval will flow towards the value
\(\mu = \epsilon\).

To complete the proof, recall that \(R(\pi,\mu_0,I)\) is smooth in all its
parameters as well as being \(O(I^2)\).
It is therefore bounded below on the compact set \(\mu_0 \in [\epsilon,\pi]\)
by some value \(\underbar{R}(I)\). 
Theorem \ref{thm:khalil-perturbation} asserts that there exists some
\(I_3 > 0\) and \(r > 0\) so that, for all \(I \in [-I_3, I_3]\),
\[
    R(\pi,\mu_0,I) \geq \underbar{R}(I) > -I^2 r
    .
\]
Note that we can assume \(I_3 \leq I_2\) without loss of generality.
Furthermore, \(Q(\mu_0)\) is a strictly increasing function, so
for \(\mu_0 \in [\epsilon,\pi[\),
\[
    P_\mathcal{O}(\mu_0) > \mu_0 + I K Q(\epsilon) - I^2 r
    .
\]
Picking a small \(\gamma > 0\) and choosing 
\(I^\star \in\, ]0, I_3]\) so that
\[
    I^\star K Q(\epsilon) - (I^\star)^2 r \geq \gamma > 0
    ,
\]
means that \( P_\mathcal{O}(\mu_0) \geq \mu_0 + \gamma\)
for all \(\mu_0 \in [\epsilon,\pi[\).

Looking at ~\eqref{eqn:muhat-discrete} we find that
\(\mu_{n+1} \geq \mu_n + \gamma\). 
This implies that all solutions of the discrete time system will flow towards
\(\mu = \pi\).

We conclude that, for all \(m\), \(g\), \(l\), \(\bar{q}_a\), 
there exists \(I > 0\) small enough that the constraint 
~\eqref{eqn:acrobot-constraint} is injecting energy on \(\mathcal{O}_1\).

By the same arguments presented in this section, the Poincar\'{e} section
satisfies
\[
    P_\mathcal{O}(\mu_0) \leq \mu_0 - \gamma
    ,
\] 
when \(I < 0\). 
In this case, the constraint dissipates energy on \(\mathcal{O}_1\).

\subsubsection*{Energy Gain on \(\bar{\mathcal{O}}_1\)}
Before moving on to the rotation analysis, we must first confirm that any
acrobot constrained by ~\eqref{eqn:acrobot-constraint} will eventually start
rotating.
The definition of energy gain states that all orbits initialized in
\(\mathcal{O}_1\) will escape compact subsets of \(\mathcal{O}_1\) in finite time. 
This is not enough to prove that all orbits will escape the boundary of
\(\mathcal{O}_1\). 

Indeed, let \(E_\pi\) be the level set with energy \(E(\pi,0)\), 
which forms the boundary of \(\mathcal{O}_1\). 
Define \(\bar{\mathcal{O}}_1 = \mathcal{O}_1 \cup E_\pi\) to be the closure of
\(\mathcal{O}_1\).
An orbit of the acrobot will only begin rotating if it escapes 
\(\bar{\mathcal{O}}_1\) by crossing through \(E_\pi\).
It is possible for orbits starting in \(\mathcal{O}_1\) to always approach
\(E_\pi\) without ever crossing into the rotation zone.
We will prove this does not happen in general.

To begin, let us analyze the upright equilibrium.
Taking the Jacobian of the constrained dynamics
~\eqref{eqn:acrobot-constrained-dynamics} at \((q_u,p_u) = (\pi,0)\) yields
\[
    J = \begin{bmatrix}
        -\frac{6mgl\bar{q}_aI}{5} & \frac{1 - 2m^2gl^3\bar{q}_a^2 I^2}{5ml^2} \\
        3mgl & mgl\bar{q}_aI
    \end{bmatrix}
    ,
\]
which has characteristic polynomial
\[
    \det\left(\lambda \Id{2} - J\right)
    = \lambda^2 + \frac{mgl\bar{q}_a I}{5} \lambda - 3g
    .
\]
By Descartes' rule of signs, this polynomial has one root with positive real
part. 
The equilibrium \((\pi,0)\) is therefore unstable, so the stable manifold \(\Pi^+\) of
initial conditions converging to \((\pi,0)\) is one-dimensional, and hence is
of measure zero in \(\SxR\).

Using \(x(t) := (q_u(t),p_u(t))\) as shorthand, let
\(x(0) \in \mathcal{O}_1\) be a nonzero initial condition of the acrobot.
Suppose the orbit \(x(\R)\) is confined within 
\(\bar{\mathcal{O}}_1\), and does not exit through \(E_\pi\) into the rotation
zone.
Since \(\bar{\mathcal{O}}_1\) is compact, the Birkhoff Theorem \cite{birkhoff}
implies:
\begin{itemize}
    \item The positive limit set \(L_+\) of \(x(t)\) is non-empty, compact, and
        invariant.
    \item The solution \(x(t)\) asymptotically tends to \(L_+\).
\end{itemize}
Since the acrobot gains energy on \(\mathcal{O}_1\), the positive limit set of
\(x(t)\) must be the largest invariant subset of \(E_\pi\). 
From our discussion on pseudo-polar coordinates (in particular the extension
~\eqref{eqn:alpha-dot-boundary} of \(\dot{\alpha}\) to \(\bar{\mathcal{O}}_1\)),
we know that \(\dot{\alpha} \geq 0\) on \(E_\pi\), with equality if and only if 
\(\alpha \in \{0,\pi\}\).
There are two possibilities: either \(E_\pi\) itself is invariant, or the
largest invariant subset of \(E_\pi\) is \(\{(\pi,0)\}\).
To rule out the first possibility, take the derivative of \(E(q_u,p_u)\) at the
\(p_u\)-axis to get
\[
    \dot{E}(0,p_u) = -\frac{g}{5l} \sin(q_a)p_u
    .
\]
This is non-zero everywhere except at the origin, which means it is non-zero on
\(E_\pi\). 
This means \(E_\pi\) is not invariant, so the positive limit set of \(x(t)\)
must be the set \(\{(\pi,0)\}\).
Hence, \(x(t)\) converges to \((\pi,0)\), which means \(x(0) \in \Pi^+\).
Since we know that \(\Pi^+\) is a set of measure zero in \(\SxR\), almost every
orbit initialized in \(\mathcal{O}_1\) must escape through the boundary
\(E_\pi\) into the rotation domain.

Escaping once into the rotation domain does not guarantee the acrobot
orbit will not return to \(\mathcal{O}_1\).
Indeed, the orbit could escape through \(E_\pi\) and return to \(\mathcal{O}_1\)
several times.
However, the orbit must eventually remain outside the closure of 
\(\mathcal{O}_1\). 
If not, the positive limit set would once again be
\(E_\pi\); by the same argument as above, the orbit must converge to 
\((\pi,0)\).

We conclude that all orbits beginning in \(\mathcal{O}_1\) will, in finite time,
escape the closure of \(\mathcal{O}_1\) and remain in the rotation domain
forever after.

\subsubsection*{Summary}

We have proven the first part of Theorem \ref{thm:acrobot-energy-stabilization},
which claims there exists a control value \(I^\star > 0\) such that, for 
\(I \in \, ]0,I^\star]\), ~\eqref{eqn:acrobot-constraint} injects energy into
the acrobot on \(\mathcal{O}_1\).
Enough energy is injected that orbits will exit the closure of \(\mathcal{O}_1\)
and enter the rotation domain.

Here is a summary of the proof:
\begin{enumerate}
    \item We found pseudo-polar coordinates \((\alpha,\mu)\) adapted to level
        sets of the nominal pendulum's mechanical energy on \(\mathcal{O}_1\),
        where \(\alpha \in \Sone\) is a pseudo-angle and \(\mu \in \, ]0,\pi[\)
        is a pseudo-radius.
    \item We showed there exists a value \(I_1\) small enough where 
        \(\dot{\alpha} > 0\) everywhere on \(\mathcal{O}_1\).
    \item Using \(\alpha\) as a time variable, we found the time-scaled
        pseudo-radius \(\hat{\mu}\) and expanded it into
        ~\eqref{eqn:acrobot-muhat-approx} using perturbation theory.
    \item Taking this expanded solution, we defined and expanded the
        Poincar\'{e} map \(P_\mathcal{O}\) to analyze the discrete-time system
        ~\eqref{eqn:muhat-discrete}.
    \item We found a value \(I_2 \leq I_1\) making the origin a repeller of 
        ~\eqref{eqn:muhat-discrete}, and a value \(I_3 \leq I_2\) which drives
        \(\mu_n\) towards \(\mu = \pi\).
    \item We showed that the upright equilibrium \((\pi,0)\) is unstable, and
        used this fact to prove that almost every orbit crosses through
        \(E_\pi\), thereby exiting the closure of \(\mathcal{O}_1\) in finite
        time.
\end{enumerate}
Choosing \(I \in \,]0,I_3]\) guarantees energy injection on
\(\mathcal{O}_1\) because all orbits approach the pseudo-radius \(\mu = \pi\).
In fact, almost all orbits escape \(\bar{\mathcal{O}}_1\).
Likewise, choosing \(I \in [-I_3,0[\) guarantees energy dissipation on
\(\mathcal{O}_1\).

% Empty proof environment to put QED square on the right of the page
\begin{proof}[\unskip\nopunct]
\end{proof}

\subsection{Perturbation Analysis for Rotations}
Figure \ref{fig:pendulum-level-sets} reminds us that rotations of the 
nominal pendulum obtained by setting \(I = 0\) look like open curves on some
rotation domain \(\mathcal{R} \subset \SxR\), where
\[
    \mathcal{R} := \left\{ (q_u,p_u) \in \SxR \mid E(q_u,p_u) > E(\pi,0)\right\}
    .
\]
Performing a similar process to what we did for oscillations,
we wish to find a new set of coordinates \((\beta,\rho)\)
where the pseudo-radius \(\rho \in \mathbb{R}\) remains constant on level sets
of \(E\) and the pseudo-angle \(\beta \in \Sone\) is always increasing on
\(\mathcal{R}\).

\subsubsection*{Pseudo-Polar Coordinates}

In the oscillation region \(\mathcal{O}_1\) we showed that the energy of the
nominal pendulum \eqref{eqn:acrobot-nominal-dynamics} is uniquely determined by
an orbit's intersection point \(\mu\) with the \(q_u\)-axis.
Likewise, on \(\mathcal{R}\) the energy is determined by the orbit's intersection
point \(\rho\) with the \(p_u\)-axis, as in Figure \ref{fig:rho-intersection}.

\begin{figure}
    \centering
    \includestandalone[]{images/rho_intersection}
    \caption{The domain \(\mathcal{R}\) (blue) where a pendulum rotates. The
        pseudo-radius \(\rho\) corresponds to the intersection of an orbit of
        rotation with the \(p_u\)-axis. The pseudo-angle \(\beta\) selects a
        point on the rotation.}
    \label{fig:rho-intersection}
\end{figure}

Note that the boundary of \(\mathcal{R}\) intersects the \(p_u\) axis when 
\(p_u^2 = 60m^2 g l^3\). 
This boundary is the homoclinic orbit for the
upright equilibrium of the pendulum, which is the level set with energy
\(E(\pi,0)\).
Hence, we must have \(\rho > \sqrt{60 m^2 g l^3}\), if a
rotation has momentum \(p_u > 0\), 
and \(\rho < -\sqrt{60 m^2 g l^3}\) if it has momentum \(p_u < 0\).
The energy level set associated with \(\rho\) is 
\[
    \left\{(q_u,p_u) \in \SxR \mid E(q_u,p_u) = \frac{\rho^2}{10ml^2}\right\}
    ,
\]
which gives the relationship
\begin{equation}\label{eqn:rotation-pu2}
    \frac{p_u^2}{10m l^2} + 30mgl(1 - c_u) = \frac{\rho^2}{10 ml^2}
    .
\end{equation}
On this level set, \(q_u\) takes all values on \(\Sone\), so our angle of
rotation is uniquely parameterized by \(\beta = q_u\).
Since \(\rho\) does not change sign along the rotation, we have the smooth
relationship
\begin{align}\label{eqn:rotation-Tinv}
    q_u &= \beta
    , \\
    p_u &= \sign{\rho}\sqrt{\rho^2 - 30 m^2 g l^3\left(1 - c_\beta \right)}
    .
\end{align}

Inverting this relationship gives our pseudo-polar coordinates
\begin{align}\label{eqn:rotation-T}
    \beta &= q_u
    , \\
    \rho &= \sign{p_u}\sqrt{p_u^2 + 30 m^2 g l^3 \left(1 - c_u \right)}
    .
\end{align}

Computing the acrobot's constrained dynamics in \((\beta,\rho)\)-coordinates and
setting \(I = 0\) yields the dynamics of the nominal pendulum.
MATLAB evaluates those dynamics as
\begin{align}\label{eqn:acrobot-rot-mu-dot-nom}
    \dot{\rho} &= 0
    , \\
    \label{eqn:acrobot-rot-alpha-dot-nom}
    \dot{\beta} &=  \sign{\rho} 
    \frac{\sqrt{\rho^2 - 30m^2gl^3(1 - c_\beta)}}{5ml^2}
    .
\end{align}
As expected, \(\dot{\beta}\) does not change sign on \(\mathcal{R}\) because the
orbits always flow clockwise.
If \(\rho > 0\), the rotation curve goes from \(\beta = -\pi\) to 
\(\beta = \pi\); 
if \(\rho < 0\), it goes from \(\beta = \pi\) to \(\beta = -\pi\).
By continuity of ~\eqref{eqn:rotation-T}, there exists \(I_1 > 0\) small enough
that \(\dot{\beta}\) is non-zero and does not change sign on \(\mathcal{R}\) for
all \(I \in [-I_1,I_1]\).

\subsubsection*{Time Scaling}

Using \(\beta\) as our new time variable (via a time reparameterization
\(t = t(\beta)\)) produces the time-scaled pseudo-radius
\(\hat{\rho}(\beta) := \rho(t(\beta))\).
This reduces the system \((\dot{\beta},\dot{\rho})\) into the scalar time-varying
ODE
\begin{equation}\label{eqn:rhohat-dot}
    \begin{cases}
        \diff{\hat{\rho}}{\beta} = \frac{\dot{\rho}}{\dot{\beta}}
        , \\
        \hat{\rho}(0) = \rho_0
        .
    \end{cases}
\end{equation}

\subsubsection*{Perturbation Analysis of the Time Scaled System}

In the spirit of perturbation analysis, we expand the time-scaled system 
\(\hat{\rho}(\beta,\rho_0,I)\). 
From ~\eqref{eqn:khalil-perturbation-nominal} we know the nominal system at
\(I = 0\) is
\[
    \begin{cases} 
        \diff{\hat{\rho}_0}{\beta} = 0
        , \\
        \hat{\rho}_0(0) = \rho_0
        ,
    \end{cases}
\]
which has solution \(\hat{\rho}_0(\beta,\rho_0) \equiv \rho_0\).

We take a first-order Taylor approximation of \(\hat{\rho}(\beta,\rho_0,I)\)
around \(\hat{\rho}_0(\beta,\rho_0)\) to get
\begin{equation}\label{eqn:acrobot-rhohat-approx}
    \hat{\rho}(\beta,\rho_0,I) = \hat{\rho}_0(\beta,\rho_0) +
    I\hat{\rho}_1(\beta,\rho_0) + R(\beta,\rho_0,I)
    ,
\end{equation}
where \(R(\beta,\rho_0,I)\) is smooth and \(O(I^2)\).

Using MATLAB's symbolic toolbox and 
~\eqref{eqn:khalil-perturbation-firstorder}, 
we discover that \(\hat{\rho}_1(\beta,\rho_0)\) is the solution to the linear
time-varying scalar ODE
\begin{equation}\label{eqn:acrobot-rho1-dot}
  \begin{cases}
      \diff{\hat{\rho}_1}{\beta} =
    \frac{5m^2 g l^3 \bar{q}_a \left(
        m^2gl^3\left(18s_\beta^2 + 30c_\beta(1 - c_\beta)\right)
        - c_\beta\rho_0^2
    \right)}{
    |\rho_0|\sqrt{\rho_0^2 - 30m^2gl^3(1 - c_\beta)}
    }
    =: b(\beta,\rho_0)
     , \\
     \hat{\rho}_1(0) = 0
     ,
 \end{cases}
\end{equation}
whose solution is
\[
    \hat{\rho}_1(\beta,\rho_0) = \int \limits_0^\beta b(\sigma,\rho_0)d\sigma
    .
\]

\subsubsection*{Poincar\'{e} Analysis}

Suppose we initialize the acrobot at \((\beta,\rho) = (0,\rho_0)\).
One full rotation amounts to \(\beta\) traversing \(2\pi\) rad in a clockwise
direction; that is, when \(\beta\) goes from \(0\) to \(\sign{\rho_0}2\pi\).
Since \(\dot{\beta}\) does not change sign on \(\mathcal{R}\),
there is a well-defined Poincar\'{e} map describing how the pseudo-radius
\(\rho\) changes each time the orbit \((q(t),p(t))\) hits the \(p_u\) axis, \ie,
every time \(\beta\) changes by \(2\pi\).
We define this Poincar\'{e} map as
\[
    P_\mathcal{R}(\rho_0) := \hat{\rho}\left(\sign{\rho_0}2\pi,\rho_0,I\right)
    ,
\]
which expands into
\[
    P_\mathcal{R}(\rho_0) = \rho_0 + I \hat{\rho}_1(\sign{\rho_0}2\pi, \rho_0)
    + R(\sign{\rho_0}2\pi,\rho_0,I)
    .
\]

Let us pause here for a moment to remember what we are trying to accomplish.
The second part of Theorem \ref{thm:acrobot-energy-stabilization} states that,
under suitable conditions on the integral of \(b(\beta,\rho_0)\), 
the acrobot will gain energy on 
\[
    \mathcal{O}_2(\bar{\rho}) := \left\{ (q_u,p_u) \in \SxR
    \mid E(q_u,p_u) < E(0,\bar{\rho}) \right\}
    ,
\]
for some \(\bar{\rho} > \sqrt{60m^2gl^3}\).
Recall from Section \ref{sec:acrobot-proof-o1} that any acrobot initialized in
\(\mathcal{O}_1\) will eventually enter the rotation region \(\mathcal{R}\).
What remains, then, is to prove that the acrobot's orbits will also gain
energy on
\[
    \mathcal{R}_{\bar{\rho}} := \mathcal{R} \cap \mathcal{O}_2(\bar{\rho})
    .
\]
This can be proven by analyzing the Poincar\'{e} map \(P_\mathcal{R}(\rho_0)\).
Looking at Figure \ref{fig:acrobot-rhobar-regions}, it is clear that
\(P_\mathcal{R}\) has the domain
\[
    D := \left[-\bar{\rho},\sqrt{60m^2gl^3}\right[ \, 
    \cup \, 
    \left]\sqrt{60m^2gl^3},\bar{\rho}\right]
    ,
\]
which is the intersection of \(\mathcal{R}_{\bar{\rho}}\) with the \(p_u\)-axis.
We label the part of \(D\) contained in the negative \(p_u\)-axis as
\[
    D^- :=\left[-\bar{\rho},\sqrt{60m^2gl^3}\right[
    ,
\]
and the part contained in the positive \(p_u\)-axis as
\[
    D^+ := \left]\sqrt{60m^2gl^3},\bar{\rho}\right]
    ,
\]
so that \(D = D^- \cup D^+\).
If \(P_\mathcal{R}(\rho_0)\) is always further from the origin than 
\(\rho_0\) (\ie~ it is ``expanding" on \(D\)), the acrobot will be gaining energy
on \(\mathcal{O}_2(\bar{\rho})\).

\begin{figure}
    \centering
    \includestandalone[width=0.5\textwidth]{images/acrobot_rhobar_regions}
    \caption{The region \(\mathcal{R}_{\bar{\rho}}\) of rotations in
        \(\mathcal{O}_2(\bar{\rho})\) is coloured
        in blue, while the domain \(D = D^- \cup D^+\) of the Poincar\'{e} map 
        is coloured in red.}
    \label{fig:acrobot-rhobar-regions}
\end{figure}

Mirroring the Poincar\'{e} analysis for oscillations, let us find a
function \(S(\rho_0)\) which is positive on \(D\).
This will be the rotational analogue to \(Q(\mu_0)\) from
~\eqref{eqn:acrobot-Qmu}.

Notice that the function \(b(\beta,\rho_0)\) is even and
\(2\pi\)-periodic in \(\beta\), which implies that
\begin{align*}
    \hat{\rho}_1(\sign{\rho_0}2\pi, \rho_0) &=
    \int \limits_0^{\sign{\rho_0}2\pi} b(\sigma,\rho_0)d\sigma
    , \\
     &=\sign{\rho_0}\int\limits_0^{2\pi}b(\sigma,\rho_0)d\sigma
     , \\
     &= \sign{\rho_0} \hat{\rho}_1(2\pi,\rho_0)
     .
\end{align*}
Defining \(S(\rho_0) := \hat{\rho}_1(2\pi,\rho_0)\),
the Poincar\'{e} map becomes
\begin{equation}\label{eqn:acrobot-poincare-r}
    P_\mathcal{R}(\rho_0) = \rho_0 + I\sign{\rho_0}S(\rho_0) 
    + R(\sign{\rho_0}2\pi,\rho_0,I)
    .
\end{equation}

Proving expansion of \(P_\mathcal{R}(\rho_0)\) becomes easier if
\(S(\rho_0)\) is positive on \(D\).
Recall the assumption that there exists
\(\epsilon > 0\) where \(S(\rho_0) \geq \epsilon\) for all 
\(\rho_0 \in D^+\).
Since \(b(\beta,\rho_0)\) is even in \(\rho_0\),
its integral \(S(\rho_0)\) is also even in \(\rho_0\),
which means \(S(\rho_0) \geq \epsilon\) for every \(\rho_0 \in D\).

\subsubsection*{Energy Gain on \(\mathcal{O}_2(\bar{\rho})\)}
The Poincar\'{e} section \(P_\mathcal{R}\) allows us understand the evolution of
\(\hat{\rho}(\beta,\rho_0,I)\) by studying the evolution of the discrete time
ODE
\begin{equation}\label{eqn:rhohat-discrete}
    \rho_{n+1} := P_\mathcal{R}(\rho_n) 
    = \rho_n + I\sign{\rho_n}S(\rho_n) + 
        R(\sign{\rho_n}2\pi,\rho_n,I)
    .
\end{equation}
Here, \(\rho_n\) represents the distance along the
\(p_u\)-axis when an orbit of the constrained dynamics intersects the
\(p_u\)-axis for the \(n\)th time, assuming the orbit was initialized at
\((q_u,p_u) = (0,\rho_0)\).

To prove energy gain on \(\mathcal{R}_{\bar{\rho}}\), it is
enough to prove that \(\abs{\rho_n}\) eventually reaches \(\bar{\rho}\).
This is true if there exists some \(\gamma > 0\)
whereby \(\abs{P_\mathcal{R}(\rho_0)} \geq \abs{\rho_0} + \gamma\) for all 
\(\rho_0 \in D\)

We begin with \(\rho_0 \in D^+\).
Recall that \(R(\beta,\rho_0,I)\) is \(O(I^2)\) and smooth in all
its parameters.
Therefore, it is bounded below on the compact set
\(\rho_0 \in \left[\sqrt{60m^2gl^3},\bar{\rho}\right]\) by some value
\(\underbar{R}(I)\).
Theorem \ref{thm:khalil-perturbation} asserts that there exists some 
\(I_2 > 0\) and \(r > 0\) so that, for all \(I \in [-I_2,I_2]\),
\[
    R(2\pi,\rho_0,I) \geq \underbar{R}(I) > -I^2r
    .
\]
Since \(S(\rho_0) \geq \epsilon\), we find that
\[
    P_\mathcal{R}(\rho_0) > \rho_0 + I\epsilon -I^2r
    ,
\]
for all \(\rho_0 \in D^+\).
Choosing \(I_3 \in \,]0,I_2]\) so that
\[
    I_3\epsilon - (I_3)^2 r \geq \gamma
    ,
\] 
makes \(P_\mathcal{R}(\rho_0) \geq \rho_0 + \gamma\) for all 
\(\rho_0 \in D^+\)

Now take \(\rho_0 \in D^-\), where we wish to show that 
\(P_\mathcal{R}(\rho_0) \leq \rho_0 - \gamma\).
The remainder \(R(\beta,\rho_0,I)\) is bounded above on the compact set 
\(\rho_0 \in \left[-\bar{\rho},-\sqrt{60m^2gl^3}\right]\), and 
Theorem \ref{thm:khalil-perturbation} asserts that there exists some 
\(I_4 > 0\) and \(r^\prime > 0\) so that, for all \(I \in [-I_4,I_4]\), 
\[
    R(-2\pi,\rho_0,I) < I^2 r^\prime
    .
\]
Hence,
\begin{align*}
    P_\mathcal{R}(\rho_0)
    &= \rho_0 - IS(\rho_0) + R(-2\pi,\rho_0,I)
    , \\
    &\leq \rho_0 - I\epsilon + I^2r^\prime
    .
\end{align*}
Choosing \(I_5 \in ]0,I_4]\) so that 
\[
    I_5 \epsilon - (I_5)^2r^\prime \geq \gamma
    ,
\] 
makes \(P_\mathcal{R}(\rho_0) \leq \rho_0 - \gamma\) for all 
\(\rho_0 \in D^-\).

To connect the two subregions of \(D\) together, choose
\(I^\star \in\, ]0,\min\left\{I_3,I_5\right\}]\).
Then \(\abs{P_\mathcal{R}(\rho_0)} \geq \abs{\rho_0} + \gamma\) for all
\(\rho_0 \in D\), so the constrained dynamics are gaining energy on 
\(\mathcal{R}_{\bar{\rho}}\).

By the same arguments presented in this section, the Poincar\'{e} section
satisfies 
\[
    \abs{P_\mathcal{R}(\rho_0)} \leq \abs{\rho_0} - \gamma
    ,
\]
when \(I < 0\).

\subsubsection*{Conclusion}
We have proven the second part of Theorem
\ref{thm:acrobot-energy-stabilization}: if the acrobot satisfies the assumptions
on \(S(\rho_0)\), there exists \(I > 0\) small enough that the constraint
~\eqref{eqn:acrobot-constraint} injects energy into the acrobot on
\(\mathcal{O}_2(\bar{\rho})\).

Here is a summary of the proof:
\begin{enumerate}
    \item We found pseudo-polar coordinates \((\beta,\rho)\) adapted to level
        sets of the nominal pendulum's mechanical energy on \(\mathcal{R}\),
        where \(\beta \in \Sone\) is a pseudo-angle and
        \(\rho \in D\) is a pseudo-radius.
    \item We showed there exists a value \(I_1\) small enough where
        \(\dot{\beta} > 0\) everywhere on \(\mathcal{R}\).
    \item Using \(\beta\) as a time variable, we found the evolution of the
        time-scaled pseudo-radius \(\hat{\rho}\) and expanded it into
        ~\eqref{eqn:acrobot-rhohat-approx} using perturbation theory.
    \item Taking this expanded solution, we defined and expanded the
        Poincar\'{e} map \(P_\mathcal{R}\) on the domain 
        \(D = D^- \cup D^+\) which shows how the acrobot's orbits behave on
        \(\mathcal{R}_{\bar{\rho}}\).
    \item Using the Poincar\'{e} map to analyze the discrete-time system
        ~\eqref{eqn:rhohat-discrete}, we found values \(I_2,I_4\) bounding the
        remainder term \(R(\beta,\rho_0,I)\) on \(D^+\) and \(D^-\)
        (respectively), then found values \(I_3, I_5\) which drive all orbits of
        ~\eqref{eqn:rhohat-discrete} towards \(\abs{\rho} = \bar{\rho}\).
\end{enumerate}
Recall from Section \ref{sec:acrobot-proof-o1} that there exists \(I^\star > 0\)
making orbits escape \(\bar{\mathcal{O}}_1\) into 
\(\mathcal{R}\). 
Choosing \(I \in \,]0,\min\{I_3,I_5,I^\star\}]\)
guarantees energy injection on \(\mathcal{O}_2(\bar{\rho})\).
Likewise, choosing \(I \in [-\min\{I_3,I_5,I^\star\},0[\) guarantees energy
dissipation on \(\mathcal{O}_2(\bar{\rho})\).

This analysis, together with that of section \ref{sec:acrobot-proof-o1},
completes the proof of Theorem \ref{thm:acrobot-energy-stabilization}.

% Draw the proof square only
\begin{proof}[\unskip\nopunct]
\end{proof}

\section{Experimental Results}
In this section, we test our VNHC on the physical acrobot built by
\citet{xingbo_thesis} (Figure \ref{fig:xingbo-acrobot}).
Since it is not a simple acrobot, this experiment will test whether VNHCs are
robust to model uncertainties.
As we will see, our VNHC can inject energy on the general acrobot model.

\begin{figure}
    \centering
    \includegraphics[width=0.5\linewidth]{images/xingbo_acrobot.png}
    \caption{The acrobot built by Wang \cite{xingbo_thesis}.}
    \label{fig:xingbo-acrobot}
\end{figure}

Wang's acrobot  has inertia matrix as in ~\eqref{eqn:general-acrobot-inertia}
and potential function as in ~\eqref{eqn:general-acrobot-potential}.
The physical parameters are written explicitly in Table
\ref{tab:acrobot-parameters}.

\begin{table}
    \centering
    \caption{Physical parameters for the real acrobot, as measured by
    \citet{xingbo_thesis}.}
    \label{tab:acrobot-parameters}
    \begin{tabular}{ccccccccc}
        \toprule
        $m_t$ & $m_l$ & $d_t$ & $d_l$ & $l_t$ & $l_l$ & $J_t$ & $J_l$ & $g$ \\
        \midrule
        0.2112 & 0.1979 & % mass
        0.148 & 0.145 & % total length
        0.073 & 0.083 & % COM distance
        0.00129 & 0.00075 & % moment of inertia
        9.81 \\ % gravity
        \bottomrule
    \end{tabular}
\end{table}

This acrobot's actuator is a servo motor with a built-in PID controller, so we
can set \(q_a = \arctan\left(I p_u\right)\) directly without needing to solve
for the VNHC's torque input \(\tau\).
The control parameter \(I\) needs to be small, but it also needs to be large
enough to overcome friction in the servo motor.
Since the acrobot's mass and length values are small, its momentum is
also small.
Through various experiments, we determined that \(I = 10\) is sufficient to
ensure the actuator can fully rotate in its allowable range
\(q_a \in \left[ -\frac{\pi}{2}, \frac{\pi}{2}\right]\).

\subsection{Simulations}

To compute the energy of the nominal pendulum, we evaluate the system at the
VNHC \(q_a = 0\), which yields
\[
    E(q_u,p_u) \approx 396.5501 p_u^2 + 0.5997(1 - \cos(q_u))
    .
\]
Hence, the level set of \(E_\pi\) of the energy \(E(\pi,0)\) is parameterized by
\begin{align*}
    p_u &\approx \pm \sqrt{\frac{1.1994}{793.1001}(2 - (1 - \cos(q_u))}
    ,
\end{align*}
for each \(q_u \in ]-\pi,\pi]\).

\begin{figure}[ht]
    \centering
    \includegraphics[width=0.8\linewidth]{images/acrobot_orbit.png}
    \caption{A simulation of the acrobot from \cite{xingbo_thesis}.}
    \label{fig:acrobot-orbit}
\end{figure}

We initialize the simulation for this acrobot at 
\((q_u,p_u) = \left(\frac{\pi}{32},0 \right)\) and plot the resulting orbit in 
Figure \ref{fig:acrobot-orbit}.
The nominal energy level set \(E_\pi\) is outlined in black. 
Because this acrobot is not simple, \(E_\pi\) is not the boundary between
rotations and oscillations.
The oscillation domain is actually much larger: orbits rotate once they hit the
\(p_u\)-axis at \(\abs{p_u} \approx 0.15\), while \(E_\pi\) intersects the
\(p_u\)-axis at \(\abs{p_u} \approx 0.055\).
Despite this, we see that eventually the acrobot remains outside \(E_\pi\).
The points where the orbit exits \(E_\pi\) are highlighted with black stars,
with the final departure highlighted in red.
At this final departure, the acrobot begins rotating and continues to gain
energy over time.

To verify that the acrobot will consistently begin rotating, we use a Monte
Carlo method \cite{montecarlo} to simulate this
acrobot \(1000\) times.
At each iteration, we initialize the acrobot randomly inside the sublevel set
\[
    \left\{(q_u,p_u) \in \SxR \mid
    E(q_u,p_u) \leq E\left(\frac{\pi}{32},0\right)\right\}
    ,
\] 
and measure how long it takes the acrobot to begin rotating.
The results are plotted in Figure \ref{fig:acrobot-mc}.
The acrobot always rotates within 10--35 seconds.

\begin{figure}
    \centering
    \includegraphics[width=0.8\textwidth]{images/acrobot_mc.png}
    \caption{Monte Carlo simulation results for the acrobot from
    \cite{xingbo_thesis}.}
    \label{fig:acrobot-mc}
\end{figure}

\subsection{Physical Experiments}
To complete this chapter, we perform a four experiments with the physical
acrobot:
\begin{enumerate}
    \item An undisturbed test to validate that the acrobot reaches rotations.
    \item A test where we stop the acrobot once it has started rotating.
    \item A test where we push the acrobot in its direction of motion after it has
        started rotating.
    \item A test where we push the acrobot against its direction of motion.
\end{enumerate}

The acrobot has an encoder at the pivot which measures \(q_u\) and
\(\dot{q}_u\), and the servo provides a measurement of \(q_a\).
We estimate \(\dot{q}_a\) through sequential values of \(q_a\) and
compute \(p_u\) through \(p_u = e_1^T M(q) \dot{q}\).
The actuator value at iteration \(k \in \mathbb{Z}_{> 0}\) is assigned through
the equation
\[
    q_a^{k} = \arctan(I p_u^{k-1})
    .
\]

\subsubsection*{Test 1: No Disturbances}
For this test we initialize the acrobot at 
\((q_u,p_u) \approx \left(\frac{\pi}{8},0\right)\). 
The resulting orbit is shown in Figure \ref{fig:acrobot-unperturbed-orbit},
which is clearly gaining energy over time.

\begin{figure}[ht]
    \centering
    \includegraphics[width=0.5\linewidth]{images/acrobot_unperturbed_orbit.png}
    \caption{The orbit of the physical acrobot during an unperturbed test.}
    \label{fig:acrobot-unperturbed-orbit}
\end{figure}

\subsubsection*{Test 2: Stopping the Acrobot}
For this test we initialize the acrobot at 
\((q_u,p_u) \approx \left(\pi,0\right)\) and let it run for 15
seconds, then stop the acrobot as it reaches the bottom of its arc.
The resulting orbit is shown in Figure \ref{fig:acrobot-stopped-orbit}.
The blue curves correspond to the orbit before the disturbance, while the red
spiral shows that the acrobot continues to oscillate after it is stopped.
Despite the disturbance, it gains energy and eventually reaches a rotation
again.

\begin{figure}[ht]
    \centering
    \includegraphics[width=0.5\linewidth]{images/acrobot_stopped_orbit.png}
    \caption{The orbit of the physical acrobot before (blue) and after (red) it
    is stopped.}
    \label{fig:acrobot-stopped-orbit}
\end{figure}

\subsubsection*{Test 3: Pushing the Acrobot Forwards}
To see how the acrobot responds when pushed in its direction of motion, we allow
the acrobot to rotate freely for 15 seconds and then give it a push in its
direction of motion.
The orbit in Figure \ref{fig:acrobot-fpush-orbit} shows that the acrobot speeds
up to rotate with energy \(E(0,0.22)\), but then slows down until it reaches a
stable rotation with energy \(E(0,-0.195)\).

\begin{figure}[ht]
    \centering
    \includegraphics[width=0.5\textwidth]{images/acrobot_fpush_orbit.png}
    \caption{The orbit of the physical acrobot before (blue) and after (red) it
    is pushed in its current direction of motion.}
    \label{fig:acrobot-fpush-orbit}
\end{figure}

\subsubsection*{Test 4: Pushing the Acrobot Backwards}
In this final experiment, we test whether the acrobot can easily change
directions when pushed against its current direction of motion.
We again allow the acrobot to rotate freely for 15 seconds, then push it the
opposite way.
The orbit in Figure \ref{fig:acrobot-rpush-orbit} demonstrates that the acrobot
responds by readily changing direction, and quickly achieves its maximum speed
with energy \(E(0,0.195)\).

\begin{figure}[ht]
    \centering
    \includegraphics[width=0.5\textwidth]{images/acrobot_rpush_orbit.png}
    \caption{The orbit of the physical acrobot before (blue) and after (red) it
    is pushed against its current direction of motion.}
    \label{fig:acrobot-rpush-orbit}
\end{figure}

\subsection{Summary of Experimental Results}
The simulations and experiments in this section show that VNHCs are excellent
tools for injecting energy even in non-simple acrobots. 
Furthermore, the experimental results show that the energy gain is robust
against a variety of disturbances.
Finally, the two push tests suggest that Wang's acrobot constrained by our VNHC
will gain energy on \(\mathcal{O}_2(0.195)\).

%/========== /Acrobot ==========/%
% vim: set tw=80 ts=4 sw=4 sts=0 et ffs=unix :
