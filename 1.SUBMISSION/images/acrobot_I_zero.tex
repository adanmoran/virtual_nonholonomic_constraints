%! TEX program = lualatex

% Use :VimtexToggleMain to compile this file alone with Vimtex
\documentclass[tikz]{standalone}

\usepackage{sty/adantikz}

\begin{document}
\begin{tikzpicture}
    % Define coordinates
    \coordinate (origin) at (0,0);
    % Define our lengths, etc for the acrobot
    \pgfmathsetmacro{\len}{3};
    \pgfmathsetmacro{\qu}{30};
    \pgfmathsetmacro{\qa}{0};
    \pgfmathsetmacro{\quc}{\len*cos(\qu)};
    \pgfmathsetmacro{\qus}{\len*sin(\qu)};
    \pgfmathsetmacro{\qac}{\len*cos(\qa+\qu)};
    \pgfmathsetmacro{\qas}{\len*sin(\qa+\qu)};
    % Positions of the VLP representation
    \pgfmathsetmacro{\xe}{\len*(sin(\qu)+0.5*sin(\qu+\qa))};
    \pgfmathsetmacro{\ye}{-\len*(cos(\qu)+0.5*cos(\qu+\qa))};
    % The base of the pivot
    \draw[thick] (-2,0) coordinate (leftwall) -- (2+\len,0) coordinate (rightwall);
    \fill[black] (origin) circle (0.05);
    % Fill a cross-hatched wall above the pivot
    \fill[pattern=north east lines] (leftwall) rectangle ($(rightwall) + (0,0.1)$);
    % The gravity arrow
    \draw[draw=gray, ->] ($(leftwall) + (0.5,-0.5)$) -- ($(leftwall) +(0.5,-1)$) node [right,black] {$g$};

    % The dashed line at the vertical of the origin
    \draw[draw=gray, dashed] (origin) -- ++(0,-2) coordinate (vert);

    % The first link as a thick rod
    \draw[draw=black, ultra thick] (origin) -- 
        ($(origin) + (\qus,-\quc)$) coordinate (m1);
    \fill[black] (m1) circle (0.05);

    % The second link
    \draw[draw=black, ultra thick] (m1) -- 
        ($(m1) + (\qas,-\qac)$) coordinate (m2);
    % The effective COM
	\coordinate (me) at ($(origin) + (\xe,\ye)$);

    % The angle q_u
    \pic [draw,->, "$q_u$", angle eccentricity = 1.2,angle radius=25, pic text
    options={shift={(-2pt,-2pt)}}] {angle=vert--origin--me};

    % The first center of mass
    % label uses {angle:label_name}
    \node[draw=none, rotate=\qu, inner sep = 0pt, label={[label
        distance=-2pt]\qu+180:$m$}] at (m1) 
    {\centerofmass};
    % The second center of mass
    \node[draw=none, rotate=\qa+\qu, inner sep=0pt, label={[label distance=-1pt]110:$m$}] at (m2)
        {\centerofmass};
\end{tikzpicture}
\end{document}
% vim: set tw=80 ts=4 sw=4 sts=0 et ffs=unix :
