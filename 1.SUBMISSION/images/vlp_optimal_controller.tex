%! TEX program = lualatex

% Use :VimtexToggleMain to compile this file alone with Vimtex
\documentclass[tikz]{standalone}

\usepackage{sty/adantikz}

\begin{document}
\begin{tikzpicture}
% Set the underline{l} and overline{l} positions
\pgfmathsetmacro{\lu}{2}
\pgfmathsetmacro{\lb}{4}

\begin{axis}[ 
 axis line style = thick,
 % Set the axes to be centered nicely
 axis x line = center, 
 axis y line = middle,
 xlabel = $\theta$,
 x label style={at={(axis description cs:1.08,0)},anchor = east},
 xmin = -3.1415, xmax = 3.1415,
 ylabel = $l$,
 y label style={at={(axis description cs:0.5,1.1)},anchor = north}, 
 ymin = 0, ymax = 5,
 % Custom tick labels at specified positions
 xticklabels = {$-\pi$, $-\pi/2$, $0$, $\pi/2$, $\pi$},
 xtick={-3.1415, -1.571, 0, 1.571, 3.1415},
 yticklabels = {$\underline{l}$,$\overline{l}$},
 ytick = {\lu,\lb}, % Choose the plot values for underline{l} and overline{l}
 % Move the \overbar{l} label to the right
 yticklabel style={xshift={(\ticknum == 1)*(0.5cm)}}
]
    % Draw the l(theta) function 
    \addplot[red,domain=-pi:(-pi/2),smooth]{\lu};
    \addplot[blue,domain=(-pi/2):0,smooth]{\lb};
    \addplot[red,domain=0:(pi/2),smooth]{\lu};
    \addplot[blue,domain=(pi/2):pi,smooth]{\lb};
    % Draw the vertical dashed lines at theta = +- pi/2
    \draw[dashed, color=gray!40] (-pi/2,0) -- (-pi/2,5);
    \draw[dashed, color=gray!40] (pi/2,0) -- (pi/2,5);
\end{axis}
\end{tikzpicture}
\end{document}
% vim: set tw=80 ts=4 sw=4 sts=0 et ffs=unix :
