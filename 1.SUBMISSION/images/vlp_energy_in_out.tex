%! TEX program = lualatex

% Use :VimtexToggleMain to compile this file alone with Vimtex
\documentclass[tikz]{standalone}

\usepackage{sty/adantikz}

\begin{document}
\begin{tikzpicture}
% Set the underline{l} and overline{l} positions
\pgfmathsetmacro{\lu}{2}
\pgfmathsetmacro{\lb}{4}
\pgfmathsetmacro{\dl}{(\lb - \lu)/2}
\pgfmathsetmacro{\lavg}{(\lb + \lu)/2}
% Define the figure
\begin{axis}[ 
 axis line style = thick,
 % Set the axes to be centered nicely
 axis x line = center, 
 axis y line = middle,
 xlabel = $\theta$, 
 x label style={at={(axis description cs:1.08,0)},anchor = east},
 xmin = -3.1415, xmax = 3.1415,
 ylabel = $l$,
 y label style={at={(axis description cs:0.5,1.1)},anchor = north}, 
 ymin = 0, ymax = 5,
 % Custom tick labels at specified positions
 xticklabels = {$-\pi$, $-\frac{\pi}{2}$, $0$, $\frac{\pi}{2}$, $\pi$},
 xtick={-3.1415, -1.571, 0, 1.571, 3.1415},
 yticklabels = {$\underline{l}$,$l_{avg}$,$\overline{l}$},
 ytick = {\lu,\lavg,\lb}, % Choose the plot values for underline{l} and overline{l}
 % Move the \overbar{l} label to the right
 yticklabel style={xshift={(\ticknum==1)*(1cm)},yshift={(\ticknum==0)*(-0.25cm)+(\ticknum==1)*(0.25cm) + (\ticknum==2)*(0.25cm)}},
 legend pos=south east,
 axis on top
]
    % Add named boundaries for each of the blocked regions. We use "forget
    % plot" so they don't appear on the legend
    \addplot[draw=none, forget plot, name path global=I1U,domain=-pi:(-pi/2)]{\lu};
    \addplot[draw=none, forget plot, name path global=I2U,domain=(-pi/2):0]{\lu};
    \addplot[draw=none, forget plot, name path global=I3U,domain=0:(pi/2)]{\lu};
    \addplot[draw=none, forget plot, name path global=I4U,domain=(pi/2):pi]{\lu};

    \addplot[draw=none, forget plot, name path global=I1A,domain=-pi:(-pi/2)]{\lavg};
    \addplot[draw=none, forget plot, name path global=I2A,domain=(-pi/2):0]{\lavg};
    \addplot[draw=none, forget plot, name path global=I3A,domain=0:(pi/2)]{\lavg};
    \addplot[draw=none, forget plot, name path global=I4A,domain=(pi/2):pi]{\lavg};

    \addplot[draw=none, forget plot, name path global=I1B,domain=-pi:(-pi/2)]{\lb};
    \addplot[draw=none, forget plot, name path global=I2B,domain=(-pi/2):0]{\lb};
    \addplot[draw=none, forget plot, name path global=I3B,domain=0:(pi/2)]{\lb};
    \addplot[draw=none, forget plot, name path global=I4B,domain=(pi/2):pi]{\lb};
    % Fill in the region where the VNHC needs to be for injection in green
    \addplot[draw=none, forget plot, fill=green!40] fill between[of=I1U and I1A];
    \addplot[draw=none, forget plot, fill=green!40] fill between[of=I2A and I2B];
    \addplot[draw=none, forget plot, fill=green!40] fill between[of=I3U and I3A];
    \addplot[draw=none, fill=green!40] fill between[of=I4A and I4B];
    % Fill in the region where the VNHC needs to be for dissipation in yellow
    \addplot[draw=none, forget plot, fill=yellow!40] fill between[of=I1A and I1B];
    \addplot[draw=none, forget plot, fill=yellow!40] fill between[of=I2U and I2A];
    \addplot[draw=none, forget plot, fill=yellow!40] fill between[of=I3A and I3B];
    \addplot[draw=none, fill=yellow!40] fill between[of=I4U and I4A];

    % Draw the horizontal lines at lu, lb
    \draw[dashed, color=red] (-pi,\lu) -- (pi,\lu);
    \draw[dashed, color=blue] (-pi,\lb) -- (pi,\lb);
    % Draw the horizontal line at lavg
    \draw[dashed, color=black] (-pi,\lavg) -- (pi,\lavg);
    % Draw the vertical lines at +- pi/2
    \draw[dashed, color=gray!40] (-pi/2,0) -- (-pi/2,5);
    \draw[dashed, color=gray!40] (pi/2,0) -- (pi/2,5);

    \legend{injection, dissipation}; 
\end{axis}
\end{tikzpicture}
\end{document}
% vim: set tw=80 ts=4 sw=4 sts=0 et ffs=unix :
