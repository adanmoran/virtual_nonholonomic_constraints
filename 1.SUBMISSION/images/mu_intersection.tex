%! TEX program = lualatex

% Use :VimtexToggleMain to compile this file alone with Vimtex
\documentclass[tikz]{standalone}

\usepackage{sty/adantikz}
%TODO: Add a figure with one oscillation mark at a distance mu from
    % the origin to the intersection on the q-axis. We also draw the boundary of
    % mathcalO and shade the interior, and draw a line to the oscillation with
    % clockwise angle alpha

\begin{document}
\begin{tikzpicture}
    % Define the value of m2gl3 making sqrt(60m2gl3)=15
    \pgfmathsetmacro{\mgl}{15^2/60};
    % mu intersection with qu axis
    \pgfmathsetmacro{\mux}{4*pi/6};
    % orbit's intersection with pu axis
    \pgfmathsetmacro{\muy}{sqrt(30*\mgl*(cos(0)-cos(deg(\mux))))};
    % x coordinate of where arrow to orbit points
    \pgfmathsetmacro{\xalpha}{pi/2};
    % y coordinate of where arrow to orbit points
    \pgfmathsetmacro{\yalpha}{-sqrt(30*\mgl*(cos(deg(\xalpha))-cos(deg(\mux))))};

% Define the figure
\begin{axis}[
 axis line style = thick,
 % Set the axes to be centered nicely
 axis lines  = center,
 % Turn off arrows
 axis line style={-},
 % Set the axes
 xlabel = $q_u$, 
 x label style={right},
 xmin = -3.5, xmax = 3.5,
 ylabel = $p_u$,
 y label style={above},
 ymin = -20, ymax = 20,
 % Custom tick labels at specified positions
 xtick = {-3.1415, \mux, 3.1415},
 xticklabels = {$-\pi$, $\mu$, $\pi$},
 xticklabel style={xshift={(\ticknum==0)*(-0.3cm) + (\ticknum==1)*(-0.25cm) + (\ticknum==2)*(0.25cm)}},
 ytick = {-15, 15},
 yticklabels = {$-\sqrt{60m^2gl^3}$,$\sqrt{60m^2gl^3}$},
 % Move the label so it looks good
 yticklabel style={yshift={(\ticknum == 0)*(-0.3cm) + (\ticknum == 1)*(0.25cm)}}
]
    % Define coordinates for drawing alpha
    \coordinate (origin) at (0,0);
    \coordinate (pizero) at (pi,0);
    \coordinate (alphapoint) at (\xalpha, \yalpha);

    % Colour in the region for oscillations
    \addplot[black,->,dashed,thick,name path=upper,samples=100,domain=-pi:pi]
        {sqrt(30*\mgl*(cos(deg(x))+1))};
    \addplot[black,<-,dashed,thick,name path=lower, samples=100,domain=-pi:pi]
        {-sqrt(30*\mgl*(cos(deg(x))+1))};
    \addplot[fill=blue,opacity=0.2] 
        fill between[of=upper and lower];

    % Draw an oscillation contained entirely in the oscillation region, with the
    % intersection point of the q_u axis labelled mu
    \addplot[red!70,thick,samples=300,domain=-\mux:\mux]
        {sqrt(30*\mgl*(cos(deg(x))-cos(deg(\mux))))};
    \addplot[red!70,thick,samples=300,domain=-\mux:\mux]
        {-sqrt(30*\mgl*(cos(deg(x))-cos(deg(\mux))))};


    % Draw an arrow (clockwise, bottom right) with angle alpha pointing to the
    % oscillation orbit.
    \draw[black, ->] (0,0) -- (alphapoint);
    \pic [draw, <-, "$\alpha$", angle eccentricity=1.5]
    {angle=alphapoint--origin--pizero};

    % Draw vertical dashed lines at +-pi
    \draw[gray, dashed] (-pi,-20) -- (-pi,20);
    \draw[gray, dashed] (pi,-20) -- (pi,20);

\end{axis}
\end{tikzpicture}
\end{document}
% vim: set tw=80 ts=4 sw=4 sts=0 et ffs=unix :
