\documentclass[11pt,letter]{article}
\usepackage{amsfonts}
\usepackage{amsmath}
\usepackage{amssymb}
\usepackage{amstext,pdfsync}
\usepackage[dvips]{color}
\usepackage{tcolorbox}
\usepackage{mathpazo}
\usepackage{amssymb,xspace,enumerate}
\definecolor{LineColor}{rgb}{0.4,0.4,0.4}


%--------------------------------------------------%
\definecolor{shadecolor}{gray}{.9}
\definecolor{darkblue}{rgb}{0.0,0.0,0.3}
\newenvironment{myquote}{%
\bigskip%
\begin{tcolorbox}[]\footnotesize}{%
\end{tcolorbox}%			
\bigskip}

%\headsep 0 true cm
%\textwidth 15.0 true cm
%\textheight 24.0 true cm
%\topmargin 0 true cm
%\oddsidemargin 0 true cm
%\evensidemargin 0 true cm
%\headsep 0 true cm
\usepackage{geometry}

%\setlength{\parindent}{2ex}

\newcommand*{\vhc}{\textsc{vhc}\xspace}
\newcommand*{\vhcs}{\textsc{vhc}s\xspace}
\newcommand*{\vcg}{\textsc{vcg}\xspace}
\newcommand*{\vcgs}{\textsc{vcg}s\xspace}
\newcommand*{\R}{\mathbb{R}}
\newcommand*{\Q}{\mathcal{Q}}
\newcommand*{\X}{\mathcal{X}}
\newcommand*{\specone}{\textsc{spec 1}\xspace}
\newcommand*{\spectwo}{\textsc{spec 2}\xspace}
\newcommand{\NC}{\textsc{nlc}\xspace}
\DeclareMathOperator{\Image}{Im}



\newcommand{\drawline}
{   \hspace*{-\marginparsep minus \marginparwidth}
	\begin{minipage}[t]{\textwidth+\marginparwidth+\marginparsep}%
		{\large \bfseries {} \hfill {}}\\[-0.15\baselineskip]%
		\textcolor{LineColor}{\rule{\columnwidth}{2 pt}}%}
	\end{minipage}}

\newcommand{\myver}{\vspace{0.5cm}}

\newcommand{\mybox}[1]{{\color{blue}\fbox{#1}}}

%\include{Bibtex_DistTeleop}

\title{\vspace{-1cm} {\Large \bf \color{darkblue}{Reply to the Associate
      Editor and Reviewers}\\[0.6cm] \large {\bf From gymnastics to virtual nonholonomic constraints: energy
injection, dissipation, and regulation for the acrobot
      }\\[0.2cm] {(Revision of  IEEE TCST 22-0089)\\ A.~Moran-MacDonald, M.~Maggiore, X.~Wang}}} \author{}
\date{\today}

\begin{document}
\vspace{-4cm}
\maketitle

%\drawline
\pagenumbering{gobble}

\indent We sincerely thank the Associate Editor for collecting two thoughtful reviews and for the 
constructive remarks on this submission. We have revised the paper addressing the reviewers'
comments. Below are detailed replies to the Associate Editor and the reviewers.
\pagenumbering{arabic}
%---------------------------------------------------------------------------%
%\vspace{-1cm}
\subsection*{\textcolor{darkblue}{Reply to the Associate Editor}} 
\vspace{0cm}



\begin{myquote}
We have received only two reviews for this paper, which are - however -
sufficiently informative for a recommendation  to be  issued. Overall,
the reviewers praise the high quality of the paper, but request that a
sharper focus is placed on its contribution with respect not only on
the literature on virtual nonholonomic constraints, but the  landscape
of energy-based nonlinear control. While this may appear a daunting
task, the authors are invited to put their work into perspective as
suggested by reviewer no.2.  The remaining issues appear to be well
within the authors' reach.  
\end{myquote}

In this revision, we have significantly expanded the introduction to include a \emph{detailed} overview of the literature on orbital stabilization mentioned by reviewer 2, contrasting our ideas to existing ones. We have also included a new subsection titled ``Contributions of this article'' in which we better showcase the main results of the paper.  We have also addressed all other issues raised by the two reviewers.


\newpage
%---------------------------------------------------------------%
\vspace{0.2cm}
\subsection*{\textcolor{darkblue}{Reply to Reviewer 2}} 
\vspace{0.2cm}

We wish to sincerely thank the reviewer for the constructive comments,
which we have  addressed in this revision. Below are
detailed replies to the reviewer's remarks.



\begin{myquote}
The problem of designing a feedback control for the two-link acrobot
system through virtual nonholonomic constraints is addressed in this
paper. \((\cdots)\) 
The paper is technically correct and the proofs of Theorem 4 and
Theorem 5 are mathematically rigorous and consistent with the
statements.
\end{myquote}

We thank the reviewer for this assessment.

\begin{myquote}
However, my major concern is that neither the benefit nor
the novelty of considering stabilization by assigning virtual
nonholonomic constraints are clearly articulated. The literature on how
to design nonlinear controllers, possibly passive, for achieving either
swing-up and orbital stabilization or swing-up and asymptotic
stabilization is extensive. In terms of the acrobot, most of the
existing approaches rely on the construction of Casimir functions
(conserved quantities) to help shape the energy of the closed-loop
system via Control-by-Interconnection (CbI) as in [R9], [R8], the
potential energy shaping of the closed-loop system via Interconnection
and Damping Assignment (IDA-PBC) as in [R1], [R2], [R3], [R7], orbital
stabilization via Immersion and Invariance (I\&I) as in [R4], [R5], and
finally orbital stabilization via energy shaping and with the injection
of both pumping and damping as in [R6]. As an application-based
research paper, I would have expected an application of established
research findings and a comparison with other existing approaches. In
addition to showing the effectiveness of implementing virtual
nonholonomic constraints, it would be valuable to discuss the benefit
of the method when compared with the approaches mentioned beforehand. 
\end{myquote}


The reviewer makes a good point, and we completely agree that not placing our paper in a wider context was an omission. This revision \emph{significantly expands} the literature treatment and placed our work in a wider context. The introduction has been significantly expanded with a new subsection titled ``Related literature'' in which we review other techniques for orbital stabilization: I\&I, IDA, VHC-based, and the Fantoni-Lozano approach. We also stress the differences between these approaches and the ideas of this submission.

\begin{myquote}
Furthermore, the following issues should be addressed:

1.	Please define q before the Hamiltonian function given in by the
equations (6). I assume that the generalized coordinate q is composed
of (qu, qa), however, the structure of q is not defined anywhere in the
manuscript.
\end{myquote}

Done, and thank you for raising this point.

\begin{myquote}
 2.	Please enumerate the Hamiltonian function H(q,p) in (6) with
(6a) and the dynamics associated with H(q, p) with (6b). It is quite
uncommon to write "The dynamics of the acrobot in (q,p) coordinates are
given in (6)î. I would suggest, instead, writing îThe dynamics of the
acrobot with Hamiltonian function (6a) are given in (6b)î.  
\end{myquote}

Done, and thank you.

\begin{myquote}
3.	The manuscript deals with the acrobot model, which yields a
Hamiltonian system described by an inhomogeneous differential equation
with a constant input matrix. What is the benefit of introducing the
general Hamiltonian system as in (8) with a nonconstant input matrix B
and then assuming B constant? For ease of presentation, I would suggest
considering directly the inhomogeneous Hamiltonian system used in the
paper.  
\end{myquote}

Done. As suggested by the reviewer, we now directly consider a model with \(B\) constant.


\begin{myquote}
4.	If Lemma 1 and Theorem 1 are recalled from another paper,
please cite the paper which contains the statements and proofs, and
remove the proofs from this paper. Otherwise, if Lemma 1 and Theorem 1
are original results or enhance some other previous findings, then keep
the proofs but emphasise the originality of the results.  
\end{myquote}

We did not take Lemma 1 and Theorem 1 from the literature. However,
the results are relatively obvious and for this reason we do not
showcase them. We clarified this point in a footnote.

\begin{myquote}
5.	Please restate Definition 2. The expression îthe output e =
h(q,p) is of relative degree 2, 2, ..., 2î is unusual. The output is
not of relative degree, rather the system associated with the output e
= h(q,p) has a relative degree. In addition, relative degree 2, 2, ...,
2 is unclear. I think the authors wanted to say that the system has
relative degree 2 for each e\_i.
\end{myquote}

Fixed, thank you. The correct term is \emph{vector relative degree}. We also added a reference to the book by Isidori where this notion is defined.

\begin{myquote}
6.	From Definition 2 results that, if you define the output as e =
h(q,p), where h satisfies the relation h(q,p) = 0, you are looking for
a special characterization of the zero dynamics of your closed-loop
system. This approach reminds me of the construction of the invariant
manifold used to construct a stabilizer through I\&I [R5]. Is there any
relation between the two?  
\end{myquote}

In this revision we have given a detailed comparison between the  I\&I technique and the approach proposed in this paper. Both approaches involve the stabilization of a submanifold of the state space on which one has ``desirable'' behaviour. While in the I\&I literature related to orbital stabilization one assumes that the dynamics on the submanifold are those of an oscillator, in this paper we show that the dynamics on the constraint manifold induced by the proposed VNHC are ``expansive'' or ``contractive'' in that orbits gain energy or lose energy, depending on the sign of a design parameter. In particular the dynamics on our constraint manifold do not contain closed orbits. 
\begin{myquote}
  7.	The output e = h(q,p), which is here assumed relative degree 2,
rules out the most common passive outputs because the relative degree
of the passive output is at most 1. Hence, the virtual nonholonomic
constraints seem to be weak compared to the IDA-PBC approach in [R6].
Please, clarify their relationship. Moreover, the original results in
Section IV.A also seems to have much in common with the
pumping-and-damping technique in [R6].
\end{myquote}

Having relative degree two rather than one does not imply any weakness per se. There is indeed a philosophical  similarity between pumping and damping and the ideas of this submission. This is clarified in the revised introduction. Despite the philosophical similarity, the two techniques are very different and we point this out.


\begin{myquote}
8.	In the sentence before equation (14), what type of stability do
you refer to? 
\end{myquote}

We were referring to asymptotic stability of the constraint manifold. We have modified the language to make this clear.

\begin{myquote}
9.	As I have already mentioned, the subsection îComparison with
existing literatureî only discusses previous work on virtual
nonholonomic constraints and a comparison with other consolidated
methods such as I\&I, CbI, and IDA-PBC is missing. Moreover, a
comparison should be given also in terms of simulations showing the
effectiveness of the virtual nonholonomic constraints when compared
with the existing methods.   
\end{myquote}


We have addressed this point in great detail. We have not however included a simulation comparison because for it to be meaningful, the comparison would require  detailed discussions and an overview of the various techniques  which are beyond the scope of this paper. We hope that the detailed comparison provided in the introduction will at least partially address the reviewer's concern.



\begin{myquote}
10.	Some general comments: the introduction should be improved by
motivating and properly discussing the main results achieved within the
paper; it would be preferable to conclude the introduction with the
organization of the paper; the reading of the article would be easier
if the problem formulation is presented after the preliminary section.  
\end{myquote}

We agree. We have added two new subsections titled ``Contributions of this paper'' and ``Paper organization'' in which we do exactly what the reviewer recommended.

\newpage
%---------------------------------------------------------------------------------%
\vspace{0.2cm}
\subsection*{\textcolor{darkblue}{Reply to Reviewer 4}}
\vspace{0.2cm}


We wish to sincerely thank the reviewer for the constructive comments. Below are
detailed replies to the reviewer's remarks.


\begin{myquote}
My main concern: The topic of virtual (nonlinear) nonholonomic
constraints has its roots in the works of Grizzle and Griffin in CDC
2015. Since then that notion was used and employed in a very
interesting context as the one in the present paper. Nevertheless, a
mathematical foundation of the subject as happens with virtual
holonomic constraints (done by the second author of the paper under
review) was not provided yet. Some effort has been done recently to
formulate virtual nonholonomic constraints from a geometric framework
in the linear (https://arxiv.org/abs/2207.01299) and affine cases
(https://arxiv.org/abs/2301.03890) that I guess they merit to be
mentioned in the literature review of the paper.
\end{myquote}

Thank you very much for flagging this paper. We are now citing it and discussing it both in the introduction and in Section IIIB on virtual nonholonomic constraints.



\begin{myquote}
  My recommendation is based on the next aspects of the paper:

(1)The paper is mathematically correct and self-contained.

(2)The paper is accessible to a wide range of researchers in control
and robotics

(3)The paper demonstrates the theory with real-life experimental.

(4) The experiments are understandable for a wide range of researchers
in control and robotics

(5) The paper is original and the topic is on current research.

I recommend acceptance of the manuscript.
\end{myquote}


We sincerely appreciate the positive assessment of our work.
\end{document}
